\documentclass[12pt]{article}
\usepackage[czech]{babel}
\usepackage[utf8]{inputenc}
\usepackage{amssymb}
\usepackage{amscd,amsmath}
\usepackage{amssymb,amsmath,latexsym}
\DeclareMathAlphabet{\bsf}{OT1}{cmss}{bx}{n}

\newenvironment{poznamka}{\par\begingroup\setlength{\leftskip}{2cm}\setlength{\parindent}{0pt}}{\par\endgroup}

\usepackage{graphics}
\usepackage[all]{xypic}
\newcommand{\boldgreek}[1]{\mbox{\boldmath$#1$}} 
\renewcommand{\labelenumi}{{\rm(\theenumi)}}

\begin{document}

\def\sint{\text{(II)}\!\!\!\int}    \def\fint{\text{(I)}\int}
\def\setof#1#2{\{#1 \, |\, #2\} }  \def\set#1{\{#1 \} } \def\ve#1{\bsf#1}
\def\const{\text{\sf const}}

\def\non{\text{\rm non}}

\def\pad#1#2{\frac{\partial#1}{\partial#2}}
\def\padd#1#2#3{\frac{\partial^2#1}{\partial#2\partial#3}}
\def\padr#1#2#3#4{\frac{\partial^{r}#1}{\partial#2\partial#3\dots\partial#4}}

\def\der#1#2{\frac{\d #1}{\d #2}}

\def\vol{\text{\sf vol}}

\def\rrar{\rightrightarrows}
\def\arctg{\text{arctg}}
\def\cotg{\text{\rm cotg}\,}

\def\lint{\underline\int}   \def\uint{\overset{\_\_\_}{\int}}
\def\tg{\text{\rm tg}}
\def\epsilon{\varepsilon}
\def\ho{\!\rightarrow \!}
                      \def\bs{\bigvee\setof}
 \def\alg{\text{\sf Alg}}
 \def\typ{\Delta=(\Delta_t)_{t\in T}}
  \def\typm{(\Delta_t)_{t\in T}}
        \def\rel{\text{\sf Rel}}
\def\ineq{\underset{{\mbox{$=$}}}{\in}}
\def\edm{\text{\sf End}}

\def\Rbb{\mathbb R}

\def\Ebb{\mathbb E}

\def\Cbb{\mathbb C}

\def\Nbb{\mathbb N}
\def\Qbb{\mathbb Q}
\def\Zbb{\mathbb Z}



\def\aut{\text{\sf Aut}}
 \def\nr#1{\hbox{$|\!|#1{|\!|}$}}
\def\ho{\!\rightarrow \!}
                      \def\bs{\bigvee\setof}
 \def\alg{\text{\sf Alg}}
 \def\typ{\Delta=(\Delta_t)_{t\in T}}
  \def\typm{(\Delta_t)_{t\in T}}
        \def\rel{\text{\sf Rel}}
\def\ineq{\underset{{\mbox{$=$}}}{\in}}
\def\edm{\text{\sf End}}


\def\aut{\text{\sf Aut}}

\def\open{\text{\sf Open}}



                    \def\id{\text{\rm id}}     \def\Id{\text{\rm Id}}
\def\dr{\downarrow\!\! }
\def\ur{\uparrow\!\! }   \def\sq{$\quad\square$}
\def\mmap{\multimap}
\def\c{^{\text{\sf c}}}
\def\2{\text{\bf 2}}
     
\def\sue{\subseteq}\def\supe{\supseteq}     \def\wt{\widetilde}  
   \def\wh{\widehat}
\def\ems{\emptyset}         
\def\m{\wedge}   \def\qtq#1{\quad\text{#1}\quad}

\def\pprec{\hbox{$\prec\!\!\prec$}}

\def\Ds{\text{\sf D}}

\def\I{\mathfrak I} \def\impl{\Rightarrow}
\def\Pf{\mathfrak P_{\text{fin}}}


\def\M{\mathcal M}

\def\Ibb{\Bbb I}

\def\inff{\text{\sf Inf}}
\def\P{\mathfrak P} 
\def\bvd{\bigvee\!\!\!{}_{\text{\rm dir}}}
\def\bcd{\bigcup\!{}_{\text{\rm dir}}}

\def\US{\vartriangleleft}  \def\D{\mathcal D}     \def\J{\mathcal J}     
\def\USf{\vartriangleleft\!\!}  

\def\precf{\prec\!\!}  

\def\B{\mathfrak B} \def\AF{\mathfrak A}

\def\BF{\mathfrak B}
\def\b{\mathfrak b}

\def\UC{\mathcal U}   \def\SC{\mathcal S}
\def\VC{\mathcal V}
\def\AC{\mathcal A} \def\BC{\mathcal B} 

\def\Pc{\mathcal P}  \def\Qc{\mathcal Q}



\def\ent{\dashv}   \def\entf{\dashv\!\!}

\def\nid{\noindent}
    
\def\ol{\overline}
              \def\smin{\setminus} 
\def\precs{\precsim}

\def\A{\mathcal A}

\def\ub{\text{\sf ub}}

\def\lb{\text{\sf lb}}

\def\di{\text{\sf diam}}

\def\on{\text{\sf Ord}}

\def\can{\text{\sf Card}}
\def\gen{\text{\sf Gen}}



                    \def\id{\text{\rm id}}     \def\Id{\text{\rm Id}}
\def\dr{\downarrow\!\! }
\def\ur{\uparrow\!\! }   \def\sq{$\quad\square$}
\def\mmap{\multimap}
\def\c{^{\text{\sf c}}}
\def\2{\text{\bf 2}}
     
\def\sue{\subseteq}\def\supe{\supseteq}     \def\wt{\widetilde}  
   \def\wh{\widehat}
\def\ems{\emptyset}         
\def\m{\wedge}   \def\qtq#1{\quad\text{#1}\quad}
\def\oin#1{(#1)}
\def\pprec{\hbox{$\prec\!\!\prec$}}
\def\uin#1{\langle#1\rangle}
\def\rin#1{\rangle#1\rangle}
\def\lin#1{\langle#1)}


\def\I{\mathfrak I} \def\impl{\Rightarrow}
\def\Pf{\mathfrak P_{\text{fin}}}

\def\HH{\text{\sf H}}

\def\SS{\text{\sf S}}

\def\PP{\text{\sf P}}

\def\M{\mathcal M}




\def\Ibb{\Bbb I}

\def\inff{\text{\sf Inf}}
\def\P{\mathfrak P} 
\def\bvd{\bigvee\!\!\!{}_{\text{\rm dir}}}
\def\bcd{\bigcup\!{}_{\text{\rm dir}}}

\def\US{\vartriangleleft}  \def\D{\mathcal D}     \def\J{\mathcal J}     
\def\USf{\vartriangleleft\!\!}  

\def\precf{\prec\!\!}  

\def\B{\mathfrak B} \def\AF{\mathfrak A}

\def\BF{\mathfrak B}
\def\b{\mathfrak b}

\def\UC{\mathcal U}
\def\VC{\mathcal V}
\def\AC{\mathcal A} \def\BC{\mathcal B} \def\Cc{\mathcal C}


\def\ent{\dashv}   \def\entf{\dashv\!\!}

\def\nid{\noindent}
    
\def\ol{\overline}
              \def\smin{\smallsetminus} 
\def\precs{\precsim}

\def\A{\mathcal A}

\def\ub{\text{\sf ub}}

\def\lb{\text{\sf lb}}

\def\diam{\text{\sf diam}}

\def\on{\text{\sf Ord}}

\def\can{\text{\sf Card}}
\def\gen{\text{\sf Gen}}

\def\ver#1{\text{\boldgreek#1}}

\thispagestyle{empty}

.
\vskip40mm




 \centerline{\huge\bf Kurs analysy} 
 
 \vskip7mm
 
 

 \centerline{\huge\bf pro informatiky} 
 
 \vskip20mm
 
 \centerline{\large\bf Aleš Pultr} 
 
 \newpage
 \thispagestyle{empty}

 .
 
 \newpage
 
 \renewcommand{\thepage}{\roman{page}}
 \setcounter{page}{1}
 {\Large\bf Obsah:}
 
 \vskip10mm
 
 \hskip20mm{\bf\large První semestr}
 
 \vskip10mm
 
 
 {\bf I. Úvodem} \hskip10mm 1
  
 \hskip5mm 1. Základy \hskip10mm 1
 
 \hskip5mm 2. Čísla \hskip10mm 4
 
 \hskip5mm 3. Reálná čísla jako (euklidovská) přímka \hskip10mm 10
 
 \bigskip
 
 {\bf II. Posloupnosti reálných čísel} \hskip10mm 13
 
 \hskip5mm 1. Posloupnosti a podposloupnosti \hskip10mm 13
 
 \hskip5mm 2. Konvergence. Limita posloupnosti  \hskip10mm 14
 
 \hskip5mm 3. Cauchyovské posloupnosti  \hskip10mm 17
 
 \hskip5mm 4. Spočetné množiny. Velikost posloupnosti
 
 \hskip10mm jako nejmenší nekonečná mohutnost  \hskip10mm 18
 
 \bigskip
 
 {\bf III. Řady} \hskip 10mm 23
 
 \hskip5mm 1. Sčítání posloupnosti jako limita částečných součtů  \hskip10mm 23
 
 \hskip5mm 2. Absolutně konvergentní řady \hskip10mm 24
 
 \hskip5mm 3. Neabsolutně konvergentní řady  \hskip10mm 27
 
 \bigskip
 
 {\bf IV. Spojité reálné funkce}  \hskip10mm 31
 
 \hskip5mm 1. Intervaly \hskip10mm 31
 
 \hskip5mm 2. Spojité reálné funkce jedné reálné proměnné  \hskip10mm 32
 
 \hskip5mm 3. Darbouxova věta \hskip10mm 34
 
 \hskip5mm 4. Spojitost monotonních a inversních funkcí \hskip10mm 36
 
 \hskip5mm 5. Spojité funkce na kompaktním intervalu \hskip10mm 37
 
 \hskip5mm 6. Limita funkce v bodě   \hskip10mm 39
 
 \bigskip
 
 {\bf V. Elementární funkce} \hskip10mm 43
 
 \hskip5mm 1. Logaritmus\hskip10mm 43
 
  \hskip5mm 2. Exponenciály  \hskip10mm 44
  
 \hskip5mm 3. Goniometrické a cyklometrické funkce \hskip10mm 46
 
 \bigskip
 
 {\bf VI. Derivace} \hskip10mm 51
 
  \hskip5mm 1. Definice a charakteristika \hskip10mm 51
  
\hskip5mm 2. Základní pravidla derivování \hskip10mm 53

 \hskip5mm 3. Derivace elementárních funkcí \hskip10mm 56

 \hskip5mm 4. Derivace jako funkce. Derivace vyšších řádů \hskip10mm 59
 
 \bigskip
 
 {\bf VII. Věty o střední hodnotě} \hskip10mm  61
 
  \hskip5mm 1. Lokální extrémy \hskip10mm 61
  
   \hskip5mm 2. Věty o střední hodnotě  \hskip10mm 62
   
    \hskip5mm 3. Tři jednoduché důsledky  \hskip10mm 64
    
    \bigskip
    
    {\bf VIII. Několik aplikací derivování} \hskip10mm 67
    
\hskip5mm 1. První a druhá derivace ve fysice \hskip10mm 67

 \hskip5mm 2. Hledání lokálních extrémů  \hskip10mm 68
 
  \hskip5mm 3. Konvexní a konkávní funkce \hskip10mm 68
  
   \hskip5mm 4. Newtonova metoda \hskip10mm 70
   
    \hskip5mm 5. L'H\^{o}pitalova pravidla \hskip10mm 73
    
     \hskip5mm 6. Kreslení grafů funkcí  \hskip10mm 78
     
      \hskip5mm 7. Taylorův polynom a zbytek  \hskip10mm 79
      
       \hskip5mm 8. Osculační kružnice. Křivost \hskip10mm 83
 
 
 
 \vskip15mm
 
 \hskip20mm{\bf\large Druhý semestr}
 
 \vskip10mm
 
 
 {\bf IX. Polynomy a jejich kořeny} \hskip10mm 85
 
 \hskip5mm 1. Polynomy \hskip10mm 85
 
 \hskip5mm 2. Základní věta algebry.
 Kořeny a rozklady polynomů \hskip10mm 86
 
 \hskip5mm 3. Rozklady polynomů s reálnými koeficienty \hskip10mm 88
 
 \hskip5mm 4. Součtové rozklady racionálních funkcí \hskip10mm 89
 
 \bigskip
 
 {\bf X. Primitivní funkce (neurčitý integrál)} \hskip10mm 93
 
 \hskip5mm 1. Obrácení derivace \hskip10mm 93
 
 \hskip5mm 2. Několik jednoduchých formulí  \hskip10mm 94
 
 \hskip5mm 3. Integrace per partes  \hskip10mm 96
 
 \hskip5mm 4. Substituční metoda  \hskip10mm 98
 
 \hskip5mm 5. Integrál racionální funkce \hskip10mm 99
 
 \hskip5mm 6. Několik standardních substitucí \hskip10mm 102
 
 \newpage
 
 {\bf XI. Riemannův integrál} \hskip10mm 105
 
 \hskip5mm 1. Obsah rovinného obrazce \hskip10mm 105
 
 \hskip5mm 2. Definice Riemannova integrálu \hskip10mm 106
 
 \hskip5mm 3. Spojité funkce  \hskip10mm 110
 
 \hskip5mm 4. Základní věta analysy \hskip10mm 112
 
 \hskip5mm  5. N\v ekolik jednoduch\'ych fakt \hskip10mm 113
 
 \bigskip
 
 {\bf XII. Několik aplikací Riemannova integrálu} \hskip10mm 117
 
  \hskip5mm 1.  Obsah rovinného obrazce, znovu \hskip10mm 117
  
   \hskip5mm 2. Objem rotačního tělesa \hskip10mm 118
   
    \hskip5mm 3. Délka rovinné křivky
    
    \hskip10mm  a povrch rotačního tělěsa  \hskip10mm 119
    
     \hskip5mm 4. Logaritmus \hskip10mm 129
     
      \hskip5mm 5. Integrální kritrium konvergence řady \hskip10mm 121
      
      \bigskip
      
{\bf XIII. Metrické prostory: základy} \hskip10mm 123

 \hskip5mm 1.  Příklad \hskip10mm 123
 
 \hskip5mm 2.  Metrické prostory, podprostory, spojitost \hskip10mm 123
 
 \hskip5mm 3.  Několik topologických pojmů \hskip10mm 126
 
 \hskip5mm 4.  Equivalentní a silně ekvivalentní metriky \hskip10mm 131
 
 \hskip5mm 5.  Produkty (součiny) \hskip10mm 132
 
 \hskip5mm 6. Cauchyovské posloupnosti. Úplnost\hskip10mm 134
 
 \hskip5mm 7.  Kompaktní metrické prostory \hskip10mm 135
 
 \bigskip
 
 {\bf XIV. Parciální derivace a totální diferenciál.
  
\hskip12mm Řetězové pravidlo} \hskip10mm 139
 
 \hskip5mm 1. Úmluva \hskip10mm 139
 
 \hskip5mm 2.  Parciální derivace \hskip10mm 140
 
 \hskip5mm 3. Totální diferenciál \hskip10mm 141
 
 \hskip5mm 4.  Parciální derivace vyšších řádů. Záměnnost \hskip10mm 145
 
 \hskip5mm 5.  Složené funkce a řetězové pravidlo \hskip10mm 147
 
 \bigskip
 
 {\bf XV. Věty o implicitních funkcích} \hskip10mm 153
 
 \hskip5mm 1.  Úloha \hskip10mm 153
 
 \hskip5mm 2. Jen jedna rovnice \hskip10mm 154
 
 \hskip5mm 3. Na rozcvičení: dvě rovnice \hskip10mm 157
 
 \hskip5mm 4.  Obecný případ \hskip10mm 159
 
 \hskip5mm 5.  Dvě jednoduché aplikace: regulární zobrazení \hskip10mm 162
 
 \hskip5mm 6.  Lokální extrémy a vázané extrémy\hskip10mm 164
 
 \bigskip
 
 {\bf XVI.  Riemannův integrál ve více proměnných} \hskip10mm 169
 
 \hskip5mm 1. Intervaly a rozklady \hskip10mm 169
 
 \hskip5mm 2. Horní a dolní součty, definice Riemannova integrálu
\hskip10mm 170
 
 \hskip5mm 3.  Spojitá zobrazení \hskip10mm 173
 
 \hskip5mm 4. Fubiniho věta \hskip10mm 174
 
 \vskip15mm
 
 \hskip20mm{\bf\large Třetí semestr}
 
 \vskip10mm
 
 
 {\bf XVII. Více o metrických prostorech} \hskip10mm 177
 
 \hskip5mm 1. Separabilita a spočetné base \hskip10mm 177
 
 \hskip5mm 2. Totálně omezené metrické prostory \hskip10mm 179
 
 \hskip5mm 3. Heine-Borelova věta  \hskip10mm 182
 
 \hskip5mm 4. Bairova věta o kategorii \hskip10mm 184
 
 \hskip5mm 5.  Zúplnění \hskip10mm 186
 
 \bigskip
 
 {\bf XVIII. Posloupnosti a řady funkcí} \hskip10mm 191
 
 \hskip5mm 1. Bodová a stejnoměrná konvergence  \hskip10mm 191
 
 \hskip5mm 2. Více o stejnoměrné konvergenci: 
 
 \hskip10mm derivace, Riemannův integrál \hskip10mm 192
 
 \hskip5mm 3. Prostor spojitých funkcí \hskip10mm 197
 
 \hskip5mm 4. Řady spojitých funkcí \hskip10mm 199
 
 \bigskip
 
 {\bf XIX. Mocninné řady} \hskip10mm 201
 
 \hskip5mm 1. Limes superior \hskip10mm 201
 
 \hskip5mm 2. Mocninná řada a poloměr konvergence \hskip10mm 202
 
 \hskip5mm 3. Taylorova řada \hskip10mm 205
 
 \bigskip
 
 {\bf XX. Fourierovy řady} \hskip10mm 211
 
 \hskip5mm 1. Periodické a po částech hladké funkce \hskip10mm 211
 
 \hskip5mm 2. Něco jako skalární součin \hskip10mm 213
 
 \hskip5mm 3. Dvě užitečná lemmata \hskip10mm 215
 
 \hskip5mm 4. Fourierova řada \hskip10mm 216
 
 \hskip5mm 5. Poznámky \hskip10mm 219
 
 \bigskip
 
 {\bf XXI. Křivky a křivkové integrály} \hskip10mm 221
 
 \hskip5mm 1. Křivky \hskip10mm 221
 
 \hskip5mm 2. Křivkové integrály \hskip10mm 224
 
 \hskip5mm 3. Greenova věta \hskip10mm 228
 
 \bigskip
 
 {\bf XXII. Základy komplexní analysy} \hskip10mm 233
 
 \hskip5mm 1. Komplexní derivace \hskip10mm 233
 
 \hskip5mm 2. Cauchy-Riemannovy podmínky \hskip10mm 234
 
 \hskip5mm 3. Více o komplexním křivkovém integrálu. Primitivní funkce \hskip10mm 237
 
 \hskip5mm 4. Cauchyova formule \hskip10mm 240
 
 \bigskip
 
{\bf XXIII. Několik dalších fakt z komplexní analysy} \hskip10mm 243

\hskip5mm 1. Taylorova formule \hskip10mm 243

\hskip5mm 2.  Věta o jednoznačnosti \hskip10mm 245

\hskip5mm 3. Liouvilleova věta a Základní věta algebry \hskip10mm 248

\hskip5mm 4.  Poznámka o konformním zobrazení \hskip10mm 250
 
 
 
 \newpage
 
 .
 
 \newpage
 
       
 
 
 
 
 
 
 
 \setcounter{page}{1} 
 
\renewcommand{\thepage}{\arabic{page}}

\centerline{\huge\bf První semestr} 
 
 \vskip10mm
 
 \centerline{\Large\bf I. Úvodem} 
 
 \vskip10mm
 
 
 \def\d{\text{d}}
 
 {\large\bf 1. Základy}
 
 \bigskip
 
 {\bf 1.1. Logika.} Logické spojky  "a (zároveň)" a "nebo" budou zpravidla vyjadřovány slovy zatím co pro implikaci budeme používat
standardní symbol ``$\Rightarrow$''. Negace tvrzení $A$ bude značena
``$\non A$''.  Čtenář jistě ví, že
 $$
 ``A\ \Rightarrow\ B\text{''}\qtq{je ekvivalentní s}  ``\non B\ \Rightarrow\ \non A\text{''}.
 $$
Toho se v důkazech běžně využívá.
 
 \smallskip
 
Kvantifikátor $\exists$ v ``$\exists x\in M, A(x)$'' indikuje že existuje  $x\in M$ takové, že $A(x)$ platí; množina $M$ je často zřejmá a píšeme pak prostě jen
 $\exists x A(x)$. Podobně kvantifikátor $\forall$ v ``$\forall x\in M, A(x)$'' indikuje že  $A(x)$ platí pro všechna $x\in M$ a opět, je-li obor $M$ zřejmý, píšeme často jen $\forall x A(x)$.
 
 \bigskip
 
 {\bf 1.2. Množiny.} $x\in A$ znamená že $x$ je prvkem množiny $A$.
 
 Budeme užívat standardní symboly pro sjednocení
 $$ 
 A\cup B, \ \ A_1,\cup\cdots\cup A_n,\ \ \bigcup_{i\in J}A_i 
 $$
 a pro průniky
 $$ 
 A\cap B, \ \ A_1,\cap\cdots\cap A_n,\ \  \bigcap_{i\in J}A_i.
 $$
 Rozdíl množin $A$ a $B$, t.j. množina těch prvků z $A$
 které nejsou v $B$ bude označována
 $$
 A\smin B.
 $$
 Připomeňte si De Morganovy formule
 $$
 A\smin\bigcup_{i\in J}B_i=\bigcap_{i\in J}(A\smin B_i) \qtq{and}
 A\smin\bigcap_{i\in J}B_i=\bigcup_{i\in J}(A\smin B_i).
 $$
 
 Množina všech $x$ splňujících podmínku $P$ se značí
 $\setof{x}{P(x)}$.

 Tak na příklad $A\cup B=\setof{x}{x\in A\ \text{nebo}\ x\in B}$, nebo
 $\bigcap_{i\in J}A_i=\setof{x}{\forall i\in J, \ x\in A_i}$.
 
  {\em Kartézský součin (produkt)} 
 $$
 A\times B
 $$
 je množina dvojic $(a,b)$ kde $a\in A$ a $b\in B$. Budeme též pracovat s kartézskými součiny
 $$
 A_1\times\cdots\times A_n,
 $$
 systémy $n$-tic $(a_1,\dots,a_n)$, $a_i\in A_i$, a později též s
 $$
 \prod_{i\in l}A_i=\setof{(a_i)_{i\in J}}{ a_i\in A_i}.
 $$
 
Formule $A\sue B$ (čti ``$A$ je podmnožina $B$'') značí že $a\in A$ implikuje  $a\in B$.
 
 Množina všech podmnožin množiny $A$ (``{\em potenční množina množiny $A$}'') se často označuje
 $$
 \exp A \qtq{nebo} \P(A).
 $$
 
 \bigskip
 
 {\bf 1.3. Ekvivalence.  Rozklad na třídy ekvivalence.}  {\em Ekvivalence} $E$  na množině $X$ je {\em reflexivní, symetrická a a transitivní} relace $E\sue X\times X$, t.j. relace taková že
 $$
 \begin{aligned}
 &\forall x, \ xEx  &\text{(reflexivita)}\\
 &\forall x,y, \ xEy\ \text{implikuje} \ yEx  &\text{(symetrie)}\\
 &\forall x,y,z \ xEy\ \text{a} \ yEz \ \text{implikuje} \ xEz  &\text{(transitivita)}.
 \end{aligned}
 $$
 (Píšeme $xEy$ místo $(x,y)\in E$). Označme
 $$
 Ex=\setof{y}{yEx}.
 $$
 Takové množiny se nazývají {\em třídy ekvivalence} této $E$. Platí 
 
 \medskip
 
 {\bf 1.3.1. Tvrzení.} {\em Každá ekvivalence na množině $X$  vytváří disjunktní rozklad na třídy ekvivalence. 
Na druhé straně, k disjunktnímu rozkladu
 $$
 X=\bigcup_{i\in J}X_i
 $$
máme ekvivalenci definovanou formulí
$$
xEy  \qtq{právě když}  \exists i,\ x,y\in X_i.
$$
 
 Důkaz.} Druhé tvrzení je zřejmé. Pro první potřebujeme dokázat, že pro kterékoli dva prvky $x,y$ máme buď  $Ex=Ey$ nebo $Ex\cap Ey=\ems$. Je-li $z\in  Ex\cap Ey$ máme $xEzEy$, tedy $xEy$,a potom, znovu z transitivity, 
 $z\in Ex$ právě když $z\in Ey$. \sq
 
 \medskip
 
 {\bf  Poznámka.} Všimněte si že zde vlastně jde o vzájemně jednoznačný vztah mezi všemi ekvivalencemi na  $X$ a všemi disjunktními rozklady množiny $X$.
 
 
 
 \bigskip
 
 {\bf 1.4. Zobrazení.}  {\em  Zobrazení} $f:A\to B$ sestává z těchto dat:
 \begin{enumerate}
 \item množina $X$,  {\em definiční obor} zobrazení $f$,
 \item množina $Y$,  {\em obor hodnot} množiny $f$,
 \item a podmnožina $f\sue X\times Y$ taková, že
  \begin{itemize}
  \item[-] pro každé $x\in X$ existuje $y\in Y$ takové, že $(x,y)\in f$, a
  \item[-] je-li $(x,y)\in f$ a $(x,z)\in f$ je $x=y$.
  \end{itemize}
\end{enumerate}  
Jednoznačně dané $y$ z podmínky (3) se obvykle značí $f(x)$ (někdy mluvíme o hodnotě  $f$ v argumentu $x$). Zobrazení může být často vyjádřeno formulí v argumentu (jako třeba $f(x)=x^2$); mějme však na mysli, že definiční obor a obor hodnot jsou podstatné údaje: pošleme-li celé číslo $x$ do celého čísla $x^2$ popisujeme jinou funkci než zobrazování reálného čísla $x$ do $x^2$ v reálném oboru, a omezíme-li v druhém případě obor hodnot na nezáporná čísla budeme mít další, jinou, funkci.

Zobrazení $f:X\to Y$ je {\em prosté} jestliže
$$
\forall x,y\in X, \  (x\neq y\ \Rightarrow\ f(x)\neq f(y));
$$
je {\em na} jestliže
$$
\forall y\in Y \exists x\in X\ \ f(x)=y.
$$
Všimněte si důležitosti informaci o oboru hodnot $Y$ pro tuto druhou vlastnost.

 {\em Identické zobrazení} $\id_X:X\to X$ je dáno předpisem $\id(x)=x$.

\smallskip

 {\em Obraz} podmnožiny $A\sue X$ v zobrazení $f:X\to Y$, t.j., $\setof{f(x)}{x\in A}$, bude označován $f[A]$, 
a {\em vzor} $\setof{x}{f(x)\in B}$ množiny $B\sue Y$ bude označován $f^{-1}[B]$.

\medskip

{\bf 1.4.1. Skládání zobrazení.} Pro zobrazení
$
f:X\to Y, \ g:Y\to X
$
dostáváme jejich  {\em složení}
$$
g\circ f:X\to Z
$$
 formulí $(g\circ f)(x)=g(f(x))$.

{\em Inverse}  ({\em inversní zobrazení}) zobrazení $f:X\to Y$ je zobrazení $g:Y\to X$ takové, že
$$
gf=\id_X\qtq{a} fg=\id_Y.
$$
Všimněte si, že má-li $f$  inversi, je prosté a na; naopak každé prosté zobrazení na má (jednoznačně určenou)
 inversi.

\medskip

{\bf 1.4.1. Funkce.}  O zobrazeních $f:X\to Y$ s definičním oborem $Y$ který je podmnožinou nějaké množiny čísel (přirozených čísel, celých, racionálních, reálných, komplexních čísel -- viz dále) často mluvíme jako o {\em funkcích}. Zejména se budeme zabývat {\em reálným funkcemi}, případem $Y\sue\Rbb$. Ze začátku bude většinou také $X\sue \Rbb$; mluvíme pak o {\em reálných funkcích jedné reálné proměnné}.
 
 \vskip10mm
 
 {\large\bf 2. Čísla.} 
 
 \bigskip
 
 {\bf 2.1. Přirozená čísla.} Čtenář je s nimi jistě dobře seznámen, připomeňme však formální přístup na základě Peanových axiomů. Je dána množina
 $$
 \Nbb
 $$
v níž je především dán významný prvek 0 a zobrazení $\sigma:\Nbb\to\Nbb$ ({\em funkce následníka}; obvykle se píše $n'$ místo $\sigma(n)$) takové že
 \begin{enumerate}
 \item pro každé $n\neq 0$ existuje právě jedno $m$ takové, že $m'=n$,
 \item 0 není následník,
 \item platí-li tvrzení $A$ pro 0 (symbolicky, $A(0)$) platí-li že $A(n)\Rightarrow A(n')$, platí then $\forall n A(n)$.
 \end{enumerate}
 (To poslední se nazýva  {\em axiom indukce}.)
 
 Dále zde máme operace $+$ a $\cdot$ (ta druhá se běžně označuje prostě juxtaposicí, a budeme to také tak dělat) pro které platí
 $$\begin{aligned}
 &n+0=n,\quad n+m'=(n+m)',\\
 &n\cdot 0=0,\quad nm'=nm+ n.
 \end{aligned}
 $$
 Konečně zde máme uspořádání  $n\leq m$
 definované předpisem
 $$
 n\leq m\qtq{právě když} \exists k, m=n+k.
 $$
 
 \medskip
 
 {\bf 2.1.1.} Tak dostaneme systém
 $(\Nbb,+,\cdot,0,1,\leq)$ (1 je $0'$, následník 0) kde platí
 $$
 \begin{aligned}
 &n+0=n, \quad n\cdot 1=n,&\\
 &m+(n+p)=(m+n)+p, \quad m(np)=(mn)p &\text{(pravidla associativity)}\\
 &m+n=n+m. \quad mn=nm &\text{pravidla(commutativity)}\\
 &m(n+p)=mn+mp &\text{(distributivita)}\\
 &n\leq n,\quad m\leq n\ \text{and}\ n\leq m\ \text{implikuje} \ n=m &\text{(reflexivita a antisymetrie)}\\
 &m\leq n \ \text{and}\ n\leq p\ \text{implies}\ m\leq p &\text{(transitivita)}\\
 &\forall m,n \ \text{buď}\ n\leq m\ \text{nebo}\ m\leq n\\
 &m\leq n \  \text{implikuje} \ n+p\leq m+p\\
 &m\leq n \  \text{implikuje} \ np\leq mp.
 \end{aligned}
 $$
 
 Jako snadné cvičení dokažte (aspoň některá) z těchto pravidel indukcí z axiomů.
 
 \bigskip
 
 {\bf 2.2. Celá čísla.} Množina celých čísel
$$
 \Zbb
 $$
se dostane z $\Nbb$ přidáním záporných čísel. Čtenář se může pokusit o formální konstrukci (např. může přidat nové prvky $(n,-)$ kde $n\in \Nbb$, $n\neq 0$ a  vhodně dodefinovat operace a uspořádání (jediné místo, kde je opravdu potřeba dát trochu pozor je definice sčítání).
 Tak dostaneme systém
 $$
 \Zbb
 $$ 
 kde platí všechna pravidla z 1.1 kromě posledního, které je potřeba nahradit pravidlem
 $$
 x\leq y\ \text{a}\ z\geq 0 \ \Rightarrow\ xz\leq yz.
 $$
 Na druhé straně ale máme jedno navíc, totiž
$$
 \forall x\ \exists y\ \ \text{takov\'e \v ze}\ \ x+y=0
 $$
 které umožňuje kromě sčítání a násobení také odčítání.
 
 \bigskip
 
 {\bf 2.3. Racionálni čísla.}  Již umime sčítat, násobit a odčítat. Schází nám ještě neomezené dělení. To jest, úplně neomezené být nemůže: z pravidel která jsme zmínili  vidíme, že $0\cdot x=0$ a tedy dělení  $0$ nedává smysl. Ale to bude jediná výjimka v následujícím systému racionálnch čísel. Začněme třeba s množinou 
 $$
 X=\setof{(x,y)}{x,y\in\Zbb, y\neq 0}
 $$
 a definujme
 $$
 (x,y)+(u,v)=(xv+yu,uv)\ \ \text{a}\ \ (x,y)(u,v)=(xu,yv).
 $$
 Dále použijeme relaci ekvivalence
 $$
 (x,y)\sim (u,v)  \ \ \text{právě když}\ \ xv=uy
 $$
 a položíme 
 $$
 \Qbb=X/\sim.
 $$
 Snadno se dokáže, že
 $$
 (x,y)\sim(x',y')\ \ \text{a}\ \ (u,v)\sim(u',v'),
 $$
 potom, že
 $$
 (x,y)+(u.v)\sim(x',y')+(u',v')\ \ \text{a}\ \ (x,y)(u,v)\sim(x'.y')(u',v')
 $$
 (je to jednoduché cvičení) a že nyní je možné definovat sčítání a násobení na $\Qbb$, a že pro třídy ekvivalence máme
 (0 je třída prvku $(0,n)$ a 1 je třída obsahující $(n,n)$)
$$
 \begin{aligned}
 &x+0=n, \quad x\cdot 1=x,&\\
 &x+(y+z)=(x+y)+z, \quad x(yz)=(xy)z &\text{(pravidla asociativity)}\\
 &x+y=y+x. \quad xy=yx &\text{(pravidla komutativity)}\\
 &x(y+z)=xz+yz &\text{(distributivita)}\\
 &\forall x\exists y, \ x+y=0\\
 &\forall x\neq 0\exists y, \ xy=1.
  \end{aligned}
 $$
 Systémům splňujícím tato pravidla se říka {\em komutativní tělesa}.
 
 Dále můžeme definovat relaci $\leq$ předpisem
 $$
 (x,y)\leq (u,v) \ \text{pro}\ y,v>0 \ \ \text{kdy\v z}\ \ xv\leq yu
 $$
 čímž na $\Qbb$ vznikne uspořádání, pro které platí
$$
 \begin{aligned}
  &x\leq x,\quad x\leq y\ \text{a}\ y\leq x\ \text{implikuje} \ x=y &\text{(reflexivita a antisymmetrie)}\\
 &x\leq y \ \text{a}\ y\leq z\ \text{implikuje}\ x\leq z &\text{(transitivita)}\\
 &\forall x,y \ \text{buď}\ x\leq y\ \text{nebo}\ y\leq x\\
 &x\leq y \  \text{implikuje} \ x+z\leq y+z\\
 &x\leq y\ \text{and}\ z>0 \  \text{implikuje} \ xz\leq yz.
 \end{aligned}
 $$
 čímž jsme dostali  {\em uspořádané (komutativní) těleso}.
 
 Asi není nutné připomínat standardní symbol
 $$
 \frac pq
 $$
 užívaný pro třídu ekvivalence obsahující $(p,q)$.
 
 \bigskip
 
 {\bf 2.4. Racionální čísla nás ještě úplně neuspokojují.} Nyní tedy máme systém čísel, ve kterém můžeme sčítat, odčítat, násobit a dělit. Zdá se též být uspořádán uspokojivým způsobem (ukáže se však, že právě pořeby tohoto uspořádání budou klíčem k řešení obtíží).
 
Už staří Řekové si všimli vážného problému. Rádi bychom přiřadili  úsečkám vzniklým při jednoduchých úlohách délky.
A už tak základní úloha jako délka diagonály v jednotkovém čtverci není řešitelná v oblasti racionálních čísel. Nutně potřebujeme druhou odmocninu. Podívejme se, co se stane.

 Předpokládejme, že $\sqrt 2$, číslo  $x$ takové, že $x^2=2$, může být vyjádřeno racionálním číslem, že tedy máme celá čísla $p,q$ pro která
$$
 \left(\frac pq\right)^2=2.
 $$
Můžeme předpokládat, že tato čísla $p,q$ jsou nesoudělná, jinak zlomek zkrátíme.  
 
 Máme
 $$ 
 \frac{p^2}{q^2}=2,\qtq{tedy} p^2=2q^2
 $$
 a tedy $p$ musí být sudé. Potom je ale $p^2$ dělitelné čtyřmi,a následkem toho i $q$ je sudé, a  $p,q$ jsou soudělná ve sporu s předpokladem.
 
 \bigskip
 
 {\bf 2.5. Uspořádání, suprema a infima.}  {\em Lineární uspořádání} na množině $X$ je relace
 $\leq$ splňující
 $$
 \begin{aligned}
  &x\leq x &\text{(reflexivita)} \\
   &x\leq y\ \text{a}\ y\leq x\ \text{implikuje} \ x=y &\text{(antisymmetrie)}\\
 &x\leq y \ \text{a}\ y\leq z\ \text{implikuje}\ x\leq z &\text{(transitivita)}\\
 &\forall x,y \ \text{buď}\ x\leq y\ \text{anebo}\ y\leq x &\text{(linearita)}
 \end{aligned}
 $$ 
 Pokud požadujeme jen reflexivitu, antisymetrii a transitivity mluvíme o  {\em částečném uspořádání}.
 
 \medskip
 
 {\em Horní mez} podmnožiny $M$ částečně uspořádané množiny $(X,\leq)$ je   $b\in X$ pro které
$$
 \forall x\in M,\  x\leq b;
 $$
 $M$ je  {\em omezená} (shora) má-li $M$ horní mez.
  
Podobně mluvíme o  {\em dolní mezi} $b$ jestliže
 $$
 \forall x\in M,\  x\geq b,
 $$
a $M$ je {\em omezená} (zdola) má-li $M$ dolní mez.

Velmi často je z kontextu patrno zda máme na mysli omezení shora či zdola a mluvíme pak prostě o {\em omezené} množině.

{\em Supremum} podmnožiny $M\sue (X,\leq)$ je její nejmenší horní mez (nemusí existovat, samozřejmě). Existuje-li, označuje se
$$
\sup M.
$$
Explicitněji, $s\in X$ je supremum množiny $M$ jestliže
\begin{enumerate}
\item pro každé $x\in M$ je  $x\leq s$, a
\item je-li $x\leq y$ pro všechna $x\in M$ je $s\leq y$.
\end{enumerate}
V lineárně uspořádané množině je to ekvivalentní s podmínkami
\begin{enumerate}
\item pro každé $x\in M$ je $x\leq s$, a
\item je-li $y<s$  pak existuje $x\in M$ takové, že $y<x$.
\end{enumerate}
Druhá formulace má své výhody a bude užívána častěji než ta první.
 
 \medskip
 
Podobně {\em infimum} množiny $M$ je největší dolní mez $M$. Existuje-li, je označováno
$$
\inf M.
$$
Explicitněji, $i\in X$ je infimum množiny $M$ jestliže
\begin{enumerate}
\item pro každé $x\in M$ je  $x\geq i$, a
\item je-li $x\geq y$ pro všechna $x\in M$ je $i\geq y$.
\end{enumerate}
V lineárně uspořádané množině je to ekvivalentní s podmínkami
\begin{enumerate}
\item pro každé $x\in M$ je $x\geq i$, a
\item je-li $y>i$  pak existuje $x\in M$ takové, že $y>x$.
\end{enumerate}


\medskip

Je zřejmé, že supremum či infimum (pokud existuje) je jednoznačně určeno.

\medskip

{\bf 2.5.1. Příklad.} Připomeňme si obtíž s odmocninou ze 2 v bodě 2.4. Všimněte si, že v uspřádané množině racionálních čísel
$\Qbb$ množina $\setof{x}{0\leq x, \ x^2\leq 2}$ je shora omezená ale nemá supremum.
Podobně, $\setof{x}{0\leq x, \ x^2\geq 2}$ je omezená zdola a nemá infimum.

\medskip

{\bf 2.5.2. Cvičení.} Dokažte podrobně že v  {\em lineárně} uspořádaných množinách dvě zmíněné varianty  požadavků pro supremum resp. infimum jsou skutečně ekvivalentní. Jak užíváte linearitu? Proč je nutná?


\bigskip

{\bf 2.6. Reálná čísla.} Systém reálných čísel
$$
\Rbb
$$
tak jak ho budeme užívat, je zúplnění (ve více než jednom smyslu) systému  $\Qbb$. Je to {\em lineárně uspořádané komutativní těleso} ve kterém
\begin{equation}
\text{\em kažá shora omezená neprázdná množina má supremum. }\tag{sup}
\end{equation}
Při práci s reálnými čísly budeme užívat tyto vlastnosti: pravidla z bodu 2.3 a (sup) (a nic dalšího).

\medskip

{\bf 2.6.1. Tvrzení.} {\em V $\Rbb$ má každá neprázdná zdola omezená množina infimum.

Důkaz.}  Nechť $M$ je neprázdná a zdola omezená. Položme
$$
N=\setof{x}{x\ \text{je doln\'\i\ mez}\ M}.
$$
Jelikož je $M$ zdola omezená je $N$ neprázdná. Jelikož je $M$ neprázdná, $N$ je shora omezená (každé $y\in M$ je horní mez množiny $N$). Existuje tedy
$$
i=\sup N.
$$
Jelikož každé $x\in M$ je horní mez množiny $N$ je $i\leq x$ pro všechna $x\in M$. Na druhé straně, je-li $y$ dolní mez množiny $M$, je $y$ v $N$ a tedy $y\leq i=\sup N$. \sq

\vskip10mm

{\bf\large 3. Systém reálných čísel jako  (euklidovská) přímka.}

\bigskip

{\bf 3.1. Absolutní hodnota.} Připomeňme, že {\em absolutní hodnota} reálného čísla je
$$
|a|=\begin{cases} a\ \text{je-li}\ a\geq 0,\\
                  -a\ \text{je-li}\ a\leq 0
\end{cases}
$$ 

\medskip

{\bf 3.1.1.} Zřejmě platí

\smallskip

{\bf Pozorov\'an\'\i.} {\em $|a+b|\leq |a|+|b|$.}

Tato nerovnost (říká se jí {\em trojúhelníková nerovnost}) bude velmi často užívána v důkazech, často bez zvláštního připomenutí.

\bigskip

{\bf 3.2. Metrická struktura na množině $\Rbb$: reálná přímka.} Systém reálných čísel opatříme  {\em vzdáleností}
$$
d(x,y)=|x-y|
$$
a díváme se na něj (kromě všech jiných dříve zmíněných vlastností) jako na euklidovskou přímku.

Všimněte si, že to je důvod pro výraz ``trojúhelníková nerovnost'':
vezmeme-li $a=x-y$ a $b=y-z$ dostaneme z 3.1.1
$$
|x-z|\leq |x-y|+|y-z|
$$
(to jest, $d(x,z)\leq d(x,y)+d(y,z)$).

\medskip

{\bf 3.3. Poznámka: Shrnutí.} Uvědomte si, že $\Rbb$ má dost složitou kombinovanou strukturu. Je to zároveň
\begin{itemize}
\item  komutativní těleso (algebra algebra se sčítáním, odčítáním, násobením a dělením),
\item lineárně uspořádana množina, a
\item metrický prostor. 
\end{itemize}                


\bigskip

{\bf 3.4. Dodatek pro později: komplexní (Gaussova) rovina.} Trojúhelníková nerovnost na přímce je samozřejmě velmi jednoduchá záležitost. Předveďme složitější případ. Komplexní čísla sice nebudeme nějakou dobu potřebovat, ale 
všimněme si hned teď jejich geometrické struktury. 

Ke komplexnímu číslu $a=x+ iy$ máme {\em komplexně sdružené} $\ol a=x-iy$ a absolutní hodnoyu
$$
|a|=a\cdot\ol a=x^2+y^2.
$$
Díváme-li se na komplexní číslo $x+iy$ jako na bod $(x,y)$ v euklidovské rovině je  $|a|$ jeho standardní vzdálenost od  $(0.0)$, a
$$
|a-b|
$$
je standardní pythagorovská vzdálenost bodů $a$ a $b$. Systém komplexních čísel nahlížený z této perspektivy se nazývá {\em Gaussova rovina}. Máme

\medskip

{\bf 3.4.1. Tvrzení.} {\em Pro absolutní hodnotu komplexních čísel platí
$$
|a+b|\leq |a|+|b|.
$$

Důkaz.}
Buď $a=a_1+ia_2$ a $b=b_1+ib_2$. Můžeme předpokládat, že $b\neq 0$. Pro libovolné reálné $\lambda$ máme zřejmě
$0\leq (a_j+\lambda b_j)^2=a_j^2+2\lambda a_jb_j+\lambda^2b_j$, $j=1,2$. Sečteme-li tyto nerovnosti dostaneme
$$
0\leq |a|^2+2\lambda(a_1b_1+a_2b_2)+\lambda^2|b|^2.
$$
Volba $\lambda=-\frac{a_1b_1+a_2b_2}{|b|^2}$ dává
$$
0\leq |a|^2-2\frac{(a_1b_1+a_2b_2)^2}{|b|^2}+\frac{(a_1b_1+a_2b_2)^2}{|b|^4}|b|^2=
|a|^2-\frac{(a_1b_1+a_2b_2)^2}{|b|^2}
$$
a tedy $(a_1b_1+a_2b_2)^2\leq |a|^2|b|^2$. Následkem toho
$$
\begin{aligned}
|a+b|^2=(a_1+b_1)^2&+(a_2+b_2)^2=|a|^2+2(a_1b_1+a_2b_2)+|b|^2\leq\\
                &\leq   |a|^2+2|a||b|+|b|^2=(|a|+|b|)^2.\quad \square 
\end{aligned}                   
$$


\medskip

{\bf 3.4.2. Poznámka.} Setkáme se s důkazy vět o komplexních číslech které budou formálně doslovná opakování důkazů vět o reálných číslech. Přes tuto formální shodu se může jednat o podstatně hlubší fakt v případech kde byla podstatně užita trojúhelníková nerovnost.



\newpage

\centerline{\Large\bf II. Posloupnosti reálných čísel.} 
 
 \vskip10mm
 
 {\large\bf 1. Posloupnosti a podposloupnosti}
 
 \bigskip
 
 {\bf 1.1.} (Nekonečná) {\em posloupnost} je seskupení
 $$
 x_0,x_1,\dots,x_n,\dots .
 $$
 Takže to vlasně není nic jiného než zobrazení $x:\Nbb\to\Rbb$ napsané jako  ``tabulka''; mohli bychom psát $x(n)=x_n$. 
 
 \medskip
 
 {\bf Poznámka.} Indexování $0,1,2,\dots$  není podstatné, pořadí v daném seskupení však je. 
Výhoda zápisu jako seskupení je v tom, že argument je dán pořadím, ne konkretně užitým indexem. Můžeme mít posloupnost
 $$
 x_1,x_2,\dots,x_n,\dots 
 $$
 nebo třeba
 $$
 x_1,x_4,\dots,x_{n^2},\dots 
 $$
 atd.; kdybychom je chtěli representovat jako tabulky zobrazení měli bychom zde, dejme tomu, $x(n)=x_{n+1}$, nebo $x(n)=x_{(n+1)^2}$ atd.. Podposloupnosti, o kterých budeme hovořit dál, jsou tak z\v rejm\v e samy posloupnosti.
 
 
 \medskip
 
 {\bf 1.1.1,} Naše posloupnosti budou většinou nekonečné, ale je třeba poznamenat, že se hovoří též o konečných posloupnostech jako třeba
 $$
 x_1,x_2,\dots, x_n.
 $$
 a a podobně.
 
 \bigskip
 
 {\bf 1.2.  Podposloupnosti.} {\em Podposloupnost } posloupnosti
$$
 x_0,x_1,\dots,x_n,\dots 
 $$
je kterákoli posloupnost
 $$
 x_{k_0},x_{k_1},\dots,x_{k_n},\dots 
 $$
kde $k_n$ jsou přirozená čísla taková, že
 $$
 k_0<k_1<\cdots<k_n<\cdots .
 $$
 Když se na původní posloupnost díváme jako na zobrazení $x:\Nbb\to\Rbb$ jak bylo zmíněno nahoře vidíme, že podposloupnost je složené zobrazení $x\circ k$ se zobrazením $k:\Nbb\to\Nbb$ které
 {\em roste}, to jest, takové, že $m<n$ implikuje $k(m)<k(n)$.
 
 \medskip
 
 {\bf 1.2.1. Označení.} Posloupnost $x_1,x_2,\dots$ můžeme psát jako
 $$
 (x_n)_n,
 $$
 takže podposloupnost nahoře je pak
 $(x_{k_n})_n$.
 
 \bigskip
 
 {\bf 1.3.} Posloupnost $(x_n)_n$ je {\em rostoucí, neklesající, nerostoucí, resp. klesající}, jestiže
 $$
 m<n\quad{\Rightarrow}\quad x_m<x_n,\ x_m\leq x_n,\ x_m\geq x_n,\  \text{resp.}  \ x_m>x_n\ .
 $$
 
 
 
 \vskip10mm
 
  {\large\bf 2. Konvergence. Limita posloupnosti}
 
 \bigskip
 
 {\bf 2.1. Limita.} Řekneme, že číslo $L$ je {\em limita} posloupnosti $(x_n)_n$ a píšeme
 $$
 \lim_nx_n=L
 $$
jestliže
 \begin{equation}
 \forall \epsilon>0 \ \exists n_0\ \ \text{takové, že}\ \forall n\geq n_0, \ |x_n-L|<\epsilon.
 \tag{$*$}
 \end{equation}
Říkáme pak, že $(x_n)_n$ {\em konverguje} k $L$; nespecifikujeme-li $L$, řekneme, že je {\em konvergentní}. Jinak mluvíme o  {\em divergentní} posloupnosti.
 
 \smallskip
 
 Při použití sybolu $\lim_n x_n$ automaticky předpokládáme, že ta limita existuje.
 
 \medskip
 
 {\bf 2.1.1.} Následující formule je zřejmě ekvivalentí s ($*$).
 $$
 \forall \epsilon>0 \exists n_0\ \ \text{takov\'e \v ze}\ \forall n\geq n_0, \ L-\epsilon<x_n<L+\epsilon.
 $$
 Vyvolává názornou představu (pro dost velké $n$ je $x_n$  v libovolně malém ``$\epsilon$ovém okolí'' čísla $L$), a často se s ní lépe pracuje.
 
 \medskip
 
 {\bf 2.1.2. Poznámka.} Typická divergentní posloupnost není posloupnost rostoucí nade všechny meze, jako t\v reba
 $1,2,3,\dots$. Takové případy se dají spravit přidáním $+\infty$ and $-\infty$ a snadnou modifikací definice, jak uvidíme později. Představujte si raději posloupnosti jako $0,1,0,1,\dots$.
 
 \bigskip
 
 {\bf 2.2. Pozorování.} 1. {\em Limita konstantní posloupnosti $x,x,x,\dots$ je $x$.}
 
 2. {\em Existuje-li limita, je jednoznačně definována.}
 
 3. {\em Každá podposloupnost konvergentní posloupnosti konverguje, a sice k téže limitě.}
 
 (K bodu 2, jsou-li $L$ a $K$ limity posloupnosti $(x_n)_n$ je libovolně malé $\epsilon>0$ a dost velké $n$, $|L-K|=|L-x_n+x_n-K|\leq |L-x_n|+|x_n-K|<2\epsilon$.
 Pro 3 si stačí uvědomit, že $k_n\geq n$.)
 
 \medskip
 
 {\bf 2.2.1. Poznámka.} Na druhé straně, divergentní posloupnost může mít konvergentní podposloupnosti. Samozřejmě ale jestliže
 $x_p,x_{p+1},x_{p+2},\dots$ (t.j., podposloupnost s $k_n=p+n$) konverguje potom konverguje i $(x_n)_n$.
 
 \bigskip
 
 {\bf 2.3. Tvrzení.} {\em Nechť\ $\lim a_n=A$ a $\lim b_n=B$ existují. Potom $\lim(\alpha a_n)$, 
 $\lim(a_n+b_n)$, $\lim(a_n\cdot b_n)$ a, jsou-li všechna $b_n$ a $B$ nenulová, též $\lim\frac{a_n}{b_n}$ existují
a platí
 \begin{enumerate}
 \item $\lim(\alpha a_n)=\alpha\lim a_n$,
 \item $\lim(a_n+b_n)=\lim a_n+\lim b_n$,
 \item $\lim(a_n\cdot b_n)=\lim a_n\cdot\lim b_n$,
 \item $\lim\frac{a_n}{b_n}=\frac{\lim a_n}{\lim b_n}$.
 \end{enumerate} }
 
 \smallskip
 
 {\bf Poznámka před důkazem.} 1. Uvědomte si roli čísla $\epsilon>0$ v definici limity jako
 ``libovolně malého kladného reálného čísla'' kde jeho přesná hodnota ani není tak moc důležitá. Takže například stačí dokázat, že pro každé $\epsilon>0$ existuje $n_0$ takové že pro $n\geq n_0$ máme $|x_n-L|<100\epsilon$ (to  $n_0$ jsme mohli vzít pro $\frac1{100}\epsilon$ msto toho počátečního $\epsilon$. 
 
 2. V následujícím bodu (3) si zapamatujte trik přičtení $0$ ve formě $x-x$ (bude to tam $x=a_nB$). Užívá se často.
 
 \smallskip
 
 {\em Důkaz.} (1): Máme $|\alpha a_n-\alpha A|=|\alpha||a_n-A|$. Tedy, je-li $|a_n-A|<\epsilon$
máme $|\alpha a_n-\alpha A|<|\alpha|\epsilon$.
 
 (2) Jestliže $|a_n-A|<\epsilon$ a $|b_n-B|<\epsilon$ potom $|(a_n+b_n)-(A+B)|=|a_n-A+b_n-B|\leq
 |a_n-A|+|b_n-B|<2\epsilon$.
 
 (3) Jestliže $|a_n-A|<\epsilon$ a $|b_n-B|<\epsilon$ dostaneme
 $$
 \begin{aligned} 
 |a_nb_n&-AB|=|a_nb_n-a_nB+a_nB-AB|\leq\\
 &\leq |a_nb_n-a_nB|+|a_nB-AB|=
 |a_n||b_n-B|+|B||a_n-A|<\\
 &<
 (|A|+1)|b_n-B|+|B||a_n-A|<(|A|+|B|+1)\epsilon
 \end{aligned}
 $$
 (užili jsme zřejmý fakt, že při $\lim a_n=A$ je pro dost velké  $n$, $|a_n|<|A|+1$).
 
 (4) Máme již (3), takže stačí dokázat, že $\lim\frac{1}{b_n}=\frac{1}{\lim b_n}$. Buď
 $|b_n-B|<\epsilon$. Potom
 $$
 \left|\frac{1}{b_n}-\frac{1}{B}\right|=\left|\frac{b_n-B}{b_nB}\right|=\left|\frac1{b_nB}\right||b_n-B|\leq
\left|\frac2{BB}\right||b_n-B|<\left|\frac2{BB}\right|\epsilon.
 $$
protože zřejmě je-li $\lim b_n=B\neq 0$ je pro dost velké $n$, $|b_n|>\frac12 |B|$. \sq
 
 \bigskip
 
 {\bf 2.4. Tvrzení.} {\em Nechť $\lim a_n=A$ a $\lim b_n=B$ existují a nechť $a_n\leq b_n$ pro všechna $n$.
 Potom $A\leq B$.
 
 Důkaz.} Předpokládejme opak a zvolme $\epsilon=A-B>0$. Zvolme $n$ takové, že $|a_n-A|<\frac12\epsilon$
 a $|b_n-B|<\frac12\epsilon$; potom $a_n>A+\frac{\epsilon}2$ a $b_n<B-\frac{\epsilon}2$, a tedy $a_n>b_n$ ve sporu s předpokladem.\sq
 
 \bigskip
 
 {\bf 2.5. Tvrzení.} {\em Nechť $\lim a_n=A=\lim b_n$  a $a_n\leq c_n\leq b_n$ pro každé $n$.
 Potom $\lim c_n$ existuje a je rovna $A$.
 
 Důkaz.} Zvolme $n_0$ tak aby pro $n\geq n_0$ bylo $|a_n-A|<\epsilon$ a $|b_n-A|<\epsilon$. Potom
 $$
 A-\epsilon<a_n\leq c_n\leq b_n<A+\epsilon.
 $$
 Užijme 2.1.1. \sq
 
 \bigskip
 
 {\bf 2.6. Tvrzení.} {\em Shora omezená neklesající posloupnost reálných čísel konverguje ke svému supremu.

 Důkaz.} Množina $\setof{x_n}{n\in\Nbb}$ je omezená a neprázdná a tedy zde  nějaké supremum $s$ existuje. Je-li $\epsilon$ větší než nula, musí být pro nějaké $n_0$,  $s-\epsilon<x_{n_0}$ a potom pro všechna $n\geq n_0$,
 $$
 s-\epsilon<x_{n_0}\leq x_n\leq s.
 $$
 Užijme 2.1.1. \sq
 
 \bigskip
 
 {\bf 2.7. Věta.} {\em Nechť $a,b$ jsou reálná čísla taková, že $a\leq x_n\leq b$ pro všechna $n$. Potom existuje podposloupnost $(x_{k_n})_n$ posloupnosti $(x_n)_n$ která konverguje v $\Rbb$, a platí $a\leq \lim_nx_{k_n}\leq b$.
 
 Důkaz.} Vzměme
 $$
 M=\setof{x}{x\in\Rbb,\ x\leq x_n\ \text{pro nekonečně mnoho}\ n}.
 $$
 $M$ je nekonečná protože $a\in M$ a $b$ je horní mez $M$. Musí tedy existovat
 $
 s=\sup M$ a platí $a\leq s\leq b
 $.
 
Pro každé $n$ je množina
 $$
 K(n)=\setof{k}{s-\frac1n<x_k<s+\frac1n}
 $$
nekonečná: skutečně, podle 2.5 (druhá formulace definice suprema) máme $x>s-\epsilon$ takové, že $x_k> x$ pro nekonečně mnoho $k$, zatím co podle definice množiny $M$ je jen konečně mnoho $k$ takových, že
 $x_k\geq s+\epsilon$.
 
 Zvolme $k_1$ tak aby
 $$
  s-1 < x_{k_1}< s+1
  $$.
 Mějme zvolena $k_1< k_2<\cdots< k_n$ taková, že $j=1,\dots,n$
  $$
  s-\frac1j < x_{k_j}< s+\frac1j.
  $$
 Jelikož $K(n+1)$ je nekonečná, můžeme zvolit $k_{n+1}> k_n$ tak aby
  $$
  s-\frac1{n+1} < x_{k_{n+1}}< s+\frac1{n+1}.
  $$
  Takto zvolená podposloupnost $(x_{k_n})_n$ naší $(x_n)_n$ zřejmě konverguje k $s$. \sq
 
  
  
 \vskip10mm
 
  {\large\bf 3. Cauchyovské posloupnosti}
 
 \bigskip
 
 {\bf 3.1.} Řekneme, že posloupnost $(x_n)_n$ je {\em Cauchyovská} jestliže
 $$
 \forall \epsilon>0 \ \exists n_0\ \ \text{takové že}\ \forall m,n\geq n_0, \ |x_m-x_n|<\epsilon.
 $$
 
 \medskip
 
 {\bf 3.1.1. Pozorování.} {\em Každá konvergentní posloupnost je Cauchyovská.}
 
 (Je-li $|x_n-L|<\epsilon$ pro $n\geq n_0$ je pro $m,n\geq n_0$,
 $$
 |x_n-x_m|=|x_n-L+L-x_m| \leq|x_n-L|+|L-x_m|<2\epsilon.)
 $$
  
 \bigskip
 
 {\bf 3.2. Lemma.} {\em Má-li Cauchyovská posloupnost Cauchyovskou podposloupnost, konverguje celá.
 
 Důkaz.} Nechť v Cauchyovské posloupnosti $(x_n)_n$ máme $\lim x_{k_n}=x$ pro nějakou podposloupnost. Zvolme pro $\epsilon>0$
 
$n_1$ takové, že pro $m,n\geq n_1$ je $|x_m-x_n|<\epsilon$, a $n_2$ takové, že pro
 $n\geq n_2, \ |x_{k_n}-x|<\epsilon$. Položme $n_0=\max(n_1,n_0)$.
 
 Když teď $n\geq n_0$ je
 $$
 |x_n-x|=|x_n-x_{k_n}+x_{k_n}-x|\leq|x_n-x_{k_n}|+|x_{k_n}-x|\leq 2\epsilon
 $$
protože $k_n\geq n\geq n_1$. \sq
 
 \bigskip
 
 {\bf 3.3. Lemma.} {\em Každá Cauchyovská posloupnost je omezená.
  
 Důkaz.} Zvolme $n_0$ tak aby $|x_n-x_{n_0}|<1$ pro všechna $n\geq n_0$. Potom máme
 $$
 a=\min\setof{x_j}{j=1,2,\dots,n_0} -1\leq x_n\leq b=\max\setof{x_j}{j=1,2,\dots,n_0}+1
 $$
 pro všechna $n$. \sq
 
 \bigskip
 
 {\bf 3.4. Věta.} (Bolzano-Cauchyova Věta) {\em Posloupnost reálných čísel konverguje právě když je Cauchyovská.

 Důkaz.}  Cauchyovská posloupnost je podle Lemmatu 3.3 omezená, a tedy, podle Věty 2.7 má konvergentní podposloupnost. Použij
 Lemma 3.2. 
 
 Druhá implikace byla již pozorována v 3.1.1. \sq
 
 \medskip
 
 {\bf 3.4.1. Poznámky.} 1.  Důkaz byl velmi krátký,  ale to proto, že vše bylo již připraveno ve Větě 2.7.
 
 2. Bolzano-Cauchyova Věta je velice důležitá. Uvědomte si, že zde máme kriterium konvergence které lze použít bez předchozí znalosti hodnoty limity, nebo hodnot předem spočítaných.
 
 
   
 \vskip10mm
 
  {\large\bf 4. Spočetné množiny: velikost posloupnosti
  
  \hskip7mm jako nejmenší nekonečno}
 
 \bigskip
 
 \def\card{\text{card}}
 
Tato sekce je o obecných posloupnostech, nejen o posloupnostech reálných čísel.

\bigskip
 
 {\bf 4.1. Srovnávání mohutností (kardinalit).}  Dvě množiny $X,Y$ jsou stejně veliké (říkáme, že mají stejnou mohutnost nebo kardinalitu a píšeme
 $$
 \card X=\card Y \ )
 $$
existuje-li vzájemně jednoznačné zobrazení $f:X\to Y$. Dále píšeme
 $$
 \card X\leq \card Y
 $$
existuje-li prosté zobrazení $f:X\to Y$. Znamená to, že množina $Y$ je nejméně tak velká jako množina $X$.
 
 \medskip
 
 {\bf Poznámka.} Přirozeně vzniká otázka zda $\card X\leq \card Y$ a $\card Y\leq \card X$
 implikují, že $\card X= \card Y$.To je zřejmé pro konečné množiny, a ne zcela zřejmé pro nekonečné, ale platí to, je to známá
 Cantor-Bernsteinova Věta.
 
 \bigskip
 
 {\bf 4.2. Tvrzení.} {\em Mohutnost množiny přirozených čísel je nejmenší z nekonečných mohutností. Formálně,
 je-li $X$ nekonečná, je $\card\Nbb\leq \card X$.
 
 Důkaz.} Prosté zobrazení $f:\Nbb\to X$ můžeme kostruovat induktivně takto. Zvolme $f(0)\in X$ libovolně.
 Jsou-li hodnoty $f(0),\dots,f(n)$ zvoleny, je jich konečně mnoho a tedy je ještě $X\smin\set{f(0),\dots,f(n)}$ nekonečná
a můžeme zvolit $f(n+1)\in X\smin\set{f(0),\dots,f(n)}$.\sq
 
 \bigskip
 
 {\bf 4.3. Spočetné množiny.} Množina $X$ je  {\em spočetná} je-li $\card X=\card\Nbb$. Jinými slovy, je spočetná existuje-li vzájemně jednoznačné zobrazení $f:\Nbb\to X$, tedy, právě když ji můžeme seřadit do (prosté) posloupnosti
 $$
 X:\ x_0,x_1,\dots,x_n\dots
 $$
 (vezmeme $x_n=f(n)$).
 
Chceme-li říci, že $X$ je konečná nebo spočetná, říkáme, že je {\em nejvýš spočetná}.
 
Uvědomte si, že
 
 {\bf 4.3.1.} {\em na to, abychom zjistili, že je množina spočetná stačí vědět, že je nekonečná a seřadit ji do jakékoli posloupnosti: po vynechání případných opakování stále zbývá nekonečná posloupnost.}
 
 \bigskip
 
 {\bf 4.4. Tvrzení.} {\em  Jsou-li $X_n$, $n\in\Nbb$, nejvýš spočetné je množina
 $$
 X=\bigcup_{n=0}^\infty X_n
 $$
nejvýš spočetná.
 
 Důkaz.} Seřaďme množiny $X_n$ do posloupností
 $$
 X_n: \ x_{n0},x_{n1},\dots,x_{nk},\dots \ .
 $$
$X$ nyní můžeme seřadit do poslupnosti
 $$
 \begin{aligned} x_{00},\ \ x_{01},x_{10},\ \ &x_{02},x_{11},x_{20},\ \ x_{03},x_{12},x_{21},x_{30},\dots,\\
 &\dots x_{0,k},x_{1,k-1},x_{2,k-2},\dots,x_{k-2,2},x_{k-1,1},x_{k,0},\dots .
 \end{aligned}
 $$
 \sq
 
 \bigskip
 
 {\bf 4.5. Důsledek.} {\em Je-li $X$ spočetná, je $X\times X$ spočetná.}
 
 (Máme $X\times X=\bigcup_{x\in X}X\times\set{x}$.)
 
 \bigskip
 
  {\bf 4.6. Důsledek.} {\em Množina $\Qbb$ všech racionálních čísel je spočetná.}
  
  \bigskip
  
   {\bf 4.7. Důsledek.} {\em Je-li $X$ spočetná, je každá  kartézská mocnina $X^n$ spočetná, a tedy též
   $$
   \bigcup_{n=0}^\infty X^n
   $$
   je spočetná.
   
   Následkem toho je množina všech konečných podmnožin spočetné množiny spočetná.}
   
   
   \bigskip
   
    {\bf 4.8. Fakt.} {\em Množina $\Rbb$ všech reálných čísel spočetná {\em není}.
    
    Důkaz.} Representujme  reálná čísla mezi nulou a jednotkou v dekadických rozvojích 
    $$
    r: \ 0.r_1r_2\cdots r_n\cdots \ .
    $$ 
	Předpokládejme, že je můžeme seřadit do posloupnosti (vertikálně)
	$$
    \begin{aligned}
    &r_1: \ 0.r_{11}r_{12}r_{13}\cdots r_{1n}\cdots \\
    &r_2: \ 0.r_{21}r_{22}r_{23}\cdots r_{2n}\cdots \\
    &r_3: \ 0.r_{31}r_{32}r_{33}\cdots r_{3n}\cdots \\
    &\quad\dots \\
    &r_k: \ 0.r_{k1}r_{k2}r_{k3}\cdots r_{kn}\cdots \\
    &\quad\dots
    \end{aligned}
    $$
    Definujme nyní
    $$
    x_n=\begin{cases} 1 \ \text{jestliže}\  r_{nn}\neq 1,\\
                     2 \ \text{jestliže} \  r_{nn}= 1. \end{cases}
   $$
   Reálné číslo $r=0.x_1x_2\cdots x_n\cdots$ se potom v naší vertikální posloupnosti neobjeví -- spor.\sq   
   
   \bigskip
    
   {\bf 4.9. Cantorova Diagonalizační Věta.} Procedura z 4.8 je speciální případ slavné Cantorovy diagonalizace.
   
   \medskip
   
   {\bf Věta.} (Cantor) {\em Mohutnost množiny $\P(X)$ všech podmnožin množiny $X$ je ostře větší než mohunost množiny $X$.
   
   Důkaz.} Předpokládejme, že $\card X=\card\P(X)$. Máme tedy vzájemně jednoznačné zobrazení $f:X\to\P(X)$ (stačilo by sobrazení na). Položme
   $$
   A=\setof{x}{x\in X,\ x\notin \ f(x)}
   $$
   a vezměme  $a\in X$ takové, že $A=f(a)$. Nemůže být $a\notin A=f(a)$ protože pak by $a\in A$ podle definice $A$. Ale nemůže být ani $a\in A$ protože pak by ze stejného důvodu bylo $a\notin A$.\sq
   
   
   
   \newpage
  .
   \newpage

\centerline{\Large\bf III. Řady.} 
 
 \vskip10mm
 
 {\large\bf 1. Sčítání posloupnosti jako limita částečných součtů }
 
 \bigskip
 
 {\bf 1.1.} Buď $(a_n)_n$ posloupnost reálných čísel. K ní přiřazená {\em řada}
 $$
 \sum_{n=0}^\infty a_n \qtq{or} a_0+a_1+a_2+\cdots
 $$
 je limita $\lim_n\sum_{k=0}^n a_k$, pokud existuje. 
 
 Přesněji, existuje-li ta limita, mluvíme o konvergentní řadě, jinak se říká, že jde o řadu  
 {\em divergentní}.
 
 \bigskip
 
 {\bf 1.2. Snadno sčítatelná řada: \v rada geometrická.} Buď $q$ reálné číslo, $0\leq q<1$.
Pro konečné součty
 $$
 s(n)=1+q+q^2+\cdots +q^n.
 $$
máme
 $$
 q\cdot s(n)=q+q^2+\cdots +q^{n+1} =s(n)-1 +q^{n+1}
 $$
takže
 $$
 s(n)=\frac{1-q^{n+1}}{1-q} 
 $$
 a jelikož $\lim_n q^n=0$ (jinak by bylo $a=\inf_nq^n>0$ a proto $\frac{a}{q}>a$ takže pro nějaké $k$,
 $q^k<\frac{a}{q}$ a $q^{k+1}<a$ --  spor) dostaneme
 $$
 \sum_{n=0}^\infty q^n = \lim_ns(n)= \frac1{1-q}. \quad\quad \square
 $$
 
 \bigskip
 
 {\bf 1.3. Tvrzení.} {\em Nechť řada $\sum_{n=0}^\infty a_n$ konverguje. Potom $\lim_na_n=0$.
 
 Důkaz.} Nechť ne. Potom existuje $b>0$ takové, že pro každé $n$ existuje $p_n> n$ takové, že 
 $|a_{p_n}|\geq b$. Tedy
 $$
 \left|\sum_{k=0}^{p_n}a_k-\sum_{k=0}^{p_n-1} a_k\right|=|a_{p_n}|\geq b
 $$
 a posloupnost $(\sum_{k=0}^n a_k)_n$ nen\'\i\ ani Cauchyovská.   \sq
 
 \bigskip
 
 {\bf 1.4. Jeden divergentní případ:  harmonická řada.} Nutná podmínka z 1.3 není postačující. Zde je příklad, t.zv. {\em harmonická řada}
 $$
 1+\frac12+\frac13+\cdots+\frac1n+\cdots .
 $$
 Vezměme konečné součty
 $$
 S_n=\sum_{k=10^n+1}^{10^{n+1}}\frac1k
 $$
 (tedy,
 $$
 S_0=\frac12+\cdots+\frac1{10},\ S_1=\frac1{11}+\cdots+\frac1{100},\ S_2=\frac1{101}+\cdots+\frac1{1000}, \ \ \text{atd.}).
 $$ 
 $S_n$ má $9\cdot10^{n}$ sčítanců, každý z nich $\geq \frac1{10^{n+1}}$ takže $S_n\geq \frac9{10}$ a tedy
 $$
 \sum_{k=0}^{10^{n+1}}\frac1k= 1+S_0+\cdots S_n\geq1+n\frac9{10}.
 $$
 
 \medskip
 
 {\bf 1.4.1.} Z téhož důvodu máme divergentní řady
 $$
 \frac12+\frac14+\frac16+\cdots\qtq{a} 1+\frac13+\frac15+\frac17+\cdots
 $$
 
 
 \vskip10mm
 
  {\large\bf 2. Absolutně konvergentní řady}
 
 \bigskip
 
 {\bf 2.1.} Řada $\sum_{n=1}^\infty a_n$ je {\em absolutně konvergentní} konverguje-li  řada
 $$
 \sum_{n=1}^\infty|a_n|.
 $$

 
 \bigskip
 
 {\bf 2.2. Tvrzení.} {\em Absolutně konvergentní řada konverguje.
 
Obecněji, pokud $|a_n|\leq b_n$ pro všechna $n$ a  $\sum_{n=1}^\infty b_n$
 converguje potom $\sum_{n=1}^\infty a_n$ konverguje.
 
 Důkaz.} Položme
 $$
 s_n=\sum_{k=1}^n a_k\qtq{a} \ol s_n=\sum_{k=1}^n b_k
 $$
 a připomeňme si II.3. Posloupnost $(\ol s_n)_n$ konverguje a je tedy Cauchyovská. Pro $m< n$ máme
 $$
 |s_n-s_m|=|\sum_{k=m+1}^n a_k|\leq \sum_{k=m+1}^n |a_k|\leq \sum_{k=m+1}^n b_k=|\ol s_n-\ol s_m|;
 $$
takže i posloupnost $(s_n)_n$ je Cauchyovská, a  tedy konvergentní. \sq
 
 \medskip
 
 {\bf Poznámka.}  Je to příklad důležitého důsledku Bolzano-Cauchyovy Věty. Všimněte si, že zde máme zaručenu existenci sumy o jejíž hodnotě nemáme žádnou informaci.
 
 \bigskip
 
 {\bf 2.3. Věta.} {\em Řada $\sum_{n=0}^\infty a_n$ konverguje absolutně právě když pro každé $\epsilon>0$ existuje  $n_0$ takové, že
pro každou konečnou $K\sue \setof{n}{n\geq n_0}$ je $\sum_{k\in K}|a_k|<\epsilon$.
 
 Důkaz.} Pro posloupnost $(x_n)_n$ kde $x_n=\sum_{k=0}^n |a_k|$ a $n_0\leq n\leq m$ máme $|x_n-x_m|=\sum_{m\leq k\leq n}|a_k|$. Podmínka o konečných podmnožinách $K$ (připomeňme si, že sčítance jsou nezáporné), je jen přeformulování požadavku aby $(x_n)_n$ byla Cauchyovská. \sq
 
 \medskip
 
 {\bf 2.3.1. Poznámka.} Podle Věty 2.3 vidíme, že součet absolutně konvergentní řady je s libovolnou přesností
 aproximován součty přes konečné podmnožiny indexů: pro každé $\epsilon$ máme konečnou podmnožinu množiny $\Nbb$ takovou, že přes žádnou konečnou množinu ve zbytku členů $|a_k|$ nedostaneme v absolutní hodnotě větší součet než  $\epsilon$. V následující větě uvidíme další aspekt tohoto faktu:  absolutně konvergentní řada může být libovolně přeházena a součet se nezmění.
 
Pro neabsolutně konvergentní řady tomu tak není. Tam je součet opravdu jen limita \'useků jak za sebou do sebe zapadají a výsledek závisí na pořadí  $a_1,a_2,a_3,\dots$. To uvidíme v příští sekci. 
 
 \bigskip
 
 {\bf 2.4. Věta.} {\em Nechť $s=\sum_{n=1}^\infty a_n$ absolutně konverguje. Potom hodnota součtu nezávisí na  seřazení sčítanců v posloupnosti $a_n$.  Přesněji, pro každé vzájemně jednoznačné zobrazení $p:\Nbb\to\Nbb$ konverguje řada
 $\sum_{n=1}^\infty a_{p(n)}$ ke stejnému součtu $s$.
 
 Důkaz.} Pro $\epsilon>0$ nejprve zvolme podle  2.3  $n_1$ takové, aby pro každou konečnou
  $K\sue \setof{n}{n\geq n_1}$ byl součet $\sum_{k\in K}|a_k|<\epsilon$. Dále zvolme $n_2\geq n_1$ tak aby
 $|\sum_{k=1}^{n_2}a_k-s|<\epsilon$. Konečně pak zvolme $n_0\geq n_2$ takové že pro $n\geq n_0$ je
 $$
 \set{p(1),\dots,p(n)}\supe\set{1,2,\dots,n_2}.
 $$
Buď nyní $n\geq n_0$. Položme $K=\set{p(1),\dots,p(n)}\smin\set{1,2,\dots,n_2}$. Máme
 $$
 \begin{aligned}
 |\sum_{k=1}^{n}a_{p(k)}&-s|=|\sum_{k=1}^{n_2}a_k + \sum_{k\in K}a_k-s|=\\
 &= |\sum_{k=1}^{n_2}a_k-s+\sum_{k\in K}a_k|\leq |\sum_{k=1}^{n_2}a_k-s|+\sum_{k\in K}|a_k|<2\epsilon. \quad\square
 \end{aligned}
 $$
 
 \bigskip
 
 {\bf 2.5. Dvě kriteria absolutní konvergence.} Sčítatelnost geometrické řady (viz 1.2 a Tvrzení 2.2) vede k následujícím jednoduchým kriteriím absolutní konvergence.
 
 \medskip
 
 {\bf 2.5.1. Tvrzení.} (D'Alembertovo Kriterium Konvergence) {\em Nechť existují $q<1$ a $n_0$ taková,že pro všechna $n\geq n_0$,
 $$
 \left|\frac{a_{n+1}}{a_n}\right|\leq q.
 $$
 Potom $\sum_{n=1}^\infty a_n$ absolutně konverguje. Existuje-li $n_0$ takové, že pro $n\geq n_0$ 
 $$
 \left|\frac{a_{n+1}}{a_n}\right|\geq 1 
 $$
řada $\sum_{n=1}^\infty a_n$ diverguje.
 
 Důkaz.} Platí-li první, máme pro $n\geq n_0$, $|a_{n+1}| \leq q|a_n|$ takže
 $|a_{n+k}|\leq |a_{n_0}|\cdot q^k$.
 
 Druhé tvrzení je triviální.\sq
 
 \medskip
 
 {\bf 2.5.2. Tvrzení.} {Cauchyovo Kriterium Konvergence) {\em Nechť existují $q<1$ a $n_0$ taková,že pro všechna $n\geq n_0$,
 $$
 \sqrt[n]{|a_n|}\leq q.
 $$
 Potom $\sum_{n=1}^\infty a_n$ absolutně konveguje. Existuje-li $n_0$ takové, že pro $n\geq n_0$
$\sqrt[n]{|a_n|} \geq 1 $
Potom $\sum_{n=1}^\infty a_n$ diverguje.
 
 Důkaz.} To je ještě snadnější: je-li $\sqrt[n]{|a_n|}\leq q$ je
 $|a_n|\leq q^n$.\sq
 
 \medskip
 
 {\bf 2.5.3.} Tato kriteria jsou častom presentována v trochu slabší, ale průhlednější formě:
 
 \smallskip
 
 {\em Jestliže\  $\lim_n \left|\frac{a_{n+1}}{a_n}\right|<1$ resp. $\lim_n\sqrt[n]{|a_n|}<1$
 potom $\sum_{n=1}^\infty a_n$ konverguje absolutně,
      jestliže $\lim_n \left|\frac{a_{n+1}}{a_n}\right|>1$ resp. $\lim_n\sqrt[n]{|a_n|}>1$
 potom řada $\sum_{n=1}^\infty a_n$ nekonverguje vůbec.}
 
 \smallskip
 
 V této formulaci je zřejmá mezera: co se stane je-li ta limita 1? 
 Cokoli: taková řada může být i absolutně konvergentní, nebo konvergentní ale ne absolutně, nebo třeba nekonverguje vůbec (to poslední jsme viděli 1.4, příklady prvních dvou uvidíme dále v 3.2).
 
 \vskip10mm
 
  {\large\bf 3. Neabsolutně konvergentní řady}
 
 \bigskip
 
 {\bf 3.1. Alternující řada.} Již jsme viděli, že $\lim a_n=0$ obecně nestačí k tomu, aby řada konvegovala. V následujícím důležitém případě však ano.
 
 \medskip
 
 {\bf Tvrzení.} {\em Nechť $a_n\geq a_{n+1}$ pro všechna $n$. Potom řada
 $$
 a_1-a_2+a_3- a_4+\cdots
 $$
 konverguje právě když $\lim_na_n=0$.
 
 Důkaz.} Definujme $s_n=\sum_{k=0}^n(-1)^{n+1}a_k$. Máme
 $$
 s_{2n+2}=s_{2n} + a_{2n+1}-a_{2n+2}\leq s_{2n}
 \qtq{and} s_{2n+3}=s_{2n+1} - a_{2n+2}+a_{2n+3}\geq s_{2n+1}.
 $$
 Takže zde jsou dvě posloupnosti
$$
 \begin{aligned}
 &s_1\geq s_3\geq\cdots\geq s_{2n+1}\geq\cdots,\\
 &s_2\leq s_4\leq\cdots\leq s_{2n}\leq\cdots,
 \end{aligned}
 $$
 obě z nich konvergentní podle
 II.2.6. Máme $s_{2n+1}-s_{2n}= a_{2n+1}$ Takže obě konvegují k témuž číslu (a tedy k $\lim_ns_n$) právě když $\lim_na_n=0$.\sq
 
 \bigskip
 
 {\bf 3.2. Poznámky.} 1. Speciálně máme konvergentní řadu
 \begin{equation}
 1-\frac12+\frac13-\frac14+\frac15-\cdots.
 \tag{$*$}
 \end{equation}
 Ta podle 1.4 ale není absolutně konvergentní. Všimněte si, že zde $\lim_n\left|\frac{a_{n+1}}{a_n}\right|=1$ (viz též 2.5.3)
 
 \smallskip
 
 2. Vezměme řadu ($*$) a přepišme ji na
 $$ \left(1-\frac12\right)+\left(\frac13-\frac14\right)+\left(\frac15-\frac16\right)+\cdots,
 $$
 tedy na
 $$
 \frac1{1\cdot 2}+\frac1{3\cdot 4}+\frac1{5\cdot 6}+\cdots .
 $$
 To je posloupnost positivních čísel se stejným součtem jako ($*$). Je tedy absolutně konvergentní a máme zde též $\lim_n\left|\frac{a_{n+1}}{a_n}\right|=\lim_n\left|\frac{(2n+1)(2n+2)}{(2n+3)(2n+4)}\right|=1$ (viz opět 2.5.3).
 
 \bigskip
 
 {\bf 3.3.} Na konec ještě ukážeme, že součet konvergentní ale nikoli absolutně konvergentní řady je pouze ta limita
z definice, a nemůže být nahlížena jako
 ``spočetný součet''.
 
  Mějme tedy $\sum_{n=1}^\infty a_n$ konvergentní, ale nikoli absolutně.  Posloupnost $(a_n)_n$ rozdělme na dvě
 $$
 \begin{aligned}
 &B:\quad b_1,b_2,b_3,\dots,\\
 &C:\quad c_1,c_2,c_3,\dots,
 \end{aligned}
 $$
 první sestávající ze všech kladných $a_n$, druhá z těch záporných, obě v tom pořadí jak se objevují v původní $(a_n)_n$.
 
 \medskip
 
 {\bf 3.3.1. Lemma.} {\em Žádná z posloupností $(\sum_{k=1}^nb_k)_n$,
 $(\sum_{k=1}^n(-c_k))_n$ nemá horní mez.
 
 Důkaz.} 1. Nechť ji mají obě. Potom $\sum_{n=1}^\infty b_n$ i 
 $\sum_{n=1}^\infty c_n$ jsou absolutně konvergentní. Pro $\epsilon>0$ zvolme $n_1$
takové, že pro každou $K\sue \setof{n}{n\geq n_1}$ je $\sum_{k\in K}|b_k|<\epsilon$ a $\sum_{k\in K}|c_k|<\epsilon$. Zvolíme-li nyní $n_0$ tak aby $\set{a_1,\dots,a_{n_0}}$ již obsahovala $\set{b_1,\dots,b_{n_1}}$ a $\set{c_1.\dots,c_{n_1}}$ máme pro každou konečnou $K\sue \setof{n}{n\geq n_0}$ součet $\sum_{k\in K}|a_k|<2\epsilon$ a vidíme, že $\sum_{n=1}^\infty a_n$ absolutně konverguje.
 
 \smallskip
 
 2. Buď $(\sum_{k=1}^n(-c_k))_n$ omezená a $(\sum_{k=1}^nb_k)_n$ ne. Potom $\sum_{n=1}^\infty c_n$ je absolutně konvergentní; zvolme $n_1$
tak aby pro každou  $K\sue \setof{n}{n\geq n_1}$ bylo $\sum_{k\in K}|c_k|<1$. Je-li $n_0$ takové, že $\set{a_1,\dots,a_{n_0}}$ obsahuje úsek $\set{c_1,\dots,c_{n_1}}$ máme pro $n\geq n_0$  $\sum_{k=1}^na_k>
 \sum_{k=1}^nb_k -\sum_{k=1}^{n_1}|c_k| -1$ a tedy $(\sum_{k=1}^na_k)_n$ není omezená a nemůže konvergovat. \sq
 
 \medskip
 
 {\bf 3.3.2. Tvrzení.} {\em Buď $\sum_{n=1}^\infty a_n$ konvergentní ale ne absolutně. Buď  $r$ libovolné reálné číslo. Potom může být řada přeházena tak, že $\sum_{n=1}^\infty a_{p(n)}$ ($p:\Nbb\to\Nbb$ je vhodné vzájemně jednoznačné zobrazení) je rovna $r$.
 
 Důkaz.} Buď, dejme tomu,  $r\geq 0$. Buď $n_1$ první přirozené číslo takové, že
 $\sum_{k=1}^{n_1}b_k > r$. Dále vezměme první $m_1$ takové, že
 $\sum_{k=1}^{n_1}b_k +\sum_{k=1}^{m_1}b_k <r$. Potom $n_2$ první takové, že
 $$
 \sum_{k=1}^{n_1}b_k +\sum_{k=1}^{n_1}b_k +\sum_{k=n_1+1}^{n_2}b_k >r
 $$
  a $m_2$ první takové,že
 $$
 \sum_{k=1}^{n_1}b_k +\sum_{k=1}^{n_1}b_k +\sum_{k=n_1+1}^{n_2}b_k +
 \sum_{k=m_1+1}^{m_2}c_k<r.
 $$
 Když takto pokračujeme a vezmeme v úvahu to, že obě $(b_n)_n$ and $(c_n)_n$ (podposloupnosti $(a_n)_n$)
konvergují k nule, vidíme, že
$$
 \begin{aligned}
 b_1&+\cdots+b_{n_1}+c_1+\cdots+c_{m_1}+b_{n_1+1}+\cdots+b_{n_2}+c_{m_1+1}+\cdots+
 c_{m_2}+\cdots\\
 &\cdots+b_{n_k+1}+\cdots+b_{n_{k+1}}+c_{m_k+1}+\cdots+
 c_{m_{k+1}}+\cdots =r     \quad\quad\quad\square
 \end{aligned}
 $$
 

\newpage
. 
 
 
 \newpage
 
 \centerline{\Large\bf IV. Spojité reálné funkce} 
 
 \vskip10mm
 
 {\large\bf 1. Intervaly}
 
 \bigskip
 
 {\bf 1.1. Označení a terminologie.} Připomeňme si standardní značení. Pro $a\leq b$ položíme
 $$
 \begin{aligned} 
 &(a,b)=\setof{x}{a<x<b}\\
 &\langle a,b)=\setof{x}{a\leq x<b}\\
 &(a,b\rangle=\setof{x}{a<x\leq b}\\
 &\langle a,b\rangle=\setof{x}{a\leq x\leq b}\\
 &(a,+\infty)=\setof{x}{a< x}\\
 &\langle a,+\infty)=\setof{x}{a\leq x}\\
 &(-\infty,b)=\setof{x}{x<b}\\
  &(-\infty,b\rangle=\setof{x}{x\leq b}
 \end{aligned}
 $$
 Tyto podmnožiny $\Rbb$, a navíc $\ems$ a $\Rbb$, se nazývají (reálné) {\em intervaly}. O prvních čtyřech $\ems$ se mluví jako o {\em omezených intervalech}.
 
Dále, v případech 1, 5, 7, $\ems$ a $\Rbb$ mluvíme o  {\em otevřených intervalech}, a
 v případech 4, 5, 8, $\ems$ a $\Rbb$ o {\em uzavřených intervalech}. Všimněte si, že $\ems$ and
 $\Rbb$ (a jen ty) jsou zároveň otevřené i uzavřené.
 
 \medskip
 
 {\bf 1.1.1. Pozor možn\'e nedorozumnění.} Symbol ``$(a,b)$'' jsme již užívali pro uspořádanou dvojici prvků. Budeme to dělat i dále: čtenář jistě rozezná z kontextu jde li o interval nebo o dvojici.
 
 \bigskip
 
 {\bf 1.2. Obecná charakteristika intervalu.} Podmnožina $\Rbb$ se nazývá 
{\em interval} jestliže
 \begin{equation}
 \forall a,b\in J\ \ (a\leq x\leq b\ \Rightarrow\ x\in J). \tag{int}
 \end{equation}
 
 \medskip
 
 {\bf 1.2.1. Tvrzení.} {\em Podmnožina $J\sue\Rbb$ je interval ve smyslu definice {\em (int)} právě když je to  jeden z případů uváděných v 1.1 (zahrnujeme opět $\ems$ a $\Rbb$).
 
 Důkaz.} Každá z množin z 1.1 zřejmě splňuje (int).
 
 Nechť nyní $J$ splňuje (int) a nechť je neprázdná. 
 
 \smallskip
 
 (a) Nechť  $J$ má horní i dolní mez. Potom existují $a=\inf J$ a
 $b=\sup J$. 
 
 (a1) Jestliže $a,b\in J$ potom zřejmě $J=\langle a,b\rangle$.
 
 (a2) Jestliže $a\in J$ ale $b\notin J$ a je-li $a\leq x< b$ máme podle definice infima $y\in J$ takové, že $x< y$ a tedy podle (int) $x\in J$ a vidíme, že $J= \langle a,b)$.
 
 (a3) Podobně v případě, že $a\notin J$ a $b\in J$ zjistíme, že $J=(a,b\rangle$.
 
 (a4) Není-li žádné z $a,b$ v $J$ a je-li $a<x<b$ zvolíme podle definic suprema a infima $y,z\in J$ taková, že $a<y<x<z<b$ a zjistíme, že  $J=(a,b)$.
 
 \smallskip
 
 (b) Má-li $J$ dolní, ale ne horní mez položme $a=\inf J$.
 
 (b1) Je-li $a\in J$ uvažujme jako v (a2), s $y\in J$ takovým, že $a\leq x<y$ získaným 
z nedostatku horní meze, a snadno zjistíme, že $J=\langle a,+\infty)$.
 
 (b2) Jestliže $a\notin J$ uvažujeme jako v (a4) volbou $y$ z definice  infima and $z$ z neexistence horní meze. Dostaneme $J=(a,+\infty)$.
 
 \smallskip
 
 (c) Má-li $J$ hérní ale ne dolní mez vezmeme  $b=\sup J$a analogicky jako v (b) zjistíme, že $J$ je buď $(+\infty,b \rangle$ nebo $(+\infty,b)$.
 
 \smallskip
 
 (d) Konečně jestliže $J$ nemá horní ani dolní mez zjistíme snadno (podobně jako v (a4)) že $J=\Rbb$. \sq
 
\bigskip

{\bf 1.3. Kompaktní intervaly.} Omezené uzavřené intervaly $\langle a,b\rangle$
mají zvláště dobré vlastnosti. Budeme o nich mluvit jako o {\em kompaktních intervalech} (jsou to speciální případy velmi důležitých {\em kompaktních prostorů} o nichž bude řeč později). Zejména budeme  užívat větu II.2.7 v následující reformulaci.

\medskip

{\bf 1.3.1. Věta.} {\em Každá posloupnost v kompaktním intervalu  $J$ obsahuje podposloupnost konvergující v $J$.}

\vskip10mm
 
 {\large\bf 2. Spojité reálné funkce jedné reálné proměnné}
 
 \bigskip
 
 {\bf 2.1.} Budou nás zajímat funkce $f:D\to\Rbb$ s definičním oborem $D$ typicky intervalem nebo názorným sjednocením intervalů. Pokud nezdůrazníme jinak, budeme o těchto {\em reálných funkcích jedné reálné proměnné} mluvit prostě
 jako o  {\em funkcích}.
 
 \bigskip
 
 {\bf 2.2. Spojitost.} Řekneme, že funkce $f:D\to\Rbb$ je  {\em spojitá v bodě} $x\in D$ jestliže
 $$
 \forall \epsilon>0\ \exists \delta>0\ \ \text{takové, že}\ \ (y\in D \ \text{a}\ |y-x|<\delta)\ \Rightarrow\ |f(y)-f(x)|<\epsilon.
 $$ 
Funkce $f:D\to\Rbb$ je {\em spojitá} je-li spojitá v každém bodě $x\in D$, tedy jestliže
 $$
 \forall x\in D\  \forall \epsilon>0 \ \exists \delta>0\ \ \ \ ((y\in D \ \text{a}\ |y-x|<\delta)\ \Rightarrow\ |f(y)-f(x)|<\epsilon).
 $$ 
 
 \medskip 
 
 {\bf 2.2.1. Konstanty a identita.} Například konstantní funkce $f:D\to\Rbb$ definovaná předpisem
 $f(x)=c$ \ pro všechna $x\in D$, nebo $f:D\to\Rbb$ definovaná předpisem $f(x)=x$ jsou spojité.
 
 \bigskip
 
 {\bf 2.3. Aritmetické operace s funkcemi.} Pro $f,g:D\to\Rbb$ a $\alpha\in \Rbb$ definujeme
 $$
 f+g,\ \alpha f,\ fg\ \text{a pokud $g(x)\neq 0$ pro $x\in D$}, \ \frac{f}{g}
 $$
 předpisy
 $$
 \begin{aligned}
 (f+g)(x)=f(x)&+g(x),\ \ (\alpha f)(x)=\alpha f(x),\\ 
 &(fg)(x)=f(x)g(x)\ \ \text{a} \ \ \left(\frac{f}{g}\right)(x)= \frac{f(x)}{g(x)}.
 \end{aligned}
 $$
 
 \medskip
 
 {\bf 2.3.1. Tvrzení.} {\em Nechť  $f,g:D\to\Rbb$ jsou spojité v $x$ a nechť
$\alpha$ je reálné číslo. Potom
 $f+g$, $\alpha f$, $fg$, a je-li  $g(x)\neq 0$ pro $x\in D$, též  $\frac{f}{g}$,
 jsou spojité v $x$.
 
 Důkaz.} Je to zcela analogické důkazu v II.2.3 - jediný rozdíl je v hledání čísel $\delta$ místo $n_0$. Jen pro ilustraci důkaz provedeme, tentokrát krajně pedantsky, pro součin $fg$. Všimněte si však, že ta přílišná pečlivost směřující k upravenému  $\epsilon$, místo toho abychom prostě mysleli v představě  ``libovolně malého'', jen zamlžuje podstatu věci. Jako cvičení si důkaz zopakujte bez pedantských úprav.

 \smallskip
 
 Vezměme $\epsilon>0$. Zvolme
 $$
 \begin{aligned}
 &\delta_1> 0 \ \text{takové, že}\  |y-x|<\delta_1 \ \Rightarrow\  
 |f(y)|\leq |f(x)|,\\
 &\delta_2> 0 \ \text{takové, že}\  |y-x|<\delta_2 \ \Rightarrow\  
 |f(y)-f(x)|< \frac{\epsilon}{2(|g(x)|+1)},\\
 &\delta_3> 0 \ \text{takové, že}\  |y-x|<\delta_3 \ \Rightarrow\  
 |g(y)-g(x)|< \frac{\epsilon}{2(|f(x)|+1)}
 \end{aligned}
 $$
a položme $\delta=\min(\delta_1,\delta_2,\delta_3)$. Je-li $|y-x|<\delta$ máme
 $$
 \begin{aligned}
 |f(x)g(x)-&f(y)g(y)|=|f(x)g(x)-f(y)g(x)+f(y)g(x)-f(y)g(y)|=\\
 &=|(f(x)-f(y))g(x)+f(y)(g(x)-g(y))|\leq \\
 &\leq|g(x)||f(x)-f(y)|+|f(y)||g(x)-g(y)|< \\
 &<(|g(x)|+1)\frac{\epsilon}{2(|g(x)|+1)} +
 (|f(x)|+1)\frac{\epsilon}{2(|f(x)|+1)}=\epsilon.\quad\square
 \end{aligned} 
 $$
 
 
 \medskip
 
 {\bf 2.3.2.} Následující tvrzení mohu ponechat čtenáři jako lehké cvičení.

\smallskip
 
 {\bf Tvrzení.} {\em Pro $f,g:D\to \Rbb$ definujme $\max(f,g)$, $\min(f,g)$ a
 $|f|$ předpisy
 $$
 \begin{aligned}
 \max(f,g)(x)&=\max(f(x),g(x)), \ \min(f,g)=\min(f(x),g(x)) \\ 
 &\text{a}\ \  
 |f|(x)=|f(x)|.
 \end{aligned}
 $$
Jsou-li $f$ and $g$ spojité v $x$ jsou in $\max(f,g)$, $\min(f,g)$ a 
 $|f|$ spojité v $x$.}
 
 \bigskip
 
 {\bf 2.4. Skládání reálných funkcí.} Buďte $f:D\to\Rbb$ a $g:E\to\Rbb$ reálné funkce a nechť $f[D]=\setof{f(x)}{x\in D}\sue E$. Definujeme pak složení funkcí $f$ a $g$, značené
 $$
 g\circ f,
 $$
 předpisem $(g\circ f)(x)= g(f(x))$.
 
 \medskip
 
 {\bf 2.4.1. Tvrzení.} {\em Buď $f:D\to\Rbb$ spojitá v $x$ a $g:E\to\Rbb$ spojitá v $f(x)$. Potom je $g\circ f$ spojitá v $x$.
 
 Důkaz.} Vezměme $\epsilon> O$, zvolme  $\eta>0$ tak aby $|z-f(x)|<\eta$ implikovalo
 $|g(z)-g(f(x))|<\epsilon$, a $\delta>0$ tak aby $|y-x|<\delta$ implikovalo
 $|f(y)-f(x)|<\eta$. Potom $|y-x|<\delta$ implikuje
 $|g(f(y))-g(f(x))|<\epsilon$, \sq
 
 
 


\vskip10mm
 
 {\large\bf 3.  Darbouxova věta}
 
 \bigskip
 
 {\bf 3.1. Věta.} {\em Buď $f:J\to \Rbb$ spojitá funkce definovaná na intervalu $J$. Buďte $a,b\in J$, $a<b$, a nechť je $f(a)f(b)<0$. Potom existuje $c\in (a,b)$ takové, že $f(c)=0$.
 
 Důkaz.} Nechť třeba $f(a)<0<f(b)$ (jinak zkoumejme $-f$ a užijme toho, že je to spojitá funkce právě když to platí o $f$).
 
 Položme
 $$
 M=\setof{x}{a\leq x\leq b,\ f(x)\leq 0}.
 $$
 Jelikož je $a\in M$, $M\neq \ems$, a $M$ má horní mez $b$ z definice. Proto existuje
 $$
 c=\sup M
 $$
 a platí $a\leq c\leq b$ takže $c\in J$ a hodnota $f(c)$ je definována.
 
 Nechť $f(c)<0$.  Vezměmě $\epsilon=-f(c)$ a $\delta >0$ takové, že pro
 $x$ kde $|c-x|\leq \delta$ je $f(c)-\epsilon<f(x)<f(c)+\epsilon$. Zvláště je pro $c\leq x<c+\delta$ dosud $f(x)<f(c)+(-f(c))=0$ takž $c$ není horní mez $M$.
 
 Nechť $f(c)>0$. Vezměme $\epsilon=f(c)$  a $\delta >0$ takové, že pro
 $x$ kde $|c-x|\leq \delta$ máme $f(c)-\epsilon<f(x)<f(c)+\epsilon$.
 Teď je zase pro $c-\delta< x$ již $0=f(c)-f(c)<f(x)$ (je-li $x>c$, je $0<f(x)$ z definice množiny $M$) a
existují horní meze menší než  $c$.
 
 Tedy $f(c)$ není ani menší ani větší než 0 a zůstává jen možnost $f(c)=0$.\sq
 
 \bigskip
 
 {\bf 3.2. Věta.} (Věta Darbouxova) {\em Buď $f:D\to\Rbb$ spojitá funkce, a nechť $J$ je interval $J\sue D$. potom jeho obraz $f[J]$ je též interval.
 
 Důkaz.} Nechť jsou $a<b$ v $J$ a nechť $f(a)<y<f(b)$ nebo $f(a)>y>f(b)$. Definujme  $g:D\to \Rbb$ předpisem $g(x)=f(x)-y$. Podle 2.2.1 a 2.3.1 je $g$ spojitá. Máme $g(a)g(b)<0$ a tedy podle 3.1 existuje $x$ mezi $a$ a $b$ (a
tedy v $x$)
 takové, že $g(x)=f(x)-y=0$, tedy $f(x)=y$. \sq
 
 \bigskip
 
 {\bf 3.3. Úmluva.} \v Rekneme, \v ze funkce $f:D\to\Rbb$ je {\em rostoucí, neklesající, nerostoucí} resp. {\em klesající} jestliže
 $$
 x<y\quad{\Rightarrow}\quad f(x)<f(y),\ f(x)\leq f(y), \ f(x)\geq f(y),\   \ \text{resp.}\ 
 f(x)>f(y).
 $$
 Na rozdíl od zvyklosti v teorii usořádaných množin (kde rozlišujeme monotonní a antitonní zobrazení), v analyse
se užívá termín {\em monotonní zobrazení} jako společný pro všechny tyto případy.
 
 Jestliže $x<y$ implikuje $f(x)<f(y)$ resp. $f(x)>f(y)$ mluvíme o {\em ryze monotonních zobrazeních}.
 
 \bigskip
 
 {\bf 3.4. Tvrzení.} {\em Buď $J$  interval a buď $f:J\to\Rbb$ prosté spojité zobrazení. Potom je $f$ ryze monotonní.
 
 Důkaz.} Jinak existují $a<b<c$ taková, že $f(a)<f(b)> f(c)$ nebo 
 $f(a)>f(b)< f(c)$. Diskutujme první případ, druhý je zcela analogický.
 Zvolme $y$ takové, že $\max(f(a),f(c))< y<f(b)$. Užitím Věty 3.2 pro interval $\langle a,b \rangle$ dostaneme $x_1$, $a<x_1<b$, s hodnotou $f(x_1)=y$, a když totéž učiníme v intervalu
  $\langle b,c \rangle$ získáme $x_2$, $b<x_2<c$ též s $f(x_2)=y$. $f$ tedy není prosté. \sq 
 
 
 
 \vskip10mm
 
 {\large\bf 4. Spojitost monotonních a inversních funkcí}
 
 \bigskip
 
 {\bf 4.1. Věta.} 
 {\em  Monotonní funkce $f:J\to \Rbb$ na intervalu $J$ 
je spojitá právě když $f[J]$ je interval.
 
 Důkaz.} I. Kdyby $f[J]$ nebyl interval $f$ by nebyla spojitá podle 3.2.
 
 II. Nyní buď $f[J]$ interval,  $x\in J$; necht to není krajní bod toho intervalu, takže existují $x_1<x<x_2$ stále v $J$. Zvolme $\epsilon>0$
 aby bylo
 $|f(x)-f(y)|=0$ pro $x-\delta<y<x+\delta$.
 
 Jestliže $f(x_1)<f(x)=f(x_2)$ zvolme $u$ takové, že $\min(f(x_1),f(x)-\epsilon)<u<f(x)$  a, podle 3.2, $x_1'$ takové, že $f(x_1')=u$. 
 Pokud nyní je $0<\delta\leq x-x_1',x_2-x$ dostáváme z monotonie
 $f(x)-\epsilon<f(y)\leq f(x)$ pro $x-\delta<y<x+\delta$.
 
 
 Jestliže $f(x_1)=f(x)<f(x_2)$ zvolme $v$ tak, aby $f(x)<v<\min(f(x_2),f(x)+\epsilon)$ a podle 3.2, $x_2'$ takové že $f(x_2')=v$.
Je-li $0<\delta\leq x-x_1,x_2'-x$ dostáváme z monotonie
 $f(x)\leq f(y)<f(x)+\epsilon$ pro $x-\delta<y<x+\delta$.
 
 Jestliže $f(x_1)<f(x)<f(x_2)$ zvolme $u,v$ taková, že $\max(f(x_2),f(x)-\epsilon)<u<f(x)<v<\min(f(x_2),f(x)+\epsilon)$ a podle 3.2,  $x_1',x_2'$ tak, aby $f(x_1')=u$ a $f(x_2')=v$.
  Je-li $0<\delta\leq x-x'_1,x_2'-x$ dostáváme z monotonie
 $f(x)-\epsilon< f(y)<f(x)+\epsilon$ pro $x-\delta<y<x+\delta$.
 
 \smallskip
 
 Případy krajních bodů intervalu (má-li je) jsou zcela analogické, vlastně jednodušší protože se sta\v c\'\i\  zabývat jen jednou stranou okolí bodu $x$. \sq
 
 \medskip
 
 {\bf Poznámka.} Museli jsme zvlášť diskutovat případy $f(x_1)=f(x)=f(x_2)$, $f(x_1)< f(x) =f(x_2)$ a $f(x_1)=f(x)<f(x_2)$ protože předpokládáme $f$ jen monotonní, ne nutně ryze monotonní. Čtenář ovšem vidí, že podstata je patrná z případu $f(x_1)<f(x)<f(x_2)$; při prvním čtení doporučuji první tři případy přeskočit, důkaz bude (ještě) průhlednější.
 
 \bigskip
 
 
 {\bf 4.2. Inverse  reálné funkce $f:D\to \Rbb$.} Inverse funkce $f:D\to \Rbb$ je reálná funkce $g:E\to\Rbb$, kde $E=f[D]$, taková, že $g\circ f$ a $f\circ g$ existují a
 $f(g(x))=x$ and $g(f(x))=x$ pro všechna $x\in E$ resp.  $x\in D$.
 
 \smallskip
 
 {\bf 4.2.1. Pozorování.} {\em Je-li $g:E\to\Rbb$ inverse funkce $f:D\to\Rbb$ je
 $f:D\to\Rbb$ inverse funkce $g:E\to\Rbb$, platí $f[D]=E$ a $g[E]=D$, a $f,g$ omezené na $D,E$ jsou vzájemně inversní zobrazení.}
 
 (První tvrzení je zřejmé. Dále, položíme-li pro $y\in E$ $x=g(y)$ máme $f(x)=y$. Tedy jsou restrikce  $D\to E$ and $E\to D$ vzájemně jednoznačné.)
 
 \bigskip
 
 {\bf 4.3. Tvrzení.} {\em Buď  $f:J\to\Rbb$ funkce definovaná na intervalu $J$. Potom má inversi $g:J'\to\Rbb$ právě když je ryze monotonní, a toto zobrazení $g$ je potom spojité.
 
 Důkaz.} $f$ musí být podle 3.4 prosté a tedy ryze monotonní. To podle 2.3 znamená, že $J'=f[J]$ je interval, a inverse $g:J'\to\Rbb$ je také ryze monotonní. Máme $g[J']=J$ interval, a tedy je $g$ spojitá podle 4.1. \sq
 
 \bigskip
 
 {\bf 4.4. Poznámka.} Nyní již začínáme mít zásobu spojitých funkcí. Z 2.2.1 a 2.3.1 hned vidíme, že jsou spojité funkce dané polynomiálními formulemi
 $$
 f(x)=a_0+a_1x+a_2x^2+\cdots+a_nx^n
 $$
 a též funkce
  $$
  f(x)=\frac{a_0+a_1x+a_2x^2+\cdots+a_nx^n}{b_0+b_1x+b_2x^2+\cdots+b_mx^m}
  $$
  (říká se jim {\em racionální funkce})
  pokud ovšem definiční obor neobsahuje $x$ pro která
  $b_0+b_1x+b_2x^2+\cdots+b_mx^m=0$.
  
Dále, podle 4.3 jsou zde spojité funkce dané formulemi
  $$
  f(x)=\sqrt{x},\ \  f(x)=\sqrt[n]{x}
  $$
(se zřejmými předpoklady o definičních oborech) a všechny funkce získané ze všech zmíněných v konečně mnoha krocích skládáním, aritmetickými operacemi, a operacemi z 2.3.2. Další dostaneme v příští kapitole.

\vskip10mm
 
 {\large\bf 5. Spojité funkce na kompaktních intervalech}
 
 \bigskip
 
 {\bf 5.1. Věta.} {\em Funkce $f:D\to\Rbb$ je spojitá právě když pro každou posloupnost konvergentní v $D$ je $\lim_nf(x_n)=f(\lim_nx_n)$.
 
 Důkaz.} I. Buď $f$ spojitá a $\lim_nx_n=x$. Pro $\epsilon>0$ zvolme ze spojitosti $\delta>0$ takové,že $|f(y)-f(x)|<\epsilon$ pro $|y-x|<\delta$. Podle definice konvergence máme $n_0$ takové, že pro $n\geq n_0$, $|x_n-x|<\delta.$ Tedy, je-li $n\leq n_0$ je $|f(x_n)-f(x)|<\epsilon$ a vidíme, že
  $\lim_nf(x_n)=f(\lim_nx_n)$.
  
  \smallskip
  
  II. Nechť $f$ není spojitá. Potom máme $x\in D$ a $\epsilon_0>0$
 takové, že pro každé  $\delta>0$ existuje $x(\delta)$ takové, že 
  $$
  |x-x(\delta)|<\delta \qtq{ale} |f(x)-f(x(\delta))|\geq \epsilon_0.
  $$
  Položme $x_n=x(\frac1n)$. Potom $\lim_nx_n=x$ ale $(f(x_n))_n$ nemůže konvergovat k $f(x)$. \sq
 
 \bigskip
 
 {\bf 5.2. Věta.} {\em Spojitá funkce $f:\langle a,b\rangle\to \Rbb$ na kompaktím intervalu nabývá maxima i minima.
 To jest, existují $x_0,x_1\in\langle a,b\rangle$ taková, že pro všechna $x\in\langle a,b\rangle$,
 $$
 f(x_0) \leq f(x)\leq f(x_1).
 $$
 
 Důkaz} provedeme pro maximum. Položme
 $$
 M=\setof{f(x)}{x\in\langle a,b\rangle}
 $$
 
 I. Nejprve předpokládejme, že $M$ není shora omezená. Potom můžeme pro každé $n$ zvolit $x_n\in\langle a,b\rangle$ takové, že $f(x_n)>n$.  Podle 1.3.1 existuje podposloupnost $x_{k_n}$ s limitou $\lim_nx_{k_n}=x\in\langle a,b\rangle$. Podle 5.1, $\lim_nf(x_{k_n})=f(x)$ ve sporu s tím, že $f(x_{k_n})$ mohou být libovolně velká čísla.
 
 \smallskip
 
 II. Tedy zřejmě neprázdná množina $M$ je shora omezená a má tedy supremum 
 $s=\sup M$. Podle definice suprem máme
 $x_n\in\langle a,b\rangle$ taková, že
 \begin{equation}
 s-\frac1n < f(x_n)\leq s. \tag{$*$}
 \end{equation}
Zvolme opět podposloupnost $x_{k_n}$ s limitou $\lim_nx_{k_n}=x\in\langle a,b\rangle$. Podle 5.1 je $\lim_nf(x_{k_n})=f(x)$ a podle ($*$) je tato limita $s$. Máme tedy $f(x)=\sup M=\max M$.\sq

\bigskip


{\bf 5.4. Důsledek.} {\em Nechť jsou všechny hodnoty spojité funkce na kompaktním intervalu $J$ kladné. 
Potom existuje $c> 0$ takové, že pro všechny hodnoty $f(x)$ je $\geq c$.}

(Totiž $c=\min_Mf(x)$.)

\bigskip

{\bf 5.5. Důsledek.} {\em Nechť $f:J\to\Rbb$ je spojitá a nechť $J$ je kompaktní. Potom je $f[J]$ kompaktní interval.

Obecněji, je-li $f:D\to\Rbb$ spojitá  a $J\sue D$  kompaktní interval je $f[J]$ kompaktní interval.}

\medskip

{\bf 5.5.1. Poznámka.} Kompaktní intervaly a  $\ems$ jsou jediné intervaly jejichž typ je zachován libovolným spojitým obrazem. Pro ty ostatní je $f[J]$ opět interval, ale typ se může změnit.


 \vskip10mm
 
 {\large\bf 6. Limita funkce v bodě}
 
 \bigskip
 
 {\bf 6.1.} V následujícím, abychom se vyhnuli zbytečným písmenům, budeme vynechávat specifikaci definičních oborů v některých formulích
(např., víme-li již, že naše funkce je $f:D\to\Rbb$ a mluvíme-li o spojitosti píšeme jen  ``$\forall \epsilon>0\ \exists \delta>0\ \ \text{takové že}\ \  |y-x|<\delta\ \Rightarrow\ |f(y)-f(x)|<\epsilon$''.

\medskip


Řekneme, že funkce $f:D\to\Rbb$ {\em má limitu $b$ v bodě} $a$, a píšeme
$$
\lim_{x\to a}f(x)= b
$$
jestliže
 $$
 \forall \epsilon>0\ \exists \delta>0\ \ \text{takové, že}\ \ (0< |x-a|<\delta)\ \Rightarrow\ |f(x)-b|<\epsilon.
$$

\medskip

{\bf Poznámka.} Všimněte si nápadné podobnosti s definicí spojitosti, ale též zásadního rozdílu:
\begin{itemize}
\item[] {\em v této definici není zmínka o konkretní hodnotě funkce $f$ v bodě $a$. Bod $a$ ani nemusí být v definičním oboru
 $D$, a  pokud je, hodnota $f(a)$ nehraje žádnou roli a nemá co dělat s hodnotou  $b$.}
\end{itemize}

\bigskip

{\bf 6.2. Jednostranné limity.} Řekneme, že $f:D\to\Rbb$ {\em má limitu $b$  v bodě} $a$ {\em zprava}, a píšeme
$$
\lim_{x\to a+}f(x)= b
$$
jestliže
 $$
 \forall \epsilon>0\ \exists \delta>0\ \ \text{takové, že}\ \ (0< x-a<\delta)\ \Rightarrow\ |f(x)-b|<\epsilon.
$$
{\em Má limitu $b$  v bodě} $a$ {\em zleva}, psáno
$$
\lim_{x\to a-}f(x)= b,
$$
jestliže
 $$
 \forall \epsilon>0\ \exists \delta>0\ \ \text{takové, že}\ \ (0< a-x<\delta)\ \Rightarrow\ |f(x)-b|<\epsilon.
$$

\medskip

{\bf 6.2.1. Poznámka.} Čtenář si jistě všiml, že jsme formálně mohli dostat jednostranné limity jako speciální případy základní definice prostě změnou oboru hodnot definujíce 
$f$ jen pro $x>a$ u limity zprava, a podobně v druhém případě. Ale to by kazilo intuici. Ať je definiční obor jakýkoli, smysl definic je chování funkce při přibližování k $a$ (aniž bychom tohoto bodu dosáhli), v jednostraných případech při přibližování shora nebo zdola.

\bigskip

{\bf 6.3. Pozorování.} {\em Funkce $f:D\to\Rbb$ je spojitá v bodě $a$ právě když $\lim_{x\to a}f(x)= f(a)$.}

(Srovnejte definice.)

\medskip

{\bf 6.3.1. Jednostranná spojitost.} Řekneme, že funkce $f:D\to\Rbb$  je  {\em spojitá v bodě $a$ zprava} (resp.  {\em zleva} je-li $\lim_{x\to a+}f(x)= f(a)$ ( resp. $\lim_{x\to a-}f(x)= f(a)$).

\bigskip

{\bf 6.4. Tvrzení.} {\em Nechť limity $\lim_{x\to a}(f)(x)=A$ a  $\lim_{x\to a}g(x)=B$ existují and nechť
$\alpha$ je reálné číslo. Potom $\lim_{x\to a}(f+g)(x)$, $\lim_{x\to a}(\alpha f)(x)$, $\lim_{x\to a}(fg)(x)$, a pokud $B\neq 0$ též $\lim_{x\to a}\frac{f}{g}(x)$, existují, a jsou rovny, v tomto pořadí, $A+B$, $\alpha A$, $AB$ a $\frac{A}{B}$.

Důkaz.} Užijte 6.3 a 2.3.1. Všimněte si, že je-li $B\neq 0$ existuje $\delta_0>0$ takové, že $|x-a|<\delta_0$ je $g(x)\neq 0$. \sq

\medskip

{\bf 6.4.1.} Totéž samozřejmě platí pro jedostranné limity.

\bigskip

{\bf 6.5.} Nyní bychom mohli očekávat, že analogicky s 2.4.1 bude též platit,  když
 $\lim_{x\to a}f(x)=b$ a $\lim_{x\to b}=c$ bude též $\lim_{x\to a}(g(f(x))=c$.
To skoro platí, ale musíme být opatrní.

Uvažme následující příklad. Definujme $f,g:\Rbb\to\Rbb$ předpisy
$$
f(x)=\begin{cases}&x\ \text{pro racionální}\ x,\\
                  &0\ \text{pro iracionální}\ x
                  \end{cases}
                  \qtq{a}
 g(x)=\begin{cases}&0\ \text{pro}\ x\neq 0,\\
                  &1\ \text{pro}\ x=0.
                  \end{cases} 
$$                                  
Máme zde $\lim_{x\to 0}f(x)=0$ a $\lim_{x\to 0}=0$ zatím co $\lim_{x\to 0}g(f(x))$ vůbec neexistuje.

Platí však velmi užitečné 

\medskip

{\bf 6.5.1. Tvrzení.}  {\em Nechť $\lim_{x\to a}f(x)=b$ a $\lim_{x\to b}g(x)=c$.
Nechť

\noindent buď

{\em (1)} $g(b)=c$ (je-li $g(b)$ je definováno, $g$  je spojitá v $b$)

\noindent nebo


{\em (2)} pro dostatečně malé $\delta_0>0$, $0<|x-a|<\delta_0\ \Rightarrow\ f(x)\neq b$.

\noindent Potom $\lim_{x\to a}g(f(x))$ exisuje a je rovna $c$.

Důkaz.} Pro $\epsilon>0$ zvolme $\eta>0$ takové, že
$$
0<|y-b|<\eta \quad\Rightarrow\quad |g(y)-c|<\epsilon
$$
a k tomuto $\eta$ zvolme $\delta>0$ (v druhém případě $\delta\leq\delta_0$) takové, že
$$
0<|x-a|<\delta \quad\Rightarrow\quad |f(x)-b|<\eta.
$$
Potom při $0<|x-\delta|<\epsilon$ máme v případě  (2)\ 
$|g(f(x))-c|<\epsilon$ protože $|f(x)-b|>0$. V případě (1), $|f(x)-b|=0$ nastat může, ale nic nepříjemného se nestane: mám zde $|g(f(x))-c|=0$. \sq

\bigskip

{\bf 6.6. Tvrzení.}  {\em Nechť $\lim_{x\to a}f(x)=b=\lim_{x\to a}g(x)$ and nechť
$f(x)\leq h(x)\leq g(x)$ pro $|x-a|$ menší než nějaké $\delta_0>0$. Potom
 $\lim_{x\to a}h(x)$ existuje a je rovna $b$.
 
 Důkaz.} Je to zřejmé: je-li $|f(x)-b|<\epsilon$ a $|g(x)-b|<\epsilon$ je
 $b-\epsilon<f(x)\leq h(x)\leq g(x)<b+\epsilon$.\sq


\bigskip

{\bf 6.7. Nespojitosti prvního a druhého druhu.} Není-li funkce definovaná
v bodě $a\in D$ v tomto bodě spojitá, mluvíme o nespojitosti prvního druhu existují-li zde jednostranné limity, ale buď nejsou stejné, nebo nejsou rovny hodnotě $f(a)$.

Jinak mluvíme o nespojitosti druhého druhu.

\newpage

 .
 \newpage
 
 \centerline{\Large\bf V. Elementární funkce} 
 
 \vskip10mm
 
 
 V IV.4.4 jsme zmínili některé základní spojité reálné funkce dané jednoduchými formulemi (polynomy, racionální funkce, odmocniny) a všechno co z nich je možné dostat opakovaným užitím skládání, aritmetických operací a inversí.

V této kapitole rozšíříme zásobu funkcí logaritmy, exponenciálami, goniometrickými a cyklometrickými funkcemi. O takto rošířeném systému se obvykle mluví jako o systému 
 {\em elementárnich funkcí}.
 
  Tyto nové funkce budou zavedeny s různou mírou přesnosti. Logaritmus zavedeme axiomaticky a čtenář bude muset prozatím věřit, že funkce s takovými vlastnostmi skutečně existuje. To však bude celkem brzo napraveno až budeme mít základní techniku Riemannova integrálu

Goniometrické funkce budou užívány tak jak je student již zná z dřívějška. K tomu budeme potřebovat navíc jen některá názorná fakta o limitách. P\v ri tom využijeme geometrickou intuici o délce oblouku kružnice (doufejme dostatečně přesvědčivé, ale přísnost úvahy nebude dokonalá). Přesně se k nim budeme moci vrátit až ve třetím semestru.


 \vskip10mm
 
 {\large\bf 1. Logaritmy}
 
 \bigskip.
 
 {\bf 1.1.} Funkce
 $$
 \lg:(0,+\infty)\to \Rbb
 $$
má následující vlastnosti\footnote{Existence takové funkce bude dokázána v XII.4.} 
 \begin{enumerate}
 \item $\lg$ roste na celém intervalu $(0,+\infty)$
 \item $\lg(xy)=\lg(x)+\lg(y)$
 \item $\lim_{x\to 1}\frac{\lg x}{x-1}=1$.
 \end{enumerate}
 
 \bigskip
 
 {\bf 1.2. Dvě rovnosti.} Máme
 $$
 \lg 1=0 \qtq{a} \lg\frac{x}{y}=\lg x-\lg y.
 $$
 
 ($\lg 1=\lg(1\cdot 1)=\lg 1+\lg 1$. Dále, $\lg\frac{x}{y}+\lg y=\lg(\frac{x}{y}y)=\lg x$.)
 
 \bigskip
 
 {\bf 1.3. Tři limity.} Máme
 $$
 \lim_{x\to 0}\frac{\lg(1+x)}{x}=1, \quad \lim_{x\to 1}\lg x=0,\quad
 \lim_{x\to a}\lg\frac{x}{a}=0.
 $$
 
 (Pro první užijeme IV.6.5.1 a zřejmé $\lim_{x\to 0}(x+1)=1$. Pro druhou, $\lim_{x\to 1}\lg x=\lim_{x\to 1}\frac{\lg x}{x-1}\lim_{x\to 1}(x-1)=1\cdot 0=0$; pro třetí užijeme, IV.5.1 druhou, a zřejmou $\lim_{x\to a}\frac{x}{a}=1$.)
 
 \bigskip
 
 {\bf 1.4. Tvrzení.} {\em Funkce $\lg$ je spojitá a $\lg[(0,+\infty)]=\Rbb$.
 
 Důkaz.} Pro libovolné $a>0$ máme $\lim_{x\to a}\lg x=
 \lim_{x\to a}\lg(a\frac{x}{a})=\lim_{x\to a}(\lg a+\lg\frac{x}{a})=
 \lim_{x\to a}\lg a+\lim_{x\to a}\lg\frac{x}{a}=\lg a +0=\lg a$ takže je $\lg$ spojitá podle IV.6.3.
 Nyní již podle IV.3.2 víme, že  $J=\lg[(0,+\infty)]$ je interval. Podle 1.1(1),
 $K=\lg 2>0$ a podle 1.2 máme $-K=\lg\frac12$. Tedy jsou v  $J$ libovolně velká kladná čísla, totiž $nK=\lg(2^n)$, a
 libovolně velká kladná čísla, totiž $-nK=\lg\frac{1}{2^n}$, tak\v ze podle 
  definice intervalu $x\in J$ pro všechna $x\in\Rbb$. \sq
 
 \bigskip
 
 {\bf 1.5. Logaritmus s obecným základem.} Zatím jen definice.  {\em Logaritmus o základu} $a$, kde  $a>0$ a $a\neq 1$, je
 $$
 \log_a x=\frac{\lg x}{\lg a}.
 $$
 
 \vskip10mm
 
 {\large\bf 2. Exponenciela}
 
 \bigskip.
 
 {\bf 2.1.}  Podle 1.4 (a IV.4.3), má $\lg$ spojitou inversi
 $$
 \exp:\Rbb\to\Rbb\qtq{se všemi hodnotami $\exp(x)$ kladnými.}
 $$
 Z pravidel 1.1 a 1.2 hned dostaneme
 $$
 \begin{aligned}
 &\exp 0=1,\\
 &\exp(x+y)=\exp x\cdot\exp y, \ \text{a}\\
 &\exp(x-y)=\frac{\exp x}{\exp y}.
 \end{aligned}
 $$
 
 \medskip
 
 {\bf 2.1.1.} Z 
 1.1.(3) a IV.5.5.1 získáme důležitou limitu
 $$
 \lim_{x\to 1}\frac{\exp(x)-1}{x}=1.
 $$
 
 \bigskip
 
 {\bf 2.2. Funkce $\exp$ a umocňování. Eulerovo číslo.} Číslo
 $$
 e=\exp(1).
 $$
se $e$ naývá {\em Eulerovo číslo} nebo {\em Eulerova konstanta}.
 
Pro přirozené $n$ dostáváme
 $$
 \exp{n}=\exp(\overbrace{1+1+\cdots+1}^{n})=e^n
 $$
 a podle 2.1,
 $$
 \exp(-n)=\frac{1}{\exp(n)}=e^{-n}.
$$
Dále, vzpomeňme si na standardní racionální exponenty $a^{\frac{p}{q}}$ definované jako $\sqrt[q]{a^p}$. Dostáváme 
$$
\exp(\frac{p}{q})=e^{\frac{p}{q}}
$$
jelikož $\exp(\frac{p}{q})^q=\exp(p)=e^p$ a je to jediné kladné číslo s touto vlastností. Vezmeme-li nyní v úvahu spojitost $\exp$ vidíme, že na tuto funkci je přirozené se dívat jako na
$$
\exp(x)=e^x,
$$
 $x$-tou mocninu čísla $e$.

\medskip
Limita z 2.1.1   
bude užívána ve tvaru
 $$
 \lim_{x\to 1}\frac{e^x-1}{x}=1.
 $$

\bigskip

{\bf 2.3.} Jelikož je $e^{\lg a}=\exp\lg a= a$
můžeme pro $a>0$ definovat
$$
a^x=e^{x\lg a}
$$
a snadno ověříme, že to je přirozené mocnění (exponenciace) jako u $e^x$ 
(souhlasící s klasickým $a^n=\overbrace{a a\cdots a}^{n\ \text{kr\'at}}$ atd.).

\medskip

{\bf 2.3.1.} Teď můžeme lépe osvětlit $\log_ax$ z 1.5: je to inverse k exponenciaci $a^x$ podobně jako $\lg x$ je inverse k $e^x$. Skutečně, máme $a^{\log_ax}=a^{\frac{\lg x}{\lg a}}=
e^{\frac{\lg x}{\lg a}\lg a}=e^{\lg x}=x$  a $\log_a(a^x)=\frac{\lg(a^x)}{\lg a}=\frac{\lg(e^{x\lg a})}{\lg a}=\frac{x\lg a}{\lg a}=x$.

\medskip

{\bf 2.3.2.} Nakonec ještě můžeme exponenciace užít (i když jen pro $x>0$) k definici spojité funkce
$$
x\mapsto x^a= e^{a\lg x}.
$$
Je snadné cvičení ukázat, že je to ve shodě s klasicými $x^n$ and $x^{\frac{p}{q}}$ (omezenými na $x>0$).

\vskip10mm
 
 {\large\bf 3. Goniometrické a cyklometrické funkce}
 
 \bigskip
 
 {\bf 3.1.} Připomeňme si běžné funkce 
 $$
 \sin,\cos:\Rbb\to\Rbb
 $$
  obvykle definované jako podíl protilehlé resp. přilehlé odvěsny k přeponě pravoúhlého trojúhelníka. Argument v těchto funkcích je úhel (k něnuž je strana o kterou jde protilehlá či přilehlá). K měření toho úhlu  (a tedy k získání argumentu $x$) se užívá délka úseku jednotkové kružnice (viz obrázek dole);
 budeme předpokládat, že víme co tato délka je\footnote{Rigorosní definice bude až v  XXIII.1}.
 
 \vskip8mm
  
   \centerline{
\xymatrix@R=7mm{
&&&*=0{}\ar@{-}[ddd]_{\sin x}\ar@/^/@{-}[dddr]_{x}&\\
&&&&\\
&&&&\\
*=0{}\ar@{-}[uuurrr]_1\ar@{-}[rrr]^{\cos x}&&&*=0{}\ar@{-}[r]&*=0{}
}}

\vskip12mm
  
\noindent Obě funkce definujeme na celé $\Rbb$ jako periodické s periodou $2\pi$,  viz dále (``argumentová délka se
obtáčí kolem jednotkové kružnice''). 

\medskip

{\bf 3.1.1.} Shrňme základní známé vlastnosti:
  $$
  \begin{aligned}
  &\sin^2x+\cos^2x=1,\\
  &|\sin x|,|\cos x|\leq 1.\\
  &\sin(x+2\pi)=\sin x, \quad \cos(x+2\pi)=\cos x,\\
  &\sin(x+\pi)=-\sin x, \quad \cos(x+\pi)=-\cos x,\\
  &\cos x=\sin (\frac{\pi}{2}-x),\quad \sin x=\cos (\frac{\pi}{2}-x).\\
  &\sin(-x)=-\sin x,\quad \cos(-x)=\cos x.
  \end{aligned}
  $$
  
  \medskip
  
  {\bf 3.1.2.} Dále si připomeňme důležité formule 
$$
\begin{aligned}
&\sin(x+y)=\sin x\cos y+\cos x\sin y,\\
&\cos(x+y)=\cos x\cos y-\sin x\sin y.
\end{aligned}
$$

\bigskip

{\bf 3.1.3.} Z formulí 3.1.2 snadno odvodíme běžně užívané rovnosti
$$
\begin{aligned}
&\sin x\cos y=\frac12(\sin(x+y)-\sin(x-y)),\\
&\sin x\sin y=\frac12(\cos(x-y)-\cos(x+y)),\\
&\cos x\cos y=\frac12(\cos(x-y)+\cos(x+y)).
\end{aligned}
$$

\bigskip

{\bf 3.2. Čtyři důležité limity.} 

\hskip20mm 1. $\lim_{x\to 0}\sin x=0$,

 \hskip20mm 2.  $\lim_{x\to 0}\cos x=1$,
 
 \hskip20mm 3.  $\lim_{x\to 0}\frac{\sin x}{x}=1$, 
 
\hskip20mm 4. $\lim_{x\to 0}\frac{\cos x-1}{x}=1$.

\medskip

{\em Vysvětlení.} Mluvím raději o  ``Vysvětlení'' než o ``Důkazu''. Dedukce bude založena na intuitivním chápání délky úseku kružnice $x$ jednotkové kružnice (nemělo by to dělat potíž, krátké úseky, o které jde jsou ``skoro rovné'').

Vezměme následující obrázek.


\medskip
 \centerline{
\xymatrix@R=7mm{
&&&&{D}\ar@{-}[dddd]^{\tan x=\frac{\sin x}{\cos x}}\\
&&&{B}\ar@{-}[ur]\ar@{-}[ddd]_{\sin x}\ar@/^/@{-}[dddr]_{x}&\\
&&&&\\
&&&&\\
{A}\ar@{-}[uuurrr]_1\ar@{-}[rrr]^{\cos x}&&&{C}\ar@{-}[r]&{E}}}

\vskip10mm

1. Jelikož $|\sin(-x)|=|\sin x|$ stačí vzít kladné  $x$. Zakřivený argument $x$ je delší než $\sin x$ (úsek $BC$) (je dokonce delší než rovná úsečka $BE$), takže pro malá kladná  $x$ máme
$0<\sin x< x$, a jelikož $\lim_{x\to 0}x=0$ tvrzení platí.

\smallskip
 
 2. Podle 1 máme $\lim_{x\to 0}\cos^2x=1-\lim_{x\to 0}\sin^2x=1$ a  jelikož
 $x\mapsto\sqrt x$ je spojitá v bodě 1 dostáváme $\lim_{x\to 0}\cos x=1$.
 
 \smallskip
 
 3. Srovnáním obsahu trojúhelniků $ABC$, $ADE$ a kruhové výseče nad $x$ ($ABE$), dostaneme
 $$
 \frac12\sin x\cos x\leq \frac12 x\leq \frac12\frac{\sin x}{\cos x}
 $$
 a z toho dále
 $$
 \cos x\leq \frac{\sin x}{x}\leq\frac{1}{\cos x}.
 $$
 Užijme 2 a IV.6.6.
 
 \smallskip
 
 4. Jelikož $\sin^2x=1-\cos^2x=(1+\cos x)(1-\cos x)$ máme
 $$
 \frac{1-\cos x}{x}=\frac{1}{1+\cos x}\cdot\sin x\cdot \frac{\sin x}{x}.
 $$
 Užijme 2, 1, a 3. \sq
 
 \bigskip
 
 {\bf 3.3. Tvrzení.} {\em Funkce $\sin$ a $\cos$ jsou spojité.
 
 Důkaz.} Jelikož $\cos x=\sin(\frac{\pi}{2}-x)$ stačí dokázat, že je spojitá funkce $\sin$.  Máme
 $$
 \sin x=\sin(a+(x-a))=\sin a\cdot\cos(x-a)+\cos a\cdot\sin(x-a)
 $$
 a tedy podle 3.2 a IV.6.5.1,
 $$
 \lim_{x\to a}\sin x= \sin a\cdot 1 + \cos a\cdot 0=\sin a.
 $$
 Užijme IV.6.3.  \sq
 
 \bigskip
 
 {\bf 3.4. Tangens a kotangens.}  $\sin x= 0$ právě v bodech $x=k\pi$ kde  $k$ je celé číslo, a $\cos x= 0$ právě 
když $x=k\pi+\frac{\pi}{2}$. Takže můžeme korektně definovat funkci  {\em tangens},
 $$
 \tan:D\to\Rbb\  \ \text{kde}\ D=\bigcup_{-\infty}^{+\infty}((k-\frac12)\pi,(k+\frac12)\pi)
 $$
 formulí
 $$
 \tan x=\frac{\sin x}{\cos x}.
 $$
Platí
 
 \smallskip
 
 {\bf Fakt.} {\em Funkce $\tan$ je spojitá a roste na každém intervalu $((k-\frac12)\pi,(k+\frac12)\pi)$, platí $\tan(x+\pi)=\tan x$, a 
 $\tan[((k-\frac12)\pi,(k+\frac12)\pi)]=\Rbb$.}

 Začneme s periodou $\pi$: funkce $\sin$ a $\cos$ mají periodu $2\pi$, ale zde je $\frac{\sin(x+\pi)}{\cos(x+\pi)}=
 \frac{-\sin x}{-\cos x}=\frac{\sin x}{\cos x}$.
 
Jelikož $\sin$ zřejmě roste a $\cos$ klesá na $\langle 0,\frac{\pi}{2}\rangle$, $\tan$ v tomto intervalu roste, a jelikož $\tan(-x)=
 \frac{\sin(-x)}{\cos(-x)}=\frac{-\sin x}{\cos x}=-\tan x$ usoudíme, že $\tan$ roste na celém $(-\frac{\pi}{2},\frac{\pi}{2})$. Konečně, ze spojitosti vidíme, že existuje $\delta>0$ takové, že pro $\frac{\pi}{2}-\delta<x<\frac{\pi}{2}$ máme
 $\cos x<\frac1{2n}$ and $\sin x>\frac12$ takže $\tan x> n$ a $\tan(-x)<-n$, takže dále
  $\tan[((k-\frac12)\pi,(k+\frac12)\pi)]$ (protože je to interval) musí být celé $\Rbb$.\sq
  
  \medskip
  
Podobně máme funkci {\em kotangens}
   $$
 \cot:D\to\Rbb\  \ \text{where}\ D=\bigcup_{-\infty}^{+\infty}(k\pi,(k+1)\pi)
 $$
 definovanou formulí
 $$
 \cot x=\frac{\cos x}{\sin x}
 $$
 s periodou $\pi$, spojitou a klesající na každém $(k\pi,(k+1)\pi)$, a zobrazující tento interval na $\Rbb$.
 
 \bigskip
 
 {\bf 3.5. Cyklometrické funkce.} Funkce $\sin$ omezená na $\langle-\frac{\pi}{2},\frac{\pi}{2}\rangle$ je ryze monotonní a zobrazuje tento interval na $\langle-1,1\rangle$. Jeho inverse 
 $$
 \arcsin:\langle-1,1\rangle \to\Rbb
 $$
se nazývá {\em arkussinus}. Similarly we have the function {\em arkuscosinus}
 $$
 \arccos:\langle-1,1\rangle \to\Rbb
 $$
 inversní ke $\cos$ omezenému na $\langle 0,\pi\rangle$.

 Zvlášť zajímavou úlohu hraje inverse k $\tan$ omezené na $(-\frac{\pi}{2},\frac{\pi}{2})$, nazývaná {\em arkustangens} a značená
 $$
 \arctan:\Rbb\to\Rbb,
 $$
 definovaná na celém $\Rbb$.
 
 \newpage

 
 \centerline{\Large\bf VI. Derivace} 
 
 \vskip10mm
 
 {\large\bf 1. Definice a charakteristika}
 
 \bigskip
 
 {\bf 1.1. Úmluva.} Když budeme mluvit o derivaci funkce
 $f:D\to\Rbb$ v bodě $x$ budeme přepokládat, že obor hodnot $D$ obsahuje interval $(x-\delta,x+\delta)$ pro nějaké malé $\delta>0$ (tomu se říká, že $x$ je  {\em vnitřní bod} oboru $D$).

Když budeme mluvit o derivaci funkce
 $f:D\to\Rbb$ v bodě $x$ zprava resp. zleva budeme předpokádat, že 
  $D$ obsahuje $\langle x,x+\delta)$ resp. $(x-\delta,x\rangle$.
 
 \bigskip
 
 {\bf 1.2. Derivace.}  {\em Derivace funkce $f:D\to\Rbb$ v bodě  $x_0$} je limita
 $$
 A=\lim_{h\to 0}\frac{f(x_0+h)-f(x)}{h},
 $$
 pokud existuje. Existuje-li, říkáme, že $f$ {\em má derivaci v} $x_0$.
 
 Derivace (limita $A$ nahoře) se obvykle značí
 $$
 f'(x_0).
 $$
Jiná označení jsou např., 
 $$
 \der{f(x_0)}{x}, \quad \der{f}{x}(x_0),\qtq{nebo} \left(\der{}{x}f\right)(x_0).
 $$
 (To druhé a třetí pochází z nahrazení symbolu  $f'$,  bez specifikace $x_0$, symbolem $\der{f}{x}$ or $\der{}{x}f$.)
 


 
 \medskip
  
 {\bf 1.2.1.} Z IV.6.5.1 okamžitě dostaneme tuto formuli pro derivaci
 \begin{equation}
 f'(x_0)=\lim_{x\to x_0}\frac{f(x)-f(x_0)}{x-x_0}.  \tag{$*$}
 \end{equation}
 
 \bigskip
 
 {\bf 1.3. Jednostranné derivace.} {\em Derivace $f$ v $x_0$ zprava} resp. {\em zleva} je jedostraná limita
 $$
 f'_+(x_0)=\lim_{h\to 0+}\frac{f(x_0+h)-f(x)}{h} \qtq{resp.}
 f'_-(x_0)=\lim_{h\to 0-}\frac{f(x_0+h)-f(x)}{h}.
 $$
 Většina pravidel pro jednostrannou derivaci bude stejná jako pro obyčejnou derivaci a nebude potřebovat zvláštní zmínky. Výjimka je pravidlo pro skládání 2.2  -- viz 2.2.2.
 
 \bigskip
 
 {\bf 1.4. Poznámky.} Derivace má (při nejmenším) tři různé motivace a interpretace.
 
 \smallskip
 
 1. {\em Geometrie.} Podívejme se na $f$ jako na rovnici křivky
 $$
 C=\setof{(x,f(x))}{x\in D}
 $$
 v rovině. Potom $f'(x_0)$ je sklon tečny $C$ v bodě
 $(0,f(x_0))$. Přesněji, ta tečna je dána rovnicí
 $$
 y=f(x_0)+f'(x_0)(x-x_0).
 $$
 
 \smallskip
 
 2. {\em Fysika.} Předpokládejme, že $f(x)$ je délka trajektorie prošlé pohybujícím se tělesem za dobu  $x$. Potom
 $$
 \frac{f(y)-f(x)}{y-x}
 $$
 je průměrná rychlost mesi časy $x$ a $y$, a $f'(x_0)$ je okamžitá rychlost v $x_0$.
  
	Ještě důležitější je ve fysice změna rychlosti, {\em zrychlení}. To je vyjádřeno  {\em druhou derivací}, viz dále v Sekci 4.
 
 \smallskip
 

3. {\em Přibližné hodnoty.} S lineárními funkcemi $L(x)=C+Ax$ se snadno počítá. Derivace nám dává approximaci dané funkce v malých okolích daného argumentu lineární funkcí s chybou podstatně menší než je změna argumentu dané funkce v malém okolí. Viz 1.5.1.

\bigskip

{\bf 1.5. Věta.} {\em Funkce $f$ má derivaci $A$ v bodě $x$ právě když pro dostatečně malé $\delta>0$ existuje reálná funkce  $\mu:(-\delta,+\delta)\smin\set{0}\to\Rbb$  taková, že
\begin{enumerate}
\item $\lim_{h\to 0}\mu(h)=0$, a
\item pro $0<|h|<\delta$,
$$
f(x+h)-f(x)=Ah+\mu(h)h.
$$
\end{enumerate}

Důkaz.} Nechť $A=\lim_{h\to 0}\frac{f(x+h)-f(x)}{h}$ existuje. Položme
$$
\mu(h)=\frac{f(x+h)-f(x)}{h}- A.
$$
Potom má $\mu$ zřejmě žádané vlastnosti.

Nechť naopak $\mu$ existuje. Potom pro malé $|h|$,
$$
\frac{f(x+h)-f(x)}{h}=A+\mu(h)
$$
a $f'(x)$ existuje a je rovno  $A$ podle pravidla pro limitu součtu. \sq

\medskip

{\bf 1.5.1.} Zpět k 1.4.3. Máme-li $f(x+h)-f(x)=Ah+\mu(h)h$ jako v (2) potom lineární funkce $L(y)=f(x)+A(y-x)$ approximuje $f(y)$ v malém okolí  bodu $x$ s chybou $\mu(|y-x|)|y-x|$, tedy
$\mu(|y-x|)$-krát menší než   $|y-x|$.

\bigskip

{\bf 1.6. Důsledek.} {\em Má-li $f$ v bodě $x$ derivaci, je tam spojitá.}

(Skutečně, položme $h=y-x$. potom
$$
|f(y)-f(x)|\leq |A(y-x)|+|\mu(y-x)||(y-x)|<(|A|+1)|y-x|
$$
pro malá $|y-x|$.)


\vskip10mm
 
 {\large\bf 2. Základní pravidla derivování}
 
 \bigskip
 
 {\bf 2.1. Aritmetická pravidla.} V následujících pravidlech předpokládáme, že $f,g:D\to\Rbb$ mají derivace v bodě $x$ a tvrzení obsahují implicite  existenci derivací $f+g$, $\alpha f$, $fg$ a $\frac{f}{g}$.
 
 \medskip
 
 {\bf Tvrzení.} {\em \begin{enumerate}
 \item $(f+g)'(x)=f'(x)+g'(x)$,
 \item pro každé reálné $\alpha$, $(\alpha f)'(x)=\alpha f'(x)$,
 \item $(fg)'(x)=f(x)g'(x)+f'(x)g(x)$, a
 \item jestliže $g(x)\neq 0$ potom
 $$
 \left(\frac{f}{g}\right)'(x)=\frac{f'(x)g(x)-f(x)g'(x)}{(g(x))^2}.
 $$
 \end{enumerate}
 
 Důkaz.} Přepíšeme formule tak, že pravidla budo okamžitě plynout z IV.6.4 (and 1.6).
 
 \smallskip
 
 (1) Máme
 $$
 \begin{aligned}
 \frac{(f+g)(x+h)-(f+g)(x)}{h}&=\frac{f(x+h)+g(x+h)-f(x)-g(x)}{h}=\\
 &=\frac{f(x+h)-f(x)}{h}+\frac{g(x+h)-g(x)}{h}.
 \end{aligned}
 $$
 
 \smallskip
 
 (2)
 $$
 \frac{(\alpha f)(x+h)-(\alpha f)(x)}{h}=\frac{\alpha f(x+h)-\alpha f(x)}{h}
=\alpha\frac{f(x+h)-f(x)}{h}.
$$

\smallskip

(3)
$$
\begin{aligned}
&\frac{(fg)(x+h)-(fg)(x)}{h}=\frac{f(x+h)g(x+h)-f(x)g(x)}{h}=\\
&\frac{f(x+h)g(x+h)-f(x+h)g(x) +f(x+h)g(x)-f(x)g(x)}{h}=\\
&=f(x+h)\frac{g(x+h)-g(x)}{h}+g(x)\frac{f(x+h)-f(x)}{h}.
\end{aligned}
$$ 

\smallskip

(4) Vzhledem k (3) stačí pravidlo odvodit pro $\frac{1}{g}$. Máme
$$
\begin{aligned}
&\frac{(\frac{1}{g})(x+h)-(\frac{1}{g})(x)}{h}=
\frac{\frac{1}{g(x+h)}-\frac{1}{g(x)}}{h}=
\frac{\frac{g(x)-g(x+h)}{g(x+h)g(x)}}{h}=\\
&=\frac{g(x)-g(x+h)}{g(x+h)g(x)h}=
\frac{1}{g(x+h)g(x)}\left(-\frac{g(x+h)-g(x)}{h}\right).
\end{aligned}
$$
\sq

\bigskip


{\bf 2.2. Pravidlo pro skládání.} Buďte $f:D\to\Rbb$ a $g:E\to\Rbb$ takové, že $f[D]\sue E$, takže je složení  $g\circ f$  definováno.

\smallskip

{\bf 2.2.1. Věta.} {\em Má-li $f$ derivaci v bodě $x$ a má-li $g$ derivaci v $y=f(x)$, má $g\circ f$
 derivaci v bodě $x$ 
a platí
$$
(g\circ f)'(x)=g'(f(x))f'(x).
$$

Důkaz.} Podle 1.5 existují $\mu$ a $\nu$ s limitami $\lim_{h\to 0}\mu(h)=0$ a 
$\lim_{k\to 0}\nu(k)=0$ takové, že
$$
\begin{aligned}
&f(x+h)-f(x)=Ah+\mu(h)h\ \ \text{and}\\
&g(y+k)-g(y)=Bk+\nu(k)k.
\end{aligned}
$$
Abychom mohli užít IV.6.5.1 dodefinujme $\nu(0)=0$ což limitu $\nu$ v 0 nezmění.

Nyní je
$$
\begin{aligned}
(g\circ &f)(x+h) -(g\circ f)(x)=g(f(x+h))-g(f(x))=\\
&=g(f(x)+(f(x+h)-f(x)))-g(f(x))= g(y+k)-g(y)
\end{aligned}
$$
kde $k=f(x+h)-f(x)$, a tedy
$$
\begin{aligned}
(g\circ f)&(x+h) -(g\circ f)(x)=Bk+\nu(k)k=\\
&=B(f(x+h)-f(x))+\nu(f(x+h)-f(x))(f(x+h)-f(x))=\\
&=B(Ah+\mu(h)h)+\nu(Ah+\mu(h)h)(Ah+\mu(h)h)=\\
&=(BA)h+(A\mu(h)+\nu((A+\mu(h))h)(A+\mu(h)))h.
\end{aligned}
$$
Definujeme-li nyní $\ol\mu(h)=A\mu(h)+(A+\mu(h))\nu((A+\mu(h))h)$ dostaneme
$$
(g\circ f)(x+h) -(g\circ f)(x)=(BA)h+\ol\mu(h)h
$$
a jelikož $\lim_{h\to 0}\ol\mu(h)=0$ (triviálně máme $\lim_{h\to 0}A\mu(h)=0$, a $\lim_{h\to 0}\nu((A+\mu(h))h)=0$ podle IV.6.5.1 -- připomeňte si, že jsme dodefinovali $\nu$ hodnotou $\nu(0)=0$) tvrzení plyne z 1.5. \sq

\medskip

{\bf 2.2.2. Poznámka o jednostraných derivacích.} Na rozdíl od aritmetických pravidel 2.1, a také pravidla pro inversi následujícího v  2.3, při skládání musíme být s jednostrannými derivacemi opatrní. I  když máme $x$ stále napravo nebo nalevo od  $x_0$, $f(x)$ může oscilovat kolem $f(x_0)$.

\bigskip

{\bf 2.3. Pravidlo pro inversi.}

\smallskip 

{\bf  Věta.} {\em Buď $f:D\to\Rbb$ funkce inversní k $g:E\to\Rbb$ a nechť $g$ má nenulovou derivaci v $y_0$. Poto má $f$ derivaci v $x_0=g(y_0)$ a platí
$$
f'(x_0)=\frac{1}{g'(y_0)}=\frac{1}{g'(f(x_0))}.
$$

Důkaz.} Máme $f(x_0)=f(g(y_0))=y_0$. Funkce
$$
F(y)=\frac{y-y_0}{g(y)-g(y_0)}=\frac{y-f(x_0)}{g(y)-x_0}
$$
má tedy nenulovou limitu $\lim_{y\to y_0}f(y)=\frac{1}{g'(y_0)}$. Funkce  $f$ je spojitá (viz IV.4.2) a jelikož má inversi je prostá. Můžeme tedy užít IV.6.5.1 pro $F\circ f$ a dostáváme
$$ 
\lim_{x\to x_0}F(f(x))=\frac{1}{g'(y_0)}.
$$
Nyní, protože 
$$
F(f(x))=\frac{f(x)-f(x_0)}{g(f(x))-x_0}=\frac{f(x)-f(x_0)}{x-x_0},
$$
již dostáváme naše tvrzení. \sq

\medskip

{\bf 2.3.1. Poznámka.} Smysl předchozí věty je v tom, že

\smallskip

\centerline {\em $f'(x_0)$ existuje.}

\smallskip

\noindent Její hodnota již plyne z 2.2: derivace identické funkce $\id(y)=y$ je zřejmě konstantní 1, a jelikož $\id(y)=y= f(g(y))$, máme
$1=f'(g(y))g'(y)$. Ale, samozřejmě, abychom mohli užít 2.2.1 musíme předpokládat existenci derivace $f$.

\bigskip

{\bf 2.4. Shrnutí.} V následující sekci se naučíme jak derivovat $x$, $\ln x$ a $\sin x$. Potom již 2.1, 2.2 and 2.3  
umožní derivovat libovolné elementární funkce. 


\vskip10mm
 
 {\large\bf 3. Derivace elemenárních funkcí.}
 
 \bigskip

Stačilo by presentovat derivace konstant, identity (které už tak jako tak známe, první je konstantní 0 a druhá konstantní 1), sinu a logaritmu. Z nich už je pomocí aritmetických operací, inversí a skládání možno  všechny elementární funkce získat, a pro tyto konstrukce již pravidla derivování máme. Z různých důvodů však některé případy popíšeme explicitněji.
 
 \bigskip
 
 {\bf 3.1. Polynomy.} Máme
 $$
 (x^n)' =nx^{n-1} \ \ \text{pro všechna přirozená}\ n.
 $$
 To je možno spočítat indukcí z 2.1(3), ale přímo je to také snadné.
 
 Pro $n=0$ je formule trivíální. Buď tedy $n>0$. Potom
 $$\begin{aligned}
 \lim_{h\to 0}\frac{(x+h)^n-x^n}{h}&=\lim_{h\to  0}\frac{\sum_{k=0}^n\binom{n}{k}x^{n-k}h^k-x^n}{h}=\\
 &=\lim_{h\to 0}\frac{\binom{n}{1}x^{n-1}h+h^2\sum_{k=2}^n\binom{n}{k}x^{n-k}h^{k-2}}{h}=\\
&= nx^{n-1}+\lim_{h\to 0}h\sum_{k=2}^n\binom{n}{k}x^{n-k}h^{k-2}=nx^{n-1}.
 \end{aligned}
 $$
 Následkem toho
 $$
 (\sum_{k=0}^na_kx^k)'=\sum_{k=1}^nka_kx^{k-1}.
 $$
 
 \medskip
 
 {\bf 3.1.1. Záporné mocniny.} Pro $-n$, $n$ kde je přirozené číslo, máme podle 2.1.4
 $$
 (x^{-n})'=\frac{1}{x^n}= \frac{-nx^{n-1}}{x^{2n}}= -nx^{-n-1}.
 $$
 
 \medskip
 
 {\bf 3.1.2. Odmocniny a racionální mocniny.} Podle 2.3 dostáváme pro $f(x)=\sqrt[q]x$
 (jelikož $g(y)=y^q$)
 $$
 (\sqrt[q]x)'=\frac{1}{q(\sqrt[q]x)^{q-1}}=\frac{1}{q}(\sqrt[q]x)^{1-q}.
 $$
 Tedy, užitím 2.2.1 získáme
 $$
 (x^\frac{p}{q})'=\frac{1}{q}(\sqrt[q]x^p)^{1-q}px^{p-1}=
 \frac{p}{q}x^{(\frac{p(1-q)}{q}+p-1)}=\frac{p}{q}x^\frac{p-q}{q}=
 \frac{p}{q}x^{\frac{p}{q}-1}.
 $$
 
 \bigskip
 
 {\bf 3.2. Logaritmus.} Máme
 $$
 (\lg x)'=\frac1{x}.
 $$
 Skuečně, použitím V.1.2, V.1.3 a IV.6.5.1 získáme
 $$
 \lim_{h\to 0}\frac{\lg(x+h)-\lg x}{h}=\lim_{h\to 0}\frac{\lg\frac{x+h}{h}}{h}=
 \lim_{h\to 0}\frac1{x}\frac{\lg(1+\frac{h}{x})}{\frac{h}{x}}=
 \frac1{x}\lim_{h\to 0}\frac{\lg(1+\frac{h}{x})}{\frac{h}{x}}=\frac1{x}.
 $$
 
 \bigskip
 
 {\bf 3.3. Exponenciely a obecné mocniny.} Podle 3.2 a 2.3 máme
 $$
 (e^x)'=\frac{1}{\lg'(e^x)}=\frac{1}{\frac{1}{e^x}}=e^x.
 $$
 Následkem toho podle 2.2,
 $$
 (a^x)'=(e^{x\lg a})'=\lg a\cdot e^{x\lg a}=\lg a\cdot a^x.
 $$
Pro obecný exponent $a$ (i když jen pro kladná $x$) získáme nepřekvapivé
 $$
 (x^a)'=(e^{a\lg x})'=(e^{a\lg x})a\frac{1}{x}=a x^{a-1}.
 $$
 
 \bigskip
 
 {\bf 3.4. Goniometrické funkce.} Máme
 $$
 (\sin x)'=\cos x\qtq{a} (\cos x)'=-\sin x.
 $$
Skutečně, podle V.3.1.2 a V.3.2
 $$
 \begin{aligned}
 &\lim_{h\to 0}\frac{\sin(x+h)-\sin x}{h}=
 \lim_{h\to 0}\frac{\sin x\cos h+ \sin h\cos x-\sin x}{h}=\\
 &=\lim_{h\to 0}\frac{\sin x(\cos h-1)+ \sin h\cos x}{h}=
 \sin x\cdot\lim_{h\to 0}\frac{\cos h-1}{h}+\cos x\cdot\lim_{h\to 0}\frac{\sin h}{h}=\\
 &=\sin x\cdot 0+\cos x\cdot 1= \cos x, 
 \end{aligned}
 $$
 a podle V.3.1.1 a 2.2
 $$
 (\cos x)'=(\sin (\frac{\pi}{2}-x))'=\cos(\frac{\pi}{2}-x)\cdot(-1)=
 -\sin x.
 $$
 Dále z 3.2.1(4) dostaneme
 $$
 (\tan x)'=\left(\frac{\sin x}{\cos x}\right)'=\frac{\cos x\cos x-\sin x(-\sin x)}{\cos^2x}=
 \frac{\cos^x+\sin^2x}{\cos^2x}=\frac{1}{\cos^2x}.
 $$
 
 \bigskip
 
 {\bf  3.5. Cyklometrické funkce.} Podle 2.3 získáme
 $$
 (\arcsin x)'=\frac{1}{\sin(\arcsin x)}=\frac{1}{\sqrt{1-\sin^2(\arcsin x)}}=
 \frac{1}{\sqrt{1-x^2}}.
 $$
 Následující formule je zvláště zajímavá:
 $$
 (\arctan x)'=\frac{1}{1+x^2}.
 $$
 Pro tu si nejprve uvědomte (podívejte se na pravoúhlý trojúhelník s odvěsnami $1$ a $\tan x$) že
 $$
 \cos^2x=\frac1{1+\tan^2x}
 $$
 a z 2.3 počítejte
 $$
 (\arctan\, x)'=\frac1{\tan'(\arctan x)}=\cos^2(\arctan x)=\frac1{1+\tan^2(\arctan x)}=
 \frac1{1+x^2}.
 $$
 
 
 
 \vskip10mm
 
 {\large\bf 4. Derivace jako funkce. Derivace vyšších řádů.}
 
 \bigskip
 
 {\bf 4.1.} Přísně řečeno jsme dosud mluvili o derivacích jen jako o číselných hodnotách počítaných v tom či onom bodě. Funkce $f:D\to\Rbb$ však má často derivace ve všech bodech oboru $D$, nebo na jeho podstatné části $D'$. Tak vznikne
 $$
 f':D'\to\Rbb
 $$
 a mluvíme potom o této {\em funkci} jako o derivaci funkce $f$. Jak jsme už zmínili v  1.2, tato funkce se často označuje
 $$
 \der{f}{x} \qtq{or} \der{}{x}f.
 $$
 
 \bigskip
 
 {\bf 4.2. Derivace vyšších řádů.} Funkce $f'$ může sama mít derivaci $f''$, které říkáme {\em druhá derivace} funkce $f$, a také můžeme mít dále  {\em třetí derivaci} $f'''$ a tak dále. Mluvíme o {\em derivacích vyšších řádů}. Místo  $n$ čárek píšeme obvykle 
$$
 f^{(n)}
 $$
 a symboly $
 \der{f}{x}$, $\der{}{x}f$ jsou v tomto smyslu zobecněny na
 $$
 \der{^nf}{x^n} \qtq{and} \der{^n}{x^n}f.
 $$
 
 \bigskip
 
 {\bf 4.3. Poznámka.} Čtenář si asi všiml,  (zvláště u
  $\lg$ nebo u $\arctan$) že derivace může být podstatně jednodušší než výchozí funkce. To není tak úplně dobrá zpráva jak to vypadá. Ukazuje to, že když stojíme před úkolem obráceným k derivaci, integrováním, musíme očekávat výsledky podstatně složitější než funkce z nichž vycházíme. A skutečně, prostředky k integrování jsou dost omezené, a integrály elementárních funkcí často ani elementární nejsou.
	
	\newpage 
	.
 

\newpage

\centerline{\Large\bf VII.  Věty o střední hodnotě.} 
 
 \vskip10mm
 
 {\large\bf 1. Lokální extrémy.}
 
 \bigskip
 
 {\bf 1.1. Růst a klesání v bodě.} Funkce $f:D\to \Rbb$ 
 {\em roste} (resp. {\em klesá}) v bodě $x$ existuje-li $\alpha>0$ takové, že
 $$
 \begin{aligned}
 &x-\alpha <y<x\  \Rightarrow\  f(y)<f(x)\qtq{a}
 x<y<x+\alpha\  \Rightarrow\  f(x)<f(y)\\
 &(\text{resp.}\ x-\alpha <y<x\  \Rightarrow\  f(y)>f(x)\qtq{a}
 x<y<x+\alpha\  \Rightarrow\  f(x)>f(y). 
 \end{aligned}
 $$
 
 \medskip
 
 {\bf 1.1.1. Poznámka.}  Roste-li nebo klesá  funkce na intervalu
 potom zřejmě roste nebo klesá v každém bodě toho intervalu. Naopak však
funkce může (dejme tomu) růst v bodě $x$ a přitom nerůst v žádném otevřeném  intervalu $J\ni x$.
 Např. funkce
 $$
 f(x)=\begin{cases} x+\frac12x\sin\frac1{x} \ \ \text{for} \ x\neq 0,\\
                    0 \ \ \text{for} \ x=0.\end{cases}
$$ 
(namalujte si obrázek) roste v  0, ale v žádném otevřeném intervalu tento bod obsahujícím.

Přirozeně se ptáte, zda funkce, která roste v každém bodě intervalu $J$ roste v $J$. To není úplně bezprostřední fakt. Viz ale snadný důkaz v d\'ale v 3.1


\medskip

{\bf 1.1.2. Tvrzení.} {\em Buď $f'(x)>0$ (resp. $<0$). Potom $f$ iroste (resp. klesá) v $x$.

Důkaz.} Připomeňme si VI.1.5 s $A=f'(x)$. Buď $\alpha>0$ takové, že $|\mu(x)|<|A|$ pro $-\alpha < x<\alpha$. Potom je ve výrazu
$$
f(x+h)-f(x)=(A+\mu(h))h
$$
 $A+\mu(h)$ kladné (resp. záporné) právě když takové bylo $A$, a tedy má $f(x+h)-f(x)$ stejné znaménko jako $h$ (resp. opačné).\sq  

\bigskip

{\bf 1.2. Lokální extrémy.} Funkce $f:D\to \Rbb$ má 
 {\em lokální maximum} (resp. {\em lokální minimum}) $M=f(x)$ v bodě $x$ existuje-li $\alpha>0$ takové, že pro body $y$ z $D$
 $$
 \begin{aligned}
 &x-\alpha <y<x\  \Rightarrow\  f(y)\leq f(x)\qtq{a}
 x<y<x+\alpha\  \Rightarrow\  f(x)\geq f(y)\\
 &(\text{resp.}\ x-\alpha <y<x\  \Rightarrow\  f(y)\geq f(x)\qtq{a}
 x<y<x+\alpha\  \Rightarrow\  f(x)\leq f(y). 
 \end{aligned}
 $$
 Společný termín pro lokální maximum či minimum je
 
 \smallskip
 
 \centerline{\em lokální extrém.}
 
 \medskip
 
 {\bf Poznámka.} Zdůrazňujeme, že podmínka je vyžadována jen pro body z $D$  (což obvykle nedůrazňujeme, viz úmluvu v  IV.5.1). Např. funkce $f:\langle 0,1\rangle\to\Rbb$ definovaná předpisem $f(x)=x$ má lokální minimum 0 v $x=0$ a  lokální maximum 1 v $x=1$.
 
 \bigskip
 
 {\bf 1.3.} Srovnáním  1.1 and 1.2 a užitím Tvrzení 1.1.2
 bezprostředně dostáváme
 
 \medskip
 
 {\bf Tvrzení.} {\em Pokud $f$ roste nebo klesá v bodě $x$, zejména má-li tam nenulovou derivac, nemá v něm lokální extrém. }

 
 
 \vskip10mm
 
 {\large\bf 2. Věty o střední hodnotě.}
 
 \bigskip
 
 {\bf 2.1. Věta.} (Rolleova Věta.) {\em Buď $f$ spojitá na kompaktním intervalu $J=\langle a,b\rangle$, $a<b$,   nechť má derivaci v otevřeném intervalu
  $(a,b)$ a nechť $f(a)=f(b)$. Potom existuje $c\in (a,b)$ takové, že
  $f'(c)=0$.
  
  Důkaz.} Podle věty IV.5.2 funkce $f$ nabývá maxima (a tedy lokálního maxima)  v nějakém bodě $x\in J$ a minima (a tedy lokálního minima) v bodě $y\in J$.
  
  \smallskip
  
  I. Pokud $f(x)=f(y)$ je $f$ konstantní na $J$ a má tedy derivaci $0$ všude v $(a,b)$.
 
 \smallskip
 
 II. Pokud $f(x)\neq f(y)$ potom aspoň jeden z $x,y$ není ani $a$ ani $b$. Označme ho $c$ a podle 1.3 máme $f'(c)=0$. \sq
 
 \bigskip
 
 {\bf 2.2. Věta.} (Věta o střední hodnotě, Lagrangeova věta.) {\em Buď $f$ spojitá na kompaktním intervalu $J=\langle a,b\rangle$, $a<b$ a   nechť má derivaci v otevřeném intervalu
  $(a,b)$. Potom existuje bod $c\in (a,b)$ takový, že
  $$
  f'(c)=\frac{f(b)-f(a)}{b-a}.
  $$
	
	 Důkaz.} Definujme funkci $F$ 
  předpisem
  $$
  F(x)=(f(x)-f(a))(b-a)- (f(b)-f(a))(x-a).
  $$
  Potom je $F$ spojitá na $\langle a,b\rangle$, má (podle standardních pravidel z předchozí kapitoly) derivaci a sice
  \begin{equation}
  F'(x)=f'(x)(b-a)-(f(b)-f(a)), \tag{$*$}
  \end{equation}
  a $F(a)=F(b) =0$. Můžeme tedy užít Rolleovu větu 2.1 a $(*)$ dává
 $0=f'(c)(b-a)-(f(b)-f(a))$, a tedy $f'(c)(b-a)=f(b)-f(a)$. Tvrzení získáme vydělíme-li obě strany  $b-a$. \sq
 
 \medskip
 
 {\bf 2.2.1.} Zde je geometrická interpretace: Křivka (``diagram funkce $f$'') $\setof{(x,f(x))}{x\in J}$ má  tečnu rovnoběžnou s úse\v ckou spojující $(a,f(a))$ s $(b,f(b))$. Viz obrázek:
 
   \centerline{
\xymatrix@R=7mm{
&&&&{}\\
&&{f(c)}{}\ar@{-}[urr]\ar@/^/@{-}[rr]&&{f(b)}\\
{}\ar@{-}[urr]&&&&\\
{f(a)}\ar@{-}[uurrrr]\ar@/^/@{-}[uurr]&&&&\\
{a}\ar@{--}[u]\ar@{-}[rr]&&{c}\ar@{--}[uuu]\ar@{-}[rr]&&{b}\ar@{--}[uuu]
}}

\vskip10mm

{\bf 2.2.2. Nenápadné,  ale  výhodné  reformulace.} Nejprve si všimněte, že formule 2.2 platí též při $b<a$ (potom samozřejmě mluvíme o  $c$ i v $(b,a)$. Má-li derivace smysl mezi $x$ a $x+h$ můžeme řící, že
$$
f(x+h)-f(x)=f'(x+\theta h)h \ \ \text{with} \ 0<\theta<1
$$
(srovnejte s formulí z V.1.5). To se často píše ve formě
$$
f(y)-f(x)=f'(x+\theta(y-x))(y-x) \ \ \text{s} \ 0<\theta<1.
$$

\bigskip

{\bf 2.3. Věta.} {Zobecněná věta o střední hodnotě, zobecněná Lagrangeova věta.) {\em Nechť jsou $f,g$ spojité na kompaktním intervalu $J=\langle a,b\rangle$, $a<b$, a   nechť mají derivace na otevřeném intervalu
  $(a,b)$. Nechť je $g'$ nenulová na $(a,b)$. Potom existuje $c\in (a,b)$ takové, že
  $$
  \frac{f'(c)}{g'(c)}=\frac{f(b)-f(a)}{g(b)-g(a)}.
  $$
  
  Důkaz} je prakticky stejný jako v 2.2. Definujme funkci $F:\langle a,b\rangle\to\Rbb$ předpisem
  $$
  F(x)=(f(x)-f(a))(g(b)-g(a))- (f(b)-f(a))(g(x)-g(a)).
  $$
  Potom má $F$ derivaci a to
  \begin{equation}
  F'(x)=f'(x)(g(b)-g(a))-(f(b)-f(a))g'(x), \tag{$*$}
  \end{equation}
a $F(a)=F(b) =0$. Takže můžeme opět užít  Rolleovu větu a $(*)$ dává
 $0=f'(c)(g(b)-g(a))-(f(b)-f(a))g'(c)$, to jest $f'(c)(g(b)-g(a))=(f(b)-f(a))g'(c)$.  Nyní podle 2.2, $g(b)-g(a)=g'(\xi)(b-a)\neq 0$ a naší formuli dostaneme vydělením obou stran $(g(b)-g(a))g'(c)$. \sq
 
 \vskip10mm
 
 {\large\bf 3. Tři jednoduché důsledky.}
 
 \bigskip
 
 {\bf 3.1.  Tvrzení.}  {\em Nechť je $f:D\to \Rbb$ spojitá na $\langle a,b\rangle$ a nechť má kladnou (resp. zápornou) derivaci na $(a,b)\smin\set{a_1,\dots,a_n}$ pro nějakou konečnou posloupnost $a<a_1<a_2<\cdots<a_n<b$. Potom $f$ roste (resp. klesá) na $\langle a,b\rangle$.
 
 Důkaz.} Jelikož tvrzení zřejmě platí pokud platí pro restrikce na $\langle a,a_1\rangle$, $\langle a_i,a_{i+1}\rangle$ a $\langle a_{n},b\rangle$, můžeme zapomenout na body $a_i$. Nechť $a\leq x<y\leq b$. Potom máme $c$ takové, že $f(y)-f(x)=f'(c)(y-x)$. Je-li $f'(c)$ kladné, je $f(y)>f(x)$. \sq
 
 \bigskip
 
 {\bf 3.2. Nespojitosti derivací.} Nechť derivace funkce $f:J\to\Rbb$, kde $J$ je otevřený interval, existuje na celém $J$. Funkce $f$ musí být spojitá (viz VI.1.6), ale $f'$ ne nutně. Vezměme $f:\Rbb\to\Rbb$ definovanou předpisem
 $$
 f(x)=\begin{cases}x^2\sin\frac1{x}\ \  \text{pro} \ x\neq 0,\\
                    0\ \ \text{pro} \ x=0.\end{cases}
$$
Pro $x\neq 0$ získáme použitím pravidel z VI.2 and VI3,
$$
f'(x)=2x\sin\frac1{x}+x^2\cdot\cos\frac1{x}\cdot\left(-\frac1{x^2}\right) =
2x\sin\frac1{x}-\cos\frac1{x}
$$
a tedy $\lim_{x\to 0}f'(x)$ neexistuje: hodnota $f'$ v
$\frac{1}{2k\pi+\frac{\pi}{2}}$ je $-1+\frac{2}{2k\pi+\frac{\pi}{2}}$ zatímco v $\frac{1}{2k\pi}$ je to 0.

$f'(0)$ však existuje a je rovna 0, protože 
$\left|\frac{f(h)-f(0)}{h}\right|=\left|h\sin\frac1{h}\right|\leq|h|$.

\medskip
 
 {\bf 3.2.1.} Nespojitost předvedené funkce $f'$ je druhého druhu (připomeňte si IV.6.7.), a nic jiného se nemůže stát: derivace nemůže mít nespojitost prvního druhu. Platí
 
 \smallskip
 
 {\bf Tvrzení.} {\em Nechť $\lim_{y\to x}f'(y)$ (nebo
 $\lim_{y\to x+}f'(y)$, resp. $\lim_{y\to x-}f'(y)$) existuje. Potom $f'(x)$ (nebo $f'_+(x)$,  resp. $f'_-(x)$) existuje a je rovna příslušné limitě.
 
 Důkaz} provedeme pro $f'_+$.  Podle 2.2.2 máme
 $$
 \frac{f(x+h)-f(x)}{h}=f'(x+\theta_h h), \quad 0<\theta_h<1,
 $$
 a $\lim_{h\to 0+}f'(x+\theta_h h)=\lim_{h\to 0+}f'(x+h)
 =\lim_{y\to x+}f'(y)$.\sq
 
 \bigskip
 
 {\bf 3.3. Jednoznačnost primitivní funkce.}  Později se budeme zabývat úlohou  opačnou k derivování (připomeňme VI.4.3), určení t.zv. {\em primitivní funkce} $F$, toti\v z funkce $F$ takové, že $F'=f$. Taková $F$ nemůže být jednoznačně určena (např. máme $(x+1)'=x'=1$), ale situace je zcela průhledná. Máme
 
 \medskip
 
 {\bf Tvrzení.} {\em Buďte na intervalu $J$ dány funkce $F.G:J\to\Rbb$  takové, že
 $F'=G'=f$. Potom je  $F=G+C$ pro nějakou konstantu  $C$.
 
 Důkaz. } Vezměme $H=F-G$. Potom je $H'=\text{const}_0$ a jelikož je  $H$ definována na intervalu máme podle 2.2 pro každé $x,y$,
 $$
 H(x)-H(y)=H'(c)(x-y)=0.\quad\quad \square
 $$
 
 \medskip
 
 {\bf 3.3.1. Poznámka.} Předpoklad, že definiční obor je zde interval je samozřejmě podstatný.
 
 
\newpage
.
\newpage

\centerline{\Large\bf VIII. Několik aplikací derivování.} 

\vskip10mm
 
 {\large\bf 1. První a druhá derivace ve fysice.}
 
 \bigskip
 
 Recall  VI.1.4. Jedna z prvních motivací (a aplikace) přišla z fysiky.
 
 \medskip
 
 {\bf 1.1.} Representujme pohybující se těleso v euklidovslém prostoru $\mathbb E_3$ jeho posicí v čase
$$
 (x(t),y(t),z(t))
 $$
 (souřadnice $x,y,z$ jsou zde reálné funkce které mají být analysovány, a reálný argument, representující čas, je označován $t$). Rychlost pak dostaneme jako vektorovou funkci (t.j., zobrazení $D\to\Rbb^3$ se souřadnicemi reálnými funkcemi)
 \begin{equation}
 \left(\der{x}{t}(t),\der{y}{t}(t),\der{z}{t}(t)\right). \tag{$*$}
 \end{equation}
 
 \bigskip
 
 {\bf 1.2. Zrychlení.} Jeden z nejdůležitějších pojmů Newtonovské fysiky (a fysiky vůbec),   {\em síla}, je spojen se {\em zrychlením}, druhou derivací vektorové funkce $(x,y,z)$,
 $$
 \left(\der{^2x}{t^2}(t),\der{^2y}{t^2}(t),\der{^2z}{t^2}(t)\right).
 $$
Čtenář jistě ví, že síla je dána vzorcem
 $M(\der{^2x}{t^2},\der{^2y}{t^2},\der{^2z}{t^2})$ kde $M$ je hmota.
 
 \bigskip
 
 {\bf 1.3. Tečna křivky.} Stejně jako v 1.1 můžeme získat tečnu ke křivce dané parametricky jako $(f_1,f_2,f_3)$ s reálnými funkcemi $f_i:J\to\Rbb$. Máme pak $(f'_1(x_0),f'_2(x_0),f'_3(x_0))$ vektor určující směr tečny v bodě 
  $(f_1(x_0),f_2(x_0),f_3(x_0))$, a sama tečna je parametricky popsána jako přímka
  $$
 (f_1(x_0),f_2(x_0),f_3(x_0))+ x(f'_1(x_0),f'_2(x_0),f'_3(x_0)),\quad x\in\Rbb.
 $$
 
 \bigskip
 
 {\bf 1.4. Poznámka.} V VI.1.4 jsme se ještě zmínili o dalším aspektu derivování, totiž o aproximacích. O tom více později, zejména v sekci o Taylorově formuli. 
 
 
 \vskip10mm
 
 {\large\bf 2. Určování lokálních extrémů.}
 
 \bigskip
 
 
 {\bf 2.1.  Tvrzení.} {\em Pro funkci $f:D\to \Rbb$ definujme $E(f)$ jako množinu všech $x\in D$ takových, že
 \begin{itemize}
 \item  buď $x$ není vnitřní bod $D$, 
 \item nebo $f'(x)$ neexistuje,
 \item nebo $f'(x)=0$. 
 \end{itemize}
     Potom  $E(f)$ obsahuje všechny body v nichž jsou lokální extrémy.
     
Důkaz.} Ve všech bodech mimo $E(f)$ je nenulová derivace. Užijme VII.1.2. \sq  

\medskip

{\bf 2.2. Poznámky.} 1. Když hledáme lokální extrémy nesmíme zapomínat na body které nejsou vnitřní ani na body kde derivace neexistuje. Určením těch $x$ v nichž$f'(x)=0$ úkol nekončí.

2. Tvrzení 1.3.1 poskytne seznam všech kandidátů na lokální extrém. Tím není řečeno, že by všechny body z $E(f)$ byly lokální extrémy. 
Viz následující příklady.

(a) Definujme $f:\langle 0,\infty)\to \Rbb$ předpisem
$$
 f(x)=\begin{cases} x\sin\frac1{x} \ \ \text{pro} \ x\neq 0,\\
                    0 \ \ \text{pro} \ x=0.\end{cases}
$$  
Bod 0 není vntřní, ale lokální extrém tam není.

(b) Definujme $f:(0,2)\to \Rbb$ předpisem
 $$
 f(x)=\begin{cases}x\ \  \text{pro} \ 0<x\leq 1,\\
                    2x-1 \ \ \text{pro} \ 1\leq x< 2.\end{cases}
$$ 
$f$ nemá derivaci v $x=1$, ale extrém tam není.

(c) $f(x)=x^3$ definovaná na celém $\Rbb$ nemá žádný extrém, ale máme $x=0$ 
$f'(0)=0$.

\vskip10mm
 
 {\large\bf 3. Konvexní a konkávní funkce}
 
 \bigskip
 
Z VII.3.1 víme, že znaménko (první) derivace určuje zda funkce roste nebo klesá. Druhá derivace určuje zda je
 {\em konvexní} (``vypuklá dolů'') nebo {\em konkávní} (``vypuklá nahoru").
 
 \bigskip
 
 {\bf 3.1.} Řekneme, že funkce $f:D\to \Rbb$ je
 {\em konvexní} (resp. {\em ryze konvexní}) na intervalu $J\sue D$ jestliže pro  $a,b,c$ v $J$ taková, že $a<b<c$ máme
 \begin{equation}
 \frac{f(c)-f(b)}{c-b}-\frac{f(b)-f(a)}{b-a}\geq 0\ \ (\text{resp.}\ >0). \tag{$*$}
 \end{equation}
 Řekneme,že je  {\em konkávní} (resp. {\em ryze konkávní}) na $J$ jestliže pro  $a,b,c$ v $J$ taková, že $a<b<c$ máme
 $$
 \frac{f(c)-f(b)}{c-b}-\frac{f(b)-f(a)}{b-a}\leq 0\ \ (\text{resp.}\ <0).
 $$
 
 \bigskip
 
 {\bf 3.2.} Formule pro konvexitu  vyjadřuje fakt, že hodnoty $f(b)$ funkce $f$ v bodech mezi $a,c$ leží pod úsečkou spojující body $(a,f(a))$ a $(c,f(c))$ v rovině $\Rbb^2$. Viz obrázek:
 
  
   \centerline{
\xymatrix@R=3mm{
&&&&{f(c)}\\
f(a)\ar@/_1pc/@{-}[drr]\ar@{-}[urrrr]&&&&\\
&&{f(b)}\ar@/_1pc/@{-}[uurr]&&{}\\
&&&&&\\
{a}\ar@{--}[uuu]\ar@{-}[rr]&&{b}\ar@{--}[uu]\ar@{-}[rr]&&{b}\ar@{--}[uuuu]
}}

\bigskip

\noindent Spojující úsečka je dána formulí
$$
y=f(a)+\frac{f(c)-f(a)}{c-a}(x-a), \quad a\leq x\leq b,
$$
a položíme-li $x=b$ dostaneme $y(b)=f(a)+\frac{f(c)-f(a)}{c-a}(b-a)$
takže, když třeba má hodnota $f(b)$ být pod tou úsečkou , tedy $f(b)<y(b)$, bude
\begin{equation}
 \frac{f(c)-f(a)}{c-a}-\frac{f(b)-f(a)}{b-a} >0. \tag{$**$}
 \end{equation}
Pro $x,y>0$ máme $\frac{X}{x}>\frac{Y}{y}$ právě když
 $\frac{X+Y}{x+y}>\frac{Y}{y}$ (první říka, že $Xy>Yx$, druhé, že $Xy+Yy>Yx+Yy$) takže formule $(**)$ je ekvivalentní s $(*)$ (pro ryzí konvexitu).
 
 \bigskip
 
 {\bf 3.3. Tvrzení.}  {\em Buď $f:D\to \Rbb$ spojitá na $\langle a,b\rangle$ a nechť má druhou derivaci na $(a,b)\smin\set{a_1,\dots,a_n}$ pro nějakou konečnou množinu bodů $a<a_1<a_2<\cdots<a_n<b$. Buď $f''(x)>0$ ($\geq 0,\  \leq 0,\ <0$, resp.) v
 $(a,b)\smin\set{a_1,\dots,a_n}$. Potom je $f$ ryze konvexní (konvexní, konkávní, ryze konkávní, resp.) na $\langle a,b\rangle$.
 
 Důkaz.} Podobně jako v VII.3.1 můžeme zapomenout na ty vyloučené body
 $a_i$ a dokazovat větu pro $f$ spojitou na $\langle a,b\rangle$ s druhou derivací na $(a,b)$. Budeme pracovat třeba s $f''(x)>0$ na tomto otevřeném intervalu.
 
Podle věty o střední hodnotě máme pro $x<y<z$ v $\langle a,b\rangle$
 $$
 V=\frac{f(z)-f(y)}{z-y}-\frac{f(y)-f(x)}{y-x}=f'(v)-f'(u)
 $$
 pro nějaké $x<u<y<v<z$. Použitím téže věty podruhé dostaneme
 $$
 V=f''(w)(v-u)
 $$
 s $u<w<v$, tedy $v-u>0$ a $w\in(a,b)$ takže je $f''(w) >0$
 a $V>0$. \sq
 
 
  \bigskip
 
 {\bf 3.4. Inflexe.}  {\em Inflexní bod} funkce $f:D\to\Rbb$ je
prvek $x\in D$ pro který existuje $\delta>0$ s $(x-\delta,x+\delta)\sue D$  takové, že
 
 -- buď je $f$ konvexní na $(x-\delta,x\rangle$ a konkávní na $\langle x,x+\delta)$,
 
  -- nebo je $f$ konkávní na $(x-\delta,x\rangle$ a konvexní na $\langle x,x+\delta)$.
  
  \medskip
  
 Z 3.3 dostáváme
  
  \smallskip
  
  {\bf 3.4.1. 
 Důsledek.} {\em Buď   $f:J\to \Rbb$ funkce na otevřeném intervalu $J$ se spojitou druhou derivací na $J$. Potom je $f''(x)=0$ ve všech inflexních bodech funkce $f$.}
  
  \medskip
  
  {\bf 3.4.2. Poznámka.} Takže pro funkci s druhou derivací na intervalu máme seznam $\setof{x}{f''(x)=0}$ obsahující všechny inflexní body. Ne všechny $x$ s $f''(x)=0$ ovšem musí být inflexní. Třeba funkce $f(x)=x^{2n}$ jsou konvexní na celém  $\Rbb$ ačkoli $f''(0)=0$.
  
  \vskip10mm
 
 {\large\bf 4. Newtonova metoda}
 
 \bigskip
   
 (Známá též jako Newton-Raphsonova Metoda.) Jde o proceduru která dává posloupnost přibližných řešení rovnice $f(x)=0$. Může být velmi efektivni
 -- viz 4.3.
 
 \bigskip
 
 {\bf 4.1.} Chceme vyřešit rovnici
 \begin{equation}
 f(x)=0 \tag{$*$}
 \end{equation}
 kde $f$ je reálná funkce s první derivací $f'$.  Předpokládejme, že hodnoty funkcí $f$ a  $f'$ se dají celkem snadno počítat. Potom následující procedura často vede k velmi rychlé konvergenci k řešení.
 
 \smallskip
 Pro  $b\in D$ vezměme $(b,f(b))$ bod grafu $\Gamma=\setof{(x,f(x))}{x\in D}$ funkce $f$. Dále vezměme tečnu ke křivce $\Gamma$ v tomto bodě. Ta je grafem lineární funkce
 $$
 L(x)=f(b)+f'(b)(x-b). 
 $$
V rozumně malém okolí bodu $b$ je funkce $L(x)$  dobrá aproximace funkce $f$ a tedy můžeme předpokládat, že
řešení rovnice
 \begin{equation}
 L(x)=0 \tag{$**$}
 \end{equation}
aproximuje řešení rovnice ($*$). Řešení rovnice $(**)$ se spočte snadno: je to
$$
\wt b=b-\frac{f(b)}{f'(b)}.
$$
Nakreslete si obrázek!

%\begin{figure}
%\caption{The Newton method}
%\end{figure}

\bigskip

\noindent Bod $\wt b$ je mnohem blíže k řešení rovnice  $(*)$ než $b$, a opakujeme-li postup, výsledné číslo $\wt{\wt b}$ je opět mnohem blíže.

\bigskip

{\bf 4.2.} To vede k postupu zvaném {\em Newtonova metoda}.
K řešení rovnice $(*)$
\begin{itemize}
\item nejprve zvolíme nějaké přiblžné $a_0$ (ne nutně dobrou aproximaci, prostě něco pro začátek), a
\item potom definujeme
$$
a_{n+1}= \wt{a_n}=a_n-\frac{f(a_n)}{f'(a_n)}.
$$
\end{itemize}
Výsledná posloupnost
$$
a_0,a_1,a_2,\dots
$$
(za vhodných okolností) konverguje k řešení, a to často velmi rychle -- viz 4.3.

\medskip

{\bf 4.2.1. Příklad.} Počítejme druhou odmocninu ze 3, tedy řešení rovnice
$$
x^2-3=0.
$$
Dostáváme
$$
a_{n+1}=a_n-\frac{a_n^2-3}{2a_n}=\frac{a_n^2+3}{2a_n}.
$$
 Začneme-li třeba s $a_0=2$, dostaneme
$$
\begin{aligned}
&a_1=1.75,\\
&a_2=1.732142657,\\
&a_3=1.73205081
\end{aligned}
$$
Tak $a_1$ souhlasí s $\sqrt 3$ (dané v tabulkách jako $1.7320508075$) na dvě desetinná místa. $a_2$ na čtyři, a $a_3$ již na osm desetinných míst!

\bigskip

{\bf 4.3.} Příklad naznačuje, že za vhodných okolností se může chyba zmenšovat velmi rychle. Předvedeme jednoduchý odhad za předpokladu, že existuje druhá derivace.

Označme $a$ přesné řešení, tedy platí $f(a)=0$. Máme
$$
a_{n+1}-a=a_n-a-\frac{f(a_n)}{f'(a_n)}= a_n-a-\frac{f(a_n)-f(a)}{f'(a_n)},
$$
 a podle věty o střední hodnotě existuje $\alpha$ mezi $a_n$ a $a$ takové, že
$$
a_{n+1}-a=(a_n-a)-(a_n-a)\frac{f(\alpha)}{f'(a_n)}= (a_n-a)\left(1-\frac{f(\alpha)}{f'(a_n)}\right),
$$
a dále, užívše tuto větu ještě jednou, tentokrát pro funkci $f'$, získáme $\beta$ mezi $a$ a $\alpha$ takové, že
$$
a_{n+1}-a= (a_n-a)\left(\frac{f'(a_n)-f'(\alpha)}{f'(a_n)}\right)=
(a_n-a)(a_n-\alpha)\frac{f''(\beta)}{f'(a_n)}
$$
takže, protože $\alpha$ je mezi $a_n$ a $a$ dostaneme pro nějaký horní odhad $K$ čísel $|\frac{f''(\beta)}{f'(a_n)}|$ (který zpravidla není moc velký),
$$
|a_{n+1}-a|\leq |a_n-a|^2K.
$$
Začneme-li tedy s přiblížením na $10^{-1}$ máme v dalšim kroku odhad pod $K\cdot10^{-2}$, tdále $K^2\cdot10^{-4}$, $K^3\cdot10^{-8}$,
$K^4\cdot10^{-16}$, atd., což může být skutečně velmi rychlá konvergence, jak jsme ostatně před chvílí viděli na $\sqrt 3$.

\bigskip

{\bf 4.4. Poznámka.}  Není asi třeba zdůrazňovat, že vhodný výběr počátečního $a_0$ je podstatný. Někdy  ale již první krok automaticky zlepší velmy hrubý odhad na docela dobrý: v případě 4.2.1 jsme začali $a_0=2$, byli jsme ovšem ``na správné straně konvexity''. Kdybychom začali
``na nesprávné straně'', dejme tomu číslem $1$, dostali bychom $a_1=2$ takže první krok by nás teprve dostal na ``správnou stranu'', a byli bychom o krok zpožděni (nakreslete si obrázek).

Ale můžeme též začít velmi špatně. Vezměme
$
f(x)=-\frac74x^4+\frac{15}{4}x^2 -1$. Potom $f(1)=f(-1)=1$, $f'(1)=-f'(-1)=\frac12$ a kdybychom začali s $a_0=1$ dostali bychom
$$
a_1=-1, a_2=1, a_3=-1, a_4=1. \ \text{atd.}
$$

 \vskip10mm
 
 {\large\bf 5. L'H\^{o}pitalovo Pravidlo}
 
 \bigskip
  
 (T\'e\v z L'Hospitalovo Pravidlo; věří se, že bylo objeveno Johannem Bernoullim.)
 
 \bigskip
 
 {\bf 5.1. Jednoduché L'H\^{o}pitalovo Pravidlo.} Později příjde obtížnější; toto je velmi jedoduché.}
 
 \medskip
 
 {\bf Tvrzení.} {\em Buď $\eta>0$. Nechť $f,g$ mají derivace ve všech $x$ takových, že $0<|x-a|<\eta$. Buď $\lim_{x\to a}f(x)=\lim_{x\to a}g(x)=0$. Nechť $\lim_{x\to a}\frac{f'(x)}{g'(x)}$ existuje.
Potom též $\lim_{x\to a}\frac{f(x)}{g(x)}$ existuje a platí
$$
\lim_{x\to a}\frac{f(x)}{g(x)}=\lim_{x\to a}\frac{f'(x)}{g'(x)}.
$$

Důkaz.} Můžeme dodefinovat $f(a)=g(a)=0$ a dostaneme spojité funkce na $\langle a,x\rangle $ resp. $\langle x,a\rangle$ pro $|x-a|$ dostatečně malé. Dále, protože $\lim_{x\to a}\frac{f'(x)}{g'(x)}$ existuje, je-li $|x-a|$ dostatečně malé, máme derivace na $(a,x)$ resp. $(x,a)$, and  a derivace $g'$ je nenulová. Můžeme tedy aplikovat VII.2.3 a dostaneme
$$
\frac{f(x)}{g(x)}=\frac{f(x)-f(a)}{g(x)-g(a)}=\frac{f'(c)}{g'(c)}
$$
pro nějaké $c$ mezi $a$ a $x$. Takže, je-li $0<|x-a|<\delta$ máme též
$0<|c-a|<\delta$ a tedy zvolíme-li $\delta>0$ tak aby pro $0<|c-a|<\delta$ bylo $|\frac{f'(c)}{g'(c)}-L|<\epsilon$, máme pro
$0<|x-a|<\delta$ též
$|\frac{f'(x)}{g'(x)}-L|<\epsilon$.\sq

\medskip

{\bf 5.1.1. Poznámka.} V předchozím důkazu, pokud  $x>a$ bude $c>a$ a pokud $x<a$ bude $c<a$. Takže jsme vlastně dokázali, za odpovídajících podmínek, t\'e\v z že
$$
\lim_{x\to a+}\frac{f(x)}{g(x)}=\lim_{x\to a+}\frac{f'(x)}{g'(x)}
\qtq{a}
\lim_{x\to a-}\frac{f(x)}{g(x)}=\lim_{x\to a-}\frac{f'(x)}{g'(x)}.
$$

\medskip

{\bf 5.1.2. Příklady.} Připomeňme limity z
V.1.3 and V.3.2
$$
\lim_{x\to 0}\frac{\lg(1+x)}{x}=\lim_{x\to 0}\frac{\frac1{1+x}}{1}=1,\quad
\lim_{x\to 0}\frac{\sin x}{x}=\lim_{x\to 0}\frac{\cos x}{1}=1
$$
(samozřejmě bychom neznali ty derivace pokud bychom již dříve neměli ty limity, je to pouze ilustrace).
Nebo můžeme počítat
$$
\lim_{x\to 0}\frac{\cos x-1}{x^2}=\lim_{x\to 0}\frac{-\sin x}{2x}=
\lim_{x\to 0}\frac{-\cos x}{2}=\frac{-1}2.
$$

\bigskip

{\bf 5.2. Nekonečné limity a limity v nekonečnu.} Abychom mohli rozšířit L'H\^{o}pitalovo pravidlo do úplné obecnosti musíme nejprve rozšířit pojem limity funkce. 

\smallskip

Řekneme, že funkce $f:D\to\Rbb$ {\em má limitu $+\infty$ (resp. $-\infty$) v bodě} $a$, a píšeme
$$
\lim_{x\to a}f(x)= +\infty\ \ \text{(resp.} \ -\infty)
$$
jestliže
 $
 \forall K\ \exists \delta>0\ \ \text{takové, že}\ \ (0< |x-a|<\delta)\ \Rightarrow\ f(x)>K \ \ \text{(resp.} \ <K).
$

\smallskip

Funkce $f:D\to\Rbb$ {\em má limitu $b$  v} $+\infty$ (resp. $-\infty$), psáno
$$
\lim_{x\to +\infty}f(x)= b \qtq{(resp.} \lim_{x\to -\infty}f(x)= b)
$$
jestliže
 $
 \forall \epsilon>0\ \exists K\ \ \text{takové, že}\ \ x>K \ \text{(resp.}\ x<K)\  \Rightarrow\ |f(x)-b|<\epsilon.
$

\smallskip 

Funkce $f:D\to\Rbb$ {\em má limitu $+\infty$  v  $+\infty$}, psáno
$$
\lim_{x\to +\infty}f(x)= +\infty
$$
jestliže
 $
 \forall K\ \exists K'\ \ \text{takové, že}\ \  
 x>K'\ \Rightarrow\ f(x)>K
$
(podobně pro limity $+\infty$ v $-\infty$, $-\infty$ v $-\infty$ a $-\infty$ v $+\infty$).

\medskip

{\bf 5.2.1. Poznámka.} Jednostranné varianty předchozích definic jsou zřejmé. Všimněte si, že limity v $+\infty$ a v $-\infty$ jsou jednostranné tak jak jsou.


\bigskip

{\bf 5.3.} Při podrobném pohledu na  důkaz  5.1 vidíme, že toto tvrzení platí též pro nekonečné limity v konečných bodech, a též pro ty jednostranné.

\bigskip

{\bf 5.4. Tvrzení.} {\em Buď $\eta>0$, nechť $f,g$ mají derivace ve všech $x$ takových, že $0<|x-a|<\eta$ a nechť $\lim_{x\to a}|g(x)|=+\infty$. Nechť $\lim_{x\to a}\frac{f'(x)}{g'(x)}$ existuje (ať už konečná či nekonečná).
Potom též $\lim_{x\to a}\frac{f(x)}{g(x)}$ existuje a máme
$$
\lim_{x\to a}\frac{f(x)}{g(x)}=\lim_{x\to a}\frac{f'(x)}{g'(x)}.
$$

Důkaz.} Tento důkaz není tak průhledný jako důkaz 5.1, i když princip je podobný. Samozřejmě nemůžeme užít obratu s doplněním hodnot v $a$ nulami. 

\smallskip
Můžeme psát
$$
\frac{f(x)}{g(x)}=\left(\frac{f(x)-f(y)}{g(x)-g(y)}+\frac{f(y)}{g(x)-g(y)}\right)\frac{g(x)-g(y)}{g(x)}.
$$
Tak máme pro vhodné $\xi$ mezi $x$ a $y$
\begin{equation}
\frac{f(x)}{g(x)}=\left(\frac{f'(\xi)}{g'(\xi)}+\frac{f(y)}{g(x)-g(y)}\right)\frac{g(x)-g(y)}{g(x)}. \tag{$*$}
\end{equation}
Z technických důvodu rozlišíme tři případy.

\smallskip

I. $\lim_{x\to a}\frac{f'(x)}{g'(x)}=0$: 

Zvolme $\delta_1>0$ tak aby pro $0<|x-a|<\delta_1$ bylo
$|\frac{f'(x)}{g'(x)}|<\epsilon$. Zvolme nyní pevně $y$ tak aby $0<|y-a|<\delta_1$. Dále zvolme $\delta$ s $0<\delta<\delta_1$ takové, že
$$
0<|x-a|<\delta\quad\Rightarrow\quad \left|\frac{f(y)}{g(x)-g(y)}\right|<\epsilon\ \ \text{and}\ \ \left|\frac{g(y)}{g(x)}\right|<1.
$$
Potom podle $(*)$ máme pro $0<|x-a|<\delta$
$$
\left|\frac{f(x)}{g(x)}\right|<(\epsilon +\epsilon)2=4\epsilon
$$ 
a tedy $\lim_{x\to a}\frac{f(x)}{g(x)}=0$

\smallskip

II. $\lim_{x\to a}\frac{f'(x)}{g'(x)}=L$ konečné.

Položme $h(x)=f(x)-Lg(x)$. Potom $h'(x)=f'(x)-Lg'(x)$ a máme
$\frac{h(x)}{g(x)}=\frac{f(x)}{g(x)}-L$ a $\frac{h'(x)}{g'(x)}=
\frac{f'(x)}{g'(x)}-L$. Užijme předchozí případ pro $\frac{h(x)}{g(x)}$.

\smallskip

III. $\lim_{x\to a}\frac{f'(x)}{g'(x)}=+\infty$ ($-\infty$ je zcela analogické):

Pro $K$ zvolíme $\delta_1>0$ tak aby pro $0<|x-a|<\delta_1$ bylo
$\frac{f'(x)}{g'(x)}>2K$. Zvolme pevn\v e $y$ takov\'e \v ze $0<|y-a|<\delta_1$. Zvolme $\delta$ s $0<\delta<\delta_1$ tak aby
$$
0<|x-a|<\delta\quad\Rightarrow\quad \left|\frac{f(y)}{g(x)-g(y)}\right|<K\ \ \text{a}\ \ \left|\frac{g(y)}{g(x)}\right|<\frac12.
$$
Potom máme podle $(*)$ pro $0<|x-a|<\delta$
$$
\frac{f(x)}{g(x)}>(2K-K)(1-\frac12)>\frac12 K
$$ 
a tvrzení je dokázáno. \sq

\bigskip

{\bf 5.5.} V následujícím označuje ``$\square$'' cokoli z $a,a+,a-,+\infty$ nebo $-\infty$. Míti derivaci ``blízko u $\square$`` znamená, že příslušná funkce má derivaci
v $(a-\delta,a+\delta)\smin\set{a}$ pro nějaké $\delta>0$, 
$(a,a+\delta)$ pro nějaké $\delta>0$, $(a-\delta,a)$ pro nějaké $\delta>0$,
$(K,+\infty)$ pro nějaké $K$, nebo $(-\infty,K)$ pro nějaké $K$, v tomto pořadí.

\medskip

{\bf Věta.} (L'H\^{o}pitalovo Pravidlo) {\em Nechť $\lim_{x\to\square}f(x)=
\lim_{x\to\square}g(x)=0$ nebo $\lim_{x\to\square}|g(x)|=+\infty$. Nechť
$f,g$ mají derivace blízko u $\square$  a nechť $\lim_{x\to\square}\frac{f'(x)}{g'(x)}=L$ (konečná nebo nekonečná) existuje. Potom 
$\lim_{x\to\square}\frac{f(x)}{g(x)}$ existuje a je rovna $L$.

Důkaz.} Případy $\square=a,a+$ a  $a-$ jsou obsažené v 5.1, 5.3 a 5.4. Zbývá tedy $+\infty$ a $-\infty$. jsou zcela analogické a budeme tedy diskutovat jen první z nich.

Podle IV.6.5.1 pro upraveného limity v $+\infty$, 
$$
\lim_{x\to+\infty}H(x)= \lim_{x\to 0+}H(\frac{1}{x}).
$$
Položíme-li tedy $F(x)=f(\frac1{x})$ a $G(x)=g(\frac1{x})$ dostaneme
$F'(x)=f(\frac1{x})\cdot\frac{1}{x^2}$ and $G'(x)=g(\frac1{x})\cdot\frac{1}{x^2}$, 
a
$$
\lim_{x\to 0+}\frac{F'(x)}{G'(x)}=
\lim_{x\to 0+}\frac{f'(\frac{1}{x})\cdot\frac{1}{x^2}}{g'(\frac{1}{x})\cdot\frac{1}{x^2}}
=\lim_{x\to 0+}\frac{f'(\frac{1}{x^2})}{g'(\frac{1}{x^2})}=\lim_{x\to +\infty}\frac{f'(x)}{g'(x)}=L.
$$
Takže podle předchozího,
$$
\lim_{x\to+\infty}\frac{f(x)}{g(x)}=\lim_{x\to 0+}\frac{F(x)}{G(x)}=L. \quad\quad\square
$$

\medskip

{\bf 5.5.1. Příklad.}  Buď $a>1$. Podle 5.5,
$$
\lim_{x\to+\infty}\frac{a^x}{x^n}=\lim\frac{\lg a\cdot a^x}{nx^{n-1}}
=\lim\frac{(\lg a)^2\cdot a^x}{n(n-1)x^{n-2}}=\cdots
=\lim_{x\to+\infty}\frac{(\lg a)^n\cdot a^x}{n!}=+\infty.
$$
Takže, pro libovolné $\epsilon>0$ exponenciální funkce $(1+\epsilon)^x$ roste do nekonečna rychleji než kterýkoli polynom.

Nebo, pro každé $b>0$,
$$
\lim_{x\to+\infty}\frac{x^b}{\lg x}=\lim\frac{bx^{b-1}}{\frac{1}{x}}
=\lim_{x\to+\infty}bx^b=+\infty.
$$
Tedy, pro libovolně malé kladné $b$ funkce $x^b$ (např. kterákoli odmocnina $\sqrt[n]x$) roste do nekonečna rychleji než logaritmus.

\bigskip

{\bf 5.6. Neurčité výrazy.} To je společný termín pro limity funkcí získaných jako jednoduché výrazy z funkcí $f,g$ kde známe $\lim f$ a $\lim g$, ale kde aritmetické výpočty selžou.
Indikují se symbolickými výrazy poukazujícími na to, kde je potíž.  L'H\^{o}pitalovo pravidlo zde často pomůže.

\medskip

{\bf 5.6.1. Typy $\frac{0}{0}$  a $\frac{\infty}{\infty}$.} Tady nám často pomůže věta 5.5: úloha v $f,g$ může být neurčitá, odpovídající výraz v $f',g'$ však ne.

\smallskip

{\bf Poznámka.} Není třeba připomínat, že zjištění derivace je úloha typu $\frac00$.

\medskip



{\bf 5.6.2. Typ $0\cdot\infty$.} To může být převedeno na typ
$\frac{0}{0}$ nebo $\frac{\infty}{\infty}$ přepisem $f(x)g(x)$ jako 
$$
\frac{f(x)}{\frac{1}{g(x)}} \qtq{nebo} \frac{g(x)}{\frac{1}{f(x)}},
$$
podle toho co je výhodnější.

\medskip

{\bf 5.6.3. Typ $\infty-\infty$.} To je trochu těžší. Často pomůže následující přepis:
$$
f(x)-g(x)=\frac{1}{\frac{1}{f(x)}}-\frac{1}{\frac{1}{g(x)}}=
\frac{\frac{1}{g(x)}-\frac{1}{f(x)}}{\frac{1}{f(x)g(x)}}.
$$

\medskip

{\bf 5.6.4. Typy $0^0$, $1^\infty$ a $\infty^0$.} Užijeme faktu, že $f(x)^{g(x)}= e^{g(x)\cdot\lg f(x)}$ a že $e^x$ je spojitá. Stačí tedy spočítat $\lim(g(x)\cdot\lg f(x))$; v prvním případě jsme úlohu převedli na $0\cdot(-\infty)$, v druhém na $\infty\cdot 0$, a v posledním na
$0\cdot(+\infty)$.


 \vskip10mm
 
 {\large\bf 6. Kreslení grafu funkce}
 
 \bigskip
 
  Dejme tomu, že bychom si chtěli udělat představu o chování funkce $f$ dané nějakou formulí. To je obvykle vidět z grafu funkce $f$,
 $$
 \Gamma=\setof{(x,f(x))}{x\in D},
 $$
 umíme-li ho nakreslit.
 
 K tomu jsou nám velkou pomocí fakta která jsme se dosud naučili. 
 
 \bigskip
  
 {\bf 6.1.} Za prvé, formule nám řekne jak je to se spojitostí. L'H\^{o}pitalovo může pomoci s limitami (též jednostranými) v kritických bodech, a s asymptotickým chováním, není-li obor omezený.
 
 \bigskip
 
 {\bf 6.2.} Potom se pokusíme najít body 
 $$
 \cdots<a_i<a_{i+1}<\cdots
 $$
 v nichž $f(a_i)=0$. V intervalech $(a_i,a_{i+1})$ si všimáme, zda je tam funkce kladná nebo záporná.
 
 \bigskip
 
 {\bf 6.3.} Dále, vezmeme první derivaci a body
 $$
 \cdots<b_i<b_{i+1}<\cdots
 $$
v nichž je $f'(b_i)=0$ nebo v nichž derivace neexistuje. V intervalech $(b_i,b_{i+1})$ si všimneme znaménka a zjistíme tak, zda tam funkce roste či klesá. V bodech $b_i$ kde se znaménko mění máme lokální extrémy.
 
 Určíme $f(b_i)$ a pokud $f'(b_i)=0$ vyznačíme si tečnu $(b_i,f(b_i))$ (rovnoběžnou s osou $x$). Ať už zde byl extrém nebo ne, hodí se to jako přímka o níž se křivka $\Gamma$ opírá. Neexistuje-li $f'(b_i)$ ale jsou-li zde (odlišné) jednostrané derivace, nakreslete ``polo-tečny''.

 Mohou též pomoci tečny v $(a_i,0)$ -- čím víc tečen máme, o to snazší bude konečné nakreslení křivky (grafu) 
 
 \bigskip
 
 {\bf 6.4.} Nyní vezměme druhou derivaci a pokusme se najít body
 $$
 \cdots<c_i<c_{i+1}<\cdots
 $$
 v nichž je $f''(c_i)=0$ nebo v nichž druhá derivace neexistuje. V intervalech $(c_i,c_{i+1})$ podle znaménka zjistíme je-li graf konvexní
 (vypuklý dolů) nebo konkávní (vypuklý nahoru). V $(c_i,f(c_i))$ kde $f''(c_i)=0$ nakreslete tečny (obvykle aproximují naší křivku velmi dobře).
 
 \bigskip
 
 {\bf 6.5.} Nyní je již obvykle velmi snadné nakreslit mezi vyznačenými tečnami žádanou křivku (sledujeme při tom konvexitu and konkavitu).
 
 \bigskip
 
 {\bf 6.5. Poznámka.} 1. Třeba nebude snadné zjistit všechny hodnoty o kterých jsme mluvili. Ale už část z nich může pomoci docela dobré p\v redstav\v e. 
 
 2. K tomu abychom řešili rovnice $f(x)=0$, $f'(x)=0$ a
 $f''(x)=0$ můžeme užít Newtonovy metody. Často ale postačí jen rozumný odhad. Pro potřebné limity a asymptotiky užijeme 
 L'H\^{o}pitalovo pravidlo.
 
 \bigskip
 
 {\bf 6.7. Příklad.} 1. Nakreslete graf funkce $f$ z 4.4 a ujasněte si proč se Newtonova metoda se špatně zvoleným $a_0$ nepovedla.
 
 \smallskip
 
 2. Nakreslete graf funkce  $f(x)=\frac{4x}{1+x^2}$ (obor hodnot je celé $\Rbb$).
 
 \smallskip
 
 2. Nakreslete graf funkce  $f(x)= e^{\frac{1}{x}}$ (obor hodnot je celé  $\Rbb\smin\set{0})$.
 
 
 \vskip10mm
 
 
 {\large\bf 7. Taylorův polynom a zbytek}
 
 \bigskip
 
 {\bf 7.1.} Podle VI.1.5, funkci s derivací v bodě $a$ je možno aproximovat lineární funkcí (polynomem prvního stupně)
 $$
 p(x)=f(a)+f'(a)(x-a).
 $$
 Tento polynom $p$ je charakterisován tím, že se shoduje s $f$ v  $p^{(0)}(a)=f^{(0)}(a)=f(a)$ a $p^{(1)}(a)=f^{(1)}(a)$.
 
 Přirozeně předpokládáme, že polynom $p$ stupně $n$
takový, že
 \begin{equation}
 p^{(0)}(a)=f^{(0)}(a),\  p^{(1)}(a)=f^{(1)}(a),\ \dots\ ,\  p^{(n)}(a)=f^{(n)}(a) \tag{$*$}
 \end{equation}
 (o $f$ samé uvažujeme jako o její $0$-té derivaci) dostaneme při rostoucím $n$ stále lepší shodu, to jest, že zbytek $R(x)$ v
 $$
 f(x)=p(x)+R(x)
 $$
bude \v cím dále tím menší. Tomu je (s výjimkami) skutečně tak, jak brzy uvidíme.
 
 \bigskip
 
 {\bf 7.2. Taylorův polynom.} Nejprve ukážeme, že podmínky $(*)$ jednoznačně určují polynom $p$ stupně $n$. Pokud \hbox{$p(x)=\sum_{k=0}^nb_k(x-a)^k$}
 máme
 $$
 p'(x)=\sum_{k=1}^nkb_k(x-a)^{k-1},\ p''(x)=\sum_{k=2}^nk(k-1)b_k(x-a)^{k-2}, \dots,
 p^{(n)}(x)=n!b_n,
 $$
 to jest,
 $$
 \begin{aligned}
 &p^{(1)}(x)=1\cdot b_1+(x-a)\sum_{k=2}^nkb_k(x-a)^{k-2},\\
 &p^{(2)}(x)=1\cdot 2\cdot b_2+(x-a)\sum_{k=3}^nk(k-1)b_k(x-a)^{k-3},\\
 &p^{(3)}(x)=1\cdot 2\cdot 3\cdot b_3+(x-a)x\sum_{k=4}^nk(k-1)(k-2)b_k(x-a)^{k-4},\\
 &\dots,\\
 &p^{(n)}(x)=n!\cdot b_n
 \end{aligned}
 $$
 takže když $p^{(k)}(a)= f^{(k)}(a)$  pro $k=0,\dots,n$ máme
$$
 b_k=\frac{1}{k!}p^{(k)}(a)=\frac{f^{(k)}(a)}{k!}, \quad k=0,\dots,n.
 $$
 Výsledný polynom
 $$
 \sum_{k=0}^n\frac{f^{(k)}(a)}{k!}(x-a)^k
 $$
 se nazývá {\em Taylorův polynom stupně $n$} funkce $f$ (v $a$).
 
 \bigskip
 
 {\bf 7.3. Věta.} {\em Nechť má funkce $f$ derivace $f^{(k)}$, $k=0,\dots,n+1$ na intervalu $J=(a-\Delta,a+\Delta)$. Potom máme pro všechna $x\in J$
$$
 f(x)=\sum_{k=0}^n\frac{f^{(k)}(a)}{k!}(x-a)^k+\frac{f^{(n+1)}(\xi)}{(n+1)!}(x-a)^{n+1}
 $$
kde $\xi$ je reálné číslo mezi $x$ and $a$.
 
 Důkaz.} Definujme funkci reálné proměnné $t$ ($x$ je přitom konstanta)
 $$
 R(t)=f(x)-\sum_{k=0}^n\frac{f^{(k)}(t)}{k!}(x-t)^k.
 $$
Tedy, $R(x)=0$  a $R(a)=f(x)-\sum_{k=0}^n\frac{f^{(k)}(a)}{k!}(x-a)^k$
je zbytek, chyba při nahrazení funkce  $f$ jejím Taylorovým polynomem.
 
 Pro derivaci $R$ získáme, užitím pravidel pro derivování součtů a součinů (a také pravidla pro skládání, berouce v úvahu, že $\der{}{t}(x-t)=-1$),
 $$
 \der{R(t)}{t}=-\sum_{k=0}^n\frac{f^{(k+1)}(t)}{k!}(x-t)^k+
 \sum_{k=1}^n\frac{f^{(k)}(t)}{(k-1)!}(x-t)^{k-1}.
  $$
  Nahrazením $k$ v prvním sčítanci  a $k-1$ v druhém
  symbolem $r$ získáme
   $$
 \der{R(t)}{t}=-\sum_{r=0}^n\frac{f^{(r+1)}(t)}{r!}(x-t)^r+
 \sum_{r=0}^{n-1}\frac{f^{(r+1)}(t)}{r!}(x-t)^{r}=-\frac{f^{(n+1)}(t)}{n!}(x-t)^n.
$$
  Nyní zvolme libovolné $g$ takové, že $g'$ je nenulová mezi $a$ a $x$. Jelikož je $R(x)=0$, dostaneme z VII.2.3 
  $$
  \frac{R(a)}{g(a)-g(x)}=-\frac{f^{(n+1)}(\xi)}{n!g'(\xi)}(x-\xi)^n
  $$
 pro nějaké  $\xi$ mezi $a$ a $x$.
  
  Položíme-li nyní $g(t)= (x-t)^{n+1}$ máme $g'(t)=-(n+1)(x-t)^n$ a $g(x)=0$
  takže
  $$
  R(a)=-(x-t)^{n+1}\frac{f^{(n+1)}(\xi)}{-n!(n+1)(x-\xi)^n}(x-\xi)^n=
  (x-t)^{n+1}\frac{f^{(n+1)}(\xi)}{(n+1)!},
  $$
 což je zbytek z tvrzení. \sq
 
 \bigskip
 
 {\bf 7.4. Poznámky.} 1. Volba funkce $g(t)(x-t)^{n+1}$ je skvělý Lagrangeův trik, a o zbytku z naší formula se hovoří jako o {\em zbytku v
  Lagrangeově  tvaru}. Všimněte si, že je velmi snadný k zapamatování: vezmeme jen jeden sčítanec navíc a dáme v něm $f^{(n+1)}(\xi)$ místo $f^{(n+1)}(a)$.
 
 Je možno užít jednodušší $g$, ale výsledek není tak uspokojivý. Položíme-li $g(t)=t$ dostaneme
 $$
 R(a)=\frac{f^{(n+1)}(\xi)}{n!}(x-\xi)^n(x-a),
 $$
tak zvanou {\em Cauchyovu zbytkovou formuli}, ne tak průhlednou.
 
 2. Pro $n=0$ dostáváme 
 $$
 f(x)=f(a)+f'(\xi)(x-a),
 $$ 
 větu o střední hodnotě.
 
3. Zbytek se často rychle zmenšuje (viz příklady dále), někdy však je to pomalejší  (pokusíme-li se např.počítat  logaritmus $\lg$ kolem $a=1$).

Také se může stát,  že celá funkce zůstane ve zbytku. Třeba u  
$$
f(x)=\begin{cases} e^{-\frac{1}{x^2}}\ \ \text{for}\ x\neq 0,\\
                       0\ \ \text{for}\ x=0.\end{cases}
$$                       
Máme derivace všech řádů, ale $f^{(k)}(0)=0$ pro všechna $k$.
 
 \bigskip
 
  
 {\bf 7.5. Příklady.} Např. pro exponencielu dostaneme
 $$
 e^x=1+\frac{x}{1!}+\frac{x^2}{2!}+ \cdots +\frac{x^n}{n!}+
 e^\xi\frac{x^{n+1}}{(n+1)!},
 $$
 nebo pro sinus,
 $$
 \sin x= \frac{x}{1!}-\frac{x^3}{3!}+\frac{x^5}{5!}-\cdots\pm\frac{x^{2n+1}}{(2n+1)!}
 \pm\cos\xi\frac{x^{2n+2}}{(2n+2)!}.
 $$
 V obou případech se zbytek zmenšuje rychle.


\vskip10mm
 
 {\large\bf 8. Osculační kružnice. Křivost.}
 
 \bigskip
 
 {\bf 8.1.} Hodnota $f'(x)$ první derivace určuje jak rychle funkce
 roste nebo klesá v $x$, ať již jsou další data o
  $f$ či $x$ jakákoli.
 
Jelikož druhá derivace $f''$ určuje, zda je funkce $f$ konvexní či konkávní mohlo by se, aspoň na okamžik, zdát, že $f''(x)$ určuje zakřivení, že nám říká jak moc je graf funkce v okolí bodu $x$ zakulacen.
 
 Ale již nejprimitivnější příklady ukazují, že to není tak jednoduché. Vezměme třeba $f(x)=x^2$. Druhá derivace je stále 2, ale ohnutí není stejné: křivka je silně zakulacena  kolem $x=0$ ale při velkých $x$ je skoro rovná.
 
  \bigskip
 
 {\bf 8.2. Oskulační kružnice.} Podobně jako sklon je vidět na tečně (a tedy z první derivace),  z přímky aproximující $f$, můžeme k problému křivosti přistoupit aproximací grafu funkce kružnicí. Měla by to být kružnice
se společnou tečnou v daném bodě, a navíc se stejnou druhou derivací. Taková kružnice se nazývá

 \smallskip
 
 \centerline{\em oskulační kružnice.}
 
 \smallskip
 
 {\bf 8.2.1.} Uvažujme tedy bod $x_0$ a předpokládejme, že
 \begin{itemize}
 \item $f$ má v $x_0$ druhou derivaci, a
 \item $f''(x_0)\neq 0$ (``$f$ je ryze konvexní nebo konkávní v okolí $x_0$'').
 \end{itemize}
 Pro zjednodušení značení pišme
 $$
 y_0=f(x_0),\quad y'_0=f'(x_0)\qtq{a} y''_0=f''(x_0).
 $$
 Rovnice kružnice se středem $(a,b)$ a poloměrem $r$ je
\begin{equation}
 (x-a)^2+(y-b)^2=r^2 \tag{$*$}
 \end{equation}
 a tedy je-li $k$  funkce definovaná v okolí $x_0$ a je-li její graf část kružnice
 $(*)$ máme
 \begin{equation}
 (x-a)^2+(k(x)-b)^2=r^2 \tag{1}
 \end{equation}
 a vezmeme-li první a druhé derivace v rovnici (1) (a v prvním případě ještě vydělíme  2) dostaneme
 \begin{align}
 &(x-a)+(k(x)-b)k'(x)=0  \tag{2}\\
 &1+ (k'(x))^2+ (k(x)-b)k''(x)=0. \tag{3}
 \end{align}
 Jestliže nyní $k$ souhlasí s $f$ tak jak si přejeme, je $k(x_0)=y_0$, $k'(x_0)=y'_0$ a
 $k''(x_0)=y''_0$, a z (1), (2) and (3) dostáváme následující soustavu rovnic.
  \begin{align}
 &(x_0-a)^2+(y_0-b)^2=r^2 \tag{1y}\\ 
 &(x_0-a)+(y_0-b)y'_0=0  \tag{2y}\\
 &1+ (y'_0)^2+ (y_0-b)y''_0=0. \tag{3y}
 \end{align}
 Z (2y) dostáváme
 $$
 (x_0-a)=-(y_0-b)y'_0
 $$
 takže podle (1y),
 $$
 (y_0-b)^2(1+(y'_0)^2)=r^2
 $$
 a jelikož máme podle (3y),  $(y_0-b)=-\frac{1+ (y'_0)^2}{y''_0}$,
 můžeme uzavřít takto:
 
 \medskip
 
 {\bf 8.2.2. Tvrzení.} {\em Poloměr oskulační kružnice funkce $f$ v bodě $x_0$ je
 $$
 r=\frac{(1+(f'(x_0))^2)^\frac32}{|f''(x_0)|}.
 $$}\sq
 
 \medskip
 
 {\bf Pznámka.} Nyní je také  snadn\'e spočítat souřadnice $a,b$ středu. To můžeme ponechat čtenáři jako jednoduché cvičení.
 
 \bigskip
 
 {\bf 8.3. Křivost.} {\em Křivost}  (grafu) funkce $f$ je převrácená hodnota $\frac1{r}$ poloměru $r$ oskulační kružnice. Máme tedy

 \medskip
 
 {\bf 8.3.1. Tvrzení.} {\em Křivost grafu funkce  $f$ v bodě $x$ je
 $$
 r=\frac{|f''(x)|}{(1+(f'(x))^2)^\frac32}.
 $$}\sq
 
 \medskip
 
 {\bf Poznámka.} Teď vidíme, že  domněnka o hodnotě $f''(x)$ určující křivost nebyla konec konců tak špatná. Křivost skutečně lineárně závisí na druhé derivaci, její hodnota jen musí být upravena  pomocí
 $\frac{1}{(1+(f'(x))^2)^\frac32}$.
 
 \newpage
 

 \centerline{\huge\bf Druhý semestr} 
 
 \vskip10mm
 
 \centerline{\Large\bf IX. Polynomy a jejich kořeny} 
 
 \vskip10mm
 
 
 \def\d{\text{d}}
 
 
 
 {\large\bf 1. Polynomy}
 
 \bigskip
 
 {\bf 1.1.} Zabýváme se reálnou analysou, ale budeme potřebovat též některá základní fakta o polynomech s koeficienty a proměnnými v tělese
 $$
 \Cbb
 $$
 komplexních čísel.
  
  \smallskip
  
  Z kapitoly I, 3.4,  si připomeňme absolutní hodnotu $|a|=\sqrt{a_1^2+a_2^2}$  \ komplexního čísla $a=a_1+a_2i$ a trojúhelníkovou nerovnost
  $$
  |a+b|\leq |a|+|b|.
  $$
 Dále pak číslo komlexně sdružené $\ol a=a_1-a_2i$ \v c\'\i slu $a=a_1+a_2i$, a to, že
 $$
 \ol{a+b}=\ol {a}+\ol b, \quad \ol{ab}=\ol a\ol b \qtq{a}|a|=\sqrt{a\ol a}.
 $$
 
 \medskip
 
 {\bf 1.1.1.} Všimněte si též, že
 
 \smallskip
 
 \centerline{\em čísla $a+\ol a$ and $a\ol a$ jsou vždy reálná.}
 
 \bigskip
 
 {\bf 1.2. Stupeň polynomu.} Není-li koeficient $a_n$ v polynomu 
 $$
 p\ \equiv\ a_nx^n+\cdots+a_1x+a_0
 $$
 nula říkáme, že stupeň  $p$ je $n$ a píšeme
 $$
 \deg(p)=n.
 $$
 To nechává stranou $p=\const_0$ pterému se obvykle žádný stupeň nepřiřazuje.
 
 \medskip
 
 {\bf 1.2.1.} Okamžitě vidíme, že
 $$
 \deg(pq)=\deg(p)+\deg (q).
 $$
 
 \bigskip
 
 {\bf 1.3. Dělení polynomů.} Uvažme polynomy $p,q$ se stupni $n=\deg(p)\geq k=\deg(q)$,
 $$
 \begin{aligned}
 &p\ \equiv\ a_nx^n+\cdots+a_1x+a_0,\\
 &q\ \equiv\ b_kx^k+\cdots+b_1x+b_0.
 \end{aligned}
 $$
Odečteme-li $\frac{a_n}{b_k}x^{n-k}q(x)$ od $p(x)$ získáme nulu nebo polynom $p_1$ pro který
 $\deg(p_1)<n$,
a
 $$
 p(x)=c_1x^{n_1}q(x)+ p_1(x).
 $$
 Je-li $\deg(p_1)\geq \deg(q)$ dostaneme obdobně $p_1(x)=c_2x^{n_2}q(x)+ p_2(x)$ a opakováním této proccedury
skončíme u 
 $$
 p(x)=s(x)q(x)+r(x)
 $$
s $r=\const_0$ nebo $\deg(r)<\deg(q)$. O $r$ mluvíme jako o  {\em zbytku} při dělení
 $p$ polynomem $q$.
 
 \medskip
 
 {\bf 1.3.1. Důležité pozorování.} {\em Jsou-li koeficienty v $p$ and $q$ reálné, jsou takové i koeficienty  $s$ a $r$.}
 
  
 \vskip10mm
 
 {\large\bf 2. Základní Věta  Algebry.}
 
 \smallskip
 
  \hskip7mm{\large\bf Kořeny a rozklady.} 
 
 \bigskip
 
 {\bf 2.1.}  {\em Kořen} polynomu $p$ je číslo $x$ takové, že $p(x)=0$. Polynom s reálnými koeficienty nemusí mít reálný kořen  (viz např. $p\equiv x^2+1$) ale v tělese komplexních čísel platí
 \medskip
 
 {\bf Věta.} (Základní Věta Algebry) {\em Každý polynom $p$ se stupně $>0$ s komplexními koeficienty má komplexní kořen.}\footnote{Je to spíš věta analysy nebo geometrie, než algebry. Má řadu důkazů založených na různých principech. Jeden z nich najdete v XXIII.3.}
 
 \medskip
 
 {\bf 2.2. Rozklady komplexních polynomů.} Připomeňte si zřejmou formuli
 $$
  x^k-\alpha^k=(x-a)(x^{k-1}+x^{k-2}\alpha+\cdots+x\alpha^{k-2}x+\alpha^{k-1})
 $$
a označte polynom $x^{k-1}+x^{k-2}\alpha+\cdots+x\alpha^{k-2}x+\alpha^{k-1}$ (v $x$) stupně $k-1$
 symbolem $s_k(x,\alpha)$. Je-li $\alpha_1$ kořen v $p(x)=\sum_{k=0}^na_kx^k$ stupně $n$ máme
 $$
 \begin{aligned}
 p(x)=p(x)-p(\alpha_1)&= \sum_{k=0}^na_kx^k-\sum_{k=0}^na_k\alpha_1^k=\\
 &=\sum_{k=0}^na_k(x^k-\alpha_1^k)=(x-\alpha_1)\sum_{k=0}^na_ks_k(x,\alpha_1)
 \end{aligned}
 $$
 kde polynom $p_1(x)=\sum_{k=0}^na_ks_k(x,\alpha)$ má podle 1.2.1 stupeň přesně $n-1$.
 Opakováním této procedury dostaneme
 $$
 p_1(x)=(x-\alpha_2)p_2(x), \quad p_2(x)=(x-\alpha_3)p_3(x),\qtq{atd.}
 $$ 
 s $\deg (p_k)=n-k$, a konečně
\begin{equation}
 p(x)=a(x-\alpha_1)(x-\alpha_2)\cdots(x-\alpha_n)  \tag{$*$}
 \end{equation}
s $a\neq 0$.
 
 \bigskip
 
 {\bf 2.3. Tvrzení.} {\em Polynom stupně $n$ má nejvýše $n$ kořenů.
 
 Důkaz.} Buď $x$ kořen v $p(x)=a(x-\alpha_1)(x-\alpha_2)\cdots(x-\alpha_n)$. Potom $(x-\alpha_1)(x-\alpha_2)\cdots(x-\alpha_n)=0$ a tedy některé $x-\alpha_k$ musí být nula, a tedy, $x=\alpha_k$.\sq
 
 \medskip
 
 {\bf 2.3.1. Jednoznačnost koeficientů.} Zatím jsme s polynomem jednali jako s výrazem
 $p(x)= a_nx^n+\cdots+a_1x+a_0$. Nyní můžeme dokázat, že tento výraz je určen funkcí $p$. Máme
 
 \medskip
 
 {\bf Tvrzení.} {\em Koeficienty $a_k$ ve výrazu $p(x)= a_nx^n+\cdots+a_1x+a_0$ jsou jednoznačně určeny
 funkcí $(x\mapsto p(x))$. V důsledku toho tato funkce určuje i $\deg (p)$.
 
 Důkaz.} Nechť je $p(x)= a_nx^n+\cdots+a_1x+a_0= b_nx^n+\cdots+b_1x+b_0$ (při tom kterékoli z $a_k,b_k$ může být nula). Potom
 $a_nx^n+\cdots+a_1x+a_0- b_nx^n-\cdots-b_1x-b_0=(a_n-b_n)x^n+\cdots+(a_1-b_1)x+(a_0-b_0)$ má nekonečně kořenů a tedy nemá stupeň. Tedy je $a_k=b_k$ pro každé $k$.\sq
 
 \medskip
 
 {\bf 2.3.2. Tvrzení.} {\em Polynomy $s,r$ získané při dělení polynomu $p$ polynomem $q$ jako v 1.3
 jsou jednoznačně určené.
 
 Důkaz.} Buď $p(x)=s_1(x)q(x)+r_1(x)=s_2(x)q(x)+r_2(x)$. Potom $q(x)(s_1(x)-s_2(x))+(r_1(x)-r_2(x))$ je nulový polynom a jeliko\v z $\deg(q)>\deg(r_1-r_2)$ (je-li poslední vůbec dán)  je $s_1=s_2$. Potom $r_1-r_2\equiv 0$ a tedy také $r_1=r_2$. \sq
 
 \bigskip
 
 {\bf 2.4. Násobné kořeny.} Na druhé straně $p(x)$ nemusí mít $\deg (p)$ různých kořenů: viz na příklad
 $p(x)=x^n$ s jediným kořenem, totiž nulou. Kořeny $\alpha_k$ v rozkladu $(*)$  se mohou několikrát opakovat, a po vhodném přeřazení součnitelů může být $(*)$ přepsán
 \begin{equation}
 p(x)=a(x-\beta_1)^{k_1}(x-\beta_2)^{k_2}\cdots(x-\beta_r)^{k_r} \ \ \text{kde $\beta_k$ jsou různá.} \tag{$**$}
 \end{equation}
 Mocnina $k_j$ se nazývá {\em násobnost} kořenu $\beta_j$ a platí $\sum_{j=1}^rk_j=n$.
 
 \medskip
 
 {\bf 2.4.1. Tvrzení.} {\em Násobnost kořenu je jednoznačně určena. Následkem toho je rozklad $(**)$ určen až na permutaci součinitelů.
 
 Důkaz.} Mějme $p(x)=(x-\beta)^kq(x)=(x-\beta)^\ell r(x)$ takové,že $\beta$ není kořen  $q$ ani $r$.
 Nechť $k<\ell$. Dělíme li $p(x)$ výrazem \hbox{$(x-\beta)^k$} dostaneme (s užitím jednoznačnosti dělení, viz 2.3.2)  
 $q(x)=\hbox{$(x=\beta)^{\ell-k}r(x)$}$ takže $\beta$ je kořen $p$, spor.\sq
 
 \bigskip
 
 {\bf 2.5. Poznámka.} Množina všech komplexních polynomů je obor integrity (podobně jako množina celých čísel. Máme  $q|p$ ($q$ dělí $p$) jestliže $p(x)=s(x)q(x)$ a platí $q|p$ i $q|p$ právě když existuje číslo $c\neq 0$ takové, že $p(x)=c\cdot q(x)$. Prvočinitele jsou zde (třídy ekvivalence) binomů $x-\alpha$. V tvrzeních nahoře jsme se dozvěděli, že v oboru integrity komplexních polynomů je jednoznačný rozklad na prvočinitele.
 
 
 \vskip10mm
 
 {\large\bf 3. Rozklady polynomů  }
 
 
 \smallskip
 
  \hskip7mm{\large\bf s reálnými koeficienty.} 
 
 
 
 \bigskip
 
 {\bf 3.1. Tvrzení.} {\em Nechť jsou koeficienty $a_n$ polynomu 
 $p(x)= a_nx^n+\cdots+a_1x+a_0$ reálné. Buď $\alpha$ kořen $p$. Potom k němu komplexně sdužené číslo $\ol\alpha$
 je též kořen $p$.
 
 Důkaz,} Máme (viz 1.1) $p(\ol \alpha)= a_n\ol\alpha^n+\cdots+a_1\ol\alpha+a_0=
\ol a_n\ol\alpha^n+\cdots+\ol a_1\ol\alpha+\ol a_0=\ol{a_n\alpha^n}+\cdots+\ol{a_1\alpha}+\ol a_0=
\ol{a_n\alpha^n+\cdots+a_1\alpha+a_0}=\ol 0=0$.\sq

\bigskip

{\bf 3.2. Tvrzení.} {\em Nechť je násobnost kořenu $\alpha$ polynomu s reálnými koeficienty $p$ rovna $k$. Potom je násobnost kořenu $\ol\alpha$ také $k$.

Důkaz.} Je-li $\alpha$ reálné není co dokazovat. Nechť nyní $\alpha$ není reálné. Potom máme
$$
p(x)=(x-\alpha)(x-\ol\alpha)q(x)= (x^2- (\alpha+\ol\alpha)x+\alpha\ol\alpha)q(x)
$$
a jelikož $x^2-(\alpha+\ol\alpha)x+\alpha\ol\alpha$ má reálné koeficienty (viz  1.1.1), $q$ má také reálné koeficienty
(viz 1.3.1). Je-li $\alpha$ znovu kořen $q$ máme tu zde také nový $\ol\alpha$  polynomu $q$, a tvrzení plyne z indukce. \sq

\bigskip

{\bf 3.3.} Trinomy $x^2+\beta x+\gamma=x^2-(\alpha+\ol\alpha)x+\alpha\ol\alpha$ nemají reálné kořeny: mají již kořeny $\alpha$ an $\ol\alpha$, a víc jich mít nemohou podle 2.3. Mluvíme o nich jako o {\em ireducibilních} trinomech.

\bigskip
 
 
 {\bf 3.4.} Z 2.4, 3.1 a 3.2 nyní dostáváme
 
 \medskip
 
 {\bf 3.4.1.  Důsledek.} {\em Buď $p$ polynom stupně $n$ s reálnými koeficienty. Potom
 $$
 p(x)=a(x-\beta_1)^{k_1}(x-\beta_2)^{k_2}\cdots(x-\beta_r)^{k_r}(x^2+\gamma_1 x+\delta_1)^{\ell_1}\cdots
 (x^2+\gamma_s x+\delta_s)^{\ell_s}
 $$
s $\beta_j,\gamma_j,\delta_j$ reálnými, $x^2+\gamma_j x+\delta_j$ ireducibilními a $\sum_{j=1}^rk_j+2\sum_{j=1}^s\ell_j=n$
 ($s$ může být 0).}
 
 \medskip
 
 {\bf 3.4.1. Poznámka.} V oboru integrity polynomů s reálnými koeficienty je tedy větší různorodost prvočinitelů. Kromě  $x-\beta$ zde jsou též ireducibilní $x^2+\gamma x+\delta$.
 
 
 \vskip10mm
 
 {\large\bf 4. Součtový rozklad racionálních funkcí.}
 
 
 
 \bigskip
 
 {\bf 4.1.} Už jsme užili termín  {\em obor integrity} v poznámkách 2.5 a 3.4.1. Připomeňme si, že se jedná o  komutativní okruh  J s jednotkou $1$ takový, že $a,b\in J$, $a,b\neq 0$ implikuje $ab\neq 0$
 
 Jako v oboru $\Zbb$  celých čísel, v obecném oboru integrity (speciálně pak v oborech polynomů
 s koeficienty v $\Cbb$ resp. $\Rbb$) říkáme, že $a$ dělí $b$ a píšeme $a|b$ existuje-li $x$ takové, že $b=xa$. $a$ a $b$ jsou ekvivalentní je-li $a|b$ a $b|a$; píšeme pak $a\sim b$. 
 
  {\em Největší společný dělitel} prvků $a,b$ je $d$ takové, že $d|a$ a $d|b$ a takové, že když $x|a$ a $x|b$ máme $x|d$. Jednotka dělí každé $a$; elementy $a$ a $b$  jsou {\em nesoudělné} nemají-li (až na ekvivalenci) nejednotkový společný dělitel.
 
 \bigskip
 
 {\bf 4.2. Věta.} {\em Buď $J$ obor integrity a mějme funkci $\nu:J\to\Nbb$ a pravidlo dělení se zbytkem
  pro $a,b\neq 0$ a $b$  nedělící $a$,
  $$
  a=sb+r \qtq{s} \nu(r)<\nu(b).
  $$
 Potom pro každá $a,b\neq 0$ existují $x,y$ taková,že $xa+yb$ je největší společný dělitel $a,b$.
 
 Důkaz.} Buď $d=xa+yb$ s nejmenším možným $\nu(d)$. Nechť $d$ nedělí $a$. Potom
 $$
 a=sd+r \qtq{s}  \nu(r)<\nu(d).
 $$
 Nyní ale $(1-sx)a- syb=r$ a $\nu((1-sx)a- syb)=\nu(r)<\nu(d)$, spor. Je tedy $d|a$ a z téhož důvodu $d|b$. Jestliže na druhé straně $c|a$ a $c|b$ potom zřejmě $c|(xa+yb)$. $d$ je tedy největší společný dělitel. \sq
 
 \medskip
 
 {\bf 4.2.1. Poznámka.} Pro celá čísla (s $\nu(n)=|n|$) to bylo dokázáno Bachetem (16.-17. století), v obecnější podobě -- speciálně též pro naše polynomy -- to pochází od B\'ezouta (18. století). Obvykle se mluví o B\'ezoutově lemmatu; 
 Bachet-B\'ezoutova věta by bylo správnější.
 
 
 \bigskip
 
 {\bf 4.3.}  {\em Racionální funkce (v jedné proměnné)} je komplexní nebo reálná funkce jedné proměnné kterou můžeme napsat jako
 $$
 P(x)=\frac{p(x)}{q(x)} 
 $$
 kde $p,q$ jsou polynomy.
 
 \medskip
 
 
  {\bf 4.3.1. Věta.} {\em  Komplexní racionální funkci $P(x)=\frac{p(x)}{q(x)}$ můžeme napsat jako
  $$ 
    P_1(x) + \sum_jV_j(x)
   $$
   kde $P_1(x)$ je polynom a ostatní výrazy jsou tvaru
   $$
   \frac{A}{(x-\alpha)^k}
   $$
   kde $A$ je číslo a $\alpha$ je kořen polynomu $q$ s násobností nejméně $k$.
   
   Důkaz} indukcí podle $\deg (q)$. Tvrzení je triviální pro $\deg (q)=0$. Pro $\deg(q)=1$ (a tedy $q(x)=C(x-\alpha)$) dostaneme z 1.3 že
     $$
   p(x)=s(x)q(x)+B
   $$
   a 
   $$
   \frac{p(x)}{q(x)}=s(x)+\frac{B'}{x-\alpha}\qtq{kde} B'=\frac{B}{C}.
   $$
   
   Nechť nyní tvrzení platí pro $\deg (q)<n$. Stačí je nyní dokázat pro $\frac{p(x)}{(x-\alpha)q(x)}$ s $\deg q<n$.
  Podle indukčního předpokladu to můžeme psát jako
    $$
   \frac{P_1(x)}{x-\alpha} + \sum_j\frac{V_j(x)}{x-\alpha}.
   $$
   Jestliže $V_j=\frac{A}{(x-\alpha)^k}$ bude příslušný sčítanec $\frac{A}{(x-\alpha)^{k+1}}$. Je-li to 
   $\frac{A}{(x-\beta)^k}$ s $\beta\neq\alpha$ uvědomíme si nejprve, že největší společný dělitel $(x-\alpha)$ a 
   $(x-\beta)^k$ je 1 a tedy podle 4.2 existují polynomy $u$, $v$ takové, že
   $$
   u(x)(x-\alpha) +v(x)(x-\beta)^k=1
   $$
   takže
   $$
   \frac{A}{(x-\alpha)(x-\beta)^k} = \frac{A(u(x)(x-\alpha) +v(x)(x-\beta)^k)}{(x-\alpha)(x-\beta)^k}=
  \frac{Au(x))}{(x-\beta)^k}+\frac{Av(x)}{(x-\alpha)}
   $$
   a podle indukční hypotézy o posledních sčítancích může být přepsán do žádaného tvaru. \sq
   
   \bigskip
   
   
  {\bf 4.3.2. Věta.} {\em  Reálnou racionání funkci  $P(x)=\frac{p(x)}{q(x)}$ můžeme napsat jako
  $$
    P_1(x) + \sum_jV_j(x)
   $$
   kde $P_1(x)$ je polynom a ostatní výrazy jsou tvaru
     $$
   \frac{A}{(x-\alpha)^k}
   $$
   kde $A$ je číslo a $\alpha$ je kořen polynomu $q$ násobnosti nejméně $k$
   nebo tvaru
   $$
   \frac{Ax+B}{(x^2+ax+b)^k}
   $$
   kde $x^2+ax+b$ je některý z ireducibilních trinomů z 3.4.1 a $k$ je menší nebo rovno příslušnému  $\ell$.
   
   Důkaz} může být proveden jako důkaz v 4.3.1, jen je potřeba rozlišit víc případů nesoudělnosti $x-\alpha$ and $x^2+ax+b$. 
   
   Může to být ale též vyvozeno z 4.3.1: neni-li totiž kořen $\alpha$ reálný, máme s každým 
   $$
    \frac{A}{(x-\alpha)^k}
    $$
   též sčítanec
    $$
    \frac{B}{(x-\ol\alpha)^k}
    $$
   se stejnou mocninou $k$: jinak by součet nevyšel reálný. Nyní máme
    $$    \frac{A}{(x-\alpha)^k}+\frac{B}{(x-\ol\alpha)^k}=\frac{A(x-\ol\alpha)+B(x-\alpha)}{(x^2-(\alpha+\ol\alpha)x+\alpha\ol\alpha)^k}=
    \frac{A_1x+B_1}{(x^2+ax+b)^k}
    $$
   a znova ověříme, že $A_1,B_1$ musí být reálné.
    
    První varianta může být méně pracná ale ve druhé (i když třeba nebudeme ověřovat všechny detaily) lépe vidíme co se děle. \sq
    
    \medskip
    
    {\bf 4.3.3. Poznámka.} Při praktickém výpočtu prostě bereme v ůvahu, že nějaký takový rozklad je možný a koeficienty $A$ resp. $A$ a $B$ dostaneme řešením lineárních rovnic.
    
    \newpage
		
			

    
    
     \centerline{\Large\bf X. Primitivní funkce (neurčitý integrál).} 
 
 \vskip10mm
 
 
 \def\d{\text{d}}
 
 
 
 {\large\bf 1. Obrácení derivace}
 
 \bigskip
 
 {\bf 1.1.} V kapitole VI byla definována derivace
 a naučili jsme se derivovat elementární funkce.
 
 Nyní úlohu obrátíme. Je-li dána funkce $f$ budeme se zajímat o funkci $F$ pro kterou $F'=f$. Taková funkce $F$ se nazývá
 {\em primitivní funkce}, nebo {\em neurčitý integrál} dané $f$ (v další kapitole pak budeme diskutovat nejzákladnější z určitých integrálů, Riemannův integrál).
 
 Při derivaci jsme nejprve mluvili o derivaci funkce v bodě, což bylo číslo, a potom jsme přešli
k derivaci funkce $f$ jako funkce
 $f':D\to\Rbb$, měla-li $f$ derivaci $f'(x)$ v každém bodě $x$ oboru
  $D$. Při určování primitivní funkce se nic takového neděje.
 Vždy půjde o hledání funkce (té zmíněné $F$) k dané funkci.
 
 \bigskip
 
 {\bf 1.2.} Na rozdíl od derivace $f'$ jednoznačně určené funkcí $f$, primitivní funkce jednoznačně určená není, ze zřejmého důvodu: derivace konstanty $C$ je nula takže je-li $F(x)$ primitivní funkce k $f(x)$ je jí též kterákoli
 $F(x)+C$. Ale, jak jsme již dokázali v VIII.3.3, situace není o mnoho horší než toto. Máme
 \medskip
 
 {\bf 1.2.1. Fakt.} {\em Jsou-li $F$ a $G$  primitivní funkce k $f$ na intervalu $J$ potom je pro nějakou konstantu $C$ 
 $$
 F(x)=G(x)+C
 $$
 pro všechna $x\in J$.}
 
 \bigskip
 
 {\bf 1.3. Značení.} Primitivní funkce funkce $f$ se často označuje
 by
 $$
 \int f
 $$
 Místo tohoto stručného symbolu se neméně často píše explicitněji
 $$
 \int f(x)\d x.
 $$
 Toto druhé není jen trochu redundantní indikace toho o jakou proměnou konkretně jde (kdyby třeba šlo o $\int f(x,y)\d x$). V sekci 4 to bude velice výhodné při výpočtu integrálu substituční metodou. Ale ještě víc bude význam této symboliky  patrný ve spojení s určitým integrálem v následující kapitole. Viz XI.2.5, XI.2.6 a XI.5.5.1.
 
 Jelikož není primitivní funkce jednoznačně definována má být výraz ``$F=\int f $ '' chápan jako zkratka pro ``$F$ je primitivní funkce k $f$\ '', ne jako rovnost dvou entit (máme
 $\frac12 x^2=\int x\d x$ a $\frac12 x^2 +5=\int x\d x$ z kterýchžto ``rovností'' nemůžeme vyvozovat, že $\frac12 x^2=\frac12 x^2+5$). 
 Pro jistotu se někdy píše
 $$
 \int f(x)\d x=F(x)+C \qtq{nebo} \int f=F(x)+C, 
 $$
 ale i to má háček: tvrzení 1.2.1 platí jen pro intervaly, a definiční obor i velmi jednoduchých přirozeně definovaných funkcí nemusí být interval; viz 2.2. Musíme být opatrní.
 
  \vskip10mm
 
 {\large\bf 2. Několik jednoduchých formulí.} 
 
  
 \bigskip
 
 {\bf 2.1.} Obrácením základního pravidla pro derivaci dostáváme 
 
 \medskip
 
 {\bf Tvrzení.} {\em Buďte $f,g$ funkce definované na stejném oboru $D$ a buďte $a,b$ čísla. Nechť $\int f$ a $\int g$ na $D$ existují. Potom existuje 
 $\int(af+bg)$  a máme
 $$
 \int(af+bg)=a\int f+ b\int g.
 $$}
 
 \medskip
 
 {\bf 2.1.1. Poznámka.} To je jediné aritmetické pravidlo pro integraci. Ze zásadních důvodů nemůže být obecné pravidlo pro $\int f(x)g(x)\d x$ nebo pro 
  $\int \frac{f(x)}{g(x)}\d x$, viz 2.2.2.1 a 2.3.1.
  
  \bigskip
  
  {\bf 2.2.} Obrácením pravidla pro derivaci $x^n$ s $n\neq-1$ dostaneme
  $$
  \int x^n\d x=\frac{1}{n+1}x^{n+1}.
  $$
  (To ve skutečnosti neplatí jen pro celá čísla $n$. Pro $D=\setof{x\in\Rbb}{x>0}$ máme podle VI.3.3 formuli
  $$
  \int x^a\d x=\frac{1}{a+1}x^{a+1} \ \ \text{pro každé reálné $a\neq -1$}. )
  $$
  
  \smallskip
  Tedy podle 2.1 máme pro polynom $p(x)=\sum_{k=0}^na_kx^k$,
  $$
  \int p(x)\d x= \sum_{k=0}^n\frac{a_k}{k+1}x^{k+1}.
  $$
  
  \medskip
  
  {\bf 2.2.1.} Pro $n=-1$ (a obor hodnot $\Rbb\smin\set{0}$) máme formuli
  $$
  \int\frac{1}{x}\d x= \lg |x|.
  $$
 (Skutečně, pro $x>0$ máme $|x|=x$ a tedy $(\lg |x|)'=\frac{1}{x}$. 
  Pro $x<0$ máme $|x|=-x$ a tedy opět $(\lg |x|)'=(\lg(-x))'=\frac{1}{-x}\cdot(-1)=\frac{1}{x}$.)
  
  \medskip
  
  {\bf 2.2.2. Poznámka.} 1. Tato posledn\'\i\ formule naznačuje, že není možno očekávat jednoduché pravidlo integrování   $\frac{f(x)}{g(x)}$ v termínech $\int f$ a $\int g$: to bychom museli mít formuli vytvořující $\lg x$ z
  $x=\int 1$ a $\frac12 x^2=\int x$.
  
  2. Obor hodnot funkce $\frac{1}{x}$ není interval. Všimněte si, že máme, kromně jiného, třeba
  $$
  \int\frac{1}{x}\d x=\begin{cases}&\lg |x|+2\ \text{for}\ x<0,\\
  &\lg |x|+5\ \text{for}\ x>0.
  \end{cases}
  $$
  což ukazuje , že užívání výrazu $\int f(x)\d x=F(x)+C$ není bez nebezpečí.
  
  \bigskip
  
  {\bf 2.3.} Pro goniometrické funkce dostáváme
  $$
  \int\sin x=-\cos x\qtq{a} \int\cos x=\sin x.
  $$
  
  \medskip
  
  {\bf 2.3.1. Poznámka.} V obecnosti, primitivní funkce k elementární funkci (třebaže vždy existuje, jak uvidíme v další kapitole) nemusí být elementární. Jedna taková je
  $$
  \int \frac{\sin x}{x}
  $$
  (dokázat to je nad naše možnosti, musíte mi to věřit). Máme ovšem snadné
  $\int\frac{1}{x}$ a $\int\sin x$; Nemůže tedy platit pravidlo pro počítání
   $\int f(x)g(x)\d x$ termínech $\int f$ and $\int g$.
  
  \bigskip
  
  {\bf 2.4.} Pro exponencielu máme, triviálně,
  $$
  \int e^x\d x=e^x \qtq{a podle VI.3.3 obecněji} \int a^x\d x= \frac{1}{\lg a}a^x.
  $$
  
  
  \bigskip
  
  {\bf 2.5.} Přidejme ještě dvě zřejmé formule
  $$
  \int\frac{\d x}{1+x^2}=\arctan x \qtq{a} \int\frac{\d x}{\sqrt{1-x^2}}=\arcsin x.
  $$
  
  
  \vskip5mm
  
  \centerline{\bf ------------------}
  
  \vskip5mm
  
  V dalších dvou  sekcích se naučíme dvě užitečné metody pro hledání primitivních funkcí v složitějších případech.
  
  \vskip10mm
 
 {\large\bf 3. Integrace per partes.}
 
  
 \bigskip
 
 {\bf 3.1.} Nechť $f,g$ mají derivace. Z pravidla o derivaci součinu okamžitě získáme pravidlo
 \begin{equation}
 \int f'\cdot g= f\cdot g -\int f\cdot g'. \tag{$*$}
 \end{equation}
Na první pohled to nevypadá jako bychom něco získali: chceme integrovat $f'\cdot g$ a chce se od nas, abychom inegrovali podobnou $f\cdot g'$.
 Ale
 \begin{enumerate}
 \item $\int f\cdot g'$ může být mnohem jednodušší než $\int f'\cdot g$, nebo
 \item z té rovnice můžeme dostat užitečnou rovnici, z níž již integrál vypočteme,
 nebo 
 \item můžeme dostat rekursivní postup vedoucí k cíli.
 \end{enumerate} 
 Užití formule $(*)$ se nazývá {\em integrace per partes}.
 
 \bigskip
 
 {\bf 3.2. Příklad: Ilustrace prípadu 3.1.(1).} Počítejme
 $$
 J=\int x^a \lg x \ \ \text{s $x>0$ a $a\neq -1$}.
 $$
 Položíme-li $f(x)=\frac{1}{a+1}x^{a+1}$ a $g(x)=\lg x$ dostaneme $f'(x)=x^a$ a $g'(x)=\frac{1}{x}$ takže
 $$
 \begin{aligned}
 J&=\frac{1}{a+1}x^{a+1}\lg x-\frac{1}{a+1}\int x^{a+1}\cdot\frac{1}{x}=
 \frac{1}{a+1}(x^{a+1}\lg x-\int x^a)=\\
  &=\frac{1}{a+1}(a^{a+1}\lg x-\frac{1}{a+1} x^{a+1})=
  \frac{x^{a+1}}{a+1}(\lg x-\frac{1}{a+1})
 \end{aligned} 
 $$
 a tedy např. pro $a=1$  zjistíme, že 
$$
 \int\lg x\d x= x(\lg x-1).
 $$
 
 \bigskip
 
 {\bf 3.3. Příklad: Ilustrace prípadu 3.1.(2)} Počítejme
 $$
 J=\int e^x\sin x\d x.
 $$
 Položíme-li $f(x)=f'(x)=e^x$ a $g(x)=\sin x$ získáme
 $$
 J=e^x\sin x-\int e^x\cos x\d x.
 $$
 Integrál na levé straně je asi tak stejně složitý jako ten, se kterým jsme začli. Ale zopakujme to tentokrát s $g(x)=\cos x$. Dostaneme
 $$
 \int e^x\cos x\d x=e^x\cos x-\int e^x(-\sin x)\d x
 $$
 a tedy
 $$
 J=e^x\sin x-(e^x\cos x-\int e^x(-\sin x)\d x)=e^x\sin x-e^x\cos x-J
 $$
 a z toho zjišťujeme
 $$
 J=\frac{e^x}{2}(\sin x-\cos x).
 $$
 
 \bigskip
 
 {\bf 3.4. Příklad: Ilustrace prípadu 3.1.(3).} Počítejme
 $$
 J_n=\int x^ne^x\d x \qtq{pro celá čísla $n\geq 0$.}
 $$
Když položíme $f(x)=x^n$ and $g(x)=g'(x)=e^x$ dostaneme
 $$
 J_n=x^ne^x-\int nx^{n-1}e^x= x^ne^x-nJ_{n-1}.
 $$
 Iterujíce tento postup dostaneme
 $$
 \begin{aligned}
 J_n&=x^ne^x-nx^{n-1}e^x+n(n-1)J_{n-2}=\dots=\\
 &=x^ne^x-nx^{n-1}+n(n-1)x^{n-2}e^x+\cdots\pm n!J_0
 \end{aligned}
 $$
 a jelikož $J_0=\int e^x=e^x$ dostaneme z toho
 $$
 J_n=e^x\cdot\sum_{k=0}^n\frac{n!}{(n-k)!}(-1)^k\cdot x^{n-k}.
 $$
 
 \vskip10mm
 
 {\large\bf 4. Substituční metoda.}
 
  
 \bigskip
 
 {\bf 4.1.} Pravidlo pro derivaci složené funkce VI.2.2 může být pro naše účely reinterpretováno následujícím způsobem.
 
 \medskip
 
 {\bf Fact.} {\em Buď $\int f=F$, nechť má funkce $\phi$ derivaci $\phi'$, a nechť složení $F\circ\phi$ dává smysl. Potom
 $$
 \int f(\phi(x))\cdot\phi'(x)\d x=F(\phi(x)).
 $$}
 
 \medskip
 
 {\bf 4.1.1.} Tedy, abychom získali $\int f(\phi(x))\cdot\phi'(x)\d x$ vypočteme
 $\int f(y)\d y$ a ve výsledku substituujeme $\phi(x)$ ve všech výskytech $y$. Užití tohoto triku se nazývá {\em substituční metoda}.
 
 \smallskip
 
 Zde je značení
 $$
 \int f(x)\d x
 $$
 místo jednoduchého $\int f$ velká pomoc. Připomeňme si značení
 $$
 \frac{\d\phi(x)}{\d x}\ \ \text{pro derivaci}\ \ \phi'(x).
 $$
 Výraz  $\frac{\d\phi(x)}{\d x}$ sice není skutečně zlomek s čitatelem $\d\phi(x)$ a jmenovatelem $\d x$, ale na okamžik předstírejme, že je. Potom máme
 $$
 \d\phi(x)=\phi'(x)\d x \qtq{nebo} ``\ \d y=\phi'(x)\d x \ \text{kde}\ \phi(x)\ \text{je substituováno za $y$ ''}.
 $$
 Tedy, užití substituční metody (subtituce $\phi(x)$ za $y$) spočívá ve výpočtu
 $$
 \int f(y)\d y
 $$
 jako integrálu v proměnné $y$, a potom nahrazení $y$ výrazem $\phi(x)$ tak že píšeme
 $$
 \d y=\phi'(x)\d x\ \ \text{jak je získáno z}\ \ \frac{\d y}{\d x}=\phi'(x).
 $$
 To se snadno pamatuje.
 
 \bigskip
 
 {\bf 4.2. Příklad.} Abychom získali 
 $
 \int\frac{\lg x}{x}\d x
 $
 substituujme $y=\lg x$.Potom $\d y=\frac{\d x}{x}$ a máme tedy
 $$
 \int\frac{\lg x}{x}\d x=\int y\d y=\frac12 y^2=\frac12 (\lg x)^2.
 $$ 
 
 \bigskip
 
 {\bf 4.3. Příklad.} Abychom spočetli $\int \tan x\d x$ připomeňme si, že
 $\tan x=\frac{\sin x}{\cos x}$ a že $(-\cos x)'=\sin x$. Tedy, substitucí $y=-\cos x$ získáme
 $$
 \int\tan x\d x=\int\frac{\sin x}{\cos x}\d x=\int \frac{\d y}{-y}=
 -\lg|y|=-\lg|\cos x|.
 $$
 
 \bigskip
 
 Víc příkladů najdeme v další sekci.
 
  
 \vskip10mm
 
 {\large\bf 5. Integrály racionálních funkcí.}
 
  
 \bigskip
 
 {\bf 5.1.} Vzhledem k 2.1 a IX.4.3.2 stačí najít integrály
 \begin{equation}
 \int\frac{1}{(x-a)^k}\d x \tag{5.1.1}
\end{equation}
 a
\begin{equation}
 \int\frac{Ax+B}{(x^2+ax+b)^k}\d x\quad\text{s}\ \ x^2+ax+b\ \ \text{ireducibilním}  \tag{5.1.2}
 \end{equation}
 pro přirozená čísla $k$.
 
 \bigskip
 
 
 
 {\bf 5.2.} První, (5.1.1), je velmi jednoduché. Substituujeme-li $y=x-a$ bude $\d y=\d x$ a náš integrál spočítáme jako 
 $\int\frac{1}{y^k}$ a podle 2.2 a 2.2.1  (substituujeme zpet $x-a$ za $y$)
 $$
    \int\frac{1}{(x-a)^k}\d x=\begin{cases}&\frac{1}{1-k}\cdot\frac{1}{(x-a)^{k-1}}\ \ \text{pro}\ \ k\neq 1,\\
    &\lg|x-a|\ \ \text{pro}\ \ k=1.
  \end{cases}
  $$
  \bigskip
  
  {\bf 5.3. Lemma.} {\em Položme
  $$
  J(a,b,x,k)=\int\frac{1}{(x^2+ax+b)^k}\d x.
  $$
  Potom máme
 $$
    \int\frac{Ax+B}{(x^2+ax+b)^k}\d x=\begin{cases}&\frac{A}{2(1-k)}\cdot\frac{1}{(x^2+ax+b)^{k-1}}+(B-\frac{Aa}{2})J(a,b,x,k)\ \ \text{pro}\ \ k\neq 1,\\
    &\frac{A}{2}\lg|x^2+ax+b|+(B-\frac{Aa}{2})J(a,b,x,k)\ \ \text{pro}\ \ k=1.
  \end{cases}
  $$
  
  Důkaz.} Máme
  $$
  \frac{Ax+B}{x^2+ax+b}=\frac{A}{2}\frac{2x+a}{x^2+ax+b}+(B-\frac{Aa}{2})\frac{1}{x^2+ax+b}
  $$
  V prvním sčítanci počítáme
  $$
  \int\frac{2x+a}{x^2+ax+b}\d x
		$$
		substitucí $y=x^2+ax+b$; potom máme $\d y =(2x+a)\d x$ a úloha se redukuje, jako v 5.2, na určení $\int\frac{1}{y^k}\d y$.\sq
  
  \bigskip
  
  {\bf 5.4.} Tady bude (5.1.2) vyřešeno vypočtením
  $$
  \int\frac{1}{(x^2+ax+b)^k}\d x
  $$
  s ireducibilním $x^2+ax+b$.
  
  \medskip
  
  {\bf 5.4.1.} Nejprve pozorujeme, že následkem ireducibility je $b-\frac{a^2}{4}>0$ (jinak by
  $x^2+ax+b$ měl reálné kořeny). Existuje tedy reálné $c$ pro které
  $$
  c^2=b-\frac{a^2}{4}
  $$
  a
  $$
  x^2+ax+b=c^2\left(\left(\frac{x+\frac12 a}{c}\right)^2+1\right).
  $$
  Takže, substitucí $y=\frac{x+\frac12 a}{c}$ (tedy, $\d y= \frac{1}{c}\d x$) v $
  \int\frac{1}{(x^2+ax+b)^k}\d x
  $ dostaneme
  $$
  \frac{1}{c^{2k-1}}\int\frac{1}{(y^2+1)^k}\d y
  $$
 a náš úkol jsme zredukovali na $\int\frac{1}{(x^2+1)^k}\d x$.
  
  \medskip
  
  {\bf 5.4.2. Tvrzení.} {\em Integrál
  $$
  J_k=\int\frac{1}{(x^2+1)^k}\d x
  $$
  můžeme rekursivně spočítat formulí
  \begin{equation}
  J_{k+1}=\frac{1}{2k}\cdot \frac{x}{x^2+1}+\frac{2k-1}{2k}J_k \tag{$*$}
  \end{equation}
  kde $J_1=\text{\rm arctg}\, x$.
  
  Důkaz.} Nejprve položme
  $$
  f(x)=\frac{1}{(x^2+1)^k}\qtq{a} g(x)=x.
  $$
  Potom
  $$
  f'(x)=-k\frac{2x}{(x^2+1)^{k+1}}\qtq{a} g'(x)=1
  $$
 a z formule per partes
  $$
  \begin{aligned}
  J_k&=\frac{x}{(x^2+1)^{k}}+2k\int\frac{x^k}{(x^2+1)^{k+1}}=\\
  =\frac{x}{(x^2+1)^{k}}+&2k\left(\int\frac{x^k+1}{(x^2+1)^{k+1}}-\int\frac{1}{(x^2+1)^{k+1}}\right)=\\
  &=\frac{x}{(x^2+1)^{k}}+2kJ_k-2kJ_{k+1}
  \end{aligned}
  $$
  a výraz ($*$) máme; integrál $J_1=\arctan x$ byl již zmíněn v 2.5 \sq
  
   
 \vskip10mm
 
 {\large\bf 6. Několik standardních substitucí.}
 
  
 \bigskip
 
 {\bf 6.1.} Nejprve rozšířime terminologii z kapitoly IX. O výrazu
 $$
 \sum_{r,s\leq n}a_{rs}x^ry^s
 $$
 budeme mluvit jako o {\em polynomu ve dvou proměnných} $x,y$. Jsou-li $p(x,y)$, $q(x,y)$ polynomy v proměnných $x,y$ mluvíme o
 $$
 R(x,y)=\frac{p(x,y)}{q(x,y)}
 $$
jako o {\em racionální funkci ve dvou proměnných}.
 
 \medskip
 
 {\bf 6.1.1. Úmluva.} Ve zbytku této sekce  bude $ R(x,y)$ vždy racionální funkce ve dvou proměnných.
 
 
 \medskip
 
 {\bf 6.1.2. Pozorování.} {\em Buďte $P(x),Q(x)$ racionální funkce jako v kapitole IX. Potom je $S(x)=R(P(x),Q(x))$ racionální funkce.}
 
 \bigskip
 
 {\bf 6.2. Integrál $\int R\big(x,\sqrt{\frac{ax+b}{cx+d}}\big)\d x$.} Použijeme substituci $y=\sqrt{\frac{ax+b}{cx+d}}$. Potom je
 $y^2=\frac{ax+b}{cx+d}$ z čehož získáme
$$
 x=\frac{b-dy^2}{ay^2+a}
 $$
 a tedy
 $$
 \frac{\d x}{\d y}=S(y)
 $$
 kde $S(y)$ je racionální funkce (explicitní formule se získá snadno).  Naše substituce tedy transformuje
 $$
 \int R\bigg(x,\sqrt{\frac{ax+b}{cx+d}}\bigg)\d x\qtq{do} \int R\bigg(\frac{b-dy^2}{ay^2+a},y\bigg)S(y)\d y
 $$
 a to je již možno spočíst procedurami z předchozích sekcí.
 
 \bigskip
 
 {\bf 6.3. Eulerova substituce: integrál} $\int R(x,\sqrt{ax^2+bx+c})\d x$. Nejprve se zbavíme případu 
$a\leq 0$. Přepokládáme-li, že funkce má smysl, musí mít $ax^2+bx+c\geq 0$ na oboru hodnot (v případě $a\leq 0$)  reálné kořeny $\alpha,\beta$ a
 $$
 \begin{aligned}
  R(x,\sqrt{ax^2+bx+c})&=R(x,\sqrt{-a}\sqrt{(x-\alpha)(x-\beta)})=\\
 &=R\big(x,\sqrt{-a}(x-\alpha)\sqrt{\frac{x-\beta}{x-\alpha}}\big)
 \end{aligned}
  $$
  a tento případ byl již pojednán v 5.2.
  
  \smallskip
  
  Ale v případě $a>0$ je situace nová. Potom substituujeme $t$ z rovnice
  $$
  \sqrt{ax^2+bx+c}=\sqrt{a}x+t
  $$
  (to je {\em Eulerova substituce}). Zdvojmocnění obou stran dá rovnici
  $$
  ax^2+bx+c=ax^2+2\sqrt{a}xt + t^2
  $$
  a z ní získáme 
  $$
  x=\frac{t^2-c}{b-2t\sqrt{a}} \qtq{a tedy} \frac{d x}{\d t}=S(t)
$$
kde $S(t)$ je racionální funkce.
Náš integrál tedy můžeme spočítat jako
$$
\int R\left(\frac{t^2-c}{b-2t\sqrt{a}},\sqrt{a}\frac{t^2-c}{b-2t\sqrt{a}}+t\right)S(t)\d t.
$$

\bigskip

{\bf 6.4. Goniometrické funkce v racionální funkci: $\int R(\sin x,\cos x)\d x$.} Abychom spočetli
$$
\int R(\sin x,\cos x)\d x
$$
si pomůžeme  substitucí
$$
y=\tan\frac{x}{2}.
$$
Ze standardní formule
$$
\cos^2x=\frac{1}{1+\tan^2x}
$$
získáme
$$
\begin{aligned}
&\sin x=2\sin\frac{x}{2}\cos\frac{x}{2}=2\tan\frac{x}{2}\cos^2\frac{x}{2}=
\frac{2\tan\frac{x}{2}}{1+\tan\frac{x}{2}^2}=\frac{2y}{1+y^2},\\
&\cos x=\cos^2\frac{x}{2}-\sin^2\frac{x}{2}=2\cos^2\frac{x}{2}-1=
\frac{2}{1+y^2}-1=\frac{1-y^2}{1+y^2}.
\end{aligned}
$$
Dále máme
$$
\frac{\d y}{\d x}=\frac12\cdot\frac{1}{\cos^2\frac{x}{2}}=
\frac12\cdot(1+\tan^2\frac{x}{2})=\frac12(1+y^2)
$$
a tedy
$$
\d x-\frac{2}{1+y^2}\d y
$$
takže úlohu můžeme vyřešit počítáním
$$
\int R\left(\frac{2y}{1+y^2},\frac{1-y^2}{1+y^2}\right)\frac{2}{1+y^2}\d y.
$$

\bigskip

{\bf 6.5. Poznámka.} Procedury ze sekcí 4 a 5 jsou nepochybně velmi pracné a náročné na čas. To je zčásti proto, že zde
pokrýváme značně obecné případy. V konkretním případě někdy můžeme najít kombinaci substituce a metody per partes která vede k cíli mnohem rychleji. Srovnejte třeba
 $\int\tan x\d x$ spočtené v 4.3 s 6.4.

\newpage


   
    \centerline{\Large\bf XI. Riemannův integrál} 
 
 \vskip10mm
 
 
 \def\d{\text{d}}
 
 
 
 {\large\bf 1. Obsah rovinného obrazce.}
 
 \def\vol{\text{\sf vol}}
 
 \bigskip
 
 {\bf 1.1.} Označme symbolem $\vol(M)$ obsah rovinného obrazce $M\sue \Rbb^2$. Obrazec může být příliš exotický, aby se o obsahu mohlo snadno mluvit, ale o takové zde nepůjde
Když symbol $\vol$ použijeme, implicite máme na mysli, že obsah dává smysl.
 
 
 
 \bigskip
 
 {\bf 1.2.} O následujících požadavcích se asi snadno dohodneme.
 \begin{enumerate}
 \item $\vol(M)\geq 0$ má-li smysl,
 \item je-li $M\sue N$ je $\vol(M)\leq\vol(N)$,
 \item jsou-li $M$ a $N$ disjunktní je $\vol(M\cup N)=\vol(M)+\vol(N)$, a
 \item je-li $M$ obdélník se stranami $a,b$ je $\vol(M)=a\cdot b$.
 \end{enumerate}
 
 \bigskip
 
 {\bf 1.3. Pozorování.} 1. {\em $\vol(\ems)=0$.
 
 {\em 2.} Buď $M$ úsečka. Potom $\vol(M)=0$.
 
 Důkaz.} 1: $\ems$ je podmnožinou kteréhokoli obdélníka,
 tvrzení tedy platí z (1),(2) a (4)
 
 2 dostaneme podobně: úsečka délky $a$ je podmnožinou obdélníka se stranami $a,b$
 při čemž může být $b$ libovolně malé.\sq
 
 \medskip
 
 {\bf 1.3.1. Poznámka.} Vidíme tedy, že 
 v 1.2(4) nebylo potřeba mluvit o tom, zahrnujeme-li do obdélníka okraje nebo jejich části.
 
 \bigskip
 
 {\bf 1.4. Tvrzení.} {\em Dávají-li obsahy smysl platí o nich
 $$
 \vol(M\cup N)=\vol(M)+\vol(N)-\vol(M\cap N).
 $$
 Zvláště pak máme
 $$
 \vol(M\cup N)=\vol(M)+\vol(N)\qtq{kdykoli} \vol(M\cap N)=0.
 $$
 
 Důkaz.} Plyne 1.2(3) vezmeme-li v úvahu disjunktní sjednocení
 $$
M\cup N=M\cup(N\smin M)\qtq{a} N=(N\smin M)\cup(N\cap M).
$$
\sq

\bigskip

{\bf 1.5.} V dalším budou hrát zvláštní roli obrazce následujícího typu
\bigskip
$$
\centerline{
\xymatrix @R=10pt @C=3pt{
&&(x_1,y_1)\ar@{-}[rrr]\ar@{-}[dd]&&&.\ar@{-}[dddd]&&&&&\\
&&&&&&&&&(x_3,y_3)\ar@{-}[r]\ar@{-}[ddd]&.\ar@{-}[dddddddd]\\
(x_0,y_0)\ar@{-}[rr]\ar@{-}[ddddddd]&&.\ar@{-}[ddddddd]&&&&&&&&\\
&&&&&&&&&\\
&&&&&(x_2,y_2)\ar@{-}[rrrr].\ar@{-}[ddddd]&&&&.\ar@{-}[ddddd]&\\
&&&&&&&&&&\\
&&&&&&&&&&\\
&&&&&&&&&&\\
&&&&&&&&&&\\
(x_0,0)\ar@{-}[rr]&&(x_1,0)\ar@{-}[rrr]&&&(x_2,0)\ar@{-}[rrrr]&&&&(x_3,0)\ar@{-}[r]&(x_4,0)
}
}
$$
\vskip10mm
\noindent Podle předchozích triviálních tvrzení jsou jejich obsahy prostě součty obdélníků z nichž jsou sestaveny. Např. obrazec nahoře má tedy obsah
$$
y_0(x_1-x_0) +y_1(x_2-x_1) +y_2(x_3-x_2) +y_3(x_4-x_3).
$$



 
 \vskip10mm
 
 {\large\bf 2. Definice Riemannova integrálu.} 
 
 \bigskip
 
 {\bf 2.1. Úmluva.} V této kapitole se  budeme zabývat {\em omezenými} reálnými funkcemi $f:J\to \Rbb$ definovanými na kompaktních  intervalech $J$, to jest funkcemi pro které existují čísla $m,M$ taková, že pro všechna $x\in J$ je $m\leq f(x)\leq M$. Připomeňme si, že vzhledem ke kompaktnosti
 je spojitá funkce na  $J$ vždy omezená. Naše funkce ale nebudou nutně spojité.
 
 \bigskip
 
 
 {\bf 2.2.}  {\em Rozklad} kompaktního intervalu $\langle a,b\rangle$ je posloupnost
 $$
 P:\ a=t_0< t_1<\cdots<t_{n-1}< t_n=b.
 $$
O jiném rozkladu
 $$
 P':\ a=t'_0< t'_1<\cdots<t'_{n-1}< t_m=b
  $$
  řekneme, že {\em zjemňuje} $P$ (nebo že {\em je zjemnění} toho $P$) je-li množina
  $\setof{t_j}{j=1,\dots,n-1}$ obsažena v $\setof{t'_j}{j=1,\dots,m-1}$.
  
{\em Jemnost}  rozkladu $P$, označená $\mu(P)$, je definována jako maximum rozdílů $t_{j}-t_{j-1}$.
  
  \bigskip
  
  {\bf 2.3.} Pro omezenou funkci $f:J=\langle a,b\rangle\to \Rbb$ a rozklad $P:
   a=t_0< t_1<\cdots<t_{n-1}< t_n=b$  definujeme
  {\em dolní} resp. {\em horní součet}  $f$ v $P$ jako
  $$
  s(f,P)=\sum_{j=1}^nm_j(t_j-t_{j-1}) \qtq{resp.} S(f,P)=\sum_{j=1}^nM_j(t_j-t_{j-1})
  $$
  kde $m_j=\inf\setof{f(x)}{t_{j-1}\leq x\leq t_j}$ a  $M_j=\sup\setof{f(x)}{t_{j-1}\leq x\leq t_j}$.
  
  \medskip
  
  {\bf 2.3.1. Tvrzení.} {\em Buď $P'$ zjemnění $P$. Potom je
  $$
  s(f,P)\leq s(f,P') \qtq{a} S(f,P)\geq S(f,P')
  $$
 
  
  Důkaz} pro horní součet: Nechť $t_{k-1}=t'_l< t_{l+1}'<\cdots< t'_{l+r}=t_k$. Pro $M'_{l+j}=
  \sup\setof{f(x)}{t_{l+j-1}'\leq x\leq t'_{l+j}}$ a $M_k=\sup\setof{f(x)}{t_{k-1}\leq x\leq t_k}$ máme
  $\sum_jM'_j(t'_{l+j}- t'_{l+j-1})\leq \sum_jM_k(t'_{l+j}- t'_{l+j-1})=M_k(t_k-t_{k-1})$. \sq
  
  
  \medskip
  
  {\bf 2.3.2. Tvrzení.} {\em Pro libovolné rozklady $P_1,P_2$ máme
  $$
  s(f,P_1)\leq S(f,P_2).
  $$
    
  Důkaz.} Zřejmě je $s(f,P)\leq S(f,P)$ pro každý rozklad.
  Dále, pro každé dva $P_1, P_2$ máme společné zjemnění $P$: stačí vzít sjednocení mno\v zin dělících bodu těchto zjemnění. Podle 2.3.1 tedy
  $$
  s(f,P_1)\leq s(f,P)\leq S(f,P)\leq S(f,P_2).
  $$
  \sq
  
  \bigskip 
  
  {\bf 2.4.} Podle 2.3.2 je množina reálných čísel $\setof{s(f,P)}{P\ \text{rozklad}}$ shora omezená a
  $\setof{S(f,P)}{P\ \text{rozklad}}$  zdola omezená. Máme tedy konečná čísla
  $$
  \begin{aligned}
  &\lint_a^bf(x)\d x=\sup\setof{s(f,P)}{P\ \text{rozklad}} \ \ \text{a}\\
  &\uint_a^bf(x)\d x=\inf\setof{S(f,P)}{P\ \text{rozklad}}.
  \end{aligned}
  $$
  První se nazývá {\em dolní Riemannův integrál} funkce $f$ přes $\langle a,b\rangle$, druhé je {\em horní Riemannův integrál} funkce $f$.
  
  Z 2.3.2 dále vidíme, že
  $$
  \lint_a^bf(x)\d x\leq \uint_a^bf(x)\d x;
  $$
  Pokud je $\lint_a^bf(x)\d x = \uint_a^bf(x)\d x$ nazýváme společnou hodnotu
  $$
  \int_a^bf(x)\d x
  $$
   {\em Riemannův integrál} funkce $f$ přes $\langle a,b\rangle$.
  
  \medskip
  
  {\bf 2.4.1. Pozorování.} {\em Buď $m=\inf\setof{f(x)}{a\leq x\leq b}$ a
  $M=\sup\setof{f(x)}{a\leq x\leq b}$. Potom máme
  $$
  m(b-a)\leq \lint_a^bf(x)\d x\leq \uint_a^bf(x)\d x\leq M(b-a).
  $$}
  
  \medskip
  
  {\bf 2.4.2. Tvrzení.} {\em Riemannův integrál $\int_a^bf(x)\d x$ existuje právě když pro každé $\epsilon >0$ existuje rozklad  $P$ takový, že
  $$
  S(f,P)-s(f,P)<\epsilon.
  $$
 
 Důkaz.} I. Nechť $ \int_a^bf(x)\d x$ existuje; zvolme $\epsilon>0$. Potom existují rozklady $P_1$ a $P_2$ takové, že
 $$
 S(f,P_1)<  \int_a^bf(x)\d x+\frac{\epsilon}{2} \qtq{a} s(f,P_2)>  \int_a^bf(x)\d x+\frac{\epsilon}{2}.
 $$
 Potom podle  2.3.1 máme pro společné zjemnění $P$ těchto $P_1,P_2$,
 $$
 S(f,P)-s(f,P)<\int_a^bf(x)\d x+\frac{\epsilon}{2}- \int_a^bf(x)\d x+\frac{\epsilon}{2}=\epsilon.
 $$
 
 \smallskip
 
 II. Nechť tvrzení platí. Zvolme $\epsilon>0$  a $P$ pro které $S(f,P)-s(f,P)<\epsilon$. Potom
 $$
  \uint_a^bf(x)\d x\leq S(f,P)<s(f,P)+\epsilon\leq \lint_a^bf(x)\d x +\epsilon,
 $$
a jelikož $\epsilon$ bylo libovolné vidíme, že $\uint_a^bf(x)\d x= \lint_a^bf(x)\d x$.\sq
 
 \bigskip
 
 {\bf 2.5. Poznámky.} 1. Co se děje vidíme nejlépe analysou chování nezáporných funkcí $f$. Vezměme
 $F$ = $\setof{(x,y)}{x\in\langle a,b\rangle,\ 0\leq f(x)}$, tedy obrazec omezený $x$-ovou osou, grafem funkce $f$ a vertikálními přímkami procházejícími $(a,0)$ and $(b,0)$. Vezměte největší sjednocení obdélníků $F_l(P)$
s dolními vodorovnými hranami  $\langle t_{j-1},t_j\rangle$ (připomeňte si obrázek v  1.5) obsažený v
 $F$; zřejmě je $\vol(F_l(P))=s(f,P)$. Pro podobný nejmenší  obsah $F_u(P)$ obrazce obsahujícího $F$ máme
 $\vol(F_u(P))=S(f,P)$. Tedy, má-li obsah  $F$ smysl, musí být
 $$
 s(f,P)=\vol(F_l(P))\leq\vol(F)\leq\vol(F_u(P))=S(f,P),
 $$
a pokud $\int_a^bf(x)\d x$ existuje  je toto číslo jediný kandidát na $\vol(F)$ a je přirozené považovat ho za
ten obsah.
 
 \smallskip
 
 2. Značení $\int_a^bf(x)\d x$ pochází z ne úplně korektní, ale užitečné intuice. Představte si $\d x$ jako velmi malý interval (rádi bychom řekli ``nekonečně malý, ale s nenulovou délkou'', což není takový nesmysl jak to zní); představujeme si, že  $\d x$ jsou disjunktní a pokrývají úsečku $\langle a,b\rangle$, a$\int$ znamená  ``součet'' obsahů  ``velmi tenkých obdélníků'' s vodorovnými hranami $\d x$ a výškami $f(x)$. Uvědomme si, jak  blízko je tato představa od korektnějšího pohledu nahoře v bodě 1, vezmeme-li 
  $P$ s velmi malou jemností.
 
 \bigskip
 
 {\bf 2.6. Značení.} Není-li nebezpečí nedorozumění  zkracujeme značení (analogicky jako kapitole X) 
 $$
 \lint_a^bf(x)\d x,\ \uint_a^bf(x)\d x,\ \int_a^bf(x)\d x\qtq{na}\lint_a^bf,\ \uint_a^bf,\ \int_a^bf.
 $$
 
 \vskip10mm
 
 {\large\bf 3. Spojité funkce.} 
 
 \bigskip
 
 {\bf 3.1. Stejnoměrná spojitost.} Řekneme,že reálná funkce $f:D\to\Rbb$ je {\em stejnoměrně spojitá} platí-li
 $$
 \forall \epsilon>0 \ \exists \delta>0\ \ \text{takové,že}\ \ \forall x,y\in D, \ \ |x-y|<\delta\ \Rightarrow\ 
 |f(x)-f(y)|<\epsilon.
 $$
 
 \medskip
 
 {\bf 3.1.1. Poznámka.} Všimněte si jemného rozdílu mezi spojitostí a stejnoměrnou spojitostí. V první závisí $\delta$ nejen na $\epsilon$ ale také na $x$, v druhé ne. Stejnoměrně spojitá funkce je zřejmě spojitá, ale opačná implikace neplatí. Vezměme třeba
 $$
 f(x)=(x\mapsto x^2):\Rbb\to\Rbb.
 $$
 Máme $|x^2-y^2|=|x-y|\cdot|x+y|$;  chceme-li tedy mít $|x^2-y^2|<\epsilon$ v okolí bodu $x=1$ stačí volit $\delta$ blízké číslu $\epsilon$, v okolí bodu $x=100$ potřebujeme $\delta$ kolem $\frac{\epsilon}{100}$.
 
 \medskip
 
 {\bf 3.1.2.} Ale, možná trochu překvapivě, na kompaktním oboru se tyto pojmy shodují. Platí
 
 \medskip
 
 {\bf Věta.} {\em Funkce $f:\langle a,b \rangle \to\Rbb$ je spojitá právě když je stejnoměrně spojitá.
 
 Důkaz.} Nechť $f$ není stejnoměrně spojitá. Dokážeme, že pak není ani spojitá.
 
 Jelikož formule pro stejnoměrnou spojitost neplatí, máme  $\epsilon_0>0$ takové, že pro každé $\delta>0$ existují $x(\delta), y(\delta)$ takov\'a, že $|x(\delta)- y(\delta)|<\delta$ a přitom
 $|f(x(\delta))-f(y(\delta))|\geq \epsilon_0$. Položme $x_n=x(\frac{1}{n})$ a $y_n=y(\frac{1}{n})$. Podle IV.1.3.1 můžeme najít konvergentní podposloupnosti $(\wt x_{n})_n$, $(\wt y_{n})_n$  (nejprve zvolíme konvergentní podposloupnost $(x_{k_n})_n$ posloupnosti $(x_n)_n$ a potom konvergentní podposloupnost  $(y_{k_{l_n}})_n$ posloupnosti $(y_{n_k})_k$ a konečně položíme $\wt x_n=x_{k_{l_n}}$ a 
 $\wt y_n=y_{k_{l_n}}$). Potom je $|\wt x_n- \wt y_n|<\frac1{n}$ a tedy $\lim \wt x_n=\lim \wt y_n$. Jelikož ale
$|f(\wt x_n)- f(\wt y_n)|\geq\epsilon_0$, nemůže být $\lim f(\wt x_n)=\lim f(\wt y_n)$ takže podle IV.5.1 není $f$  spojitá. \sq

\bigskip

{\bf 3.2. Věta.} {\em Pro libovolnou spojitou  $f:\langle a,b \rangle \to\Rbb$ existuje  Riemannův integrál
$\int_a^bf$.

Důkaz.} Jelikož je $f$ podle 3.1.2 stejnoměrně spojitá, můžeme pro $\epsilon>0$ najít $\delta>0$ takové, že
$$
|x-y|< \delta \quad\Rightarrow\quad |f(x)-f(y)|< \frac{\epsilon}{b-a}.
$$
Připomeňme si jemnost $\mu(P)=\max_j(t_j-t_{j-1})$ rozkladu $P: t_0<t_1<\cdots<t_k$. Je-li $\mu(P)<\delta$ máme $t_j-t_{j-1}<\delta$ pro každé $j$,
a tedy
$$
\begin{aligned}
M_j-m_j&= \sup\setof{f(x)}{t_{j-1}\leq x\leq t_j}-\inf\setof{f(x)}{t_{j-1}\leq x\leq t_j}\leq \\
 &\leq\sup\setof{|f(x)-f(y)|}{t_{j-1}\leq x,y\leq t_j}\leq\frac{\epsilon}{b-a}
 \end{aligned}
 $$
 takže
 $$
 \begin{aligned}
 S(f,P)-s(f,P)&=\sum(M_j-m_j)(t_j-t_{j-1})\leq\\
 &\leq\frac{\epsilon}{b-a}\sum(t_j-t_{j-1})=\frac{\epsilon}{b-a}(b-a)=\epsilon.
 \end{aligned}
 $$
 Použijme 2.4.2 \sq
 
 \medskip
 
 {\bf 3.2.1.} Když si promyslíme tento důkaz do detailu, dostaneme poněkud silnější větu.
 
 \smallskip
 
 {\bf Věta.} {\em  Nechť je $f:\langle a,b \rangle \to\Rbb$ a nechť je $P_1,P_2,\dots$ posloupnost rozkladů takových, že
 $\lim_n\mu(P_n)=0$. Potom
 $$
 \lim_n s(f,P_n)=\lim_n S(f,P_n)=\int_a^b f.
 $$}
 
 (Pro $\epsilon$ and $\delta$ nahoře zvolme $n_0$ takové, že pro $n\geq n_0$ máme $\mu(P_n)<\delta$.)
 
 \bigskip


{\bf 3.3. Věta.} (Integrální věta o střední hodnotě) {\em Buď $f:\langle a,b\rangle\to\Rbb$ spojitá.
Potom existuje $c\in\langle a,b\rangle$ takové, že
$$
\int_a^bf(x)\d x= f(c)(b-a).
$$

Důkaz.} Položme $m=\min\setof{f(x)}{a\leq x\leq b}$ a
  $M=\max\setof{f(x)}{a\leq x\leq b}$ (viz IV.5.2). Potom
  $$
  m(b-a)\leq \int_a^bf(x)\d x\leq M(b-a).
  $$
  Existuje tedy $K$ takové,že $m\leq K\leq M$ a že $\int_a^bf(x)\d x=K(b-a)$.
  Podle IV.3.2 existuje $c\in\langle a.b\rangle$ takové, že $K=f(c)$. \sq
 


 \vskip10mm
 
 {\large\bf 4. Základní  věta analysy.} 
 
 \bigskip
 
 {\bf 4.1. Tvrzení.} {\em Buďte $a<b<c$ a nechť je $f$  omezená na $\langle a,c\rangle$. Potom
 $$
 \lint_a^bf+\lint_b^cf=\lint_a^cf \qtq{and} \uint_a^bf+\uint_b^cf=\uint_a^cf.
$$

Důkaz} pro dolní integrál. Označme $\mathcal P(u,v)$ množinu všech rozkladů $\langle u,v\rangle$. Pro $P_1\in\mathcal P(a,b)$ a $P_2\in\mathcal P(b,c)$ definujme 
$P_1+P_2\in\mathcal P(a,c)$ jako sjednocení těch dvou posloupností. Potom zřejmě
$$
s(f,P_1+P_2)=s(f,P_1)+s(f,P_2)
$$
a tedy
$$
\begin{aligned}
\lint_a^bf+\lint_b^cf &=\sup_{P_1\in\mathcal P(a,b)}s(f,P_1)+\sup_{P_2\in\mathcal P(b,c)}s(f,P_2)=\\
&=\sup\setof{s(f,P_1)+s(f,P_2)}{P_1\in\mathcal P(a,b),P_2\in\mathcal P(b,c)}=\\
&=\sup\setof{s(f,P_1+P_2)}{P_1\in\mathcal P(a,b),P_2\in\mathcal P(b,c)},
\end{aligned}
$$
Stačí prostě přidat $b$ do jeho posloupnosti.
Takže je podle 2.3.1 toto poslední supremum rovno
$$
\sup\setof{s(f,P)}{P\in\mathcal P(a,c)}=\lint_a^cf.
$$ \sq

\bigskip

{\bf 4.2. Úmluva.} Pro $a=b$ položme $\int_a^af=0$ a pro $a>b$  definujme $\int_a^bf=-\int_b^af$. Snadno ověříme, že pak platí

\medskip

{\bf 4.2.1. Pozorování.} {\em Pro libovolná $a,b,c$ je
$$
 \int_a^bf+\int_b^cf=\int_a^cf.
$$}

\bigskip

{\bf 4.3. Věta.} (Základní Věta Analysy) {\em Buď $f:\langle a,b\rangle\to\Rbb$ spojitá. Pro $x\in\langle a,b\rangle$ položme
$$
F(x)=\int_a^xf(t)\d t.
$$
 Potom je
$F'(x)=f(x)$ (přesněji, derivace v $a$ je zprava a derivace v $b$ je zleva).

Důkaz.} Podle 4.2.1 a 3.3 máme pro $h\neq 0$
$$
\frac{1}{h}(F(x+h)-f(x))=\frac{1}{h}(\int_a^{x+h}f-\int_a^xf)=\frac{1}{h}\int_x^{x+h}f=
\frac{1}{h}f(x+\theta h)h=f(x+\theta h)
$$
kde $0<\theta<1$ a jelikož $f$ je spojitá, $\lim_{h\to 0}\frac{1}{h}(F(x+h)-f(x))=
\lim_{h\to 0}f(x+\theta h)=f(x).
$
\sq

\medskip

{\bf 4.3.1. Důsledek.} {\em Každá spojitá $f:\langle a,b\rangle\to\Rbb$ má primitivní funkci na  $(a,b)$ spojitou na $\langle a,b\rangle$.
Je-li $G$ libovolná primitivní funkce k $f$ na $(a,b)$ spojitá na $\langle a,b\rangle$, je 
$$
\int_a^bf(t)\d t=G(b)-G(a).
$$.}

(Podle 4.3 máme $\int_a^bf(t)\d t=F(b)-F(a)$. Připomeňme si IX.1.2.)

\medskip

{\bf 4.3.2. Poznámka.} Povšimněte si kontrastu mezi derivacemi a primitivními funkcemi. Mít derivaci je u spojité funkce velmi silná vlastnost, ale derivování elementárních funkcí -- to jest funkcí se kterými se typicky setkáváme -- je velmi jednoduché. Na druhé straně každá spojitá funkce má funkci primitivní, ale spočítat ji je velmi těžké až nemožné.

\bigskip

{\bf 4.4.} Připomeňte si integrální větu o střední hodnotě (3.3). Základní věta analysy ji klade do úzkého vztahu s větou o střední hodnotě diferenciálního počtu. Skutečně, označíme-li
$F$ primitivní funkci k $f$, formule 3.3  dává
$$
F(b)-F(a)=F'(c)(b-a).
$$

\vskip10mm
 
 {\large\bf 5. Několik jednoduchých fakt.} 
 
 \bigskip
 
 {\bf 5.1. Tvrzení.} {\em Nechť se $g$ a $f$ liší v konečně mnoha bodech. Potom
 $$
 \lint_a^bf=\lint_a^bg \qtq{and} \uint_a^bf=\uint_a^bg.
 $$
  Zvláště pak, existuje-li $\int_a^bf$ existuje též $\int_a^bg$ a platí $\int_a^bf=\int_a^bg $.
 
 Důkaz}\hskip1mm pro dolní integrál. Použijeme $\mu(P)$ z 2.2. Jsou-li $|f(x)|$ a $|g(x)|$ menší než $A$ pro všechna  $x$ a jestliže se $f$ a $g$ liší v $n$ bodech, pak
$$
|s(f,P)-s(g,P)|\leq n\cdot A\cdot\mu(P),
$$
a $\mu(P)$ může být libovolně malé. \sq

\bigskip

{\bf 5.2. Tvrzení.} {\em Nechť má $f$  na $\langle a,b\rangle$ jen konečně mnoho bodů nespojitosti, všechny oprvního druhu, Potom Riemannův integrál $\int_a^bf$ existuje.

Důkaz.} Buďte $c_1<c_2<\cdots<c_n$ ty body nespojitosti. Potom máme
$$
\int_a^b f=\int_a^{c_1}f+\int_{c_1}^{c_2}f+\cdots+\int_{c_n}^bf.
$$
\sq

\bigskip

{\bf 5.3.  Tvrzení.} {\em Nechť $\int_a^b f$ a $\int_a^b g$ existují a nechť jsou $\alpha,\beta$ reálná čísla. Potom $\int_a^b (\alpha f+\beta g)$ existuje and máme
$$
\int_a^b (\alpha f+\beta g)=\alpha\int_a^bf+\beta\int_a^b g.
$$

Důkaz.} I. Nejdřív snadno vidíme, že $\int_a^b \alpha f=\alpha\int_a^b f$. Pro $\alpha\geq 0$ je totiž zřejmě
$s(\alpha f,P)=\alpha s(f,P)$ a $S(\alpha f,P)=\alpha S(f,P)$, a pro $\alpha\leq 0$ máme
$s(\alpha f,P)=\alpha S(f,P)$ and $S(\alpha f,P)=\alpha s(f,P)$.

\smallskip

II. Stačí tedy provést důkaz pro $f+g$. Položme 
$m_i=\inf\setof{f(x)+g(x)}{x\in\langle t_{i-1},t_i\rangle}$, 
$m'_i=\inf\setof{f(x)}{x\in\langle t_{i-1},t_i\rangle}$ a $m''_i=\inf\setof{g(x)}{x\in\langle t_{i-1},t_i\rangle}$. Zřejmě je $m_1'+m_i''\leq m_i$ a tedy
$$
s(f,P)+s(g,P)\leq s(f+g,P), \qtq{a podobně} S(f+g,P)\leq S(f,P)+S(g,P)
$$
a snadno usoudíme, že
$$
\lint_a^b f+\lint_a^b g\leq \lint_a^b (f+g) \qtq{a} \uint_a^b (f+g)\leq\uint_a^b f+\uint_a^b g
$$
a tedy
$$
\int_a^b f+\int_a^b g\leq  \lint_a^b (f+g) \leq \uint_a^b\leq \int_a^b f+\int_a^b g.
$$\sq

\bigskip

{\bf 5.4. Per partes.} Zaveďme značení
$$
[h]_a^b =h(b)-h(a).
$$
Potom z 4.3 a X.3.1 bezprostředně dostáváme
$$
\int_a^b f\cdot g'=[f\cdot g]_a^b-\int_a^b f'\cdot g.
$$

\bigskip

{\bf 5.5. Věta.} (Věta o substituci pro Riemannův integrál) {\em Buď $f:\langle a,b\rangle \to\Rbb$ spojitá a buď $\phi:\langle a,b\rangle \to\Rbb$ vzájemně jednoznačné zobrazení s derivací. Potom
$$
\int_a^bf(\phi(x))\phi'(x)\d x=\int_{\phi(a)}^{\phi(b)}f(x)\d x.
$$

Důkaz.} Připomeňte si 4.4 i s definicí $F$. Hned získáme
$$
\int_{\phi(a)}^{\phi(b)}f(x)\d x=F(\phi(b))-F(\phi(a)).
$$
Ale podle  X.4.1 a 4.4 máme též
$$
F(\phi(b))-F(\phi(a))=\int_a^bf(\phi(x))\phi'(x)\d x,
$$
a tvrzení je dokázáné.\sq

\medskip

{\bf 5.5.1.} Za touto substituční formulí je silná geometrická intuice.

 Připomeňme 2.5 a 2.6.
 Představujte si $\phi$ jako deformaci intervalu $\langle a,b \rangle$ na interval
 $\langle \phi(a),\phi(b)\rangle$. Derivace $\phi'(x)$ měří to, jak jsou malé intervaly kolem $x$
 natahovány či stlačovány. Tedy, počítáme-li integrál $\int_{\phi(a)}^{\phi(b)} f$  jako integrál přes původní $\langle a,b\rangle$ musíme upravit ``malé prvky'' délky $\d x$
tímto nataženímm či stlačením což dá korigovaný ``malý prvek'' délky $\phi'(x)\d x$.

\newpage
{.}
\newpage
 
\centerline{\Large\bf XII. Několik aplikací Riemannova integrálu} 
 
 
 \vskip10mm
 
 
 \def\d{\text{d}}
 
 V této krátké kapitole předvedeme několik aplikací Riemannnova integrálu. U některých půjde o výpočty obsahů a objemů a podobné záležitosti, ve dvou případech však půjde o aplikace teoretického rázu.
 
 \vskip10mm
 
 
 
 {\large\bf 1. Obsah rovinného obrazce, znovu.}
 
 \def\vol{\text{\sf vol}}
 
 \bigskip
 
 {\bf 1.1.} Definici Riemannova integrálu jsme motivovali představou obsahu rovinného obrazce 
 $$
 F=\setof{(x,y)}{x\in \langle a,b\rangle , 0\leq y\leq f(x)}
 $$
 kde $f$ byla nezáporná spojitá funkce. Pro rozklad $P:a=t_0<t_1\cdots<t_n=b$ intervalu $\langle a,b\rangle$ je tento $F$ minorizován sjednocením obdélníků
 $$
 \bigcup_{j=1}^n\langle t_{j-1},t_j\rangle\times\langle 0,m_j\rangle
 \qtq{kde} m_j=\inf\setof{f(x)}{t_{j-1}\leq x\leq t_j},
 $$
 s obsahem
 $$
 s(f,D)=\sum_{j=1}^n m_j(t_j-t_{j-1}),
 $$
 a majorizován sjednocením obdélníků
 $$
 \bigcup_{j=1}^n\langle t_{j-1},t_j\rangle\times\langle 0,M_j\rangle
 \qtq{kde} M_j=\sup\setof{f(x)}{t_{j-1}\leq x\leq t_j},
 $$
 s obsahem
 $$
 S(f,D)=\sum_{j=1}^n M_j(t_j-t_{j-1}).
 $$ 
Tedy (viz XI.2.5), jediný kandidát pro objem $ F$ je
 $$ 
 \vol(F)=\int_a^bf(x)\d x,
$$
společná hodnota suprema prvních a infima druhých.

\bigskip

{\bf 1.2.} Tak na příklad obsah úseku paraboly
$$
F= \setof{(x,y)}{-1\leq x\leq 1,0\leq y\leq 1-x^2}
$$
je
$$
\int_{-1}^1(1- x^2)\d x=[x-\frac13 x^3]_{-1}^1=1-\frac13+1-\frac13=\frac43.
$$

\bigskip

{\bf 1.3.} Spočítejme obsah kruhu o poloměru $r$. Jeho polovina je dána jako
$$
J=\int_{-r}^r\sqrt{r^2-x^2}\d x.
$$
Substituujme $x=r\sin y$. Potom $\d x=r\cos y\d y$ a $\sqrt{r^2-x^2}=r\cos y$
takže $J$ je transformován do
$$
J=r^2\int_{-\frac{\pi}2}^{\frac{\pi}2}\cos^2y\,\d y.
$$
Máme $\cos^2y=\frac12(\cos 2y+1)$, a pokračujeme
$$
\frac{J}{r^2}=\frac12 \int_{-\frac{\pi}2}^{\frac{\pi}2}\cos 2y\,\d y+
\frac12\int_{-\frac{\pi}2}^{\frac{\pi}2}\d y=
\frac12\left(\left[\frac12\sin 2y\right]_{-\frac{\pi}2}^{\frac{\pi}2}+
[y]_{-\frac{\pi}2}^{\frac{\pi}2}\right)=\frac12(0+\pi)
$$
takže žádaný obsah je $2J=\pi r^2$.



\vskip10mm
 
 {\large\bf 2. Objem rotačního tělesa.} 
 
 \bigskip
 
 {\bf 2.1.} Vezměme opět nezápornou spojitou $f$ a křivku
 $$
 C=\setof{(x,f(x),0)}{a\leq x\leq b}
 $$
 v třírozměrném euklidovském prostoru. Rotujme $C$ kolem $x$-ové osy
 $\setof{(x,0,0)}{x\in \Rbb}$ a uvažujme tělěso $F$ omezené z\'\i skanou plochou.
 
 Objem těles $F$ můžeme snadno spočítat takto. Místo sjednocení
 obdélníků 
 $
 \bigcup_{j=1}^n\langle t_{j-1},t_j\rangle\times\langle 0,m_j\rangle
  $
 jako v 1.1,
  budeme množinu  $F$ minorizovat sjednocením disků (válců)
   $$
 \bigcup_{j=1}^n\langle t_{j-1},t_j\rangle\times\setof{(y,z)}{y^2+z^2\leq m_i^2}
 \qtq{kde} m_j=\inf\setof{f(x)}{t_{j-1}\leq x\leq t_j}
 $$
s objemem 
 $$
 \sum_{j=1}^n\pi m_j^2(t_j-t_{j-1})
 $$
a podobně dostáváme horní odhad objemu našeho tělesa jako
 $$
 \sum_{j=1}^n\pi M_j^2(t_j-t_{j-1}) \qtq{kde} M_j=\sup\setof{f(x)}{t_{j-1}\leq x\leq t_j}.
 $$
Tedy, objem $F$ dostaneme jako
$$
\vol(F)=\pi\int_a^b f^2(x)\d x.
$$

\bigskip

{\bf 2.2.} Např. třírozměrná koule $B_3$ je omezená rotující křivkou
$\setof{(x,\sqrt{r^2-x^2})}{-r\leq x\leq r}$ a dostaneme tedy
$$
\vol(B_3)=\pi\int_{-r}^r(r^2-x^2)\d x=\pi\left[r^2x-\frac13 x^3\right]_{-r}^r=
2\pi\left(r^3-\frac13 r^3\right)=\frac43\pi r^3.
$$

\vskip10mm
 
 {\large\bf 3. Délka rovinné křivky
 
 \hskip7mm a povrch rotačního tělesa.} 
 
 \bigskip
 
 {\bf 3.1.} Uvažujme $f$ spojitou funkci $\langle a,b \rangle$
 (za chvíli budeme ještě navíc předpokládat, že má spojitou derivaci) a křivku
 $$
  C=\setof{(x,f(x))}{a\leq x\leq b}.
  $$
  Vezměme rozklad
  $$
   P:\ a=t_0< t_1<\cdots<t_{n-1}< t_n=b
   $$
intervalu $\langle a,b\rangle$,  a aproximujme $C$ systémem úseček $S(P)$ spojujících
$$
(t_{j-1},f(t_{j-1})) \qtq{s} (t_j,f(t_j)).
$$
Délka $L(P)$ takové aproximace, součet délek těchto úseček, je
$$
L(P)=\sum_{j=1}^n\sqrt{(t_j-t_{j-1})^2+(f(t_j)-f(t_{j-1}))^2}.
$$
Předpokládejme nyní, že $f$ má derivaci. Potom můžeme užít větu u střední hodnotě (VII.2.2) a získáme
$$
L(P)=\sum_{j=1}^n\sqrt{(t_j-t_{j-1})^2+f'(\theta_i)^2(t_j)-t_{j-1})^2}=\sum_{j=1}^n\sqrt{1+f'(\theta_i)^2}(t_j-t_{j-1}).
$$
Jestliže $P_1$ zjemňuje $P$ máme z trojúhelníkové nerovnosti
$$
L(P_1)\geq L(P)
$$
takže číslo
$$
L(C)=\sup\setof{L(P)}{P\ \text{rozklad intervalu} \ \langle a,b\rangle}
$$
můžeme přirozeně považovat za délku křivky $C$. Podle XI 3.2.1 tyto délky konvergují k
$$
L(C)=\int_a^b\sqrt{1+f'(x)^2}\d x.
$$

\bigskip

{\bf 3.2.} Podobně, aproximujeme-li povrch rotačního tělesa příslušnými častmi povrchů komolých jehlanů o výškách  $(t_j-t_{j-1})$ a poloměrech základen $f(t_i)$ a $f(t_{j-1})$, dostaneme formuli
$$
2\pi\int_a^b f(x)\sqrt{1+f'(x)^2}\d x.
$$


\vskip10mm
 
 {\large\bf 4. Logaritmus.} 
 
 \bigskip
 
 {\bf 4.1.} V V.1.1 byl logaritmus zaveden axiomaticky jako funkce $L$ 
 která
 \begin{enumerate}
 \item  roste v $\langle 0,+\infty)$,
 \item splňuje rovnici $L(xy)=L(x)+L(y)$, 
 \item a o níž platí, že $\lim_{x\to 0}\frac{L(x)}{x-1}=1$.
 \end{enumerate}
 Existence takové funkce (v níž jsme v V.1.1 museli věřit) bude nyní dokázána jednoduchou konstrukcí.
 
 \bigskip
 
 {\bf 4.2.} Položme
 $$
 L(x)=\int_1^x \frac1{t}\d t .
 $$
 Pro  $x>0$ je to korektní: funkce $\frac1{t}$ je definovaná a spojitá mezi $1$ a $x$.
  
 \medskip
 
 {\bf 4.2.1.} Je-li $x<y$ je $L(y)-L(x)=\int_x^y \frac1{t}\d t$
integrál kladné funkce přes $\langle x,y\rangle$ a tedy kladné číslo. $L(x)$ tedy roste.
 
 \medskip
 
 {\bf 4.2.2.} Máme
 \begin{equation}
 L(xy)=\int_1^{xy} \frac1{t}\d t=\int_1^x \frac1{t}\d t+\int_x^{xy}\frac1{t}\d t.\tag{$*$}
 \end{equation}
 v posledním sčítanci použijeme substituci $z=\phi(t)=xt$ a získáme
 $$
\int_x^{xy} \frac1{z}\d z=
\int_1^y \frac1{xt}\phi'(t)\d t=
\int_1^y \frac{x}{xt}\d t=\int_1^y \frac1{t}\d t
$$
takže $(*)$ dává
$$
L(x,y)=\int_1^x \frac1{t}\d t+\int_1^y \frac1{t}\d t=L(x)+L(y).
$$

\medskip

{\bf 4.2.3.} Končně máme
$$
\lim_{x\to 0}\frac{L(x)}{x-1}=\lim_{x\to 0}\frac{L(x)-L(1)}{x-1}=
L'(1)=\frac11=1
$$
podle XI.4.3.


\vskip10mm
 
 {\large\bf 5. Integrální kriterium konvergence řad.} 
 
 \bigskip
 
 {\bf 5.1.} Vezměme řadu $\sum a_n$ pro kterou $a_1\geq a_2\geq a_3\geq\cdots \geq 0$. Buď $f$ nerostoucí  spojitá funkce definovaná na intervalu $\langle 1,+\infty)$ taková, že
 $$
       a_n=f(n).
$$

\bigskip

{\bf 5.2. Věta.} (Integrální Kriterium Konvergence) {\em Řada $\sum a_n$ konverguje právě když je limita
$$
\lim_{n\to\infty}\int_1^nf(x)\d x
$$
konečná.

Důkaz.} Triviální odhad Riemannova integrálu dává
$$
a_{n+1}=f(n+1)\leq \int_n^{n+1}f(x)\d x\leq f(n)=a_n.
$$
Tedy je
$$
a_2+a_3+\cdots+a_n\leq\int_1^nf(x)\d x \leq a_1+ a_2+\cdots+a_{n-1}. 
   $$
Tedy dále, je-li limita $L=\lim_{n\to\infty}\int_1^nf(x)\d x$ konečná, je
$$
\sum_{1}^na_k\leq a_1+L
$$
a řada konverguje. Naopak, není-li posloupnost 
$(\int_1^nf(x)\d x )_n$ omezená, ani
$(\sum_1^na_n)_n$ není omezená.\sq

\bigskip

{\bf 5.3. Poznámka.} Všimněte si, že na rozdíl od kriterií v III.2.5, integrální kriterium
je nutná a postačující podmínka. Je tedy, samozřejmě, mnohem jemnější. To uvidíme v následujícín příkladě.

\bigskip

{\bf 5.4. Tvrzení.} {\em Pro každé reálné číslo $\alpha>1$ konverguje řada
\begin{equation}
\frac1{1^\alpha}+\frac1{2^\alpha}+\frac1{3^\alpha}+\cdots+\frac1{n^\alpha}+\cdots. \tag{$*$}
\end{equation}


Důkaz.} Máme
$$
\int_1^nx^{-\alpha}\d x=\left[\frac{1}{1-\alpha}\cdot x^{1-\alpha}\right]=\frac{1}{1-\alpha}\left(\frac1{n^{\alpha-1}}-1\right)\leq\frac1{\alpha-1}.
$$\sq

\bigskip

Všimněte si, že konvergence řady ($*$) z kriterií III.2.5 neplyne ani pro velká  $\alpha$.

\newpage


 \centerline{\Large\bf XIII. Metrické prostory: základy} 
 
 \vskip10mm
 
 
 \def\d{\text{d}}
 
 
 
 {\large\bf 1. Příklad.}
 
 \bigskip
 
{\bf 1.1.} V následujících kapitolách budeme studovat reálné funkce několika reálných proměnných. Definiční obory tedy budou podprostory euklidovských prostorů. Potřebujeme nyní rozumět lépe základním pojmům jako je konvergence či spojitost: jak uvidíme v následujícím příklad\v e, nemohou být redukovány na chování funkcí v jenotlivých proměnných. Některé pojmy si v této kapitole probereme v kontextu obecných metrických prostorů.


\bigskip

{\bf 1.2.} Uvažujme funkci $f:\mathbb E_2\to \Rbb$ dvou reálných proměnných definovanou předpisem
$$
f(x,y)=\begin{cases} &\frac{xy}{x^2+y^2}\ \ \text{pro}\ \ (x,y)\neq(0,0),\\
&0\ \ \text{pro}\ \ (x,y)=(0,0).\end{cases}
$$
Pro libovolné pevné $y_0$ je funkce $\phi:\Rbb\to\Rbb$ definovaná předpisem $\phi(x)=f(x,y_0)$ zřejmě spojitá
(pokud $y_0\neq 0$ je definovaná aritmetickým výrazem, a pro $y_0=0$ je to konstantní 0) a podobně
pro každé pevné $x_0$ též formule $\psi(y)=f(x_0)$ definuje spojitou $\psi:\Rbb\to\Rbb$. Ale funkce $f$ se vcelku chová divně: blížíme-li se k $(0,0)$ v argumentech $(x,x)$ s $x\neq 0$ jsou  hodnoty funkce $f$ stále $\frac12$ a v $x=0$ je skok do 0, zřejmá nespojitost v každém rozumném smyslu tohoto slova.



 \vskip10mm
 
 {\large\bf 2. Metrické prostory, podprostory, spojitost.}
  
 \bigskip
 
 
 {\bf 2.1.} {\em Metrika} (nebo {\em  vzdálenost})  na množině $X$ je funkce
 $$
 d:X\times X\to\Rbb
 $$
 taková, že
 \begin{enumerate}
 \item $\forall x,y$,\ \  $d(x,y)\geq 0$ a $d(x,y)=0$ právě když $x=y$,
 \item $\forall x,y$,\ \ $d(x,y)=d(y,x)$ a
 \item $\forall x,y,z$,\ \ $d(x,z)\leq d(x,y)+d(y,z)$ (trojúhelníková nerovnost).
 \end{enumerate}
 
{\em Metrický prostor} $(X,d)$ je množina $X$ opatřená metrikou $d$.
 
 \medskip
 
 {\bf Poznámka.} Požadavky (1) a (3) jsou velmi názorné: (1) požaduje, aby vzdálenost dvou různých bodů byla nenulová, (3) říká, že nejkratší dráha od $x$ do $z$ nemůže být delší než když navíc požadujeme, že na cestě musíme navštívit bod $y$. Podmínka (2) je poněkud méně uspokojivá (vezměte třeba vzdálenost mezi dvěma místy ve městě pro automobil), ale pro naše potřeby je zcela přijatelná.
 
 \bigskip
 
 {\bf 2.2. Příklady.} 1. {\bf Reálná přímka}, t.j., $\Rbb$ s vzdáleností $d(x,y)=|x-y|$.
 
 \smallskip
 
 2. {\bf Gaussova  rovina}, t.j., množina komplexních čísel $\Cbb$ se vzdáleností $d(x,y)=|x-y|$. Všimněte
 si, že tato formule v $\Cbb$ je méně triviální než $|x-y|$ v $\Rbb$.
 
 \smallskip
 
 3. {\bf $n$-rozměrný euklidovský prostor $\Ebb_n$}: množina
 $$  
 \setof{(x_1,\dots,x_n)}{x_i\in\Rbb}
 $$
  s vzdáleností
 \begin{equation}
 d((x_1,\dots,x_n),(y_1,\dots,y_n))=\sqrt{\sum_{i=1}^n(x_i-y_i)^2}. \tag{$*$}
 \end{equation}
 
 \smallskip
 
 4. Buď $J$ interval. Vezměmme množinu
 $$
 F(J)=\setof{f}{f:J\to\Rbb\ \text{omezená}}
 $$
 opatřenou vzdáleností
 $$
 d(f,g)=\sup\setof{|f(x)-g(x)|}{x\in J}.
 $$
 
 \medskip
 
 {\bf 2.2.1. Víc o $\Ebb_n$.} Euklidovský prostor $\Ebb_n$ (a jeho podmnožiny) bude
 v následujícím hrát zásadní roli. Zasluhuje si komentář.
 
 (a) Čtenář zná z lineární algebry $n$-rozměrný vektorový prostor $V_n$, skalární součin
 $x\cdot y=(x_1,\dots,x_n)\cdot(y_1,\dots,y_n)=\sum_{i=1}^nx_iy_i$, normu $\|x\|=\sqrt{x\cdot x}$, a Cauchy-Schwarzovu nerovnost
 $$
 |x\cdot y|\leq\|x\|\cdot\|y\|.
 $$
 Z té se snadno vyvodí, že $d(x,y)=\|x-y\|$ je vzdálenost na $V_n$ (udělejte to jako jednoduché cvičení).
 Prostor $\Ebb_n$ není nic jiného než $(V_n,d)$ v němž zanedbáme strukturu vektorového prostoru.
 
 (b) Gaussova rovina se geometricky shoduje s euklidovskou rovinou $\Ebb_2$. Podobně jako $V_n$ při srovnání s $\Ebb_n$ 
 má bohatší strukturu.
 
 (c) (Pythagorejská) metrika $(*)$ v $\Ebb_n$ je ve shodě s  euklidovskou geometrií. Práce s ní však může být trochu nepohodlná. Pro naše účely pohodlnější metriky (equivalentní s $(*)$) budou zavedeny dále v  4.3. 

 
 \bigskip
 
 {\bf 2.3. Spojitá a stejnoměrně spojitá zobrazení.} Buďte $(X_1,d_1)$ a $(X_2,d_2)$ metrické prostory. Zobrazení $f:X_1\to X_2$ je {\em spojité} jestliže
 $$
 \forall x\in X_1\ \forall\epsilon>0 \ \exists \delta>0\ \text{takové, že}\ \forall y\in X_1, \ 
 d_1(x,y)<\delta\ \Rightarrow d_2(f(x),f(y))<\epsilon.
 $$
Řekneme, že je {\em stejnoměrně spojité} jestliže
$$
  \forall\epsilon>0 \ \exists \delta>0\ \text{takové, že}\ \forall x\in X_1\  \forall y\in X_1, \ 
 d_1(x,y)<\delta\ \Rightarrow d_2(f(x),f(y))<\epsilon.
 $$
 Zřejmě každé stejnoměrně spojité zobrazení je spojité.
 
 \medskip
 
 {\bf 2.3.1. Pozorování.} (1) {\em Identické zobrazení$\id:(X,d)\to(X.d)$ je spojité.}
 
 (2) {\em Složení $g\circ f:(X_1,d_1)\to(X_3,d_3)$ (stejnoměrně) spojitých zobrazení
  $f:(X_1,d_1)\to(X_2,d_2)$ a  $g:(X_2,d_2)\to(X_3,d_3)$ je (stejnoměrně) spojité.}
  
 

\bigskip

{\bf 2.4. Podprostory.} Buď $(X,d)$ metrický prostor a $Y\sue X$ podmnožina.
Definujeme-li $d_Y(x,y)=d(x,y)$ pro $x,y\in Y$ získáme na $Y$ metriku; tak vytvořený $(Y,d_Y)$
 se nazývá  {\em podprostor} prostoru $(X,d)$.
  
  \medskip
  
  {\bf 2.4.1. Pozorování.} {\em Buď $f:(X_1,d_1)\to(X_2,d_2)$  (stejnoměrně) spojité zobrazení. Buďte $Y_i\sue X_i$ takové podmnožiny, že $f[Y_1]\sue Y_2$. Potom je zobrazení $g:(Y_1,{d_1}_{Y_1})\to(Y_2,{d_2}_{Y_2})$ definované předpisem $g(x)=f(x)$ (stejnoměrně) spojité.}
  
  \bigskip
  
  {\bf 2.5. Úmluvy.} 1. Nebude-li nebezpečí nedorozumění budeme často užívat stejný symbol pro různé metriky. Zejména většinou vynecháme subskript $Y$ u metriky podprostoru $d_Y$.
  
  2. Nebude-li řečeno jinak, bude podmnožina automaticky chápána s metrikou podprostoru. Budeme mluvit o podprostorech jako o příslušných pod\-mno\-žinách, a o podmnožinách jako o příslušných podprostorech. Tak mluvíme o   `` konečném podprostoru'', o ``otevřeném podprostoru'' (viz dále v 3.4) nebo, na druhé straně, o ``kompaktní podmnožině'' (viz sekci  7), atd.. 
  
 \vskip10mm
 
 {\large\bf 3. Několik topologických pojmů.}
  
 \bigskip
 
 
 {\bf 3.1. Konvergence.} Posloupnost $(x_n)_n$ v metrickém prostoru $(X,d)$ {\em konverguje} k $x\in X$ platí-li
 $$
 \forall\epsilon>0\ \exists n_0\ \text{takové že}\ \forall n\geq n_0,\ d(x_n,x)<\epsilon.
 $$
 Mluvíme pak o {\em konvergentní posloupnosti} a $x$ se nazývá její {\em limitou}; píšeme
 $$
 x=\lim_n x_n.
 $$
 
 \medskip
 
 {\bf 3.1.1. Pozorování.} {\em Buď $(x_n)_n$ konvergentní posloupnost s limitou $x$.  Potom každá podposloupnost $(x_{k_n})_n$ této $(x_n)_n$ konverguje, a platí $\lim_nx_{k_n}=x$.}
 
 \medskip

 {\bf 3.1.2. Věta.} {\em Zobrazení $f:(X_1,d_1)\to (X_2,d_2)$ je spojité právě když pro každou konvergentní  $(x_n)_n$ v $(X_1,d_1)$ posloupnost $(f(x_n))_n$ konverguje v $(X_2,d_2)$ a  platí
 $\lim_nf(x_n)=f(\lim_nx_n)$.
 
 Důkaz.} I.  Buď $f$ spojitá a nechť $\lim_nx_n=x$. Pro $\epsilon>0$ zvolme ze spojitosti $\delta>0$ tak aby $d_2(f(y),f(x))<\epsilon$ pro $d_1(x,y)<\delta$. Podle definice konvergence posloupnosti existuje $n_0$ takové, že pro $n\geq n_0$ je $d_1(x_n,x)<\delta.$ Tedy, je-li $n\leq n_0$ máme $d_2(f(x_n),f(x))<\epsilon$ a potom
  $\lim_nf(x_n)=f(\lim_nx_n)$.
  
  \smallskip
  
  II. Nechť $f$ není spojitá. Potom existují $x\in X_1$ a $\epsilon_0>0$
  takové, že pro každé $\delta>0$ existuje $x(\delta)$ takové,že 
  $$
  d_1(x,x(\delta))<\delta \qtq{ale} d_2(f(x),f(x(\delta)))\geq \epsilon_0.
  $$
 Položme $x_n=x(\frac1n)$. Potom $\lim_nx_n=x$ ale $(f(x_n))_n$ nemůže konvergovat k $f(x)$. \sq
  
  \medskip
  
  {\bf Všimněte si,} že tento důkaz je úplně stejný jako důkaz  IV.5.1, jen absolutní hodnoty $|u-v|$ jsou nahrazeny
 vzdálenostmi v daných dvou prostorech. V situaci reálných funkcí jedné reálné proměnné není v tomto ohledu nic specifického.
  
  \bigskip
  
  {\bf 3.2. Okolí.} Pro bod $x$ metrického prostoru $(X,d)$ a $\epsilon>0$ položme
  $$
  \Omega_{(X,d)}(x,\epsilon)=\setof{y}{d(x,y)<\epsilon}
  $$
  (není-li nebezpečí nedorozumnění subskript ``$(X,d)$'' vynecháváme nebo píšeme jen``$X$'').
  
  {\em Okolí} bodu $x$ v $(X,d)$ je kterákoli $U\sue X$ taková, že pro nějaké
   $\epsilon>0$ je $\Omega(x,\epsilon)\sue U$.
   
  \medskip
  
  {\bf 3.3.1. Tvrzení.} 1. {\em Je-li $U$ okolí bodu $x$ a $U\sue V$, je $V$  okolí bodu $x$.}
  
  2. {\em Jsou-li $U$ a $V$ okolí bodu $x$  je průnik $U\cap V$ okolí bodu $x$.
  
  Důkaz.} 1 je triviální. 
  
  2: Jestliže $\Omega(x,\epsilon_1)\sue U$ a $\Omega(x,\epsilon_2)\sue V$
  pak  $\Omega(x,\min(\epsilon_1,\epsilon_2))\sue U\cap V$. \sq
  
  \medskip
  
  {\bf 3.3.2. Tvrzení.} {\em Buď $Y$ podprostor metrického prostoru $(X,d)$. Potom $\Omega_Y(x,\epsilon)=\Omega_X(x,\epsilon)\cap Y$ a $U\sue Y$ je okolí bodu $x\in Y$ právě když pro nějaké okolí $V$ bodu $x$ v $(X,d)$ je $U=V\cap Y$. 
  
  Důkaz} je bezprostřední pozorování.\sq
  
  \bigskip
  
  {\bf 3.4. Otevřené množiny.} Podmnožina $U\sue(X,d)$ je {\em otevřená} je-li okolím každého svého bodu.
  
  \medskip
  
  {\bf 3.4.1. Tvrzení.} {\em Každá $\Omega_X(x,\epsilon)$ je otevřená v $(X,d)$.
  
  Důkaz.} Buď $y\in\Omega_X(x,\epsilon)$. Potom $d(x,y)<\epsilon$. Vezměme $\delta=\epsilon-d(x,y)$. Podle  trojúhelníkové nerovnosti je $\Omega(y,\delta)\sue\Omega(x,\epsilon)$.\sq
  
  \medskip
  
  {\bf 3.4.2. Pozorov\'an\'\i.} {\em Množiny $\ems$ a $X$ jsou otevřené. Jsou-li $U_i$, $i\in J$, otevřené potom $\bigcup_{i\in J}U_i$ je otevřená, and jsou-li $U$ a $V$ otevřené je $U\cap V$ otevřená.
  
  Důkaz.} První dvě tvrzení jsou zřejmá a  třetí bezprostředně plyne z 2.3.1.\sq
  
  \medskip
  
  {\bf 3.4.3. Tvrzení.} {\em Buď $Y$ podprostor metrického prostoru $(X,d)$. Potom je $U$ otevřená v $Y$ právě kdy\v z existuje $V$ otevřená v $X$ taková, že $U=V\cap Y$.
  
  Důkaz.} Pro  $V$ otevřenou v $X$ je $U\cap Y$ otevřená v $Y$ podle 3.3.2. Na druhé straně, je-li $U$ otevřená v $Y$ zvolme pro každé $x\in U$ okolí $\Omega_Y(x,\epsilon_x)\sue U$ a položme $V=\bigcup_{x\in U}\Omega_X(x,\epsilon_x)$. \sq
  
  \bigskip
  
  {\bf 3.5. Uzavřené množiny.} Podmnožina $A\sue(X,d)$ je {\em uzavřená} v $(X,d)$ je-li pro každou posloupnost $(x_n)_n\sue A$ konvergentní v $X$ limita $\lim_nx_n$ v množině $A$.
  
  \medskip
  
  {\bf 3.5.1. Tvrzení.} {\em Podmnožina $A\sue(X,d)$ je {\em uzavřená} v $(X,d)$ právě když její doplněk $X\smin A$ je otevřený.
  
  Důkaz.} I. Nechť $X\smin A$ není otevřená. Potom  existuje bod $x\in X\smin A$ takový, že pro každé  $n$ je
  $\Omega(x,\frac1n)\nsubseteq X\smin A$, to jest, $\Omega(x,\frac1n)\cap A\neq\ems$. Zvolme $x_n\in
   \Omega(x,\frac1n)\cap A$. Potom $(x_n)_n\sue A$ a ta posloupnost konverguje k $x\notin A$ a tedy $A$ není uzavřená.
   
   \smallskip
   
  
   II. Nechť je $X\smin A$ otevřená a $(x_n)_n\sue A$  konverguje k $x\in X\smin A$. Potom pro nějaké $\epsilon>0$ je
   $\Omega(x,\epsilon)\sue X\smin A$ a tdy pro dost velké $n$, $x_n\in \Omega(x,\epsilon)\sue X\smin A$, spor. \sq
   
   \medskip 
   
   Z 3.5.1, 3.4.2 a DeMorganových formulí okamžitě získáváme
   
   \medskip
   
  {\bf 3.5.2. Důsledek.} {\em Množiny $\ems$ a $X$ jsou uzavřené. jsou-li $A_i$, $i\in J$, uzavřené je i $\bigcap_{i\in J}A_i$ uzavřená, a jsou-li $A$ and $B$ uzavřené, je $A\cup B$ uzavřená.}
  
  \medskip
  
  {\bf 3.5.3. Důsledek.} {\em Buď $Y$ podprostor metrického prostoru $(X,d)$. Potom je $A$ uzavřená v $Y$ právě když existuje $B$ uzavřená v $X$ taková, že $A=B\cap Y$.}
  
  
  
  \bigskip
  
  {\bf 3.6. Vzdálenost bodu od množiny. Uzávěr.} Buď $x$ bod a $A\sue X$ podmnožina metrického prostoru $(X,d)$.  Definujme vzdálenost $x$ od $A$ jako
  $$
  d(x,A)=\inf\setof{d(x,a)}{a\in A}.
  $$
  {\em Uzávěr} množiny $A$ je
  $$
  \ol A=\setof{x}{d(x,A)=0}.
  $$
  
  \medskip
  
  {\bf 3.6.1. Tvrzení.} (1) $\ol\ems=\ems$.
  
  \hskip40mm (2) $A\sue \ol A$,
  
  \hskip40mm (3) $A\sue B\ \ \Rightarrow\ \ \ol A\sue\ol B$,
  
  \hskip40mm (4) $\ol{A\cup B}=\ol A\cup\ol B$, {\em a}
  
  \hskip40mm (5) $\ol{\ol A}= \ol A$.
  
  {\em Důkaz.} (1): $d(x,\ems)=+\infty$.
  
  (2) a (3) jsou triviální.
  
  (4): Podle (3) máme $\ol{A\cup B}\supe \ol A\cup\ol B$. Nechť nyní $x\in\ol{A\cup B}$ ale ne $x\in\ol A$. Potom $\alpha=d(x,A)>0$ a tedy všechny body $y\in A\cup B$ takové, že $d(x,y)<\alpha$ jsou v  $B$; tedy je $x\in \ol B$.
  
  (5): Buď $d(x,\ol A)=0$. Zvolme $\epsilon>0$. Potom  máme $z\in \ol A$ takové, že $d(x,z)<\frac{\epsilon}{2}$
  a pro toto $z$ můžeme zvolit $y\in A$ takové, že $d(z,y)<\frac{\epsilon}{2}$. Tedy podle trojúhelníkové nerovnosti
  $d(x,y)<\frac{\epsilon}{2}+\frac{\epsilon}{2}=\epsilon$ a vidíme, že $x\in\ol A$. \sq
  
  \medskip
  
  {\bf 3.6.2. Tvrzení.} {\em $\ol A$ je množina všech limit konvergentních posloupností  $(x_n)_n\sue A$.
  
  Důkaz.} Limita konvergentní $(x_n)_n\sue A$ je zřejmě v $\ol A$.
  
  Buď nyní $x\in \ol A$. Je-li $x\in A$ je to limita konstantní posloupnosti $x,x,x,\dots$. Je-li $x\in\ol A\smin A$ existuje pro každé $n$ nějaký bod $x_n\in A$ takový, že $d(x,x_n)<\frac1n$. Zřejmě $x=\lim_nx_n$.\sq
  
  \medskip
  
  {\bf 3.6.3. Tvrzení.} {\em $\ol A$ je uzavřená, a je to nejmenší uzavřená množina obsahující  $A$. Tedy,
  $$
  \ol A=\bigcap\setof{B}{A\sue B,\ B\ \text{uzavřená}}.
  $$
  
  Důkaz.} Nechť $(x_n)_n\sue\ol A$ konverguje k $x$. Pro každé $n$ zvolme $y_n\in A$ tak aby $d(x_n,y_n)<\frac1n$. Potom $\lim_ny_n=x$ a $x$ je v $\ol A$ podle 3.5.1.
  
  Nyní buď $B$ uzavřená a buď $A\sue B$. Je-li $x\in \ol A$ můžeme podle 3.5.1 zvolit konvergentní posloupnost $(x_n)_n$ v $A$, a tedy v $B$, takovou, že $\lim x_n=x$. Máme tedy $x\in B$. \sq
  
  \medskip
  
  {\bf 3.6.4. Důsledek.} {\em Buď $Y$ podprostor metrického prostoru $(X,d)$. Potom je uzávěr $A$ v $Y$ roven $\ol A\cap Y$ (kde $\ol A$ je uzávěr v $X$).}
  
  
  \bigskip
  
  {\bf 3.7. Věta.} {\em Buďte $(X_1,d_1), (X_2.d_2)$ metrické prostory a $f:X_1\to X_2$ 
 zobrazení. Potom jsou následující tvrzení ekvivalentní.
  \begin{enumerate}
  \item $f$ je spojité.
  \item Pro každý $x\in X_1$ a každé okolí $V$ bodu $f(x)$ existuje okolí
  $U$ bodu $x$ takové, že $f[U]\sue V$.
  \item Pro každou otevřenou $U$ v $X_2$ je vzor $f^{-1}[U]$ otevřený v  $X_1$.
  \item Pro každou uzavřenou $A$ v $X_2$ je vzor $f^{-1}[A]$ uzavřený v $X_1$.
  \item Pro každou $A\sue X_1$ je  $f[\ol A]\sue \ol{f[A]}.$
  \end{enumerate}
  
  Důkaz.} (1)$\Rightarrow$(2): Existuje $\epsilon>0$ takové, že $\Omega(f(x),\epsilon)\sue V$. Vezměme $\delta$ z definice spojitosti a položme $U=\Omega(x,\delta)$. Potom je $f[U]\sue
  \Omega(f(x),\epsilon)\sue V$.
  
  (2)$\Rightarrow$(3): Buď $U$ otevřená  a $x\in f^{-1}[U]$. Tedy je $f(x)\in U$ a $U$ je okolí bodu $f(x)$. Existuje okolí $V$ bodu $x$ takové, že $f[V]\sue U$. Následkem toho je 
  $x\in V\sue f^{-1}[U]$ a $f^{-1}[U]$ je okolí $x$.
$f^{-1}[U]$ je tedy okolí každého svého bodu.
  
  (3)$\Leftrightarrow$(4) podle 3.5.1 protože vzory zachovávají doplňky.
  
  (4)$\Rightarrow$(5): Máme $A\sue f^{-1}[f[A]]\sue f^{-1}[\ol{f[A]}]$. Podle (4) je 
  $f^{-1}[\ol{f[A]}]$ uzavřená a tedy podle 3.5.3 je $\ol A\sue f^{-1}[\ol{f[A]}]$ a konečně
  $f[\ol A]\sue \ol{f[A]}$.
  
  (5)$\Rightarrow$(1): Buď $\epsilon>0$. Položme $B=X_2\smin\Omega(f(x),\epsilon)$ a $A=f^{-1}[B]$.
  Potom $f[\ol A]\sue \ol {f[f^{-1}[B]]}\sue
  \ol {B}$. Tedy $x\notin\ol A$ (vzdálenost $d(f(x),B)$ je nejméně $\epsilon$) a tedy existuje $\delta>0$ takové, že $\Omega(x,\delta)\cap A=\ems$ a snadno vidíme, že $f[\Omega(x,\delta)]\sue\Omega(f(x),\epsilon)$.\sq
  
  \bigskip
  
  {\bf 3.8. Homeomorfismus. Topologické pojmy.} Spojité zobrazení
  $f:(X,d)\to (Y,d')$ se nazývá {\em homeomorfismus} existuje-li spojité $g:(Y,d')\to (X,d)$ takové, že $f\circ g=\id_Y$ a $g\circ f=\id _X$. Existuje-li homeomorfismus $f:(X,d)\to (Y,d')$ říkáme, že prostory $(X,d)$ a $(Y,d')$  jsou {\em homeomorfní}.
  
  Vlastnost nebo definice je {\em topologická} je-li zachovávána homeomorfismy. Máme tedy následující topologické vlastnosti a pojmy:
  \begin{itemize}
  \item konvergenci (viz 3.1.2),
  \item otevřenost (viz 3.7),
  \item uzavřenost (viz 3.7).
  \item uzávěr  (třebaže $d(x,A)$ topologická není; viz ale 3.6.3),
  \item okolí (třebaže $\Omega(x,\epsilon)$ topologické není; uvědomte si však , že $A$ je okolí $x$ existuje-li  otevřená $U$ taková, že $x\in U\sue A$),
  \item nebo spojitost sama.
  \end{itemize}
  Na druhé straně, stejnoměrná spojitost topologická vlastnost není.
  
   
  \bigskip
  
  {\bf 3.9. Isometrie} Zobrazení na 
  $f:(X,d)\to (Y,d')$ se nazývá  {\em isometrie} je-li $d'(f(x),f(y))=d(x,y)$ pro všechna $x,y\in X$. Potom je triviálně
  \begin{itemize}
  \item $f$ vzájemně jdnoznačné a spojité, a
  \item jeho inverse je také isometrie; tedy je $f$  homeomorfismus.
  \end{itemize}
  Existuje li mezi prostory  $(X,d)$ a $(Y,d')$ isometrie, říkame o nich, že jsou isometrické. Isometrie samozřejmě zachovává
   topologické pojmy, ale mnohem víc, všechno co je definováno přes vzdálenost.
  
  \vskip10mm
 
 {\large\bf 4. Ekvivalentní a silně ekvivalentní metriky.}
  
 \bigskip
 
 
 {\bf 4.1.} Řekneme, že metriky $d_1,d_2$ na téže množině jsou  {\em ekvivalentní} je-li $\id_X:(X,d_1)\to(X.d_2)$ homeomorfismus. Nahradíme-li tedy metriku nějakou ekvivalentní získáme prostor v němž jsou všechny topologické záležitosti z původního zachovány.
 
 \bigskip
 
 {\bf 4.2.} Mnohem silnější je silná ekvivalence. Řekneme,že $d_1,d_2$ na  téže množině jsou {\em silně ekvivalentní} existují-li kladné konstanty $\alpha$ a $\beta$ takové, že pro všechna  $x,y\in X$ platí
 $$
 \alpha\cdot d_1(x,y)\leq d_2(x,y)\leq\beta \cdot d_1(x,y)
 $$
 (relace je samozřejmě symetrická: vezměte $\frac1{\alpha}$ a $\frac1{\beta}$).
 
 Všimněte si, že 
 \begin{itemize}
 \item[] {\em nahrazení metriky silně ekvivalentní metrikou nezachovává jen topologické vlastnosti, ale také např. stejnoměrnou spojitost.}
 \end{itemize}
 
 \bigskip
 
 {\bf 4.3.} Silná ekvivalence nám pomůže k snadnější práci v euklidovských prostorech
 $
 \setof{(x_1,\dots,x_n)}{x_i\in\Rbb}
 $
 kde jsme zatím měli vzdálenost
 $$
 d((x_1,\dots,x_n),(y_1,\dots,y_n))=\sqrt{\sum_{i=1}^n(x_i-y_i)^2}. 
 $$
Položme
 $$ 
 \begin{aligned}
 &\lambda((x_1,\dots,x_n),(y_1,\dots,y_n))=\sum_{i=1}^n|x_i-y_i|, \ \ \text{a}\\
 &\sigma((x_1,\dots,x_n),(y_1,\dots,y_n))=\max_i|x_i-y_i|.
 \end{aligned}
 $$
 
 \medskip
 
 {\bf 4.3.1. Tvrzení.} {\em $d,\lambda$ a $\sigma$ jsou silně ekvivalentní metriky na $\Ebb_n$.
 
 Důkaz.} Že jsou $\lambda$ a $\sigma$ metriky je vidět snadno.

 Dále máme
 $$
 \lambda((x_i)_i,(y_i)_i)=\sum_{i=1}^n|x_i-y_i|\leq n\sigma((x_j)_j,(y_j)_j)
 $$
 jelikož pro každé $i$ je $|x_i-y_i|\leq \sigma((x_j)_j,(y_j)_j)$, a z téhož důvodu
 $$
 d((x_i)_i,(y_i)_i)=\sqrt{\sum_{i=1}^n(x_i-y_i)^2}\leq \sqrt n\sigma((x_j)_i,(y_j)_j).
 $$
 Na druhé straně zřejmě máme
 $$
 \sigma((x_i)_i,(y_i)_i)\leq \lambda((x_i)_i,(y_i)_i)\qtq{a}\sigma((x_i)_i,(y_i)_i)\leq d((x_i)_i,(y_i)_i).
 $$ \sq
 
 \medskip
 
 V dalším budeme obvykle pracovat s euklidovským prostorem jako s $(\Ebb_n,\sigma)$.
 
 \vskip10mm
 
 {\large\bf 5. Součiny (produkty).}
  
 \bigskip
 
 
 {\bf 5.1.} Budte $(X_i,d_i)$, $i=1,\dots,n$, metrické prostory. Na kartézském součinu
 $$
 \prod_{i=1}^nX_i
 $$
 definujeme metriku
 $$
 d((x_1,\dots,x_n),(y_1,\dots,y_n))=\max_id_i(x_i,y_i).
 $$
Získaný metrický prostor bude značen $\prod_{i=1}^n (X_i,d_i)$.
 
 \medskip
 
 {\bf 5.1.1. Značení.} Budeme též psát
  $$
 (X_1,d_1)\times (X_2,d_2) \qtq{nebo} (X_1,d_1)\times (X_2,d_2)\times (X_3,d_3)
 $$
  místo $\prod_{i=1}^2 (X_i,d_i)$ nebo $\prod_{i=1}^3 (X_i,d_i)$, a někdy též
 $$
 (X_1,d_1)\times \cdots\times  (X_n,d_n)
 $$
 místo obecného
 $\prod_{i=1}^n (X_i,d_i)$.
 
 Dále, je-li $(X_i,d_i)=(X,d)$ pro všechna $i$ píšeme
 $$
  \prod_{i=1}^n (X_i,d_i)=(X,d)^n.
  $$
 
 \medskip
 
 {\bf 5.1.2. Poznámky.} 1. Je tedy $(\Ebb_n,\sigma)$ součin $\overbrace{\Rbb\times\cdots\times\Rbb}^{n\ \text{krát}}=\Rbb^n$.
 
 \smallskip
 
 2. Pro naše potřeby jsme mohli na součinu definovat metriku také
 $$
 d((x_i)_i,(y_i)_i)=\sqrt{\sum_{i=1}^nd_i(x_i,y_i)^2}\qtq{nebo}d((x_i)_i,(y_i)_i)=\sum_{i=1}^nd_i(x_i,y_i),
 $$
 s  $d$ nahoře se však lépe pracuje.
 
 \bigskip
 
 {\bf 5.2. Lemma.} {\em Posloupnost
 $$
 (x_1^1,\dots,x_n^1),(x_1^2,\dots,x_n^2),\dots,(x_1^k,\dots,x_n^k),\dots
 $$
konverguje k $(x_1,\dots,x_n)$ v $\prod (X_i,d_i)$ právě když každá z posloupností $(x_i^k)_k$ konverguje k $x_i$ v $(X_i,d_i)$.}
 
 (Pozor: superskripty $k$ jsou indexy, ne mocniny.)
 
 {\em Důkaz.} $\Rightarrow$ okamžitě plyne z toho, že $d_i(u_i,v_i)\leq d((u_j)_j,(v_j)_j)$.
 
 \smallskip
 
 $\Leftarrow$: Nechť každá posloupnost $(x_i^k)_k$ konverguje k $x_i$. Pro $\epsilon>0$ a $i$ máme 
 $k_i$ taková, že pro $k\geq k_i$ je $d_i(x_i^k,x_i)<\epsilon$. Potom pro $k\geq\max_ik_i$ máme
 $$
 d((x_1^k,\dots,x_n^k),(x_1,\dots,x_n))<\epsilon.
 $$\sq
 
 \bigskip
 
 {\bf 5.3. Věta.} 1. {\em Projekce $p_j=((x_i)_i\mapsto x_j):\prod_{i=1}^n (X_i,d_i)\to (X_j,d_j)$ jsou spojitá zobrazení.
 
 {\em 2.} Buďte $f_:(Y,d')\to (X_j,d_j)$ libovolná spojitá zobrazení. Potom jednoznačně určené zobrazení
 $f:(Y,d')\to\prod_{i=1}^n (X_i,d_i)$ splňující $p_j\circ f=f_j$, totiž zobrazení definované předpisem
 $f(y)=(f_1(y),\dots,f_n(y))$, je spojité.
 
 Důkaz.} 1 plyne bezprostředně z toho, že $d_j(x_j,y_j)\leq d((x_i)_i,(y_i)_i)$.
 
 2: Plyne z  3.1.2 a 5.2. Je-li $\lim_k y_k=y$ v $(Y,d')$ potom je $\lim_k f_j(y_k)=f_j(y)$ v $(X_j,d_j)$ pro všechna $j$ a tedy $(f(y_k))_k$, t.j.,
 $$
 (f_1(y_1),\ \dots,f_n(y_1)),\ (f_1(y_2),\dots,f_n(y_2)),\ \dots,\ (f_1(y_k),\dots,f_n(y_k)),\ \dots
 $$
  konverguje k  $(f_1(y),\dots,f_n(y))$. \sq
  
  \bigskip
 
 {\bf 5.4. Pozorování.} {\em Zřejmě je  $\prod_{i=1}^{n+1}(X_i,d_i)$
 isometrický (viz 3.9) s $\prod_{i=1}^{n}(X_i,d_i)\times (X_{n+1},d_{n+1})$. Následkem toho obvykle stačí dokazovat tvrzení o konečných součinech pouze pro součiny dvou.}
  
  
  
  \vskip10mm
 
 {\large\bf 6. Cauchyovské posloupnosti.  Úplnost.}
  
 \bigskip
 
 
 {\bf 6.1.} Posloupnost $(x_n)_n$ v metrickém prostoru $(X,d)$ je {\em Cauchyovská} jestliže
 $$
 \forall\epsilon>0\ \exists  n_0 \ \text{takové, že}\ \ m,n \geq n_0\ \Rightarrow\ d(x_m,x_n)<\epsilon.
$$
  
\medskip

{\bf 6.1.1. Pozorování.} {\em Každá konvergentní posloupnost je Cauchyovská.}

(Stejně jako v $\Rbb$: je-li $d(x_n,x)<\epsilon$ pro $n\geq n_0$ je pro $m,n\geq n_0$,
 $$
 d(x_n,x_m) \leq d(x_n,x)+d(x,x_m)<2\epsilon.)
 $$
  
 \bigskip
 
 {\bf 6.2. Tvrzení.} {\em Nechť má Cauchyovská posloupnost konvergentní podposloupnost. Potom konverguje
 (k limitě té podposloupnosti).
 
 Důkaz.} Nechť je $(x_n)_n$ Cauchyovská a nechť $\lim_n x_{k_n}=x$. Buď $d(x_m,x_n)<\epsilon$ pro $m,n \geq n_1$ a $d(x_{k_n},x)\leq\epsilon$ pro $n\geq n_2$. Položíme-li $n_0=\max(n_1,n_2)$ máme pro $n\geq n_0$ (protože $k_n\geq n$)
 $$
 d(x_n,x)\leq  d(x_n,x_{k_n})+d(x_{k_n},x)<2\epsilon.
 $$\sq
 
 \bigskip
 
 {\bf 6.3.} Metrický prostor  $(X,d)$ je {\em úplný} jestliže v něm každá Cauchyovská posloupnost $(X,d)$ konverguje.
 
 \medskip
 
 {\bf 6.3.1.} Tedy např. podle Bolzano-Cauchyovy věty (II.3.4)je reálná přímka $\Rbb$ se standardní metrikou úplná.
 
 \bigskip
 
 {\bf 6.4. Tvrzení.} {\em Podprostor úplného prostoru je úplný právě když je uzavřený.
 
 Důkaz.} I. Buď $Y\sue(X,d)$ uzavřený. Buď $(y_n)_n$ Cauchyovská v $Y$. Potom je Cauchyovská a tedy konvergentní v  $X$, a kvůli uzavřenosti je limita v $Y$.
 
 II. Nechť $Y$ není uzavřený. Potom existuje posloupnost $(y_n)_n$ v $Y$ konvergentní v $X$ taková, že
 $\lim_ny_n\notin Y$. Potom je $(y_n)_n$ Cauchyovská v $X$, a jelikož je vzdálenost stejná, též v $Y$. Ale v $Y$ nekonverguje. \sq
 
 

\bigskip

 {\bf 6.5. Lemma.} {\em Posloupnost
 $$
 (x_1^1,\dots,x_n^1),(x_1^2,\dots,x_n^2),\dots,(x_1^k,\dots,x_n^k),\dots
 $$
 je Cauchyovská v $\prod_{i=1}^n(X_i,d_i)$ právě když každá z posloupností $(x_i^k)_k$ je Cauchyovská v $(X_i,d_i)$.

 Důkaz. } $\Rightarrow$ plyne bezproztředně z toho, že $d_i(u_i,v_i)\leq d((u_j)_j,(v_j)_j)$.
 
 \smallskip
 
 $\Leftarrow$: Nechť je každá $(x_i^k)_k$ Cauchyovská. Pro $\epsilon>0$ a $i$ zvolme 
 $k_i$ tak aby pro $k,l\geq k_i$ bylo $d_i(x_i^k,x_i^l)<\epsilon$. Potom pro $k,l\geq\max_ik_i$ máme
 $$
 d((x_1^k,\dots,x_n^k),(x_1^l,\dots,x_n^l))<\epsilon.
 $$\sq
 
 \bigskip
 
 Kombinací 5.2 a 6.5 (a samozřejmě 6.3.1) dostáváme hned
 
 \medskip
 
 {\bf 6.6. Tvrzení.} {\em Součin úplných prostorů je úplný. Speciálně, $\Ebb_n$ je úplný.}
 
 
 \bigskip
 
 Z 6.6 a  6.4 bezprostředně vyvozujeme
 
 \medskip
 
 {\bf 6.7. Důsledek.} {\em Podprostor $Y$ euklidovského prostoru $\Ebb_n$ je úplný právě když je tam uzavřený.}
 
 \bigskip
 
 {\bf 6.8. Poznámka.} Cauchyova vlastnost ani úplnost nejsou topologické vlastnosti. Uvažujme $\Rbb$ a libovolný omezený neprázdný otevřený interval   $J$ v $\Rbb$. Jsou homeomorfní (je-li např. $J=(-\frac{\pi}{2},+\frac{\pi}{2})$ máme vzájemně inversní $\tan:J\to\Rbb$ a $\text{arctg}:\Rbb\to J$). Při tom $\Rbb$ je úplný a  $J$ není.
 
Snadno se ale nahlédne, že obě vlastnosti se zachovají nahradíme-li metriku metrikou silně ekvivalentní. To se zvláště týká metrik v $\Ebb_n$ zmíněných v sekci 4.
 
 
  \vskip10mm
 
 {\large\bf 7. Kompaktní metrické prostory.}
  
 \bigskip
 
 {\bf 7.1.} Řekneme, že metrický prostor $(X,d)$ je  {\em kompaktní} obsahuje-li v něm každá posloupnost konvergentní podposloupnost.
 
 \medskip
 
 {\bf 7.1.1. Poznámka.} Tedy jsou kompaktní intervaly (uzavřené omezené intervaly $\langle a,b\rangle$)
 kompaktní v této definici, a mezi všemi typy intervalů jsou jediné takové.
 
 \bigskip
 
 {\bf 7.2. Tvrzení.} {\em Poprostor kompaktního prostoru je kompaktní právě když je uzavřený.
 
 Důkaz.} I. Buď $Y$ uzavřený podprostor kompaktního $X$ a buď $(y_n)_n$ posloupnost v $Y$. Jako posloupnost v $X$ má limitu, a z uzavřenosti je tato limita v $Y$.
 
 II. Nechť $Y$ není uzavřená. Potom existuje posloupnost  $(y_n)_n$ in $Y$ konvergentní v $X$ taková,že
 $y=\lim_ny_n\notin Y$. Potom $(y_n)_n$ nemůže mít podposloupnost konvergentní v $Y$ protože každá její podposloupnost konverguje k $y$. \sq
 
 \bigskip
 
 {\bf 7.3. Tvrzení.} {\em Buď $(X,d)$ libovolný  a nechť je podprostor $Y$ prostoru $X$ kompaktní. Potom je $Y$ uzavřený v $(X,d)$.
  
 Důkaz.}  Nechť $(y_n)_n$  posloupnost v $Y$ konverguje v $X$ k limitě $y$. Potom každá podposloupnost 
  $(y_n)_n$ konverguje k $y$ a tedy je $y\in Y$. \sq
 
 \bigskip
 
 {\bf 7.4.} Metrický prostor  $(X,d)$ je {\em omezený} jestliže pro nějaké $K$ platí, že
 $$
 \forall x,y\in X,\quad d(x,y)< K.
 $$
 
 \medskip
 
 {\bf 7.4.1. Tvrzení.} {\em Každý kompaktní prostor je omezený.

Důkaz.} Zvolme $x_1$ libovoln\v e a $x_n$ 
 tak aby
 $d(x_1,x_n)>n$. Posloupnost $(x_n)_n$ nemá konvergentní podposloupnost:
 kdyby $x$ byla limita takové podposloupnosti bylo by pro dost velk\'e $n$ nekonečně mnoho členů této podposloupnosti blíže k $x_1$ než $d(x_1,x_n)+1$, spor.\sq
 
 \bigskip
 
 {\bf 7.5. Věta.} {\em Součin konečně mnoha kompaktních prostorů je kompaktní.
 
 Důkaz.} Podle 5.4 to stačí dokázat pro součin dvou prostorů.
 
 Buďte $(X,d_1)$, $(Y, d_2)$ kompaktní a buď $((x_n,y_n))_n$ posloupnost v $X\times Y$. Zvolme konvergentní podposloupnost $(x_{k_n})_n$
 posloupnosti $(x_n)_n$ a konvergentní podposloupnost $(y_{k_{l_n}})_n$
posloupnosti$(y_{k_n})_n$. Potom je podle 5.2.
 $$
 ((x_{k_{l_n}},y_{k_{l_n}}))_n
 $$
konvergentní podposloupnost posloupnosti $((x_n,y_n))_n$. \sq
 
 \bigskip
 
 {\bf 7.6. Věta.} {\em Podprostor euklidovského prostoru $\Ebb_n$ je kompaktní právě když je omezený a uzavřený.
 
 Důkaz.} I. Kompaktní podprostor libovolného mrtrického prostoru je uzavřený podle 7.3 a omezený podle 7.4.1.
 
 
II. Buď nyní $Y\sue \Ebb_n$ omezená a uzavřená. Jelikož je omezená, je pro dostatečně velký kompaktní interval
$$
Y\sue J^n\sue\Ebb_n.
$$
Podle 7.5 je $J^n$ kompaktní a jelikož je $Y$ uzavřený v $\Ebb_n$ je též uzavřený v $J^n$a tedy kompaktní 
 podle 7.2. \sq 
 
 \bigskip
 
 {\bf 7.7. Tvrzení.} {\em Každý kompaktní prostor je úplný.
 
 Důkaz.} Cauchovská posloupnost má podle kompaktnosti konvergentní podposloupnost a tedy konverguje podle 6.2.\sq
 
 \bigskip
 
 {\bf 7.8. Tvrzení.} {\em  Buď  $f:(X,d)\to (Y,d')$ spojité zobrazení a buď $A\sue X$ kompaktní. Potom je $f[A]$ kompaktní.
 
 Důkaz.} Buď $(y_n)_n$  posloupnost v $f[A]$. Zvolme $x_n\in A$ tak aby $y_n=f(x_n)$. Buď $(x_{k_n})_n$ konvergentní podposloupnost
poslupnosti $(x_n)_n$. Potom je $(y_{k_n})_n=(f(x_{k_n}))_n$ podle 3.1.2
konvergentní podposloupnost $(x_{n})_n$. \sq
 
 \bigskip
 
 {\bf 7.9. Tvrzení.} {\em Buď $(X,d)$ kompaktní. Potom každá spojitá funkce $f:(X,d)\to\Rbb$ nabývá maxima i minima. 
 
 Důkaz.} Podle 7.8 je $Y=f[X]\sue\Rbb$ kompaktní. 
  Je tedy omezená podle 7.4.1 a musí mít supremum  $M$ a infimum $m$. Zřejmě máme $d(m,Y)=d(M,Y)=0$ a jelikož $Y$ je uzavřená, $m,M\in Y$. \sq

\medskip


{\bf 7.9.1. Důsledek.} {\em Nechť jsou všechny hodnoty spojité funkce na kompaktním prostoru kladné. Potom existuje $c> 0$ takové, že všechny hodnoty $f(x)$ jsou větší než $c$.}

 
 \bigskip
 
 Již víme, že spojitá $f$ je charakterisována tím, že všechny {\em vzory} uzavřených množin jsou uzavřené. Nyní podle  7.2 a 7.8
 vidíme, že je-li definiční obor kompaktní platí též, že {\em obrazy} uzavřených podmnožin jsou uzavřené. Z toho plyne  (m.j.) následující.
 
 \medskip
 
 {\bf 7.10. Věta.} {\em Je-li $(X,d)$ kompaktní a je-li
 $f:(X,d)\to (Y,d')$ vzájemně jednoznačné spojité zobrazení, je to homeomorfismus.
 
 Obecněji, nechť
 $f:(X,d)\to (Y,d')$ je spojité zobrazení na, nechť $g:(X,d)\to (Z,d'')$ je spojité a nechť $h:(Y,d')\to (Z,d'')$ je takové, že $h\circ f=g$. Potom je $h$ spojité.
 
 Důkaz.} Dokážeme druhé tvrzení, první z něj plyne: stačí položit $g=\id_Y$.
 
 Buď $B$ uzavřená v $Z$. Potom je $A=g^{-1}[B]$ uzavřená a tedy kompaktní v $X$ a tedy je $f[A]$ kompaktní, a proto uzavřená v $Y$.
 Jelikož je $f$ zobrazení na, máme $f[f^{-1}[C]]=C$ pro každé $C$. Proto je
 $$
 h^{-1}[B]=f[f^{-1}[ h^{-1}[B]]]=f[(h\circ f)^{-1}[B]]=
 f[g^{-1}[B]]=f[A]
 $$
uzavřená. \sq
 
 
  \bigskip
 
 {\bf 7.11. Věta.} {\em Buď $(X,d)$ kompaktní. Potom je zobrazení $f:(X,d)\to (Y,d')$ spojité právě když je stejnoměrně spojité.}
 
 \smallskip
 
 {\bf Poznámka.} Podobně jako v 3.1.2; důkaz odpovídajícího tvrzení o reálných funkcích jedné reálné proměnné můžeme opakovat prakticky doslova.
 
 \smallskip
 
{\em  Důkaz.} Nechť $f$ není stejnoměrně spojitá. Ukážeme, že není ani spojitá.
 
Jelikož formule pro stejnoměrnou spojitost neplatí, můžeme najít $\epsilon_0>0$ takové, že pro každé $\delta>0$ máme $x(\delta), y(\delta)$ takové, že $d(x(\delta),y(\delta))<\delta$ zatím co
 $d'(f(x(\delta)),f(y(\delta)))\geq \epsilon_0$. Položme $x_n=x(\frac{1}{n})$ a $y_n=y(\frac{1}{n})$. Zvolme konvergentní podposloupnosti $(\wt x_{n})_n$, $(\wt y_{n})_n$  (nejprve konvergentní podposloupnost $(x_{k_n})_n$ posloupnosti $(x_n)_n$, potom konvergentní podposloupnost $(y_{k_{l_n}})_n$ posloupnosti$(y_{n_k})_k$ a konečně klademe  $\wt x_n=x_{k_{l_n}}$ a 
 $\wt y_n=y_{k_{l_n}}$). Potom $d(\wt x_n,\wt y_n)<\frac1{n}$ a tedy $\lim \wt x_n=\lim \wt y_n$. Kdyby $f$ bylo
spojit\'e, bylo by $\lim f(\wt x_n)=\lim f(\wt y_n)$ -- spor.


\newpage

\centerline{\Large\bf XIV. Parciální derivace a totální diferenciál.} 
 
 \bigskip
 
 \hskip11mm{\Large\bf Řetězové pravidlo} 
 
 \vskip10mm
 
  
 {\large\bf 1. Úmluvy.}
 
 \bigskip
 
{\bf 1.1.}  
  Budeme pracovat s reálnými funkcemi  několika reálných proměnných, tedy se zobrazeními $f:D\to \Rbb$ jejichž definiční obor $D$ je podmnožina $\Ebb_n$. Když budeme derivovat bude  $D$ typicky otevřená. Někdy budeme mít i uzavřené definiční obory, obvykle uzávěry otevřených množin s dosti názornými hranicemi.
  
  \smallskip
  
  Víme již (viz XIII.1) že chování takových funkcí nemůže být redukováno na chování funkcí jedné proměnné  získaných fixováním všech proměnných až na jednu. To však u některých úloh dělat budeme  (zejména v definici parciálních derivací v příští sekci).


\bigskip

{\bf 1.2. Úmluva.} Pro zjednodušení značení budeme často užívat tučných písmen pro body v euklidovském prostoru $\Ebb_n$ (to jest, pro  $n$-tice reálných čísel, nebo reálné aritmetické vektory). Budeme např. psát
$$
\ve{x}\ \ \text{místo}\ (x_1,\dots,x_n)   \qtq{nebo} \ve{A}\ \ \text{místo}\ (A_1,\dots,A_n).
$$
Též budeme psát 
$$
\ve{o}\qtq{místo} (0,0,\dots,0).
$$
 
Ve vzácných případech užití subskriptů u tučných písmen, jako v $\ve{a}_1,\ve{a}_2,\dots$,  půjde vždy  o několik bodů,  nikdy o souřadnice jednoho $\ve{a}$.

\smallskip

Skalární sočin vektorů $\ve{x}$, $\ve{y}$, totiž
$\sum_{j=1}^nx_jy_j$, budeme označovat
$$
 \ve{x}\ve{y}.
$$
 
\bigskip

{\bf 1.3. Rozšíření úmluvy.}  ``Tučnou'' úmluvu budeme používat též pro {\em vektorové funkce}, tedy
$$
\ve{f}=(f_1,\dots,f_m):D\to \Ebb_m, \quad f_j:D\to\Rbb.
$$  
Všimněte si, že tady problémy se spojitostí nejsou: $\ve{f}$ je spojitá právě když jsou všechny $f_i$ 
spojité (viz XIII.5.3). 

\bigskip

{\bf 1.4. Skládání.} Vektorové funkce $\ve{f}:D\to \Ebb_m$, 
$D\sue \Ebb_n$, a  $\ve{g}:D'\to \Ebb_k$, $D'\sue \Ebb_m$ můžeme skládat
jestliže $\ve{f}[D]\sue D'$,
a píšeme 
$$
\ve{g}\circ\ve{f}:D\to\Ebb_k, \qtq{(není-li nebezpečí nedorozumnění pouze} \ve{g}\ve{f}:D\to\Ebb_k).
$$
Podobně jako u reálných funkcí nebudeme pedanticky přejmenovávat  $\ve{f}$ při restrikci na $D\to D'$. 

\vskip10mm
 
  
 {\large\bf 2. Parciální derivace}
 
 \bigskip
 


{\bf 2.1.} Buď $f:D\to\Rbb$ reálná funkce  $n$ proměnných.  Uvažujme funkce
$$
\phi_k(t)=f(x_1,\dots,x_{k-1},t,x_{k+1},\dots,x_n), \quad \text{se v\v semi}\ x_j,  \ j\neq k,\ \text{fixovanými}.
$$
{\em Parciální derivace} funkce  $f$ podle $x_k$ (v bodě $(x_1,\dots,x_n)$) je (obvyklá) derivace funkce
$\phi_k$, 
to jest, limita
$$
\lim_{h\to 0}\frac{f(x_1,\dots x_{k-1},x_k+h,x_{k+1},\dots,x_n)-f(x_1,\dots,x_n)}{h}.
$$
Někdy se mluví o {\em $k$-té parciální derivaci}  $f$ ale musíme být opatrní, aby nedošlo k nedorozumění
 s derivací vyššího řádu.

Běžně se užívá označení
$$
\pad{f(x_1,\dots,x_n)}{x_k}\qtq{nebo} \pad{f}{x_k}(x_1,\dots,x_n),
$$
označujeme li proměnné rozdílnými písmeny $f(x,y)$, píšeme samozřejmě
$$
\pad{f(x,y)}{x}\qtq{a} \pad{f(x,y)}{y},\qtq{atd.} 
$$
Toto značení není zcela konsistentní: $x_k$ ve jmenovateli $\partial x_k$ indikuje
že se soustřeďujeme na $k$-tou prom\v ennou, zatím co u $x_n$ v $f(x_1,\dots,x_n)$ v
 ``čitateli'' může jít o skutečnou hodnotu argumentu. Obvykle ale nedorozumění nevzniká. Je-li to nejasné můžeme psát např..
$$\left.\frac{\partial f(x_1,\dots,x_n)}{\partial x_k}\right|_{(x_1,\dots,x_n)=(a_1,\dots,a_n)}.$$
Ale potřeba takové specifikace je vzácná.

\bigskip

{\bf 2.2.} Podobně jako u standardní derivace se může stát (a typicky tomu tak je), že parciální derivace 
$\pad{f(x_1,\dots,x_n)}{x_k}$ existuje pro všechna $(x_1,\dots,x_n)$ v nějaké oblasti
$D'$. V takovém případě máme funkci
$$ 
\pad{f}{x_k}:D'\to\Rbb.
$$
Obvykle je z kontextu zřejmé,  když mluvíme o parciální derivaci, zdali máme na mysli funkci, nebo jen číslo  (hodnotu té limity nahoře).

\bigskip

{\bf 2.3.} Funkce $f$ z XIII.1.2 má obě parciální derivace v každém bodě $(x,y)$. Takže vidíme, že na rozdíl od standardní derivace reálné funkce jedné reálné proměnné, existence derivací zde neimplikuje spojitost. Pro kalkulus v několika proměnných budeme potřebovat silnější pojem. Tomu se budeme věnovat v příští sekci.



\vskip10mm
 
  
 {\large\bf 3. Totální diferenciál.}
 
 \bigskip
 


{\bf 3.1.}  Připomeňte si  VI.1.5. Formule $f(x+h)-f(x)=Ah$ (při zanedbání ``malé části'' $|h|\cdot\mu(h)$)
vyjadřuje přímku tečnou k $\setof{(t,f(t))}{t\in D}$ v bodě $(x,f(x))$. Nebo se na ni můžeme také dívat jako na lineární aproximaci naší funkce v okolí tohoto bodu.

Mysleme podobně o funci $f(x,y)$  (problém funkcí ve více než dvou proměnných už pak bude stejný) a uvažujme plochu 
$$
S=\setof{(t,u,f(t,u))}{(t,u)\in D}.
$$
Dvě parciální derivace vyjadřují směry dvou  tečných přímek k $S$ v bodě $(x,y,f(x,y))$, 
\begin{itemize}
\item ale ne tečnou rovinu (a teprve ta bude uspokojivé rozšíření faktu z VI.1.5),
\item a nedává lineární aproximaci celé funkce.
\end{itemize}
To spravíme pojmem totálního diferenciálu.

\bigskip

{\bf 3.2. Norma.} Pro bod $\ve{x}\in\Ebb_n$ definujeme normu $\nr{\ve{x}}$ jako jeho vzdálenost 
od $\ve{o}$. Typicky budeme užívat
$$
\nr{\ve{x}}=\max_i|x_i|
$$
(ale $\nr{\ve{x}}=\sum_{i=1}^n|x_i|$ nebo standardní pythagorejská $\nr{\ve{x}}=\sqrt{{\ve{x}}\cdot{\ve{x}}}$ by daly stejné výsledky, viz XIII.4).

\bigskip

{\bf 3.3. Totální diferenciál.} Řekneme, že funkce $f$ má {\em totální diferenciál} v bodě  $\ve{a}$ existuje-li funkce $\mu$ spojitá v okolí
 $U$ bodu  $\ve{o}$ taková, že $\mu(\ve{o})=0$ 
(v jiné, ekvivlentní, formulaci se požaduje $\mu$ definovaná v
$U\smin\set{\ve{o}}$ 
a taková, že $\lim_{\ve{h}\to\ve{o}}\mu(\ve{h})=0$),
a čísla $A_1,\dots,A_n$ pro která
$$
f(\ve{a}+\ve{h})-f(\ve{a})=\sum_{k=1}^nA_kh_k+\nr{\ve{h}}\mu(\ve{h}) .
$$

\medskip

{\bf 3.3.1. Poznámky.} 1. S použitím skalárního součinu
můžeme psát $f(\ve{a}+\ve{h})-f(\ve{a})=\ve{A}\ve{a}+\nr{\ve{h}}\mu(\ve{h})$).

2. Všimněte si, že jsme nedefinovali totální diferenciál jako entitu, pouze vlastnost funkce ``mít totální difereciál''. Ponecháme to tak.

\bigskip

{\bf 3.4. Tvrzení.} {\em Nechť má funkce
$f$ totální diferenciál v bodě $\ve{a}$. Potom platí, že

{\em 1.} $f$ je spojitá v $\ve{a}$.

{\em 2.} $f$ má všechny parciální derivace v $\ve{a}$, a to s hodnotami
$$
\pad{f(\ve{a})}{x_k}=A_k.
$$

Důkaz.} 1. Máme
$$
|f(\ve{x})-f(\ve{y})|\leq |\ve{A}(\ve{x}-\ve{y})|+|\mu(\ve{x}-\ve{y})\nr{\ve{x}-\ve{y}}
$$
a limita na pravé straně pro $\ve{y}\to\ve{x}$ je zřejmě 0.

2. M\'ame
$$
\begin{aligned}
\frac1h(f(x_1,\dots x_{k-1},&x_k+h,x_{k+1},\dots,x_n)-f(x_1,\dots,x_n))=\\
&=A_k +\mu((0,\dots,0,h,0,\dots,0))\frac{\nr{(0,\dots,h,\dots,0)}}{h},
\end{aligned}
$$
a limita na pravé straně je zřejmě $A_k$. \sq

\bigskip

{\bf 3.5.} Teď již lineární aproximaci máme: formule
$$
f(x_1+h_1\dots,x_n+h_n)-f(x_1,\dots x_n)=f(\ve{a}+\ve{h})-f(\ve{a})=
\sum_{k=1}^nA_kh_k+\nr{\ve{h}}\mu(\ve{h})
 $$ 
může bý interpretována jako dobrá aproximace funkce $f$ v bodě
$\ve{a}$ lineární funkcí
$$L(x_1,\dots,x_n)= f(a_1,\dots,a_n)+\sum A_k(x_k-a_k).$$ 
Podle požadavků na
 $\mu$ je chyba  mnohem menší než rozdíl $\ve{x}-\ve{a}$.

\smallskip

V případě funkcí jedné proměnné 
není rozdíl mezi existencí derivace v bodě  $a$ a 
vlastností mít totální diferenciál
v tomto bodě (připomeňte si VI.1.5). 
V případě více proměnných je však tento rozdíl zcela zásadní.


Geometricky se děje toto: 
Díváme-li se na funkci $f$ jako na její ``graf'',  (nad)plochu
$$
S=\setof{(x_1,\dots,x_n,f(x_1,\dots,x_n))}{(x_1,\dots,x_n)\in D}\sue\Ebb_{n+1}, 
$$
popisují parciální derivace jen tečné přímky ve směrech souřadnicových os (viz 3.1) zatímco totální diferenciál representuje celou tečnou nadrovinu.

\bigskip

{\bf 3.6.} Může být trochu překvapující, že zatímco pouhá existence parciálních derivací mnoho neznamená,
existence
 {\em spojitých} parciálních derivací je něco úplně jiného. Platí

\bigskip

{\bf Věta.} {\em Nechť má $f$ spojité parciální derivace v okolí bodu $\ve{a}$. Potom má v bodě $\ve{a}$ totální diferenciál

Důkaz.} Nechť
$$
\ve{h}^{(0)}=\ve{h},\ \ve{h}^{(1)}=(0,h_2,\dots,h_n),\
\ve{h}^{(2)}=(0,0,h_3,\dots,h_n)\qtq{etc.}
$$
(takže $\ve{h}^{(n)}=\ve{o}$). Potom máme
$$
f(\ve{a}+\ve{h})-f(\ve{a})=\sum_{k=1}^n(f(\ve{a}+\ve{h}^{(k-1)})-f(\ve{a}+\ve{h}^{(k)}))=M.
$$
Podle Lagrangeovy věty (VII.2.2) existují $0\leq\theta_k\leq 1$ takové, že
$$
f(\ve{a}+\ve{h}^{(k-1)})-f(\ve{a}+\ve{h}^{(k)})=\pad{f(a_1,\dots,a_{k-1},a_k+\theta_kh_k,a_{k+1}+h_{k+1},\dots,a_n+h_n)}{x_k}h_k
$$
a můžeme pokračovat
$$
\begin{aligned}
M&=\sum\pad{f(a_1,\dots,a_k+\theta_kh_k,\dots)}{x_k}h_k=\\
   &=\sum\pad{f(\ve{a})}{x_k}h_k +\sum\left(\pad{f(a_1,\dots,a_k+\theta_kh_k,\dots)}{x_k}-\pad{f(\ve{a})}{x_k}\right)h_k=\\
   &=\sum\pad{f(\ve{a})}{x_k}h_k+\nr{\ve{h}}\sum\left(\pad{f(a_1,\dots,a_k   +\theta_kh_k,\dots)}{x_k}-\pad{f(\ve{a})}{x_k}\right)\frac{h_k}{\nr{\ve{h}}}.
   \end{aligned}
   $$
Položme
$$
\mu(\ve{h})=\sum\left(\pad{f(a_1,\dots,a_k+\theta_kh_k,a_{k+1}+h_{k+1},\dots,a_n+h_n)}{x_k}-\pad{f(\ve{a})}{x_k}\right)\frac{h_k}{\nr{\ve{h}}}.
$$
Jelikož $\displaystyle{\left|\frac{h_k}{\nr{\ve{h}}}\right|}\leq 1$ a jelikož jsou funkce $\pad{f}{x_k}$ spojité, $\lim_{\ve{h}\to\ve{o}}\mu(\ve{h})=0$. \sq   

\bigskip

{\bf 3.7.} Můžeme tedy schematicky psát
$$
\text{spojité PD}\ \Rightarrow\ \text{TD}\ \Rightarrow\ \text{PD}
$$
(kde PD znamená parciální derivace a TD totální diferenciál). 
Žádnou z těchto implikací nelze obrátit. 
Druhou jsme již probrali; 
co se týče první, připomeňme si že pro funkci jedné proměnné derivace v bodě
je totéž jako existence totálního diferenciálu, při čemž derivace
není nutně spojitá, i když třeba existuje v každém bodě otevřeného intervalu.

\smallskip

Ve zbytku této kapitoly by samotný předpoklad existence parciálních derivací téměř nikdy nestačil.
Obvykle by stačila existence totálního diferenciálu, ale nejčastěji budeme předpokládat silnější
existenci spojitých parciálních derivací.




\vskip10mm

 
  
 {\large\bf 4. Parciální derivace vyšších řádů.}
 
 \medskip
 
 \hskip7mm {\large\bf Záměnnost}
 
 \bigskip
 



{\bf 4.1.} Připomeňte si 2.2. Máme-li $g(\ve{x})=\pad{f(\ve{x})}{x_k}$ potom podobně jako počítání druhé derivace
funkce jedné proměnné můžeme počítat druhé derivace $g(\ve{x})$, tedy
$$
\pad{g(\ve{x})}{x_l}.
$$
Výsledek, pokud existuje, se pak značí
$$
\padd{f(\ve{x})}{x_k}{x_l}.
$$
Obecněji, iterováním této procedury dostaneme
$$
\padr{f(\ve{x})}{x_{k_1}}{x_{k_2}}{x_{k_r}},
$$
{\em parciální derivace řádu} $r$.

Všimněte si, že řád je dán tím, kolikrát derivujeme, ne tím, kolikrát se to opakuje v jednotlivých proměnných.
Tak na příklad,
$$
\frac{\partial^3f(x,y,x)}{\partial x\partial y\partial z}\qtq{and}
\frac{\partial^3f(x,y,x)}{\partial x\partial x\partial x}
$$
jsou derivace třetího řádu (třebaže jsme v prním případě podle každé jednotlivé proměnné derivovali jen jednou).

Značení se zjednodušuje tak, že derivování podle téže proměnné těsně za sebou
se píše jako exponent, např.

$$
\frac{\partial^5f(x,y)}{\partial x^2\partial y^3}=\frac{\partial^5f(x,y)}{
\partial x\partial x\partial x\partial y\partial y},
$$
$$\frac{\partial^5f(x,y)}{\partial x^2\partial y^2\partial x}
=\frac{\partial^5f(x,y)}{\partial x\partial x \partial y \partial y\partial x}.$$ 

\bigskip

{\bf 4.2. Příklad.} Počítejme ``smíšené'' derivace druhého řádu funkce 
$
f(x,y)= x\sin(y^2+x).
$
Nejprve dostaneme 
$$
\pad{f(x,y)}{x}=\sin(y^2+x)+x\cos(y^2+x)\qtq{a} \pad{f(x,y)}{y}=2xy\cos(y^2+x).
$$
a potom, jako derivace druhých řádů,
$$
\padd{f}{x}{y}=2y\cos(y^2+x)-2xy\sin(y^2+x)=\padd{f}{y}{x}.
$$
Ať už to překvapí nebo ne, naznačuje to, že parciální derivace vyšších řádů možná nezávisí na pořadí derivování.  A je to v podstatě pravda -- pokud všechny derivace o které jde jsou spojité
(je ale třeba hned poznamenat, že bez tohoto předpokladu
ta rovnost platit nemusí).

\medskip

{\bf 4.2.1. Tvrzení.} {\em Buď $f(x,y)$ funkce taková, že
parciální derivace $\padd{f}{x}{y}$ a $\padd{f}{y}{x}$ jsou definovány a jsou spojité
v nějakém okolí bodu $(x,y)$. Potom máme
$$
\padd{f(x,y)}{x}{y}=\padd{f(x,y)}{y}{x}.
$$

Důkaz.} Hlavní myšlenka tohoto důkazu je snadná: pokusíme se spočíst obě derivace v jednom kroku. Jak snadno vidíme, vede to k výpočtu limity $\lim_{h\to 0}F(h)$ funkce 
$$
F(h)=\frac{f(x+h,y+h)-f(x,y+h)-f(x+h,y)+f(x,y)}{h^2}
$$
a to je co budeme dělat.

Položíme-li
$$
\begin{aligned}
&\varphi_h(y)=f(x+h,y)-f(x,y) \ \ \text{a}\\
&\psi_k(x)=f(x,y+k)-f(x,y),
\end{aligned}
$$
dostaneme pro $F(h)$  dva výrazy:
$$
F(h)=\frac{1}{h^2}(\varphi_h(y+h)-\varphi_h(y))\qtq{a} F(h)=\frac{1}{h^2}(\psi_h(x+h)-\psi_h(x)).
$$
Spočtěme první. Funkce $\varphi_h$, která je funkcí jedné proměnné,
 má derivaci
$$
\varphi_h'(y)=\pad{f(x+h,y)}{y}-\pad{f(x,y)}{y}
$$
a tedy podle Lagrangeovy formule VI.2.2 dostaneme
$$
\begin{aligned}
F(h)&=\frac{1}{h^2}(\varphi_h(y+h)-\varphi_h(y))=\frac1h\varphi_h'(y+\theta_1h)=\\
&=\pad{f(x+h,y+\theta_1h)}{y}-\pad{f(x,y+\theta_1h)}{y}.
\end{aligned}
$$
Potom, znovu podle  VI.2.2, dostaneme
\begin{equation}
F(h)=\frac{\partial}{\partial x}\left(\pad{f(x+\theta_2h,y+\theta_1h)}{y}\right) \tag{$*$}
\end{equation}
pro nějaká $\theta_1,\theta_2$ mezi 0 a 1.

Podobně když počítáme $\frac{1}{h^2}(\psi_h(x+h)-\psi_h(x))$ dostáváme
\begin{equation}
F(h)=\frac{\partial}{\partial y}\left(\pad{f(x+\theta_4h,y+\theta_2h)}{x}\right). \tag{$**$}
\end{equation}
Jelikož jsou obě $\frac{\partial}{\partial y}(\pad{f}{x})$ a
$\frac{\partial}{\partial x}(\pad{f}{y})$ spojité v bodě $(x,y)$, můžeme
$\lim_{h\to 0}F(h)$ počítat z kterékoli z  ($*$)   nebo ($**$) a  dostaneme
$$
\lim_{h\to 0}F(h)=\padd{f(x,y)}{x}{y}=\padd{f(x,y)}{y}{x}.
$$ \sq


\bigskip

{\bf 4.3.} Iterováním záměn dovolených podle 4.2.1 dostaneme snadnou indukcí

\medskip

{\bf Důsledek.} {\em Nechť má funkce $f$ v $n$ proměnných
spojité parciální derivace do řádu $k$. Potom hodnoty
těchto derivací záleží jen na tom kolikrát bylo derivováno v každé
z individuálních proměnných
 $x_1,\dots,x_n$.}


\medskip

{\bf 4.3.1.} Tedy za předpokladů věty 4.3 můžeme obecné parciální derivace
řádu $r\leq k$ psát jako
$$
\padr{f}{x_1^{r_1}}{x_2^{r_2}}{x_n^{r_n}}\qtq{kde} r_1+r_2+\cdots+r_n=r
$$
kde je samozřejmě dovoleno $r_j=0$ a indikuje absenci symbolu $\partial x_j$.



\vskip10mm
 
  
 {\large\bf 5. Složené funkce a řetězové pravidlo.}
 
 \bigskip
 
Připomeňte si důkaz pravidla pro derivaci složené funkce v VI.2.2.1.
Byl založen na ``totálním diferenciálu v jedné proměnné''. Analogickou procedurou dostaneme následující větu.

\medskip


{\bf 5.1. Věta.} (Řetězové Pravidlo v nejjednodušším tvaru) {\em Nechť má $f(\ve{x})$ totální diferenciál v bodu $\ve{a}$. 
Nechť mají $g_k(t)$ derivace v bodě $b$ 
a nechť je $g_k(b)=a_k$ pro všechna $k=1,\dots,n$. Položme
$$
F(t)=f(\ve{g}(t))=f(g_1(t),\dots,g_n(t)).
$$
Potom má $F$ derivaci v  $b$, totiž
$$
F'(b)=\sum_{k=1}^n\pad{f(\ve{a})}{x_k}\cdot g'_k(b).
$$

Důkaz.} Použítím formule  3.3 
dostaneme
$$ \begin{aligned}
\frac1h&(F(b+h)-F(b))=\frac1h(f(\ve{g}(b+h))-f(\ve{g}(b))=\\
&=\frac1h(f(\ve{g}(b)+(\ve{g}(b+h)-\ve{g}(b)))-f(\ve{g}(b))=\\
&=\sum_{k=1}^nA_k\frac{g_k(b+h)-g_k(b)}{h}+ \mu(\ve{g}(b+h)-\ve{g}(b))\max_k\frac{|g_k(b+h)-g_k(b)|}{h}.
\end{aligned}
$$
Máme $\lim_{h\to 0}\mu(\ve{g}(b+h)-\ve{g}(b))=0$ jelikož jsou funkce $g_k$ spojité v $b$. Jelikož funkce $g_k$ mají derivace, jsou
$\max_k\frac{|g_k(b+h)-g_k(b)|}{h}$  omezené v dostatečně malém okolí nuly. Limita posledního sčítance je tedy nula a máme
$$
\begin{aligned}
\lim_{h\to 0}\frac1h(F(b+h)&-F(b))
=\lim_{h\to 0}\sum_{k=1}^nA_k\frac{g_k(b+h)-g_k(b)}{h}=\\
&=\sum_{k=1}^nA_k\lim_{h\to 0}\frac{g_k(b+h)-g_k(b)}{h}=\sum_{k=1}^n \pad{f(\ve{a})}{x_k}g'_k(b).
\end{aligned}
$$
\sq

\bigskip

{\bf 5.1.1. Důsledek.} (Řetězové Pravidlo)  {\em Nechť má $f(\ve{x})$ totální diferenciál v bodě $\ve{a}$. 
Nechť mají funkce $g_k(t_1,\dots,t_r)$ parciální derivace 
v $\ve{b}=(b_1,\dots,b_r)$ a nechť je $g_k(\ve{b})=a_k$ pro všechna $k=1,\dots,n$. Potom má funkce
$$
(f\circ \ve{g})(t_1,\dots,t_r)=f(\ve{g}(\ve{t}))=f(g_1(\ve{t}),\dots,g_n(\ve{t}))
$$
všechny parciální derivace v $b$, a platí
$$
\pad{(f\circ\ve{g})(\ve{b})}{t_j}=\sum_{k=1}^n\pad{f(\ve{a})}{x_k}\cdot \pad{g_k(\ve{b})}{t_j}.
$$
}

\medskip

{\bf 5.1.2. Poznámka.} Pouhé parciální derivace u $f$ by nestačily.
Požadavek totálního diferenciálu v 5.1
je zásadní, a snadno je vidět proč. Připomeňte si geometrickou představu z 3.1 a posledního odstavce v 3.5. 
 $n$-tice funkcí $\ve{g}=(g_1,\dots,g_n)$ 
representuje parametrisovanou křivku v $D$, a $f\circ\ve{g}$ je pak křivna na
nadploše $S$. Parciální drivace $f$ (nebo, tečné přímky na  $S$ ve směru os souřadnic)
obecně nemají co dělat s chováním této křivky.

\bigskip

{\bf 5.2. Pravidla pro násobení a dělení jako důsledky řetězového pravidla.} Jak jsme již zmínili, Řetězové Pravidlo je (včetně svého důkazu) vícemméně bezprostřední rozšíření pravidla pro skládání v jedné proměnné. 
Může tedy trochu překvapit, že v sobě skrývá pravidla pro násobení a dělení.

Vezmeme-li $f(x,y)=xy$ máme $\pad{f}{x}=y$ a $\pad{f}{y}=x$ a tedy
$$
\begin{aligned}
(u(t)v(t))'=f(u(t),v(t))'=&\pad{f(u(t),v(t))}{x}v'(t)+\pad{f(u(t),v(t))}{y}u'(t)=\\
=v(t)\cdot u'(t)+ u(t)\cdot v'(t).
\end{aligned}
$$
Podobně pro $f(x,y)=\frac{x}{y}$ máme $\pad{f}{x}=\frac{1}{y}$ a $\pad{f}{y}=-\frac{x}{y^2}$, a následkem toho 
$$
\frac{u(t)}{v(t)}'= \frac{1}{v(t)}u'(t)-\frac{u(t)}{v^2(t)}=
\frac{v(t)u'(t)-u(t)v'(t)}{v^2(t)}.
$$

\bigskip

{\bf 5.3. Řetězové pravidlo pro vektorové funkce.} Udělejme ještě jeden krok a uvažme v 5.1.1 zobrazení
 $\ve{f}=(f_1,\dots,f_s):D\to\Ebb_s$. Složme je na $\ve{f}\circ\ve{g}$ 
se zobrazením $\ve{g}:D'\to\Ebb_n$ (připomeňme si úmluvu v 1.4). Potom máme
\begin{equation}
\pad{(\ve{f}\circ \ve{g})}{t_j}=\sum_k\pad{f_i}{x_k}\cdot\pad{g_k}{x_j}. \tag{$*$}
\end{equation}
Čtenářově pozornosti jistě neuniklo, že na prvé straně je součin matic
\begin{equation}
\begin{pmatrix}\pad{f_i}{x_k}\end{pmatrix}_{i,k}\begin{pmatrix}\pad{g_k}{x_j}\end{pmatrix}_{k,j}. \tag{$**$}
\end{equation}
Z lineární algebry si připomeňme roli matic v popisu lineárních zobrazení
$L:V_n\to V_m$ a zejména to, že  skládání lineárních zobrazení  odpovídá součin příslušných matic. Teď by nás formule  $(*)$ resp. $(**)$ už neměly překvapovat: 
representují fakt, který jistě očekáváme, totiž že
lineární aproximace složení  $\ve{f}\circ\ve{g}$ je složení lineárních aproximací 
pro $\ve{f}$ a $\ve{g}$ .

\medskip

{\bf 5.3.1.} Podle předchozího komentáře můžeme řetězové pravidlo psát v maticovém tvaru
takto. 
Pro $\ve{f}=(f_1,\dots,f_s):U\to\Ebb_s$, $D\sue \Ebb_n$, definujme $D\ve{f}$ jako matici
$$
D\ve{f}=\begin{pmatrix}\pad{f_i}{x_k}\end{pmatrix}_{i,k}.
$$
Potom máme
$$
D(\ve{f}\circ\ve{g})=D\ve{f}\cdot D\ve{g}.
$$
Explicitněji máme v konkretním argumentu $\ve{t}$ 
$$
D(\ve{f}\circ\ve{g})(\ve{t})=D(\ve{f}(\ve{g}))(\ve{t})\cdot D\ve{g}(\ve{t}).
$$
srovnejte to s pravidlem v jedné proměnné
$$
(f\circ g)'(t)=f'(g(t))\cdot g'(t);
$$
pro $1\times 1$ matice je samozřejmě $(a)(b)=(ab)$.

\bigskip

{\bf 5.4. Lagrangeova  formule v několika proměnných.} Jistě si pamatujete, že o podmnožině $U\sue \Ebb_n$ se mluví jako o {\em konvexní} pokud
$$
\ve{x},\ve{y}\in U\quad\Rightarrow\quad \forall t, \
 0\leq t\leq 1,\ \ (1-t)\ve{x}+t\ve{y}=\ve{x}+t(\ve{y}-\ve{x})\in U.
$$

\medskip

{\bf 5.4.1. Tvrzení.} {\em Nechť má $f$spojité parciální derivace v konvexní
otevřené množině $U\sue\Ebb_n$. Potom pro libovolné dva body
 $x,y\in D$ existuje $\theta$ pro které $0\leq \theta\leq 1$ takové, že
$$
f(\ve{y})-f(\ve{x})=\sum_{j=1}^n\pad{f(\ve{x}+\theta(\ve{y}-\ve{x}))}{x_j}(y_j-x_j).
$$

Důkaz.} Položme $F(t)=f(\ve{x}+t(\ve{y}-\ve{x}))$. Potom $F= f\circ\ve{g}$ 
kde $\ve{g}$ je definováno jako $g_j(t)=x_j+t(y_j-x_j)$, a
$$
F'(t)=\sum_{j=1}^n\pad{f(\ve{g}(t))}{x_j}g_j'(t)=\sum_{j=1}^n\pad{f(\ve{g}(t))}{x_j}(y_j-x_j).
$$
Podle VII.2.2 tedy
$$
f(\ve{y})-f(\ve{x})=F(1)-F(0)=F'(\theta)
$$
z čehož hned tvrzení dostáváme. \sq

\medskip

{\bf Poznámka.} Tato formule se často užívá ve tvaru
$$
f(\ve{x}+\ve{h})-f(\ve{x})=\sum_{j=1}^n\pad{f(\ve{x}+\theta\ve{h})}{x_j}h_j.
$$
Srovnejte ji s formulí pro totální diferenciál.

\newpage
.

\newpage 
 
\centerline{\Large\bf XV.  Věty o implicitních funkcích} 
 
 
 \vskip10mm
 
  
 {\large\bf 1. Úloha.}
 
 \bigskip
 
{\bf 1.1.} Mějme $m$ reálných funkcí $F_k(x_1,\dots,x_n,y_1,\dots y_m)$, $k=1,\dots,m$, každou z nich v $n+m$ proměnných. Uvažujeme systém rovnic
$$
\begin{aligned}
&F_1(x_1,\dots,x_n,y_1,\dots y_m)=0\\
&\dots\quad\dots\quad\dots\\
&F_m(x_1,\dots,x_n,y_1,\dots y_m)=0
\end{aligned}
$$
Rádi bychom našli řešení $y_1,\dots,y_m$. Lépe řečeno, s užitím úmluvy XIV.1, máme systém  $m$ rovnic v $m$ neznámých (počet  $n$ proměnných $x_j$ je nepodstatný)
\begin{equation}
F_k(\ve{x},y_1,\dots y_m)=0, \quad k=1,\dots,m \tag{$*$}
\end{equation}
a hledáme řešení $y_k=f_k(\ve{x})$ ($=f_k(x_1,\dots,x_n)$).

\bigskip

{\bf 1.2.} I v nejjednodušších případech nemusíme mít nutně řešení, o jednoznačném ani nemluvě.
Vezměme např. tuto jednoduchou rovnici:
$$
F(x,y)=x^2+y^2-1=0.
$$ 
Pro $|x|>1$ neexistuje $y$ vyhovující $f(x,y)=0$. Pro $|x_0|<1$, 
 máme v dostatečném okolí bodu $x_0$ dvě řešení
$$
f(x)=\sqrt{1-x^2}\qtq{a}g(x)=-\sqrt{1-x^2}.
$$
To je lepší, ale stejně zde stále máme {\em dvě} hodnoty v každém bodě takového okolí,
v rozporu s definicí funkce. Abychom dosáhli jednoznačnosti musíme nejen omezit hodnoty 
 $x$ ale {\em též hodnoty $y$} 
na nějaký interval $(y_0-\Delta,y_0+\Delta)$ (kde $F(x_0,y_0)=0$). Tedy, 
máme-li konkretní řešení $(x_0,y_0)$ máme ``okénko''
$$
(x_0-\delta,x_0+\delta)\times(y_0-\Delta,y_0+\Delta)
$$
v němž kolem něho vidíme jednoznačné řešení.

V našem příkladě
je ještě případ $(x_0,y_0)=(1,0)$, kde řešení máme (a v $x_0=1$ jediné),
ale žádné vhodné okénko jako v předchozím případě, protože
v kerémkoli okolí bodu $(1,0)$,
napravo žádné řešení není, a nalevo jsou vždy dvě.

\smallskip

Všimněte si, že v kritických bodech $(1,0)$ a $(-1,0)$ máme
\begin{equation}
\pad{F}{y}(1,0)=\pad{F}{y}(-1,0)=0. \tag{$**$}
\end{equation}


\bigskip

{\bf 1.3.}V této kapitole ukážeme, že pro funkce $F_k$ se spojitými parciálními derivacemi se nemůže stát nic horšího než v uvedeném příkladě:
\begin{itemize}
\item budeme muset mít nějaké body $\ve{x}^0$, $\ve{y}^0$ pro které $F_k(\ve{x}^0,\ve{y}^0) = 0$ kterými začneme,
\item s jistými výjimkami budeme pak mít ``okénka'' $U\times V$ taková, že pro $\ve{x}\in U$ bude přesně jedno $\ve{y}\in V$, t.j.,
$y_k=f(x_1,\dots,x_n)$, splňující daný systém rovnic;
\item a ty výjimky přirozeně rozšiřují to, co jsme viděli v $(**)$ nahoře: místo podmínky $\pad{F}{y}(x^0,y^0)\neq 0$ budeme mít
$\frac{\Ds(\ve{F})}{\Ds(\ve{y})}(\ve{x}^0,\ve{y}^0)\neq 0$ pro něco podobného, t.zv. Jakobián.
\end{itemize}
Nadto budou mít řešení spojité parciální derivace, budou-li je mít $F_j$.

 
 \vskip10mm
 
  
 {\large\bf 2. Jedna rovnice.}
 
 \bigskip
 
{\bf 2.1.
Věta.} {\em Buď $F(\ve{x},y)$ funkce  $n+1$ proměnných definovaná v okolí
 bodu $(\ve{x}^0,y_0)$. Nechť má $F$ spojité parciální derivace do řádu  
$k\geq 1$ a nechť
 $$
 F(\ve{x}^0,y_0)=0\qtq{a} \left|\pad{F(\ve{x}^0,y_0)}{y}\right|\neq 0.
 $$
 Potom existují $\delta>0$ a $\Delta>0$ takové, že pro $\ve{x}$ s
$||\ve{x}-\ve{x^0}||<\delta$ existuje právě jedno
$y$  s $|y-y_0|<\Delta$ takové, že
$$F(\ve{x},y)=0.$$ 
Nadto, píšeme-li pro toto jediné řešení $y=f(\ve{x})$,  má získaná funkce
$$
f:\oin{x^0_1-\delta,x^0_1+\delta}\times\cdots\times\oin{x^0_n-\delta,x^0_n+\delta}\to\Rbb
$$
spojité parciální derivace do řádu $k$.}

\smallskip

{\bf Před důkazem.} Čtenáři doporučujeme přepsat si následující důkaz tak jako bychom měli jen jednu reálnou proměnnou  $x$. Toto zjednodušení učiní proceduru průzračnější aniž bychom cokoli z myšlenek ztratili. Obecný $\ve{x}$ si pouze žádá složitější značení, které zatemňuje některé z kroků.
 
 {\em Důkaz.} Norma $\nr{\ve{x}}$ bude jako v IV.3.2, totiž $\max_i|x_i|$. Položme
 $$
 U(\gamma)=\setof{\ve{x}}{\nr{\ve{x}-\ve{x}^0}<\gamma}\qtq{a}
 A(\gamma)=\setof{\ve{x}}{\nr{\ve{x}-\ve{x}^0}\leq\gamma}
 $$
(``okénko'' které hledáme se ukáže jako
 $U(\delta)\times\oin{y_0-\Delta,y_0+\delta}$).
 
 Bez újmy na obecnosti buď dejme tomu 
$$\pad{F(\ve{x}^0,y_0)}{y}>0.$$ 
První parciální derivace funkce $F$
jsou spojité a $A(\delta)$ je omezená a uzavřená, a tedy kompaktní podle XIII.7.6. Tedy podle XIII.7.9 existují $a>0$, $K$,
 $\delta_1>0$ a $\Delta>0$ taková, že pro všechna
$(\ve{x},y)\in U(\delta_1)\times\uin{y_0-\Delta,y_0+\Delta}$
máme
 \begin{equation}
 \pad{F(\ve{x},y)}{y}\geq a\qtq{a} \left|\pad{F(\ve{x},y)}{x_i}\right |\leq K.  \tag{$*$}
 \end{equation}
 
 
 \smallskip
 
 I. {\em Funkce $f$} : Zvolme pevně $\ve{x}\in U(\delta_1)$,
a definujme funkci jedné proměnné $y\in\oin{y_0-\Delta,y_0+\Delta}$
předpisem
 $$
 \varphi_{\ve{x}}(y)=F(\ve{x},y).
  $$
Potom je $\varphi_{\ve{x}}'(y)=\pad{F(\ve{x},y)}{y}>0$ a tedy

\vskip1mm

$$\parbox{3in}{\em všechny $\varphi_{\ve{x}}(y)$ jsou rostoucí funkce
proměnné $y$, a \quad $\varphi_{\ve{x}_0}(y_0-\Delta)<\varphi_{\ve{x}_0}(y_0)=0
<\varphi_{\ve{x}_0}(y_0+\Delta)$.}$$

\vskip1mm

\noindent Podle XIV.3.4 a XIV.3.6, je $F$ spojitá a existuje tedy  $\delta$, $0<\delta\leq\delta_1$, takové, že
$$
\forall \ve{x}\in U(\delta),\quad \varphi_{\ve{x}}(y_0-\Delta)<0
< \varphi_{\ve{x}}(y_0+\Delta).
$$
Vidíme, že
$\varphi_{\ve{x}}$  roste a tedy je prosté.
Tedy máme podle IV.3 přesně jedno  $y\in\oin{y_0-\Delta,y_0+\Delta}$  takové, že $\varphi_{\ve{x}}(y)=0$ 
-- to jest, $F(\ve{x},y)=0$. Označme toto $y$ symbolem $f(\ve{x})$.

Všimněte si, že zatím je $f$ jen funkce; o jejích vlastnostech nic nevíme, zejména ani nevíme, je-li spojitá či ne.

\smallskip

II. {\em První derivace.} Fixujme index $j$, zapišme
posloupnost $x_1,\dots,x_{j-1}$ jako $\ve{x}_b$ a stejně zjednodušme
$x_{j+1},\dots,x_n$ na $\ve{x}_a$; máme tedy
$$
\ve{x}=(\ve{x}_b,x_j,\ve{x}_a).
$$
Budeme počítat $\pad{f}{x_j}$ jako derivaci funkce $\psi(t)=f(\ve{x}_b,t,\ve{x}_a)$.

Podle XIV.5.4.1 máme
$$
\begin{aligned}
0&=F(\ve{x}_b,t+h,\ve{x}_a,\psi(t+h))-F(\ve{x}_b,t,\ve{x}_a,\psi(t))=\\
&=F(\ve{x}_b,t+h,\ve{x}_a,\psi(t)+(\psi(t+h)-\psi(t)))-F(\ve{x}_b,t,\ve{x}_a,\psi(t))=\\
&=\pad{F(\ve{x}_b,t+\theta h,\ve{x}_a,\psi(t)+\theta(\psi(t+h)-\psi(t)))}{x_j}h\\
&+
\pad{F(\ve{x}_b,t+\theta h,\ve{x}_a,\psi(t)+\theta(\psi(t+h)-\psi(t)))}{y}(\psi(t+h)-\psi(t))
\end{aligned}
$$
a tedy
\begin{equation}
\psi(t+h)-\psi(t)=-h\cdot\frac{\displaystyle\pad{F(\ve{x}_b,t+\theta h,\ve{x}_a,\psi(t)+\theta(\psi(t+h)-\psi(t)))}{x_j}}
{\displaystyle\pad{F(\ve{x}_b,t+\theta h,\ve{x}_a,\psi(t)+\theta(\psi(t+h)-\psi(t)))}{y}}  \tag{$**$}
\end{equation}
pro nějaké $\theta$ mezi 0 a 1.

Nyní můžeme vyvodit, že $f$ je spojitá: z $(*)$ získáváme
$$
|\psi(t+h)-\psi(t)|\leq |h|\cdot\left|\frac{K}{a}\right|
$$
Užitím tohoto faktu můžeme dále z $(**)$ spočítat
$$
\begin{array}{l}\lim_{h\to 0}\frac{\displaystyle\psi(t+h)-\psi(t)}{
\displaystyle h}=\\
=-\lim_{h\to 0}\frac{\displaystyle \pad{F(\ve{x}_b,t+\theta h,\ve{x}_a,\psi(t)+\theta(\psi(t+h)-\psi(t)))}{x_j}}
{\displaystyle\pad{F(\ve{x}_b,t+\theta h,\ve{x}_a,\psi(t)+\theta(\psi(t+h)-\psi(t)))}{y}} =
-\frac{\displaystyle\pad{F(\ve{x}_b,t,\ve{x}_a,\psi(t))}{x_j}}
{\displaystyle\pad{F(\ve{x}_b,t,\ve{x}_a,\psi(t))}{y}}
\end{array}
$$

\smallskip

III. {\em Vyšší derivace.} Všimněte si, že jsme nedokázali jen 
 {\em existenci} první derivace
 $f$, ale též formuli
\begin{equation}
\pad{f(\ve{x})}{x_j}=-\pad{F(\ve{x},f(\ve{x}))}{x_j}\cdot\left(\pad{F(\ve{x},f(\ve{x}))}{y}\right)^{-1}. \tag{${**}*$}
\end{equation}
Z této můžeme induktivně počítat vyšší derivace funkce $f$ (užívajíce standardních pravidel pro derivování) 
tak dlouho pokud
$$\frac{\partial^r F}{\partial x_1^{r_1}\cdots\partial x_n^{r_n}\partial y^{r_{n+1}}}$$ 
existují a jsou spojité. \sq

\bigskip 

{\bf 2.2.} Formuli \hbox{$({**}*)$} jsme dostali jako vedlejší produkt důkazu, že $f$ derivaci má (později byla užitečná, ale v tom místě ještě ne). Všimněte si, že kdybychom věděli předem, že 
 $f$ nějakou derivaci má, mohli bychom (5.2.3) vyvodit bezprostředně z řetězového pravidla. Skutečně, máme
$$
0\equiv F(\ve{x},f(\ve{x}));
$$
derivujeme-li na obou stranách dostaneme
$$
0=\pad{F\ve{x},f(\ve{x}))}{x_j}+\pad{F\ve{x},f(\ve{x}))}{y}\cdot\pad{f(\ve{x})}{x_j}.
$$
A kdybychom derivovali dále, dostali  bychom lineární rovnice z nichž bychom mohli spočítat
 hodnoty všech derivací zaručených větou.

\medskip


{\bf 2.3. Poznámka.} Řešení $f$ v 2.1  
má tolik derivací jako výchozí $F$ -- za předpokladu, že $F$ má aspoň první.
Někdy se o funkci myslí jako o své vlastní nulté derivaci. 
 Věta však  {\em nezaručuje
spojité řešení $f$ rovnice $F(x,f(x))=0$ s pouze spojitou $F$}. 
První derivaci užíváme již k tomu, abychom dokázali existenci funkce $f$.




 \vskip10mm
 
  
 {\large\bf 3. Na rozjezd: dvě rovnice.}
 
 \bigskip
 
{\bf 3.1.}
 Vezměme dvojici rovnic
$$
\begin{aligned}
&F_1(\ve{x},y_1,y_2)=0,\\
&F_2(\ve{x},y_1,y_2)=0
\end{aligned}
$$
a pokusme se najít řešení $y_i=f_i(\ve{x})$, $i=1,2$,
v okolí bodu $(\ve{x}^0,y^0_1,y^0_2)$ (v němž jsou rovnice splněny). 
Aplikujme ``substituční metodu'' založenou na větě 2.1. 
Nejprve uvažujme druhou rovnici jako rovnici pro
 $y_2$; v okolí bodu $(\ve{x}^0,y_1^0,y_2^0)$ potom získáme
 $y_2$ jako funkci $\psi(\ve{x},y_1)$. Tu vložíme do první rovnice 
a získáme
$$
G(\ve{x},y_1)=F_1(\ve{x},y_1,\psi(\ve{x},y_1));
$$
nalezneme-li řešení $y_1=f_1(\ve{x})$  v okolí bodu  $(\ve{x}^0,y^0_1)$ můžeme ho substituovat do $\psi$ a
získáme
$y_2=f_2(\ve{x})=\psi(\ve{x},f_1(\ve{x}))$.

\bigskip

{\bf 3.2.} Když to řešení máme, shrňme co jsme vlastně předpokládali:

-- Především musíme mít spojité
parciální derivace funkcí $F_i$.

-- Potom, abychom směli získat
$\psi$ podle 2.1 tak jak jsme to udělali, potřebovali jsme
\begin{equation}
\pad{F_2}{y_2}(\ve{x}^0,y^0_1,y^0_2)\neq 0. \tag{$*$}
\end{equation}

-- Konečně také potřebujeme aby (s užitím řetězového pravidla)
\begin{equation}
0\neq\displaystyle\pad{G}{y_1}(\ve{x}^0,x^0)=
\pad{F_1}{y_1}+\pad{F_1}{y_2}\pad{\psi}{y_1}\neq 0. \tag{$**$}
\end{equation}
Potom užijeme formuli pro první derivaci
$$
\pad{\psi}{y_1}=-\left(\pad{F_2}{y_2}\right)^{-1}\pad{F_2}{y_1}
$$
z důkazu 2.1 a transformujeme $(**)$ do
$$
\left(\pad{F_1}{y_2}\right)^{-1}\left(\pad{F_1}{y_1}\pad{F_2}{y_2}-\pad{F_1}{y_2}\pad{F_2}{y_1}\right)\neq 0,
$$
to jest,
$$
\pad{F_1}{y_1}\pad{F_2}{y_2}-\pad{F_1}{y_2}\pad{F_2}{y_1}\neq 0.
$$
To je známá formule, totiž formule pro determinant. Tedy jsme vlastně předpokládali, že
$$
\left|\begin{array}{cc}
\displaystyle\pad{F_1}{y_1},&\displaystyle\pad{F_1}{y_2}\\[3ex]
\displaystyle\pad{F_2}{y_1},&\displaystyle\pad{F_2}{y_2}
\end{array}\right|
=\text{det}\left(\pad{F_i}{y_j}\right)_{i,j}\neq 0. 
$$
A tato podmínka již stačí:
 přepokládáme-li, že ten determinant je nenulový, máme  {\em buď}
$$\pad{F_2}{y_2}(\ve{x}^0,y^0_1,y^0_2)\neq 0$$ 
{\em a/nebo}
$$\pad{F_2}{y_1}(\ve{x}^0,y^0_1,y^0_2)\neq 0,$$ 
takže pokud platí druhé, začneme řešením 
$F_2(\ve{x},y_1,y_2)=0$ pro $y_1$ místo pro $y_2$. 

\vskip10mm
 
  
 {\large\bf 4. Obecný systém.}
 
 \bigskip

{\bf 4.1. Jacobiho determinant.}  Buď $\ve{F}$ soustava funkcí
$$
\ve{F}(\ve{x},\ve{y})=(F_1(\ve{x},y_1,\dots,y_m),\dots,F_m(\ve{x},y_1,\dots,y_m)).
$$
Pro toto $\ve{F}$ a $m$-tici $\ve{y}=(y_1,\dots,y_m)$
definujeme {\em Jacobiho determinant} (krátce, {\em Jacobián})
$$
\frac{\Ds(\ve{F})}{\Ds(\ve{y})}=\text{det}\left(\pad{F_i}{y_j}\right)_{i,j=1,\dots,m}
$$
Všimněte si, že pro $m=1$, tedy máme-li jednu  funkci $F$ a jedno $y$,
máme
$$
\frac{\Ds(F)}{\Ds(y)}=\pad{F}{y}.
$$


\bigskip


{\bf 4.2. Věta.} {\em Buďte $F_i(\ve{x},y_1,\dots,y_m)$, $i=1,\dots,m$, funkce
 $n+m$ proměnných se spojitými parciálními derivacemi do řádu $k\geq 1$. Buď
$$
\ve{F}(\ve{x}^0,\ve{y}^0)=\ve{o}
$$
a buď
$$
\frac{\Ds(\ve{F})}{\Ds(\ve{y})}(\ve{x}^0,\ve{y}^0)\neq 0.
$$
Potom existují $\delta>0$ a $\Delta>0$ taková, že pro každý bod
$$\ve{x}\in(x^0_1-\delta,x^0_1+\delta)\times\cdots\times(x^0_n-\delta,x^0_n+\delta)$$ 
existuje  právě jeden
$$\ve{y}\in(y^0_1-\Delta,y^0_1+\Delta)\times\cdots\times(y^0_m-\Delta,x^0_m+\Delta)$$ 
takový, že
$$\ve{F}(\ve{x},\ve{y})=0.$$ 
Píšeme-li tento $\ve{y}$ jako vektorovou funkci
 $\ve{f}(\ve{x})=(f_1(\ve{x}),\dots,f_m(\ve{x}))$, 
mají jednotlivé funkce $f_i$spojité parciální derivace do řádu $k$.}

\smallskip

{\bf Před důkazem.} Procedura sleduje substituční metodu použitou v předchozí sekci. jen budeme muset udělat trochu více s determinanty (ale to je lineární algebra čtenáři dobře známá) a na závěr
budeme muset udělat pořádek v
$\Delta$ and $\delta$ (kterým jsme zatím mnoho pozornosti nevěnovali).

\smallskip
 
{\em Důkaz} indukcí. Tvrzení platí pro $m=1$ (viz 2.1). Nechť nyní platí pro
 $m$, a nechť máme dán systém
$$
 F_i(\ve{x},\ve{y}),\ i=1,\dots,m+1
 $$
 splňující předpoklady (všimněte si, že neznámý
vektor $\ve{y}$ je také $m+1$-rozměrný). Potom speciálně v Jakobiho determinantu nemůžeme mít sloupec,
sestávající jen z nul, a tedy po případném přerovnání funkcí $F_i$'s, můžeme předpokládat, že 
 $$
 \pad{F_{m+1}}{y_{m+1}}(\ve{x}^0,\ve{y}^0)\neq 0.
 $$
Pišme $\wt{\ve{y}}=(y_1,\dots,y_m)$;  potom,
 podle indukčního předpokladu, máme $\delta_1>0$ a $\Delta_1>0$ takové, že pro
 $$ (\ve{x},\wt{\ve{y}})\in(x^0_1-\delta_1,x^0_1+\delta_1)\times\cdots\times(x^0_n-\delta_1,x^n_1+\delta_1)
 \times\cdots\times(y^0_m-\delta_1,y^0_m+\delta_1)
$$
existuje přesně jedno $y_{m+1}=\psi(\ve{x},\wt{\ve{y}})$ splňující
$$
F_{m+1}(\ve{x},\wt{\ve{y}},y_{m+1})=0\qtq{a} |y_{m+1}-y_{m+1}^0]<\Delta_1.
$$
Toto $\psi$ má spojité parciální derivace do řádu $k$ a tedy je také mají funkce
$$
G_i(\ve{x},\wt{\ve{y}})=F_i(\ve{x},\wt{\ve{y}},\psi(\ve{x},\wt{\ve{y}})),\ i=1,\dots m+1
$$  
($G_{m+1}$ je konstantně 0).
Podle řetězového pravidla získáme
\begin{equation}
\pad{G_j}{y_i}=\pad{F_j}{y_i}+\pad{F_j}{y_{m+1}}\pad{\psi}{y_i}.\tag{$*$}
\end{equation}
Nyní uvažujme determinant
$$
\frac{\Ds(\ve{F})}{\Ds(\ve{y})}
=\left|\begin{array}{cccc}
\displaystyle\pad{F_1}{y_1},&\dots,&\displaystyle\pad{F_1}{y_m},
&\displaystyle\pad{F_1}{y_{m+1}}\\[4ex]
\dots,         &\dots,&\dots,        &\dots    \\[4ex]
\displaystyle\pad{F_m}{y_1},&\dots,&\displaystyle\pad{F_m}{y_m},&
\displaystyle\pad{F_m}{y_{m+1}}\\[4ex]
\displaystyle\pad{F_{m+1}}{y_1},&\dots,&\displaystyle\pad{F_{m+1}}{y_m},&
\displaystyle\pad{F_{m+1}}{y_{m+1}}
\end{array}\right|.
$$
Násobme poslední sloupec
 $\pad{\psi}{y_i}$ a přičtěme ho k  $i$-tému. Podle ($*$), vezmeme-li v úvahu, že
$G_{m+1}\equiv 0$ a tedy 
$$\pad{G_{m+1}}{y_i}=\pad{F_{m+1}}{y_i}+\pad{F_{m+1}}{y_{m+1}}\pad{\psi}{y_i}=0,$$ 
získáme
$$
\frac{\Ds(\ve{F})}{\Ds(\ve{y})}
=\left|\begin{array}{cccc}
\displaystyle\pad{G_1}{y_1},&\dots,&\displaystyle\pad{G_1}{y_m},&
\displaystyle\pad{F_1}{y_{m+1}}\\[4ex]
\dots,         &\dots,&\dots,        &\dots    \\[4ex]
\displaystyle\pad{G_m}{y_1},&\dots,&
\displaystyle\pad{G_m}{y_m},&\displaystyle\pad{F_m}{y_{m+1}}\\[4ex]
0,&\dots,&0,&\displaystyle\pad{F_{m+1}}{y_{m+1}}
\end{array}\right|= 
\pad{F_{m+1}}{y_{m+1}}\cdot\frac{D(G_1,\dots,G_m)}{D(y_1,\dots,y_m)}.
$$
Tedy, 
$$\frac{\Ds(G_1,\dots,G_m)}{\Ds(y_1,\dots,y_m)}\neq 0$$ 
a tedy podle indukčního předpokladu
existují $\delta_2>0$ a $\Delta_2>0$ takové, že pro $|x_i-x^0_i|<\delta_2$ je jednoznačně určené
$\wt{\ve{y}}$  s $|y_i-y^0_i|<\Delta_2$ takové, že 
$$
G_i(\ve{x},\wt{\ve{y}})=0\qtq{pro} i=1,\dots,m
$$
a že výsledné $f_i(\ve{x})$ mají spojité parciální derivace do řádu $k$. Konečně pak definicí
$$
f_{i+1}(\ve{x})=\psi(\ve{x},f_1(\ve{x}),\dots,f_m(\ve{x}))
$$
získáme řešení $\ve{f}$ původní soustavy rovnic $\ve{F}(\ve{x},\ve{y})=0$.

Abychom důkaz dokončili potřebujeme omezení
$\nr{\ve{x}-\ve{x}^0}<\delta$ and $\nr{\ve{y}-\ve{y}^0}<\Delta$ v jejichž rámci je řešení korektní (to jest, jednoznačné).

Zvolme $0<\Delta\leq\delta_1,\Delta_1,\Delta_2$ a potom $0<\delta<\delta_1,\delta_2$ dost malé k tomu, aby pro  $|x_1-x_i^0|<\delta$ bylo $|f_j(\ve{x})-f_j(\ve{x}^0)|<\Delta$ (poslední
podmínka se postará o to  aby v $\Delta$-intervalu bylo {\em aspoň jedno} řešení). Nechť nyní
\begin{equation}
\ve{F}(\ve{x},\ve{y})=\ve{o},\qtq{a} \nr{\ve{x}-\ve{x}^0}<\delta\ \text{and}\ \nr{\ve{y}-\ve{y}^0}<\Delta. \tag{$**$}
\end{equation}
Musíme dokázat, že $y_i=f_i(\ve{x})$ pro všechna $i$. Jelikož
$\hbox{$|x_i-x_i^0|$}<\delta\leq\delta_1$ pro $i=1,\dots,n$,\ \  
$|y_i-y_i^0|<\Delta\leq\delta_1$ pro $i=1,\dots,m$\ \  a $|y_{m+1}-y_{m+1}^0|<\Delta\leq\Delta_1$ máme nutně $y_{m+1}=\psi(\ve{x},\wt{\ve{y}})$. Tedy podle ($**$),
$$
\ve{G}(\ve{x},\wt{\ve{y}})=\ve{o}
$$
a jelikož $|x_i-x_i^0|<\delta\leq\delta_2$ a $|y_i-y_i^0|<\Delta\leq\Delta_2$ máme skutečně
$y_i=f_i(\ve{x})$. \sq

 \vskip10mm
 
  
 {\large\bf 5. Dvě jednoduché aplikace: regulární zobrazení}
 
 \bigskip
 
{\bf 5.1.}
Nechť je $U\sue \Ebb_n$ otevřená podmnožina. Buďte $f_i$, $i=1,\dots,n$, \  zobrazení se spojitými parciálními derivacemi (a tedy sama spojitá). Výsledné (spojité) zobrazení  $\ve{f}=(f_1,\dots,f_n):U\to \Ebb_n$ se nazývá {\em regulární} jestliže je
$$
\frac{\Ds(\ve{f})}{\Ds(\ve{x})}(\ve{x})\neq 0
$$
 pro všechna $\ve{x}\in U$.

\bigskip

{\bf 5.2.} Připomeňme si, že spojitá zobrazení jsou charakterisována zachováním otevřenosti (onebo uzavřenosti)  {\em vzory} (viz XIII.3.7). Také si připomeňme velmi speciální fakt (II.7.10), že je-li definiční obor  kompaktní, jsou též {\em obrazy} uzavřených množin uzavřené. Pro regulární zobrazení máme něco podobného.

\medskip


{\bf Tvrzení.} {\em Je-li zobrazení $\ve{f}:U\to\Ebb_n$ regulární je obraz  $\ve{f}[V]$ každé otevřené $V\sue U$ otevřený.

Důkaz.} Buď $f(\ve{x}^0)=\ve{y}^0$. Definujme $\ve{F}:V\times\Ebb_n\to\Ebb_n$ předpisem
\begin{equation}
F_i(\ve{x},\ve{y})= f_i(\ve{x})-y_i. \tag{$*$}
\end{equation}
Potom $\ve{F}(\ve{x}^0,\ve{y}^0)=\ve{o}$ a $\frac{\Ds(\ve{F})}{\Ds(\ve{x})}\neq 0$, a
můžeme tedy užít 4.2  a dostaneme
$\delta>0$ a $\Delta>0$ taková, že pro každé $\ve{y}$ z $\nr{\ve{y}-\ve{y}^0}<\delta$ 
existuje $\ve{x}$ takové, že
$\nr{\ve{x}-\ve{x}^0}<\Delta$
a $F_i(\ve{x},\ve{y})=f_i(\ve{x})-y_i=0$. To znamená, že
máme $\ve{f}(\ve{x})=\ve{y}$ (nenechte se zmást výměnou rolí 
 $x_i$ a $y_i$:  $y_i$ jsou zde nezávislé proměnné), and
$$
\Omega(\ve{y}^0,\delta)=\setof{\ve{y}}{\nr{\ve{y}-\ve{y}^0}<\delta}\sue \ve{f}[V].\quad\quad\square
$$



\bigskip

{\bf 5.3. Tvrzení.} {\em Buď $\ve{f}:U\to\Ebb_n$ regulární zobrazení. 
Potom pro každý bod $\ve{x}^0\in U$ existuje otevřené okolí $V$ takové, že
restrikce $\ve{f}|V$ je vzájemně jdnoznačná. Nadto, zobrazení $\ve{g}:f[V]\to\Ebb_n$ 
inversní k $\ve{f}|V$ je réž regulární.

Důkaz.} Znovu užijeme zobrazení $\ve{F}=(F_1,\dots,F_n)$ z ($*$). Pro dostatečně malé
 $\Delta>0$  máme právě jedno $\ve{x}=\ve{g}(\ve{y})$ takové, že $\ve{F}(\ve{x},\ve{y})=0$ a $\nr{\ve{x}-\ve{x}^0}<\Delta$. Toto $\ve{g}$ má, navíc, parciální derivace. Podle XIV.5.3
máme
$$
D(\id)=D(\ve{f}\circ \ve{g})=D(\ve{f})\cdot D(\ve{g}).
$$
Podle řetězového pravidla (a věty o násobení determinantů) je
$$
\frac{\Ds(\ve{f})}{\Ds(\ve{x})}\cdot\frac{D(\ve{g})}{D(\ve{y})}=\text{det}D(\ve{f})\cdot\text{det}
D(\ve{g})=1
$$
a tedy pro každé $\ve{y}\in\ve{f}[V]$, $\frac{\Ds(\ve{g})}{\Ds(\ve{y})}(\ve{y})\neq 0$. \sq

\medskip

{\bf 5.3.1. Důsledek.} {\em Prosté regulární zobrazení $\ve{f}:U\to\Ebb_n$ má regulární inversi 
$\ve{g}:\ve{f}[U]\to\Ebb_n$.}


\vskip10mm
 
  
 {\large\bf 6. Lokální extrémy a vázané extrémy.}
 
 \bigskip
 
{\bf 6.1.} Připomeňme si hledání lokálních extrémů reálných funkcí jedné reálné proměnné
  $f$ v VII.1. Pokud $f$ byla definována na intervalu $\langle a,b\rangle$ a měla derivaci  $(a,b)$ zjistili jsme snadným užitím formule  VI.1.5   že v lokálních extrémech musela být derivace nulová.
	Potom již stačilo  podívat se na hodnoty v krajních bodech $a$ a $b$ a měli jsme úplný seznam kandidátů.

Uvažme nyní lokální extrémy funkce několika reálných proměnných
Nalezení všech možných
lokálních extrémů {\em ve vnitřku jejího definičního oboru} je stejně snadné: podobně jako u funkce jedné proměnné vyvodíme z formule pro totální diferenciál  (a ani tu bychom nepotřebovali, už parciální derivace by stačily), že v bodech lokálních extrémů  $\ve{a}$ musí být
\begin{equation} 
\pad{f}{x_i}(\ve{a})=0,\ \ i=1,\dots,n.\tag{$*$}
\end{equation} 
Ale hranice (okraj) je jiná záležitost. Ta teď  typicky není složena jen z konečně mnoha isolovanych bodů
které bychom mohli probrat jeden po druhém. 


\smallskip

{\bf 6.1.1. Příklad.} Zajímejme se o lokální extr\'emy funkce
$f(x,y)=x+2y$ na kruhu $B=\setof{(x,y)}{x^2+y^2\leq 1}$. Obor $B$ je 
kompaktní, a tedy  $f$  jistě nabývá 
minima a maxima na $B$. Ve vnitřku $B$ být nemohou: máme konstantně
 $\pad{f}{x}=1$ a $\pad{f}{y}=2$; Ty extrémy tedy musí ležet někde na
nekonečné množině
$\setof{(x,y)}{x^2+y^2=1}$, a pravidlo ($*$) nám není k ničemu.



\bigskip


{\bf 6.2.} Pokusíme se tedy hledat
lokální extrémy funkce
 $f(x_1,\dots,x_n)$ {\em vázané nějakými omezeními}
$g_i(x_1,\dots,x_n)=0$, $i=1,\dots,k$.
 Platí následující
 
 \medskip
 
 {\bf Věta.} {\em Buďte $f,g_1,\dots,g_k$ reálné funkce definované na otevřené množině
$D\sue \Ebb_n$, a a nechť mají spojité parciální derivace 
 Předpokládejme, že hodnost matice
 $$
 M=\left(\begin{array}{ccc}
 \displaystyle\pad{g_1}{x_1},&\dots,&\displaystyle\pad{g_1}{x_n}\\
 \dots,&\dots,&\dots\\
 \displaystyle\pad{g_k}{x_1},&\dots,&\displaystyle\pad{g_k}{x_n}
 \end{array}\right)
 $$
 je největší možná, tedy
$k$, v každém bodě oblasti $D$. 

Nabývá-li funkce $f$ v bodě
$\ve{a}=(a_1,\dots,a_n)$ lokálního extrému vázaného podmínkami
 $$
 g_i(x_1,\dots,x_n)=0,\ \ i=1,\dots,k 
 $$
 potom existují čísla $\lambda_1,\dots,\lambda_k$ 
 taková, že pro všechna
 $i=1,\dots,n$ \  máme
 $$
 \pad{f(\ve{a})}{x_i}+\sum_{j=1}^k\lambda_j\cdot\pad{g_j(\ve{a})}{x_i}=0.
 $$}
 
 \medskip
 
 {\bf Poznámky.} 1. Funkce $f,g_i$ jsou definovány na otevřené množině
 $D$ takže můžeme hledat derivace kdekoli je potřebujeme. 
V typických aplikacích pracujeme s funkcem, které mohou být rozšířeny 
na otevřenou množinu obsahující obor o který jde.
 
 2. Síla tvrzení je v existenci
$\lambda_1,\dots,\lambda_k$ splňujících {\em víc než} $k$ rovnic.
Viz řešení úlohy 6.1.1 v 6.3.
 
 3. Čísla $\lambda_i$ jsou známa jako {\em Lagrangeovy multiplikátory}.
 
 \medskip
 
{\em Důkaz.} Z lineární algebry víme, že  $M$ má hodnost $k$ právě když některá z $k\times k$-podmatic 
matice $M$ je regulární (a tedy má nenulový determinant). 
Bez újmy na obecnosti můžeme předpokládat, že v extrémním bodě je
 \begin{equation}
\left| \begin{array}{ccc}
 \displaystyle\pad{g_1}{x_1},&\dots,&\displaystyle\pad{g_1}{x_k}\\[4ex]
 \dots,&\dots,&\dots\\[4ex]
 \displaystyle\pad{g_k}{x_1},&\dots,&\displaystyle\pad{g_k}{x_k}
 \end{array}\right|\neq 0. \tag{1}
 \end{equation}
 Platí-li toto máme podle věty o implicitních funkcích okolí bodu 
$\ve{a}$ funkce $\phi_i(x_{k+1},\dots,x_n)$  se spojitými parciálními derivacemi takové, že (píšeme $\wt{\ve{x}}$ 
místo $(x_{k+1},\dots,x_n)$)
 $$
 g_i(\phi_1(\wt{\ve{x}}),\dots,\phi_k(\wt{\ve{x}}),\wt{\ve{x}})=0\qtq{pro} i=1,\dots,k,
 $$
 Tedy, lokální minimum či maximum 
 $f(\ve{x})$ v $\ve{a}$, vázané danými omezeními, implikuje 
odpovídající extrémní vlastnost (bez omezení) pro funkci
 $$
 F(\wt{\ve{x}})=f(\phi_1(\wt{\ve{x}}),\dots,\phi_k(\wt{\ve{x}}),\wt{\ve{x}}),
 $$
 v $\wt{\ve{a}}$, a tedy je podle 5.1
 $$
 \pad{F(\wt{\ve{a}})}{x_i}=0 \qtq{pro} i=k+1,\dots,n,
 $$
takže podle řetězového pravidla
 \begin{equation}
 \sum_{r=1}^k\pad{f(\ve{a})}{x_r}\pad{\phi_r(\wt{\ve{a}})}{x_i}+\pad{f(\ve{a})}{x_i}\qtq{pro} i=k+1,\dots,n. \tag{2}
\end{equation}
Derivováním konstantních funkcí $g_i(\phi_1(\wt{\ve{x}}),\dots,\phi(\wt{\ve{x}}),\wt{\ve{x}})=0$
dostaneme pro $j=1,\dots,k$,
\begin{equation}
 \sum_{r=1}^k\pad{g_j(\ve{a})}{x_r}\pad{\phi_r(\wt{\ve{a}})}{x_i}+\pad{g_j(\ve{a})}{x_i}\qtq{pro} i=k+1,\dots,n. \tag{3}
\end{equation}
Nyní znovu užijeme (1), pro jiný účel. Z hodnosti naší čtvercové matice má systém lineárních rovnic
$$
 \pad{f(\ve{a})}{x_i}+\sum_{j=1}^n\lambda_j\cdot\pad{g_j(\ve{a})}{x_i}=0, \quad i=1,\dots,k,
 $$ 
 jednoznačné řešení $\lambda_1,\dots,\lambda_k$. Toto jsou rovnice z tvrzení, ale zatím jen pro $i\leq k $. Zbývá dokázat, že tytéž rovnosti platí i pro
 $i>k$. Skutečně, podle (2)  (3), 
pro $i>k$ dostaneme
 $$
 \begin{aligned}
 \pad{f(\ve{a})}{x_i}&+\sum_{j=1}^n\lambda_j\cdot\pad{g_j(\ve{a})}{x_i}=
 -\sum_{r=1}^k\pad{f(\ve{a})}{x_r}\pad{\phi_r(\wt{\ve{a}})}{x_i}
 -\sum_{j=1}^n\lambda_j\sum_{r=1}^k\pad{g_j(\ve{a})}{x_r}\pad{\phi_r(\wt{\ve{a}})}{x_i}=\\
&-\sum_{r=1}^k\left(\pad{f(\ve{a})}{x_r}+\sum_{j=1}^n\lambda_j\cdot\pad{g_j(\ve{a})}{x_r}\right)\pad{\phi_r(\wt{\ve{a}})}{x_i}=-\sum_{r=1}^k\cdot\pad{\phi_r(\wt{\ve{a}})}{x_i}=0.\quad\square
 \end{aligned}
 $$

 
 \bigskip
 
 {\bf 6.3. Řešení příkladu z 6.1.1.} 
 Máme $\pad{f}{x}=1$ a $\pad{f}{y}=2$, $g(x,y)=x^2+y^2-1$ a tedy $\pad{g}{x}=2x$ and $\pad{g}{y}=2y$. Existuje {\em jedno} $\lambda$ které splňuje {\em dvě} rovnice
 $$
 1+\lambda\cdot2x=0 \qtq{a}  2+\lambda\cdot2y=0.
 $$
 To je možné jen pokud  $y=2x$. Tedy, jelikož $x^2+y^2=1$, dostáváme $5x^2=1$ a tedy $x=\pm\frac{1}{\sqrt 5}$; to lokalisuje extrémy do $(\frac{1}{\sqrt 5},\frac{2}{\sqrt 5})$ a
 $(\frac{-1}{\sqrt 5}\frac{-2}{\sqrt 5})$.
 
 \bigskip
 
 {\bf 6.4.} Vazby $g_i$ nemusí nutně vznikat z okrajů. 
Zde je jednoduchý příklad jiného typu.
 
 Ptáme se, který z pravoúhelných hranolů s daným povrchem
má největší objem.
Označíme-li délky hran $x_1,\dots,x_n$, je povrch dán formulí
 $$
 S(x_1,\dots,x_n)=2x_1\cdots x_n\left(\frac1{x_1}+\cdots+\frac{1}{x_n}\right)
 $$
 a objem je
 $$
 V(x_1,\dots,x_n)=x_1\cdots x_n.
 $$
Tedy
 $$
 \begin{aligned}
 &\pad{V}{x_i}=\frac1{x_i}\cdot x_1\cdots x_n\quad \text{a}\\ 
 &\pad{S}{x_i}=\frac2{x_i}(x_1\cdots x_n)\left(\frac1{x_1}+\cdots+\frac{1}{x_n}\right)-2x_1\cdots x_n\frac1{x^2_i}.
 \end{aligned}
 $$
Označíme-li $y_i=\frac1{x_i}$ a $s=y_1+\cdots+y_n$, a vydělíme-li rovnost z věty 
 $x_1\cdots x_n$, dostaneme
 $$
 2y_i(s-y_i)+\lambda y_i=0 \qtq{z čehož} y_i=s+\frac{\lambda}{2}.
 $$
 Tedy všechna $x_i$ jsou stejná, a řešení je krychle.

 
\newpage

. 
 
\newpage

 \centerline{\Large\bf XVI. Riemannův integrál ve více proměnných} 
 
 
 \vskip10mm
 
 
 Myšlenka Riemannova integrálu ve více proměnných je táž jako u integrálu v jedné reálné proměnné. Jediný rozdíl je v tom, že budme pracovat s $n$-rozměrnými intervaly místo s těmi standardními, a že rozklady budou dělit tyto intervaly ve všech dimensích, takže výsledné intervaly rozkladů nebudou tak přehledně uspořádány jako
 malé intervaly v $\langle t_0,t_1\rangle$,$\langle t_1,t_2\rangle$, $\dots$ .
 Ale konečný součet je konečný součet a uvidíme, že to uspořádání není moc důležité.
 
 \smallskip
 
 Opravdu nová bude Fubiniho věta (Sekce 4) umožňující počítat vícerozměrný integrál použitím integrálů v jedné proměnné. Až do toho bude vše co uděláme jen modifikace fakt z kapitoly XI.
 
 
 
 
 \vskip10mm
 
  
 {\large\bf 1. Intervaly a rozklady.}
 
 \bigskip
 
{\bf 1.1.} V této kapitole je $n$-rozměrný kompaktní interval}  součin
$$
J=\langle a_1,b_1\rangle\times\cdots\times\langle a_n,b_n\rangle
$$
(takový $J$ je skutečně kompaktní, viz  XIII.7.6); nebude-li nebezpečí nedorozumnění budeme mluvit prostě o
 {\em intervalu}. Někdy též mluvíme o  {\em cihlách}, zvlášť půjde-li o části větších intervalů.

\smallskip
 {\em Rozklad} intervalu $J$ je soustava $P=(P^1,\dots,P^n)$ rozkladů
$$
 P^j:\ a_j=t_{j0}< t_{j1}<\cdots<t_{j,{n_j-1}}< t_{j,{n_j}}=b_j, \quad j=1,\dots n.
 $$
Intervaly
 $$
 \langle t_{1,i_1},t_{1,{i_1+1}}\rangle\times\cdots\times\langle t_{n,i_n},t_{n,{i_n+1}}\rangle
 $$
 budou nazývány {\em cihlami} rozkladu $P$ a množina všech cihel rozkladu $P$ bude označována
 $$
 \BC(P).
  $$
 
 
 \bigskip
 
 {\bf 1.2.} Objem intervalu 
$
J=\langle a_1,b_1\rangle\times\cdots\times\langle a_n,b_n\rangle
$ 
je číslo
$$
\vol(J)=(b_1-a_1)(b_2-a_2)\cdots(b_n-a_n).
$$

\medskip

Jelikož různé cihly z $\BC(P)$ se zřejmě protínají v množině o objemu 0 (užijte XI.1 aplikované na obrazce ne nutně rovinné)
máme hned  

\smallskip

{\bf 1.2.1. Pozorování.} {\em $\vol(J)=\sum\setof{\vol(B)}{B\in\BC(J)}$.}

\bigskip

{\bf 1.3. Jemnost rozkladu.} Průměr intervalu
 $
J=\langle a_1,b_1\rangle\times\cdots\times\langle a_n,b_n\rangle
$
je
$$
\diam(J)=\max_i(b_i-a_i)
$$
a {\em jemnost} rozkladu $P$
je
$$
\mu(P)=\max\setof{\diam(B)}{B\in\BC(P)}.
$$

\bigskip


{\bf 1.4. Zjemnění.} Vzpomeňte si na XI.2.2, Rozklad $Q=(Q^1,\dots,Q^n)$ {\em zjemňuje}
rozklad $P=(P^1,\dots,P^n)$ pokud každý $Q^j$ zjemňuje $P^j$.

Když vezmeme v úvahu úsečky $t_{j,{k-1}}=t'_{j,l}< t_{j,{l+1}}'<\cdots< t'_{j,{l+r}}=t_{j,k}$ jemnějšího  rozkladu $Q$ dostaneme
 
 \medskip
 
 {\bf 1.4.1. Pozorování.} {\em Zjemnění $Q$ rozkladu $P$ indukuje rozklady
 $$
 Q_B\qtq{cihel} B\in \BC(P)
 $$
 a máme disjunktní sjednocení
 $$
 \BC(Q)=\bigcup\setof{\BC(Q_B)}{B\in\BC(P)}.
 $$}
 
 \bigskip
 
 {\bf 1.5. Pozorování.} {\em Každá dvě rozdělení $P,Q$  $n$-rozměrného kompaktního intervalu $J$ mají
 společné zjemnění.}
 
 (Skutečně, připomeňte si důkaz XI.2.3.2. Jsou-li $P=(P^1,\dots,P^n)$ a  $Q=(Q^1,\dots,Q^n)$ rozklady intervalu $J$
 vezměme rozklad  $R=(R^1,\dots,R^n)$ v němž $R^j$ jsou společná zjemnění $P^j$ a $Q^j$.)
 
 \vskip10mm
 
  
 {\large\bf 2. Dolní a horní součty.}
 
  \medskip
 
 \hskip7mm {\large\bf Definice Riemannova integrálu.}
 
 \bigskip
 
{\bf 2.1.} Buď $f$  omezená reálná funkce na $n$-rozměrném  kompaktním intervalu  $J$ a buď $B\sue J$ nějaký $n$-rozměrný kompaktní podinterval $J$ (cihla). Položme
$$
 m(f,B)=\inf\setof{f(\ve{x})}{\ve{x}\in B} \qtq{a}    M(f,B)=\sup\setof{f(\ve{x})}{\ve{x}\in B}.
 $$
 \smallskip
 
 Máme
 
 \smallskip
 
 {\bf 2.1.1. Fakt.}  {\em $m(f,B)\leq M(f,B)$  a je-li $C\sue B$ potom
 $$
 m(f,C)\geq m(f,B) \qtq{a} M(f,C)\leq M(f,B).
 $$}
 ($\setof{f(\ve{x})}{\ve{x}\in C}$ je podmnožina $\setof{f(\ve{x})}{\ve{x}\in B}$ a  tedy každá dolní (horní) mez druhé je dolní (horní) mez první.)
 
 \bigskip
 
 {\bf 2.2.} Buď $P$ rozklad intervalu $J$ a buď $f:J\to \Rbb$ omezená funkce. Položme
 $$
 \begin{aligned}
 &s_J(f,P)=\sum\setof{m(f,B)\cdot\vol(B)}{B\in\BC(P)}\ \ \text{a}\\ &S_J(f,P)=\sum\setof{M(f,B)\cdot\vol(B)}{B\in\BC(P)}.
 \end{aligned}
 $$
 Index $J$ budeme obvykle vynechávat.
 
 \medskip
 
 {\bf 2.2.1. Tvrzení.} {\em Nechť rozklad $Q$ zjemňuje $P$. Potom
 $$
 s(f,Q)\geq s(f,P) \qtq{a} S(f,Q)\leq S(f,P).
 $$
 
 Důkaz.} Máme (nad symboly $=$ či $\leq$ vyznačujeme které tvrzení z předchozího užíváme)
 $$ 
\begin{aligned} 
S(f,Q)&=\sum\setof{M(f,C)\cdot\vol(C)}{C\in\BC(Q)}\stackrel{1.4.1}{=}\\
&\stackrel{1.4.1}{=}\sum\setof{M(f,C)\cdot\vol(C)}{C\in\text{(disjunktní)}\bigcup\setof{\BC(Q_B)}{B\in\BC(P)}}=\\
&=\sum\setof{\sum\setof{M(f,C)\cdot\vol(C)}{C\in\BC(Q_B)}}{B\in\BC(P)}\stackrel{2.1.1}{\leq}\\
&\stackrel{2.1.1}{\leq}\sum\setof{\sum\setof{M(f,B)\cdot\vol(C)}{C\in\BC(Q_B)}}{B\in\BC(P)}=\\
&=\sum\setof{M(f,B)\sum\setof{\vol(C)}{C\in\BC(Q_B)}}{B\in\BC(P)}\stackrel{1.2.1}{=}\\
&\stackrel{1.2.1}{=}\sum\setof{M(f,B)\cdot\vol(B)}{B\in\BC(P)}= S(f,P).
\end{aligned}
$$
Podobně pro $s(f,Q)$.\sq

\bigskip

{\bf 2.2.2. Tvrzení.} {\em Buďte $P,Q$ rozklady $J$. Platí $s(f,P)\leq S(f,Q)$.

Důkaz.} Pro společné zjemnění $R$ těch $P,Q$ (viz 1.5) máme podle 2.2.1
$$
s(f,P)\leq s(f,R)\leq S(f,R)\leq S(fQ).
$$\sq

\bigskip

{\bf 2.3.}  Podle  2.2.2 je množina  $\setof{s(f,P)}{P\ \text{rozklad}}$ shora omezená a můžeme tedy definovat {\em dolní Riemannův integrál} funkce $f$ přes $J$ jako
$$
\lint_Jf(\ve{x})\d\ve{x}=\sup\setof{s(f,P)}{P\ \text{rozklad}};
$$ 
podobně,
množina $\setof{S(f,P)}{P\ \text{rozklad}}$ je omezená zdola a můžeme definovat {\em horní Riemannův integrál}  funkce $f$ přes  $J$ jako
$$
\uint_Jf(\ve{x})\d\ve{x}=\inf\setof{S(f,P)}{P\ \text{rozklad}}.
$$ 
Rovnají-li se dolní a horní integrál nazýváme společnou hodnotu {\em  Riemannův integrál funkce $f$ přes $J$} a
označujeme ji
$$
\int_Jf(\ve{x})\d\ve{x} \qtq{nebo prostě} \int_Jf
$$

\medskip

{\bf 2.3.1. Poznámka.} Integrál také můžeme psát např.  jako
$$
\int_Jf(x_1,\dots,x_n)\d x_1,\dots x_n
$$
 což jistě nepřekvapuje. Čtenář se též může setkat s
$$
\int_Jf(x_1,\dots,x_n)\d x_1\d x_2\cdots\d x_n.
$$
To může vypadat podivně, ale dává to více smyslu než je na první pohled patrno. Viz dále v 4.2.

\bigskip

{\bf 2.4.} Máme zřejmý odhad
$$
\inf\setof{f(\ve{x})}{\ve{x}\in J}\cdot\vol(J)\leq\lint_Jf\leq\uint_Jf\leq\sup\setof{f(\ve{x})}{\ve{x}\in J}\cdot\vol(J).
$$


 
 \newpage
 
  
 {\large\bf 3. Spojitá zobrazení.}
 
 \bigskip
 
{\bf 3.1.
 Tvrzení.} {\em Riemannův integrál $\int_Jf(\ve{x})\d\ve{x}$ existuje právě když pro každé $\varepsilon >0$ existuje rozklad $P$ takový, že
  $$
  S_J(f,P)-s_J(f,P)<\varepsilon.
  $$
 }
 
\medskip

{\bf Poznámka místo důkazu.} Tvrzení můžeme dokázat opakováním důkazu z  XI.2.4.2. Ale čtenář si zde může uvědomit, že spíše než snadné zobecnění věty
 IX.2.4.2, jsou obě tvrzení  speciální případy obecného jednoduchého tvrzení o supremech a infimech.
Mějme množinu $(X,\leq)$ částečně uspořádanou relací $\leq$ takovou, že pro každá $x,y\in X$ existuje $z\leq x,y$. Máme-li $\alpha:X\to\Rbb$ takové, že $x\leq y$ implikuje $\alpha(x)\geq\alpha(y)$ a $\beta:X\to\Rbb$ takové, že $x\leq y$ implikuje $\beta(x)\leq\beta(y)$, a je-li $\alpha(x)\leq\beta(y)$ pro všechna  $x,y$ potom $\sup_x\alpha(x)=\inf_x\beta(x)$ právě když pro každé $\varepsilon>0$ existuje $x$ takové, že $\beta(x)<\alpha(x)+\varepsilon$. To je triviální tvrzení o  uspořádání, které nemá co dělat se součty a takovými věcmi.  Ale to kriterium je samozřejmě velmi užitečné.

\bigskip


{\bf 3.2.} K důkazu následující věty opět užijeme stejnoměrnou spojitost spojité funkce na kompaktním prostoru (nyní v té obecnější versi z XIII.7.11).

\medskip


 {\bf Věta.} {\em Pro každou spojitou funkci  $f:J\to\Rbb$ na  $n$-rozměrném kompaktním intervalu existuje
$\int_Jf$.

Důkaz.} Budeme užívat metriku $\sigma$ v $\Ebb_n$ definovanou předpisem
$$
\sigma(\ve{x},\ve{y})=\max_i|x_i-y_i|.
$$ 
Jelikož je $f$ stejnoměrně spojitá můžeme pro $\varepsilon>0$ zvolit $\delta>0$ takové, že
$$
\sigma(\ve{x},\ve{y})< \delta \quad\Rightarrow\quad |f(\ve{x})-f(\ve{y})|< \frac{\varepsilon}{\vol(J)}.
$$
Připomeňme si  mesh $\mu(P)$ z 1.3.   Pokud je $\mu(P)<\delta$ je $\diam(B)<\delta$ pro všechny $B\in\BC(P)$
a tedy
$$
\begin{aligned}
M(f,B)-m(f,B)&=\sup\setof{f(\ve{x})}{\ve{x}\in B}-\inf\setof{f(\ve{x})}{\ve{x}\in B}\leq
\\
 &\leq\sup\setof{|f(\ve{x})-f(\ve{y})|}{\ve{x},\ve{y}\in B}<\frac{\varepsilon}{\vol(J)}
 \end{aligned}
 $$
takže
 $$
 \begin{aligned}
 S(f,P)-s(f,P)&=\sum\setof{(M(f,B)-m(f,B))\cdot\vol(B)}{B\in\BC(P)}\leq\\
&\leq\frac{\varepsilon}{\vol(J)}\sum\setof{\vol(B)}{B\in\BC(P)}=\frac{\varepsilon}{\vol{J}}\vol(J)=\varepsilon
 \end{aligned}
 $$
 podle 1.2.1. 
Nyní užijme 3.1. \sq
 
 \medskip
 
 {\bf 3.2.1.} Podobně jako v XI.3.2.1  dostaneme z předchozího důkazu též následující tvrzení.
 
 \smallskip
 
 {\bf Věta.} {\em  Nechť je   $f:J\to\Rbb$ spojitá funkce a $P_1,P_2,\dots$ posloupnost rozkladů taková, že 
 $\lim_n\mu(P_n)=0$. Potom
 $$
 \lim_n s(f,P_n)=\lim_n S(f,P_n)=\int_J f.
 $$}
 
 (S čísly $\varepsilon$ a $\delta$ jako nahoře zvolme $n_0$ takové, že pro $n\geq n_0$ bude $\mu(P_n)<\delta$.)
 
 \medskip

{\bf 3.2.2. Důsledek.} {\em Buď   $f:J\to\Rbb$ spojitá funkce na  $n$-rozměrném kompaktním
 intervalu $J$. Pro každou cihlu $B\sue J$ zvolme prvek $\ve{x}_B\in B$ a pro rozklad $P$ intervalu $J$
definujme
 $$
 \Sigma(f,P)=\sum\setof{f(\ve{x}_B)\cdot\vol(B)}{B\in\BC(P)}.
 $$
Buď $P_1,P_2,\dots$ posloupnost rozkladů taková, že
 $\lim_n\mu(P_n)=0$. Potom je
 $$
 \lim_n \Sigma(f,P_n)=\int_J f.
 $$}
 

\vskip10mm
 
  
 {\large\bf 4. Fubiniho věta.}
 
 \bigskip
 
{\bf 4.1. Věta.} {\em Vezměme
 součin $J=J'\times J''\sue\Ebb_{m+n}$
 intervalů $J'\sue\Ebb_m$, $J''\sue\Ebb_n$. Buď   $f:J\to\Rbb$ taková, že
  $\int_Jf(\ve{x},\ve{y})\d\ve{x}\ve{y}$ existuje a že pro každý $\ve{x}\in J'$ 
  (resp. každý $\ve{y}\in J''$) integrál $\int_{J''}f(\ve{x},\ve{y})\d\ve{y}$ (resp.
  $\int_{J'}f(\ve{x},\ve{y})\d\ve{x}$) existuje (což platí zejména pro každou spojitou funkci).
	Potom
 $$
 \int_Jf(\ve{x},\ve{y})\d\ve{x}\ve{y}=\int_{J'}(\int_{J''}f(\ve{x},\ve{y})\d\ve{y})\d\ve{x}=
 \int_{J''}(\int_{J'}f(\ve{x},\ve{y})\d\ve{x})\d\ve{y}.
 $$

 
 Důkaz.} Budeme zkoumat první rovnost, druhá je obdobná.
 Položme
 $$
 F(\ve{x})=\int_{J''}f(\ve{x},\ve{y})\d\ve{y}.
 $$
Dokážeme, že $\int_{J'}F$ existuje a že
$$
 \int_Jf=\int_{J'}F.
 $$
  Zvolme rozklad $P$ intervalu $J$ tak aby
 $$
 \int f-\varepsilon\leq s(f,P)\leq S(f,P)\leq \int f+\varepsilon.
 $$
 Tento rozklad $P$ zřejmě sestává z rozkladu
$P'$ intervalu $J'$ a rozkladu $P''$ intervalu $J''$.  Máme
 $$
 \BC(P)=\setof{B'\times B''}{B'\in \BC(P'), B''\in \BC(P'')},
 $$
 a každá cihla z  $P$ se objeví jako právě jeden součin $B'\times B''$. Podle 2.4 je
 $$
 F(\ve{x})\leq\sum_{B''\in\BC(P'')}\max_{\ve{y}\in B''}f(\ve{x},\ve{y})\cdot\vol B''
 $$
 a tedy
 $$
 \begin{aligned}
 S(F,P')&\leq \sum_{B'\in\BC(P')}\max_{\ve{x}\in B'}\ (\sum_{B''\in \BC(P'')}\max_{\ve{y}\in B''}f(\ve{x},\ve{y})\cdot\vol(B''))\cdot\vol(B')\leq\\
 &\leq \sum_{B'\in\BC(P')}\sum_{B''\in \BC(P'')}\max_{(\ve{x},\ve{y})\in B'\times B''}f(\ve{x},\ve{y})\cdot\vol (B'')\cdot\vol(B')\leq\\
 &\leq \sum_{B'\times B''\in\BC(P)}\max_{\ve{z}\in B'\times B''}f(\ve{z})\cdot\vol(B'\times B'')=
 S(f,P)
 \end{aligned}
 $$
 a podobně
 $$
 s(f,P)\leq s(F,P').
 $$
 Máme tedy
 $$
 \int_jf-\varepsilon\leq s(F,P')\leq{\int}_{J'}F\leq S(F,P)\leq\int_Jf+\varepsilon
 $$
 a proto je $\int_{J'}F$  roven $\int_Jf$. \sq
 
 \bigskip
 
 {\bf 4.2. Důsledek.} {\em Buď $f:J=\langle a_1,b_1\rangle\times\cdots\times\langle a_n,b_n\rangle\to\Rbb$ be spojitá funkce. Potom
 $$ \int_Jf(\ve{x})\d\ve{x}=\int_{a_n}^{b_n}(\cdots(\int_{a_2}^{b_2}(\int_{a_1}^{b_1}f(x_1,x_2,\dots,x_n)\d x_1)\d x_2)\cdots)\d x_n.
 $$}
 
 \medskip
 
 {\bf Poznámka.} Značení o kterém jsme se zmínili v 2.3 pochází, samozřejmě, z vynechání
závorek.
 

 
\newpage

 
 \centerline{\huge\bf Třetí semestr} 
 
 \vskip10mm
 
 \centerline{\Large\bf XVII. Více o  metrických prostorech} 
 
 \vskip10mm
 
 
 \def\d{\text{\rm d}}
 
 \def\sint{\text{(II)}\!\!\!\int}    \def\fint{\text{(I)}\int}
 
 
 
 {\large\bf 1. Separabilita a spočetné base.}
 
  \medskip
	
 {\bf 1.1. Hustota.} Vzpomeňte si na uzávěr z XIII.3.6.  Podmnožina $M$ metrického prostoru $(X,d)$ je  {\em hustá} je-li $\ol M=X$. Jinými slovy, $M$ je hustá pokud pro každé $x\in X$ a každé $\epsilon>0$ existuje $m\in M$ takové, že $d(x,m)<\epsilon$.
 
 \bigskip
 
 {\bf 1.2. Separabilní prostory.} Řekneme, že metrický prostor $(X,d)$ je {\em separabilní} existuje-li v něm spočetná hustá podmnožina $M\sue X$.
 
 \bigskip
 
 {\bf 1.3. Base otevřených množin.} Podmnožina $\BC$ množiny $\open(X,d)$ všech otevřených množin v $(X,d)$ se nazývá {\em basí} (otevřených množin) je-li každá otevřená množina sjednocením množin z $\BC$, tedy jestliže
 $$
 \forall U\in\open(X)\ \exists \BC_U\sue\BC \ \ \text{such that}\ \ U=\bigcup\setof{B}{B\in\BC_U}.
 $$
 Jinými slovy,
 $$
 \forall U\in\open(X)\ \ U=\bigcup\setof{B}{B\in\BC_U, \ B\sue U}.
 $$
 
 \medskip
 
 {\bf 1.3.1. Poznámky.}
 1. Tak např. množina všech otevřených intervalů $(a,b)$, nebo již množina všech intervalů $(a,b)$ s racionálními $a,b$ je base (otevřených množin) reálné přímky $\Rbb$.
 
 2. V každém metrickém prostoru je
 $$
 \setof{\Omega(x,\frac1n)}{x\in X,\ n=1,2,\dots}
 $$
 (viz XIII.3.2) base.
 
 3. Termín ``base'' je v jistém nesouhlasu se svým homonymem z lineární algebry. U base otevřených množin není žádná minimalita ani nezávislost. Pojem je spíše příbuzný s pojmem soustavy generátorů.
 
 \bigskip
 
 {\bf 1.4. Pokrytí.}  {\em Pokrytí} prostoru $(X,d)$ je podmnožina $\UC\sue\open(X,d)$ taková, že $\bigcup\setof{U}{U\in\UC}=X$.  {\em Podpokrytí} $\VC$ pokrytí $\UC$ je podmnožina $\VC\sue\UC$ taková, že je (ještě)
 $\bigcup\setof{U}{U\in\VC}=X$.
 
 \medskip
 
 {\bf Poznámka.} Přesněji bychom měli mluvit o {\em otevřených pokrytích}. Ale o jiných pokrytích než pokrytích otevřenými množinami zde mluvit nebudeme.
 
 \bigskip
 
 {\bf 1.5. Lindel\"ofova vlastnost, Lindel\"ofovy prostory.} Řekneme, že prostor $X=(X,d)$ je {\em Lindel\"ofův} nebo že má (splňuje) {\em Lindel\"ofovu vlastnost} má-li každé pokrytí prostoru $X$ spočetné podpokrytí.
 
 \bigskip
 
 {\bf 1.6. Věta.} {\em Následující tvrzení o metrickém prostoru $X$ jsou ekvivalentní.
 \begin{enumerate}
 \item $X$ je separabilní.
 \item $X$ má spočetnou basi.
 \item $X$ má Lindel\"ofovu vlastnost.
 \end{enumerate}
 
 Důkaz.} (1)$\Rightarrow$(2): Nechť je $X$  separabilní; buď $M\sue X$ spočetná hustá. Položme
 $$
 \BC=\setof{\Omega(m,r)}{m\in M,\ r\ \text{racionální}}.
 $$
 $\BC$ je zřejmě spočetná; dokážeme, že je to base. 
 
 Buď $U$ otevřená a $x\in U$. Potom existuje $\epsilon>0$ takové, že $\Omega(x,\epsilon)\sue U$. Zvolme $m_x\in M$ a racionální $r_m$ takové, že $d(x,m_x)<\frac13 \epsilon$ a že $\frac13\epsilon<r_x<\frac23\epsilon$. Potom
 $$
 x\in\Omega(m_x,r_x)\sue\Omega(x,\epsilon)\sue U.
 $$
 Skutečně, $x\in\Omega(m_x,r_x)$  triviálně, a je-li $y\in \Omega(m_x,r_x)$ potom $d(x,y)\leq d(x,m_x)+d(m_x,y)< 
 \frac13\epsilon+\frac23\epsilon= \epsilon$. Je tedy $U=\bigcup\setof{\Omega(m_x,r_x)}{x\in U}$.
 
 \smallskip
 
 (2)$\Rightarrow$(3): Buď $\BC$ spočetná base a $\UC$ pokrytí $X$.  Jelikož je
 $U=\bigcup\setof{B}{B\in \BC, B\sue U}$ pro každé $U\in\UC$ máme
 $$
 X=\bigcup\setof{B\in\BC}{\exists U_B\supe B, \ U_B\in U}.
 $$
 Pokrytí $\AC=\setof{B\in\BC}{\exists U_B\supe B, \ U_B\in U}$ je spočetné, a takové je tedy i pokrytí
 $\VC=\setof{U_B}{B\in\AC}$. 
 
 \smallskip
 
 (3)$\Rightarrow$(1): Buď $X$ Lindel\"ofův. Pro pokrytí
 $$
 \UC_n=\setof{\Omega(x,\frac1n)}{x\in X}
 $$
 zvolme spočetná podpokrytí
 $$
 \Omega(x_{n1},\frac1n),\Omega(x_{n2},\frac1n),\dots,\Omega(x_{nk},\frac1n),\dots \ .
 $$
 Potom je $\setof{x_{nk}}{n=1,2,\dots,k=1,2,\dots}$ 
hustá. \sq
 
 \bigskip
 
 {\bf 1.7. Poznámky.} 1. Často se pracuje s prostory obecnějšími než metrickými. V těch nejstandardnějších, v
 {\em topologických prostorech}, je stanoveno co jsou otevřené či uzavřené množiny, okolí, atd., aniž by musely být konstruovány z nějaké předem dané vzdálenosti
 (ve skutečnosti často taková data vůbec nejsou založena na vzdálenosti). Všechny pojmy o nichž v této sekci mluvíme mají smysl v tak zobecněném kontextu, ale jejich vztahy nejsou stejné. Vlastnost (2) (existence spočetné base) obecně implikuje separabilitu i Lindel\"ofovu vlastnost, ale žádná z ostatních implikací obecně neplatí.
 
 2. Všimněte, že existence spočetné base se dědí na každém podprostoru (viz XIII.3.4.3), takže platí (pro metrické prostory) též že
 \begin{itemize}
 \item {\em každý podprostor separabilního prostoru je separabilní}, a
 \item {\em každý podprostor Lindel\"ofova prostoru je Lindel\"ofův.}
 \end{itemize}
Zvláště druhé tvrzení je trochu překvapující (viz sekci 3 a podobnou charakteristiku kompaktnosti která se dědí jen  na uzavřených podprostorech).
 


 
 \vskip10mm
 
 {\large\bf 2. Totálně omezené metrické prostory.}
 
 \bigskip
 
 {\bf 2.1.} Metrický prostor  $(X,d)$ je {\em totálně omezený} jestliže
 $$
 \forall \epsilon>0\ \exists\ \text{konečná}\ M(\epsilon)\ \ \text{taková, že}\ \ \forall\ x\in X,\ d(x,M(\epsilon))<\epsilon.
 $$
Zřejmě platí, že
 
 \smallskip
 
 \centerline{\em každý totálně omezený prostor je omezený (viz XIII.7.4)}
 
 \smallskip
 
 \noindent (pro kažé dva $x,y\in X$ je $d(x,y)\leq \max\setof{d(a,b)}{a,b\in M(1)}+2$)  ale ne každý omezený prostor je totálně omezený:  vezměte nekonečný $X$ s $d(x,y)=1$ pro $x\neq y$).
 
 \medskip
 
 {\bf 2.1.1. Pozorování.} {\em Totální omezenost (a zrovna tak prostá omezenost) se zachovává nahradime-li metriku metrikou silně ekvivalentní (viz XIII.4) ale není to topologická vlastnost.}
 
 (Pro druhé  tvrzení vezměte interval $(a,b)$ a celou reálnou přímku $\Rbb$ a připomeňte si XIII.6.8.)
 
 \bigskip
 
 {\bf 2.2. Tvrzení.} {\em Podprostor totálně omezeného prostoru $(X,d)$ je totálně omezený.
 
 Důkaz.} Vezměme $Y\sue X$. Pro $\epsilon>0$ vezměte $M(\frac{\epsilon}{2})\sue X$ z definice a množinu
 $$
 M_Y=\setof{a\in M(\frac{\epsilon}2)}{\exists y\in Y, \ d(a,y)<\frac{\epsilon}2}.
 $$
Pro každé $a\in M_Y$ zvolme nyní $a_Y\in Y$ tak, aby $d(a,a_Y)<\frac{\epsilon}2$ a položme
 $$
 N(\epsilon)=\setof{a_Y}{a\in M_Y}.
 $$
 Potom pro každé $y\in Y$ máme $d(y,N(\epsilon))<\epsilon$.\sq
 
 \bigskip
 
 {\bf 2.3. Tvrzení.} {\em Součin $X=\prod_{j=1}^n(X_j,d_j)$ totálně omezených prostorů je totálně omezený.
 
 Důkaz.} Pro součin použijme vzdálenost $d$ z XIII.5. Vezmeme-li potom pro $X_i$ množinu $M_i(\epsilon)$ z definice bude mít $M(\epsilon)=\prod M_i(\epsilon)$ vlastnost požadovanou pro
$X$.\sq

\bigskip

{\bf 2.4. Tvrzení.} {\em Podprostor euklidovského prostoru $\Ebb_n$ je totálně omezený právě když je omezený.

Důkaz.} Vzhledem k 2.2. a 2.3 stačí dokázat, že interval $\langle a,b\rangle$ je totálně omezený. To je ale snadné: pro $\epsilon>0$ vezměme $n$ takové, že $\frac{b-a}{n}<\epsilon$ a položme
$$
M(\epsilon)=\setof{a+k\frac{b-a}{n}}{k=0,1,2,\dots}.
$$\sq

\bigskip

{\bf 2.5. Charakteristika totální omezenosti která připomíná kompaktnost.}
 
\medskip

 {\bf 2.5. Lemma.} {\em Není-li $(X,d)$ totálně omezený, obsahuje posloupnost která nemá žádnou Cauchyovskou podposloupnost.

Důkaz.} Není-li $(X,d)$ totálně omezený, existuje $\epsilon_0>0$ takové, že pro každou konečnou $M\sue X$ existuje $x_M\in X$ takové, že $d(x_M,M)\geq \epsilon_0$. Zvolme $x_1$ libovolně a máme-li již $x_1,\dots,x_n$ zvoleny
položme $x_{n+1}=x_{\set{x_1,\dots,x_n}}$. Potom každé dva členy výsledné posloupnosti jsou od sebe vzdáleny aspoň $\epsilon_0$ a tedy Cauchyovská podposloupnost neexistuje.\sq


\medskip

{\bf 2.5.2. Věta.} {\em Metrický prostor  $X$ je totálně omezený právě když každá posloupnost v $X$ obsahuje Cauchyovskou podposloupnost.

Důkaz.} Buď $(x_n)_n$ posloupnost v totálně omezeném $(X,d)$. Vezměme
$$
M(\frac1n)=\set{y_{n1},\dots,y_{nm_n}}
$$
z definice. Je-li $A=\setof{x_n}{n=1,2,\dots}$ konečná potom $(x_n)_n$ obsahuje konstantní podposloupnost. Předpokládejme tedy, že  $A$  konečná není.
Existuje $r_1$  takové, že $A_1=A\cap\Omega(y_{1r_1},1)$ je nekonečná; zvolme $x_{k_1}\in A_1$. máme-li již nekonečné 
$$
A_1\supe A_2\supe\cdots\supe A_s, \quad A_j\sue\Omega(y_{jr_j},\frac1j)
$$
a 
$$
k_1<\dots<k_s \qtq{such that} x_{k_j}\in A_j
$$
zvolme $r_{s+1}$ takové že $A_{s+1}=A_s\cap\Omega(y_{s+1,r_{s+1}},\frac1{s+1})$ je nekonečná, a $x_{k_{s+1}}\in A_{s+1}$
takové, že $k_{s+1}>k_s$. Potom je podposloupnos $(x_{k_n})_n$ Cauchyovská.

Opačná implikace je v 2.5.1. \sq



\bigskip

{\bf 2.6. Věta.} {\em Metrický prostor je kompaktní právě když je totálně omezený a úplný.

Důkaz.} Je-li $X$ kompaktní, je úplný podle XIII.7.7  a totálně omezený podle 2.5.1. 

Na druhé straně, buď $X$ totálně omezený buď  $(x_n)_n$ posloupnost v $X$. Ta obsahuje Cauchyovskou podposloupnost, a je-li $X$ navíc úplný, je to podposlounost konvergentní. \sq

\medskip

{\bf 2.6.1. Poznámky.} 1. Známe charakteristiku kompaktního podprostoru $\Ebb_n$ jako omezeného a uzavřeného (viz XIII.7.6).
Uvědomte si, že je to speciální případ věty 2.6: podmnožina $\Ebb_n$ je úplná právě když je uzavřená (viz XIII.6.6  a XIII.6.4), a je totálně omezená právě když je omezená (viz 2.4).

\smallskip

2. Všimněte si, že ani úplnost ani totální omezenost nejsou topologické vlastnosti, zatímco jejich konjunkce je.

\bigskip

{\bf 2.7. Tvrzení.} {\em Totálně omezený prostor je separabilní.

Důkaz.} Vezměme opět množiny $M(\epsilon)$ z definice. Množina
$$
\bigcup_{n=1}^\infty M(\frac1n)
$$ 
je spojitá a zřejmě hustá. \sq

\medskip

{\bf 2.7.1. Důsledek.} {\em Každý kompaktní prostor je separabilní a tedy Lindel\"ofův.}
\bigskip



\vskip10mm
 
 {\large\bf 3. Heine-Borelova věta.}
 
 \bigskip
 
 {\bf 3.1. Hromadné body.} Bod je {\em hromadný bod} množiny $A$ v prostoru $X$ jestliže každé jeho okolí obsahuje nekonečně mnoho bodů z $A$. Následující je bezprostřední, ale velmi užitečná modifikace definice kompaktnosti pomocí konvergentních podposloupností.
 
 \medskip
 
 {\bf Tvrzení.} {\em Metrický prostor $X$ je kompaktní právě když  každá nekonečná $A$ v $X$ má hromadný bod.
 
 Důkaz.} Buď $X$ kompaktní a $A\sue X$ nekonečná. Zvolme libovolnou posloupnost $x_1,x_2,\dots,x_n,\dots$  v  $A$ takovou, že $x_i\neq x_j$ pro $i\neq j$. Potom každé okolí limity $x$ podposloupnosti $(x_{k_n})_n$ obsahuje nekone\v cně mnoho členů $x_j$ a tedy je $x$ hromadný bod $A$.
 
 Naopak nechť druhé tvrzení platí a nechť je $(x_n)_n$ posloupnost v $X$. Potom je buď $A=\setof{x_n}{n=1,2,\dots}$ konečná
 a $(x_n)_n$ obsahuje konstantní podposloupnost,  nebo má $A$ hromadný bod $x$. Potom můžeme postupovat takto. Zvolme $x_{k_1}$ in $A\cap \Omega(x,1)$ a byly-li již $x_{k_1},\dots,x_{k_n}$ zvoleny vyberme $x_{k_{n+1}}$ v $A\cap\Omega(x,\frac1{n+1})$ takové, že $k_{n+1}>k_n$ (to diskvalifikuje jen konečně mnoho z nekonečně mnoha možností); potom $\lim_nx_{k_n}=x$. \sq
 
 \bigskip
 
 {\bf 3.2. Věta.} (Heine-Borelova Věta) {\em Metrický prostor je kompaktní právě když každé jeho pokrytí obsahuje konečné podpokrytí.
 
 Důkaz.} I. Buď $X$ kompaktní, ale nechť existuje pokrytí bez konečného podpokrytí. Podle 2.7.1  je $X$ is 
Lindel\"ofův  a tedy existuje {\em spočetné} pokrytí 
 \begin{equation}
 U_1,U_2,\dots, U_n,\dots \tag{$*$}
 \end{equation}
 bez konečných podpokrytí. Definujme
 $$
 V_1,V_2,,\dots, V_n,\dots
 $$
takto:
\begin{itemize}
\item za $V_1$ vezmeme první neprázdnou $U_k$, a
\item jsou-li již $V_1,V_2,,\dots, V_n$ vybrány, vezmeme za  $V_{n+1}$ první $U_k$ takové, že $U_k\nsubseteq \bigcup_{j=1}^nV_j$. Tak vynecháváme přesně ty $U_j$ které jsou v pořadí ($*$) pro pokrytí prostoru redundantní (to jest, posloupnost $(\bigcup_{k=1}^nV_n)_n$ již pokrytých částí $X$ je táž jako $(\bigcup_{k=1}^nU_n)_n$.)
\end{itemize}
 Tedy
 \begin{enumerate}
 \item $\setof{V_n}{n=1,2,\dots}$ je podpokrytí $\setof{U_n}{n=1,2,\dots}$,  
 \item procedura se nezastaví, jinak bychom měli konečné podpokrytí
 \item můžeme zvoli $x_n\in V_n\smin\bigcup_{k=1}^{n-1}V_k$.
 \end{enumerate}
  Všechna $x_n$ jsou ale různá (if $k<n$ potom $x_n\in V_n\smin V_k$ zatím co $x_k\in V_k$) a tedy máme nekonečnou množinu
  $$
  A=\set{x_1,x_2,\dots,x_n,\dots}
  $$
  a ta má hromadný bod $x$. Jelikož je $\setof{V_n}{n=1,2,\dots}$ pokrytí, existuje $n$ takové, že $x\in V_n$. To je spor, protože  $V_n$ neobsahuje žádné $x_k$ s $k>n$ takže $V_n\cap A$ is není nekonečná.
  
  \smallskip
  
  II. Nechť tvrzení o pokrytích platí a nechť existuje nekonečná $A$ bez hromadného bodu. Tedy, žádný bod $x\in X$ není hromadný bod $A$ a tedy máme otevřené $U_x\ni x$ takové, že každá $U_x\cap A$ je konečná. Zvolme konečné podpokrytí
  $$
  U_{x_1},U_{x_2},\dots, U_{x_n}
  $$
 pokrytí $\setof{U_x}{x\in X}$. Potom máme
  $$
  A=A\cap X=A\cap\bigcup_{k=1}^nU_{x_k}=\bigcup_{k=1}^n(A\cap U_{x_k})
  $$
  což je spor, protože sjednocení napravo je konečné. \sq
  
  \bigskip
  
  {\bf 3.3. Důsledek.} (Věta o konečném průniku) {\em Buď $\AC$ soustava uzavřených množin kompaktního prostoru. Je-li $\bigcap\setof{A}{A\in\AC}=\ems$ existuje $\AC_0\sue\AC$  konečná taková, že
  $\bigcap\setof{A}{A\in\AC_0}=\ems$. Následkem toho, je-li 
  $$
A_1\supe A_2\supe\cdots\supe A_n\supe\cdots
$$
klesající posloupnost neprázdných uzavřených podmnožin $X$ je
$\bigcap_{n=1}^\infty A_n\neq\ems$.

Důkaz.}  Podle De Morganova pravidla:  $\setof{X\smin A}{A\in\AC}$ je pokrytí.\sq

  
  
\vskip10mm
 
 {\large\bf 4. Baireova  věta o kategorii.}
 
 \bigskip
 
 {\bf 4.1. Průměr.} Zobecněme {\em průměr} z XVI.1.3 definujíce
v obecném metrickém prostoru $(X,d)$ pro podmnožinu $A\sue X$
 $$
 \diam(A)=\sup\setof{d(x,y)}{x,y\in A}
 $$
 $\diam(A)$ může být nekonečný: průměr $\diam(X)$ samotného prostoru je konečný jen když je ten prostor omezený.
 
 \medskip
 
 Z trojúhelníkové nerovnosti okamžitě dostáváme
 
 \smallskip
 
 {\bf 4.1.1. Pozorování.} 1. {\em $\diam(\Omega(x,\epsilon))\leq 2\epsilon$, a}
 
 2. {\em $\diam(\ol{A})=\diam(A)$.}
 
  
 \bigskip
 
 
 {\bf 4.2. Lemma.} {\em Buď $(X,d)$ úplný metrický prostor. Buď
$$
A_1\supe A_2\supe\cdots\supe A_n\supe\cdots
$$
klesající posloupnost neprázdných uzavřených podmnožin  $X$ taková, že  $\lim_n\diam(A_n)$ $=0$. Potom 
$$
\bigcap_{n=1}^\infty A_n\neq\ems.
$$

Důkaz.} Zvolme $a_n\in A_n$. Potom podle předpokladu o diametrech je $(a_n)_n$
Cauchyovská a tedy, kvůli úplnosti, má limitu $a$. Podposloupnost
$$ 
a_n,a_{n+1},a_{n+2},\dots
$$
je v {\em uzavřené} $A_n$ a její limita  $a$ je tedy v $A_n$. Jelikož $n$ bylo libovolné, $a\in \bigcap_{n=1}^\infty A_n$. \sq

\medskip

{\bf 4.2.1. Poznámky.} 1. Předpoklad zmenšujících se diametrů je podstatný: vezměme např.  uzavřené  $A_n=\langle n,+\infty)$ v úplném $\Rbb$. Na první pohled může znít trochu paradoxně, že průnik malých množin je neprázdný zatím co průnik velkých třeba ne. Princip je však snad jasný.

2. Čtenář se může ptát, zda na druhé straně není podstatné, že diametry v příkladě nahoře jsou nekonečné. V obecnějším prostoru je snadné dát příklad s $\diam(A_n)=1$, ale v $\Rbb$ nebo, obecněji, v
$\Ebb_n$ ne: viz 3.3. To má však co dělat s kompaktností, ne s úplností.

3. Samozřejmě je průnik v 4.2 nutně jednobodový.

\bigskip

{\bf 4.3. Lemma.} {\em Je-li $0<\epsilon<\eta$ je uzávěr $\ol{\Omega(x,\epsilon)}\sue\Omega(x,\eta)$

Důkaz.} To je bezprostřední důsledek trojúhelníkové nerovnosti: je-li
$d(y,\Omega(x,\epsilon))=0$ zvolme $z\in\Omega(x,\epsilon)$ s $d(y,z)<\eta-\epsilon$; potom $d(x,y)\leq d(x,z)+d(z,y)<\eta$.\sq

\bigskip

{\bf 4.4. Řídké množiny.} O podmnožině $A$ metrického prostoru  $X$ řekneme, že je  {\em řídká} je-li $X\smin\ol A$ hustá, tedy je-li
$\ol{X\smin\ol A}=X$. Všimněte si, že

\smallskip

\centerline{\em $A$ řídká právě když $\ol A$ je řídká.}

\medskip

{\bf 4.4.1. Reformulace.} {\em $A\sue X$ je řídká právě kdy\v z je pro každou neprázdnou otevřenou $U$ průnik $U\cap(X\smin\ol A)$ neprázdný.}

(Jistě, říkáme tak, že pro každé $x$ a každé $\epsilon>0$ je
průnik $\Omega(x,\epsilon)\cap(X\smin\ol A)$ neprázdný.)

\medskip

{\bf 4.4.2. Tvrzení.} {\em Sjednocení konečně mnoha řídkých množin
je řídká množina.

Důkaz.} Stačí dokázat pro dvě. Buďte $A,B$ řídké a buď $U$ neprázdná otevřená. Máme $U\cap(X\smin\ol{(A\cup B)})=U\cap(X\smin(\ol A\cup\ol B)=U\cap(X\smin\ol A)\cap(X\smin\ol B)$. 
Nyní je otevřená množina $V=U\cap(X\smin\ol A)$ neprázdná a tedy je také
$V\cap(X\smin\ol B))$ neprázdná. \sq

\bigskip

{\bf 4.5. Množiny první kategorie.} Spočetné sjednocení řídkých množin může mít k řídkosti daleko. Vezměte jednobodové podprostory prostoru $X$  racionálních čísel: máme tu již celý $X$. V úplných prostorech taková sjednocení jsou vždy jen malá část.

\medskip

Podmnožina metrického prostoru je {\em množina první kategorie} je li to spočetné sjednocení $\bigcup_{n=1}^\infty A_n$ řídkých množin $A_n$.

\medskip

{\bf 4.5.1. Věta.} (Baireova věta o kategorii) {\em Žádný úplný metrický $X$ není první kategorie v sobě.

Důkaz.} Předpokládejme, že je, to jest,
$$
X=\bigcup_{n=1}^\infty A_n\qtq{kde} X\smin \ol{A_n}\ \text{je hustá}.
$$
Můžeme předpokládat, že všechny $A_n$ jsou uzavřené;  máme tedy  $X\smin A_n$ husté otevřené.
Zvolme $U_1=\Omega(x,\epsilon)$ takovou, že $\Omega(x,2\epsilon)\sue X\smin A_1$ a $2\epsilon<1$. Tedy podle 4.1.1 a 4.3
$$
B_1=\ol U_1\sue X\smin A_1\qtq{a} \diam(B_1)<1.
$$
Mějme pro $k\leq n$ neprázdné otevřené $U_1,\dots,U_n$ takové že
\begin{equation}
  U_{k-1}\supe B_k=\ol U_k \ \text{pro}\  k\leq n,\ B_k\sue X\smin A_k,  \ \text{a}\ \diam(B_k)<\frac1k.
\tag{$*$}
\end{equation}
Jelikož $U_n\cap(X\smin A_{n+1})$ je neprázdná otevřená můžeme zvolit
 $U_{n+1}=\Omega(y,\eta)$ pro nějaké $y\in U_n\cap(X\smin A_{n+1})$ a
 $\eta$  dostatečně malé, aby $\Omega(y,2\eta)\sue U_n\cap(X\smin A_{n+1})$ a $2\eta<\frac1{n+1}$. Potom podle 4.1.1 a 4.3 máme
soustavu ($*$) prodlouženu od $n$ k $n+1$  a induktivně získáme
posloupnost neprázdných uzavřených množin $B_n$ takovou, že
\begin{enumerate}
\item $B_1\supe B_2\supe\cdots\supe B_n\supe\cdots$,
\item $\diam(B_n)<\frac1{n}$, a
\item $B_n\sue X\smin A_n$.
\end{enumerate} 
Podle (1),(2) a 4.2, $B=\bigcap_{n=1}^\infty B_n\neq\ems$, a podle (3) je
$$
B\sue \bigcap_{n=1}^\infty(X\smin A_n)=X\smin\bigcup_{n=1}^\infty A_n=X\smin X=\ems,
$$
spor. \sq

\medskip

{\bf 4.5.2. Poznámka.} Uvědomte si jak malou část úplného metrického prostoru  $X$ pokrývá. Spočetné sjednocení takových množin je stále množina první kategorie. Tedy je nejen menší než
  $X$, ale skutečně tak malá, že ani nekonečně mnoho disjunktních kopií takové množiny 
 $X$ nepokryje.

\vskip10mm
 
 {\large\bf 5. Zúplnění.}
 
 \bigskip
 
 {\bf 5.1.} Z různých důvodů, použijeme-li prostor v analyse či geometrii je lepší když je úplný.
 Již jsme viděli výhody reálné přímky $\Rbb$ proti racionální přímce $\Qbb$. Všimněte si, že rozšíření racionálních čísel n reálné je velmi uspokojivé. Neztrácíme nic z početních možností, všechno v tomto směru je spíše lepší, a $\Qbb$ je hustá v $\Rbb$
 takže všechno co chceme počítat v $\Rbb$ můžeme dobře aproximovat racionálními čísly.
 
 V této sekci ukážeme, že takto je možno rozšířit každý metrický prostor. To jest, pro každý metrický prostor $(X,d)$ máme prostor $(\wt X,\wt d)$
 takový, že
 \begin{itemize}
 \item $(X,d)$ je hustý podprostor v $(\wt X,\wt d)$ (v naší konstrukci budeme mít isometrické vložení $\iota:(X,d)\to(\wt X,\wt d)$
 takové, že $\iota[X]$ je hustý v $\wt X$), a
  \item  $(\wt X,\wt d)$ je úplný.
  \end{itemize}
  
  \bigskip
  
  {\bf 5.2. Konstrukce.} Myšlenka následující konstrukce je velmi přirozená. V původním prostoru mohou být Cauchyovské posloupnosti bez limit, tak tedy tam ty limity přidejme. To uděláme tím, že ty limity budeme representovat posloupnostmi kde limity scházely; budeme jen muset identifikovat takové posloupnosti které by měly mít stejnou limitu  -- a to uvidíte na ekvivalenci $\sim$ dále.
  
  \medskip
  
 Ozna\v cme
  $$
  \Cc(X,d), \ \ \text{krátce}\ \ \Cc(X),
  $$
  množinu všech Cauchyovských posloupností v $X$. Pro $(x_n)_n,(y_n)_n\in\Cc(X)$ definujme
  $$
  d'((x_n)_n,(y_n)_n)=\lim_nd(x_n,y_n).
  $$
  
  \medskip
  
  {\bf 5.2.1. Lemma.} {\em Limita v definici $d'$ vždy existuje a máme
  \begin{enumerate}
  \item $d'((x_n)_n,(x_n)_n)=0$,
  \item $d'((x_n)_n,(y_n)_n)=d'((y_n)_n,(x_n)_n)$, a
  \item $d'((x_n)_n,(z_n)_n)\leq d'((x_n)_n,(y_n)_n)+d'((y_n)_n,(z_n)_n)$.
  \end{enumerate}
  
  Důkaz.} První tvrzení dokážeme tak, že ukážeme, že posloupnost
$(d(x_n,y_n))_n$ je Cauchyovská v $\Rbb$. Skutečně, $(x_n)_n$ a $(y_n)_n$ jsou Cauchyovské a tedy pro $\epsilon>0$ máme $n_0$ takové, že $m,n>n_0$, $d(x_n,x_m)<\frac{\epsilon}{2}$ a $d(y_n,y_m)<\frac{\epsilon}{2}$. Potom $d(x_n,y_n)\leq d(x_n,x_m)+d(x_m,y_m)+ d(y_m,y_n)<\epsilon+d(x_m,y_m)$, tedy
$d(x_n,y_n)-d(x_m,y_m)<\epsilon$ a ze symetrie také
$d(x_m,y_m)-d(x_n,y_n)<\epsilon$, a konečně máme
$|d(x_n,y_n)-d(x_m,y_m)|<\epsilon$.

(1) a (2) jsou triviální a (3) je velmi snadné: zvolme $k$ takové, že 
$$
\begin{aligned}
&|d'((x_n)_n,(z_n)_n)-d(x_k,z_k)|<\epsilon,\quad
|d'((x_n)_n,(y_n)_n)-d(x_k,y_k)|<\epsilon \\
&\text{a} \ \ |d'((y_n)_n,(z_n)_n)-d(y_k,z_k)|<\epsilon.
\end{aligned}
$$
 Potom z trojúhelníkové nerovnosti pro $d$ dostaneme, že
$$
d'((x_n)_n,(z_n)_n)\leq d'((x_n)_n,(y_n)_n)+d'((y_n)_n,(z_n)_n)+3\epsilon
$$
a jelikož $\epsilon>0$ bylo libovolné dostáváme (3). \sq

\medskip

{\bf 5.2.2.} Definujme relaci ekvivalence $\sim$ na $\Cc(X)$ předpisem
$$
(x_n)_n\sim(y_n)_n\qtq{jestli\v ze} d'((x_n)_n,(y_n)_n)=0
$$
(z 5.2.1 bezprostředně plyne, že  $\sim$ je relace ekvivalence) a
označme
$$
\wt X=\Cc(X)/\sim,
$$
 a pro třídy $p=[(x_n)_n]$ a $q=[(y_n)_n]$ v této relaci ekvivalence položme
$$
\wt d(p,q)=d'((x_n)_n,(y_n)_n).
$$

\medskip

{\bf 5.2.3. Lemma.} {\em Hodnota $\wt d(p,q)$ nezáleží na volbě representantů  $p$ a $q$, a $(\wt X,\wt d)$ je metrický  prostor.

Důkaz.} je-li $(x_n)_n\sim(x'_n)_n$ a $(y_n)_n\sim(y'_n)_n$ máme
$$
\begin{aligned}
 d'((x_n)_n,(y_n)_n)&\leq d'((x_n)_n,(x'_n)_n)+ d'((x'_n)_n,(y'_n)_n)+ d'((y'_n)_n,(y_n)_n)=\\
  &=0+d'((x'_n)_n,(y'_n)_n)+0= d'((x'_n)_n,(y'_n)_n)
  \end{aligned}
 $$
 a ze symetrie též $ d'((x'_n)_n,(y'_n)_n) \leq d'((x_n)_n,(y_n)_n)$.
 
 Podle  5.2.1, $\wt d$ splňuje požadavky XIII.2.1(2),(3) a scházející $\wt d(p,q)=0\ \Rightarrow\ p=q$ plyne bezprostředně z definice
  $\sim$: je=li $d(p,q)=d'((x_n)_n,(y_n)_n)=0$ je $(x_n)_n\sim(y_n)_n$ a posloupnosti representují tentýž prvek množiny $\wt X$.\sq
 
 \bigskip
 
 {\bf 5.3.} Položme
$$\wt x=(x,x,\dots,x,\dots)
$$
a definujme zobrazení
 $$
 \iota=\iota_{(X,d)}:(X,d)\to (\wt X,\wt d)
 $$
 předpisem
$$
\iota(x)=[\wt x].
$$
 Potom máme 
 $$
 d'(\wt x,\wt y)=d(x,y)
 $$
 a $\iota$ je tedy isometrické vložení.
 
 \medskip
 
 {\bf Věta.} {\em Obraz isometrického vložení $\iota_{(X,d)}$ je hustý v $(\wt X,\wt d)$, a prostor $(\wt X,\wt d)$ je úplný.
 
 Důkaz.} 
  Vezměme  $p=[(x_n)_n]\in\wt X$ a zvolme $\epsilon>0$. Jelikož $(x_n)_n$ je Cauchyovská existuje $n_0$ takové, že pro $m,k>n_0$, $d(x_m,x_k)\leq\epsilon$. Ale potom je $\wt d(\iota(x_{n_0}),p)=d'(\wt{x_{n_0}},(x_k)_k)\leq d(x_{n_0},x_k)<\epsilon$.
 
 \smallskip
 
 Nyní buď
 \begin{equation}
 p_1 =[(x_{1n})_n],\  p_2 =[(x_{2n})_n],\ \dots,\  p_k =[(x_{kn})_n],\  \dots \tag{$*$}
 \end{equation}
Cauchyovská posloupnost v $(\wt X,\wt d)$.  Pro každé $p_n$ zvolme, podle již dokázané hustoty, nějaké $x_n\in X$ takové, že
 $\wt d(p_n,\iota(x_n))<\epsilon$. Pro  $\epsilon>0$ zvolme $n_0>\frac{3}{\epsilon}$ tak aby pro $m,n\geq n_0$, 
 bylo $\wt d(p_m,p_n)<\frac{\epsilon}{3}$. Potom pro $m,n\geq n_0$,
 $$
 d(x_m,x_n)=\wt d(\iota(x_m),\iota(x_n))\leq\wt d(\iota(x_m),p_m)+\wt d(p_m,p_n)+\wt d(p_n,\iota(x_n))<\frac{\epsilon}{3}+\frac{\epsilon}{3}+\frac{\epsilon}{3}=\epsilon
 $$
a vidíme, že $(x_n)_n$ je Cauchyovská. Dokážeme, že posloupnost $(*)$ konverguje k $p=[(x_n)_n]$. 
 
 
 Víme, že $\wt d(p_n,\iota(x_n))=\lim_k d(x_{nk},x_n)<\frac1n$. Zvolme $n_0>\frac{2}{\epsilon}$ takové, že pro $k,n\geq n_0$ máme $d(x_k,x_m)<\frac{\epsilon}{2}$. Potom je
 $$
d(x_{nk},x_k)\leq d(x_{nk},x_n)+d(x_n,x_k)<\frac{\epsilon}{2}+\frac{\epsilon}{2}=\epsilon
$$
a tedy 
$\wt d(p_n,p)=\lim_kd(x_{nk},x_k)\leq \epsilon.$\sq

\bigskip

{\bf 5.4. Poznámka.} Přirozeně vzniká otázka zda zúplnění  racionální přímky $\Qbb$ z n\v ehož bychom dostali
 $\Rbb$ může být konstruováno v duchu procedury právě popsané. Odpověď je opatrné ANO;  je třeba si uvědomit, že bychom měli jisté problémy s formulací co vlastně děláme. Konstrukce pracuje s metrickými prostory a vzdálenosti
již mají {\em reálné} hodnoty. To je možno obejít. Je možno mluvi o Cauchyovských posloupnostech, definovat ekvivalenci $\sim$ Cauchyovských posloupností (ne však pomocí limit, jejichž existence je založená na vlastnostech reálných čísel), a získat tak co potřebujeme. Ale většina čtenářů bude asi považovat běžně užívanou metodu Dedekindových řezů za trochu jednodušší.

\newpage

.

\newpage 


 
 \centerline{\Large\bf XVIII. Posloupnosti a řady funkcí} 
 
 \vskip10mm
 
 
 \def\d{\text{d}}
 
 
 
 {\large\bf 1. Bodová a stejnoměrná konvergence.}
 
 \bigskip
 
 {\bf 1.1. Bodová konvergence.} Buďte  $X=(X,d)$ a $Y=(Y,d')$ metrické prostory a  $f_n:X\to Y$ Posloupnost spojitých zobrazení. Máme-li pro každé $x\in X$ limitu $\lim_nf(x)=f(x)$ (v $Y$) řekneme, že posloupnost $(f_n)_n$ {\em  konverguje bodově} k zobrazení $f$ a obvykle píšeme
 $$
 f_n\to f.
 $$ 
 
 
 \medskip
 
 {\bf 1.1.1. Příklad.} Bodová konvergence  nezachovává pěkné vlastnosti funkcí $f_n$, dokonce ani spojitost, o derivovatelnosti nemluvě. Vezměme tento extrémně jednoduchý příklad. Buď $X=Y=\langle 0,1\rangle$
 a definujme $f_n$ předpisy
 $$
 f_n(x)= x^n.
 $$
 Potom $f(x)=\lim_nf_n(x)$ je $0$ pro $x<1$ kdežto $f(1)=1$.
 
 \bigskip
 
 {\bf 1.2. Stejnoměrná konvergence.} Posloupnost $(f_n:(X,d)\to (Y,d'))_n$  konverguje {\em stejnoměrně} k $f:X\to Y$
 jestliže
 $$
 \forall \epsilon>0\ \exists n_0\ \ \text{takové, že}\ \ \forall x\in X \ \ (n\geq n_0\ \Rightarrow \
 d'(f_n(x),f(x))<\epsilon).
 $$
 Mluvíme o  {\em stejnoměrně konvergentn\'\i\ posloupnosti zobrazení} a píšeme
 $$
 f_n\rrar f.
 $$
 
 \bigskip
 
 {\bf 1.3. Věta.} {\em Buďte $f_n:X\to Y$ spojitá zobrazení a nechť   $f_n\rrar f$.
 Potom je $f$ spojité.
 
 Důkaz.} Zvolme $x\in X$ a $\epsilon >0$. Zvolme pevně $n$ takové, že
 $$
 \forall y\in X, \ d'(f_n(y),f(y))<\frac{\epsilon}{3}.
 $$
 Jelikož je $f_n$ spojitá existuje $\delta>0$ takové, že
 $$
 d(x,z)<\delta\ \ \Rightarrow\ \ d'(f_n(x),f_n(z))<\frac{\epsilon}{3}.
 $$
 Tedy pro $d(x,z)<\delta$ máme
 $$
\begin{aligned}
 d'(f(x),f(z))\leq d'(f(x),f_n(x))+&d'(f_n(x),f_n(z))+d'(f_n(z),f(z))<\\
 &<\frac{\epsilon}{3}+\frac{\epsilon}{3}+\frac{\epsilon}{3}=\epsilon.
\end{aligned}
 $$
 \sq
 
 \bigskip
 
 {\bf 1.4. .} 1.
 Adjektivum ``stejnoměrná'' se vztahuje, podobně jako ve výrazu ``stejnoměrná spojitost'', k nezávislosti dané vlastnosti na umístění v definičním oboru. Na okamžik by nás mohlo napadnout, že podobně jako u stejnoměrné spojitosti, bychom mohli dostat něco zadarmo v případě kompaktního definičního oboru. Ale zde tomu tak není:
poslounost v příkladu  1.1.1 má velmi jednoduchý kompaktní definiční obor i obor hodnot a stejnoměrně konvergentní není.
 
 2. Věta 1.3 platí i pro stejnoměrnou spojitost, to jest, platí že
 \begin{itemize}
 \item[] 
 {\em jsou-li $f_n:X\to Y$ stejnoměrně spojitá zobrazení a  $f_n\rrar f$.
 potom $f$ je stejnoměrně spojité.}
 \end{itemize}
  Pro důkaz tohoto tvrzení stačí adaptovat důkaz  1.3 tím, že na začátku  $x$ nefixujeme. Čtenář to může provést v detailech jako jednoduché cvičení.
  
  \bigskip
  
  {\bf 1.5.} Řekneme, že  $(f_n)_n$ konverguje k $f$ {\em lokálně stejnoměrně} jestiže pro každé $x\in X$ existuje okolí $U$ takové, že $f_n|U\rrar f|U$ pro restrikce na  $U$. Jelikož spojitost v bodě je lokální vlastnost (t.j., $f$ je spojitá v bodě $x$ právě když je $f|U$ spojitá v $x$ pro nějaké okolí  $U$ bodu $x$) dostáváme z 1.3 okamžitě
  
  \medskip
  
  {\bf 1.5.1 Důsledek.} {\em Buďte $f_n:X\to Y$ spojitá zobrazení a nechť posloupnost $f_n$ konverguje k $f$ lokálně stejnoměrně.
 Potom je $f$ spojité.}
 
 
 \vskip10mm
 
 {\large\bf 2. Víc o stejnoměrné konvergenci:
 
 \hskip7mm  derivace, Riemannův integrál.}
 
 \bigskip
 
 {\bf 2.1. Příklad.} Třebaže stejnoměrná konvergence zachovává spojitost nezachovává existenci derivací.  Vezměme funkce
 $$
 f_n:\langle -1,1\rangle \to \langle 0,1\rangle\qtq{defined by} f_n(x)=\sqrt{(1-\frac1n)x^2+\frac1n}.
 $$
 Tyto derivovatelné funkce konvergují k $f(x)=|x|$ wkterá nemá derivaci v $x=0$: máme
 $$
 \left|\sqrt{(1-\frac1n)x^2+\frac1n}-|x|\right|=\frac{\frac1n(1-x^2)}{\left|\sqrt{(1-\frac1n)x^2+\frac1n}+|x|\right|}\leq
 \sqrt{\frac1n}.
 $$
  Derivovatelnost se ale zachovává pokud se stejnoměrnost vztahuje k derivacím. 

 
 \bigskip
 
 {\bf 2.2. Věta.} {\em Buďťe $f_n$ spojité reálné funkce definované na intervalu $J$ a nechť mají spojité derivace
 $f_n'$. Nechť $f_n\to f$ a $f_n'\rrar g$ na $J$. Potom má $f$ derivaci na $J$ a platí $f'=g$.
 
 Důkaz.} Máme
 $$
 \begin{aligned}
 &A(h)=\left|\frac{f(x+h)-f(x)}{h}-g(x)\right|=\\
 &=\left|\frac{f(x+h)-f_n(x+h)}{h}-\frac{f(x)-f_n(x)}{h}+\frac{f_n(x+h)-f_n(x)}{h}-g(x)\right|
 \end{aligned}
 $$
 a jelikož podle Lagrangeovy věty $\frac{f_n(x+h)-f_n(x)}{h}=f'_n(x+\theta h)$ pro nějaké $\theta$ s $0<\theta<1$, dostáváme dále
 $$
 \begin{aligned}
 &A(h)=\left|\frac{f(x+h)-f_n(x+h)}{h}-\frac{f(x)-f_n(x)}{h}+f'_n(x+\theta h)-\right.\\
 &\qquad\qquad-g(x+\theta h)+g(x+\theta h)-g(x)\bigg| \leq\\
 &\leq\frac1{|h|}|f(x+h)-f_n(x+h)|+\frac1{|h|}|f(x)-f_n(x)|+\\
 &\qquad\qquad+|f'_n(x+\theta h)-g(x+\theta h)|+|g(x+\theta h)-g(x)|.
 \end{aligned}
 $$
 Jelikož $f'_n\rrar g$, je funkce $g$ spojitá podle 1.3. Zvolme $\delta>0$ takové, že pro $|x-y|<\delta$ platí
 $|g(x)-g(y)|<\epsilon$; tedy, jestliže  $|h|<\delta$ je poslední sčítanec menší než $\epsilon$.
 
 Vezměme nyní $h$ pevné takové, že $|h|<\delta$ a zvolme  $n$ dost velké aby bylo
 $$
 \begin{aligned}
 &|f'_n(y)-g(y)|<\epsilon,\\
 &|f(x+h)-f_n(x+h)|<\epsilon|h|, \ \text{a}\\
 &|f(x)-f_n(x)|<\epsilon|h|
 \end{aligned}
 $$
 (všimněte si, že pro první užíváme stejnoměrnou konvergenci -- nevíme přesně kde to
 $y=x+\theta h$ je; ne tak v dalších nerovnostech, kde jde jen o pevné argumenty $x$ and $x+h$). Potom dostaneme
 $$
  A(h)=\left|\frac{f(x+h)-f(x)}{h}-g(x)\right|<4\epsilon
  $$
  a tvrzení je dokázáno. \sq
  
  \bigskip
  
  {\bf 2.3. Integrál pro limitu funkcí.} Pro Riemannův integrál nemáme obecně
  $\int_a^b\lim_nf_n=\lim_n\int_a^bf_n$ ani když integrály $\int_a^bf_n$ existují a všechny funkce $f_n$ are jsou omezeny touž konstantou. Tady je příklad.
  
  Seřaďme racionální čísla mezi $0$ a $1$ do posloupnosti
  $$
  r_1,r_2,\dots,r_n,\dots \ .
  $$
 Položme 
  $$
  f_n(x)=\begin{cases}1\ \ \text{jestliže}\ x=r_k\ \text{s}\ k\leq n,\\
                      0\ \ \text{jinak}.
                      \end{cases}
$$                      
 Potom zřejmě $\int_0^1f_n=0$ pro každé $n$. Ale limita $f$ posloupnosti $f_n$  je známá Dirichletova funkce
pro níž (zřejmě) je dolní integrál 0 and horní 1. 
 
 Pro stejnoměrnou konvergenci však platí
 
 \medskip
 
 {\bf 2.3.1. Věta.} {\em Buď $f_n\rrar f$ na $\langle a,b\rangle$ a nechť Riemannovy integrály
 $\int_a^bf_n$ existují. Potom existuje též $\int_a^bf$ a máme
 $$
 \int_a^bf=\lim_n\int_a^bf_n.
 $$
 
 Důkaz.} Pro $\epsilon>0$ zvolme $n_0$ tak, aby pro $n\geq n_0$ bylo
 \begin{equation}
 |f_n(x)-f(x)|<\frac{\epsilon}{b-a} \tag{$*$}
 \end{equation}
 pro všechny $x\in\langle a,b\rangle$. Užijme značení z XI.2. Pro rozklad $P:a=t_0< t_1<\cdots<t_{n-1}< t_n=b$ (který bude dále ještě specifikován) uvažujme
 $$
 \begin{aligned}
 &m_j=\inf\setof{f(x)}{t_{j-1}\leq x\leq t_j},\quad M_j=\sup\setof{f(x)}{t_{j-1}\leq x\leq t_j}\ \text{a}\\
 &m_j^n=\inf\setof{f_n(x)}{t_{j-1}\leq x\leq t_j},\quad M_j^n=\sup\setof{f_n(x)}{t_{j-1}\leq x\leq t_j}.
 \end{aligned}
$$
Podle $(*)$ platí pro $n,k\geq n_0$
 $$
|m_j-m_j^n|,\ |M_j-M_j^n|\leq \frac{\epsilon}{b-a} \ \ \text{a tedy také}\ \ 
|M_j^k-M_j^n|\leq \frac{2\epsilon}{b-a}
$$
 a pro dolní sumy dostaneme
 $$
 \begin{aligned}
 |s(f,P)-s(f_n,P)|&=\left|\sum(m_i-m^n_i)(t_i-t_{i-1})\right|\leq\\
 &\leq \sum|m_i-m^n_i|(t_i-t_{i-1}) \leq\epsilon
 \end{aligned}
 $$
 a podobně je pro horní sumy
 $$
 |S(f,P)-S(f_n,P)| \leq\epsilon\qtq{a} |S(f_k,P)-S(f_n,P)|\leq 2\epsilon.
  $$
	Nyní nejprve vezměme $P$ takové, že $|\int f_n-S(f_n,P)|<\epsilon$ a
	$|\int f_k-S(f_k,P)|<\epsilon$; potom usoudíme z trojúhelníkové nerovnosti
	že $|\int f_k-\int f_n|<4\epsilon$ a
  že $(\int f_n)_n$ je Cauchyovská posloupnost. Existuje tedy limita $L=\lim_n\int f_n$. Zvolme $n\geq n_0$ dost velké aby bylo
  $|\int f_n-L|<\epsilon$.
  
  Zvolíme-li nyní rozklad $P$ tak aby
  $$
  S(f_n,P)-\epsilon<\int f_n<s(f_n,P)+\epsilon
  $$
 dostaneme
  $$
  \begin{aligned}
  L-3\epsilon\leq &\int f_n-2\epsilon< s(f_n,P)-\epsilon\leq s(f,P)\leq\\
  &\leq S(f,P)\leq S(f_n,P)+\epsilon\leq \int f_n+2\epsilon\leq L+3\epsilon
  \end{aligned}
  $$
 a jelikož $\epsilon>0$ bylo libovolné vidíme konečně, že $L=\lint f=\uint f$.\sq
	
	\bigskip
	
	{\bf 2.3.2. Poznámka.} Příklad v 2.3 kde Riemannovsky integrabilní funkce bodově konvergovaly k Dirichletově funkci naznačuje, že problém by mohl být spíše v tom, že limita nemusí být integrabilní než v tom, že hodnota integrálu by byla jiná než ta limita. To je pravda jen zčásti. Skutečně, vezmeme-li mocnější Lebesgueův integrál (zhruba  řečeno, založeném na myšlence sou\v ctů přes {\em spočetné} disjunktní systémy, zatím co Riemannův  integrál založen na {\em konečných} disjunktních systémech), integrál Dirichletovy funkce je 0 (jak intuice napovídá: část intervalu na níž funkce nabývá hodnoty 1 je nekonečně menší než část s hodnotami 0). 
	
	Ale ať je integrál silný jak chce, formule
	$$\int_a^b\lim_nf_n=\lim_n\int_a^bf_n$$ nemůže platit úplně obecně. Vezměme funkce $f_n,g_n:\langle-1,1\rangle\to \Rbb\cup\set{+\infty}$ definované předpisy
	$$
	f_n(x)=\begin{cases}0\ \ \text{pro}\ \  x\leq -\frac{1}{n}\ \ \text{a}\ \ x\geq \frac{1}{n},\\
	                    n+n^2x\ \  \text{pro} \ \ -\frac{1}{n}\leq x\leq 0,\\ 
											n-n^2x\ \ \text{pro} \ \ 0\leq x\leq\frac{1}{n},
											\end{cases}											
											\quad
g_n(x)=\begin{cases}0\ \ \text{pro}\ \ x\neq 0,\\
	                    n\ \ \text{pro} \ \ x= 0	
																					\end{cases}
$$
(nakreslete si graf $f_n$).  Potom je pro každé $n$,  $\int_a^bf_n=1$ a $\int_a^bg_n=0$ zatím co $\lim_nf_n=\lim_ng_n$.	

Je to tak, že pro Lebesgueův  integrál platí formule $\int_a^b\lim_nf_n=\lim_n\int_a^bf_n$ např. je-li limita monotonní nebo jsou-li funkce stejně omezeny integrabilní funkcí. Takže v příkladě nahoře je limita $\int_a^b\lim_ng_n=\lim_n\int_a^bg_n$	korektní, limita s $f_n$ ne.	

\bigskip
	
	{\bf 2.4. Lemma.} {\em Buď $\lim_{n\to\infty}g(x_n)=A$ pro každou posloupnost $(x_n)_n$ takovou, že $\lim_nx_n=a$. Potom je $\lim_{x\to a}g(x)=A$.
	
	Důkaz.} Předpokládejme že $\lim_{x\to a}g(x)$ buď neexistuje nebo není rovna  $A$.
	Potom existuje $\epsilon>0$ takové, že pro každé $\delta>0$ existuje $x(\delta)$ pro které $0<|a-x(\delta)|<\delta$ a $|A-g(x(\delta))|\geq\epsilon$. Položme
	$x_n=x(\frac1n)$. Potom $\lim_nx_n=a$, ale $\lim_{n\to\infty}g(x_n)$ není $A$.\sq
	
	\medskip
	
	{\bf 2.4.1. Tvrzení.} {\em Buď $f:\langle a,b\rangle\times
	\langle c,d \rangle\to\Rbb$ spojitá funkce. Potom
$$
	\lim_{y\to y_0}\int_a^bf(x,y)\d x=\int_a^bf(x,y_0)\d x.
	$$
	
	Důkaz.} Jelikož $\langle a,b\rangle\times\langle c,d \rangle$ je kompaktní,
	$f$ je stejnoměrně spojitá. Tedy existuje pro každé $\epsilon>0$ číslo $\delta>0$ takové, že $\max\set{|x_1-x_2|,|y_1-y_2|}<\delta$ implikuje
	$|f(x_1,y_1)-f(x_2,y_2)|<\epsilon$. 
	
	Buď $\lim_ny_n=y_0$. Položme $g(x)=f(x,y_0)$ a $g_n(x)=f(x,y_n)$. Je-li $|y_n-y_0|<\delta$ jako nahoře,
	máme $|g_n(x)-g(x)|<\epsilon$ nezávisle na $x$, takže $g_n\rrar g$
	a dále podle 2.3, $\lim_n\int_a^bg_n(x)\d x=\int_a^b g(x)\d x$, to jest,
	$\lim_n\int_a^bf(x,y_n)\d x=\int_a^bf(x,y_0)\d x$, a tvrzení plyne z Lemmatu 2.4. \sq
	
	\bigskip
	
	{\bf 2.4.2. Tvrzení.} Buď {\em 
	 $f:\langle a,b\rangle\times
	\langle c,d \rangle\to\Rbb$ spojitá a nechť má spojitou parciální derivaci $\pad{f(x,y)}{y}$ v  $\langle a,b\rangle\times
	(c,d)$. Potom $F(y)=\int_a^bf(x,y)\d x$ má derivaci v $(c,d)$ a platí
	$$
	\frac{\d}{\d y}\int_a^bf(x,y)\d x=\int_a^b\pad{f(x,y)}{y}\d x.
	$$
	
	Důkaz.} Vezměme pevně $y\in (c,d)$ a zvolme $\alpha>0$ tak aby $c<y-\alpha<y+\alpha<d$. Položme $F(y)=\int_a^bf(x,y)\d x$ a definujme
	$$
	g(x,t)=\begin{cases} \frac1t(f(x,y+t)-f(x,y))\ \ \text{pro}\ \ t\neq 0,\\
	             \pad{f(x,y)}{y}\ \ \text{pro}\ \ t=0.
							\end{cases}
							$$
Tato funkce je spojitá $g$ na kompaktním $\langle a,b\rangle\times
\langle -\alpha,+\alpha\rangle$. To je zřejmé v bodech $(x,t)$ s $t\neq 0$, a jelikož podle Lagrangeovy věty je
$$
g(x,t)-g(x,0)=\frac1t(f(x,y+t)-f(x,y))-\pad{f(x,y)}{y}=
\pad{f(x,y+\theta t)}{y}-\pad{f(x,y)}{y},
$$						
spojitost v $(x,0)$ plyne ze spojitosti parciální derivace.

Můžeme tedy užít 2.4.1 a dostaneme
$$
	\lim_{t\to 0}\int_a^bg(x,t)\d x=\int_a^b\pad{f(x,y)}{y}\d x.
	$$
	a jelikož pro $t\neq 0$
	$$
	\int_a^bg(x,t)=\frac1t\left(\int_a^bf(x,y+t)-\int_a^bf(x,y)\right)=
	\frac1t(F(y+t)-F(y))
	$$ 
	tvrzení je dokázáno. \sq
						


 
 \vskip10mm
 
 {\large\bf 3. Prostor spojitých funkcí.}
 
 \bigskip
 
 {\bf 3.1.}	Buď $X=(X,d)$ metrický prostor. Označme
$$
C(X)
$$
množinu všech omezených spojitých reálných funkcí opatřenou metrikou
$$
d(f,g)=\sup\setof{|f(x)-g(x)|}{x\in X}
$$
(ověření, že takto definované $d$ je skutečně metrika je triviální).

\medskip

{\bf 3.1.1. Poznámka.} Připuštění nekonečných vzdáleností by nijak neuškodilo; ve skutečnosti to má výhody. 
Nicméně, dosud jsme pracovali jen s konečnými vzdálenostmi a tak už u toho zůstaneme. Poznamenejme jen, že
\begin{itemize}
\item většina toho, co bude v této sekci platí bez té omezenosti, a
\item je-li $X$ kompaktní jsou ty funkce omezené tak jako tak.
\end{itemize}

\bigskip

{\bf 3.2. Tvrzení.} {\em Posloupnost $(f_n)_n$ konverguje k $f$ v $C(X)$ právě když $f_n\rrar f$.

Důkaz.} Máme $\lim_nf_n=f_n$ v $C(X)$ jestliže pro každé $\epsilon>0$ existuje $n_0$ takové, že $d(f_n,f)=\sup\setof{|f_n(x)-f(x)|}{x\in X}\leq\epsilon$ pro $n\geq n_0$.
Jinak řečeno, pro $\epsilon>0$ existuje $n_0$ takové, že 
pro všechna $n\geq n_0$ a pro všechna $x\in X$ platí, že $|f_n(x)-f(x)|\leq\epsilon$, což je definice stejnoměrné konvergence. \sq

\bigskip

{\bf 3.3. Pozorování.} {\em Buď $a$ reálné číslo. Potom funkce $g:\Rbb\to \Rbb$  definovaná jako $g(x)=|a-x|$ je spojitá.}

(Skutečně, máme $|a-y|\leq |a-x|+|x-y|$, tedy $|a-y|-|a-x|\leq |x-y|$ a ze symetrie $||a-y|-|a-x||\leq |x-y|$.)

\medskip

{\bf 3.3.1. Věta.} {\em $C(X)$ je úplný metrický prostor.

Důkaz.} Buď $(f_n)_n$ Cauchyovská posloupnost v $C(X)$. Pro každé $\epsilon>0$ tedy existuje $n_0$ takové, že
\begin{equation}
\forall m,n\geq n_0, \ \ \forall x\in X \ \  |f_m(x)-f_n(x)|<\epsilon. \tag{$*$}
\end{equation}
takže speciálně každá posloupnost $(f_n(x))_n$ je Cauchyovská v $\Rbb$ a existuje limita $f(x)=\lim_nf_n(x)$. 

Vezměme pevné $m\geq n_0$. Vezmeme-li limitu ($*$) a  užijeme-li Pozorování	3.3 dostaneme
$$
\forall m\geq n_0, \ \  |f_m(x)-\lim_nf_n(x)|=|f_m(x)-f(x)|\leq\epsilon,
$$
nezávisle na $x$.

Je tedy $f_n\rrar f$ a tedy
\begin{itemize}
\item podle 1.3 je $f$ spojitá; je též omezená, protože fixujeme-li $m\geq n_0$ je zřejmě $|f(x)|\leq |f_m(x)|+\epsilon$ (a $f_m$ je omezená) a tedy $f\in C(X)$,
\item a podle 3.2 je $\lim_nf_n=f$ v $C(X)$.
\end{itemize}
\sq
																					
  
 \vskip10mm
 
 {\large\bf 4. Řady spojitých funkcí.}

\bigskip
 
{\bf 4.1.} S řadami spojitých funkcí
$$
\sum_{n=0}^\infty f_n(x)= f_0(x)+f_1(x)+\cdots+f_n(x)+\cdots
$$
jednáme jako s limitami
$$
\lim_n \sum_{k=0}^n f_k(x)
$$
částečných součtů.
Jako u řad čísel jsou však, ze zřejmých důvodů, skutečně důležité {\em absolutně konvergentní řady funkcí}, a sice ty, pro které je $\sum_{n=0}^\infty f_n(x)$ absolutně konvergentní pro každé $x$ v definičním oboru. Zejména  (viz III.2.4) máme, že
\begin{itemize}
\item[] {\em je-li $\sum_{n=0}^\infty f_n(x)$ absolutně konvergentní, nezávisí na pořadí sčítanců.
}
\end{itemize}
 
 \bigskip
 
 {\bf 4.2.} Řekneme, že řada $\sum_{n=0}^\infty f_n(x)$  {\em konverguje stejnoměrně}
(resp. {\em konverguje lokálně stejnoměrně}) jestliže je
$$
(\sum_{k=0}^n f_k(x))_n
$$
konvergentní (resp.lokálně stejnoměrně konvergentní) posloupnost funkcí.

 V prvním případě budeme někdy užívat symbol
 $$
\sum_{n=0}^\infty f_n(x)\rrar f(x) \qtq{or} f_0(x)+f_1(x)+\cdots+f_n(x)+\cdots\rrar f(x).
$$

\bigskip

Z 1.3 okamžitě dostáváme

\medskip

{\bf 4.3. Tvrzení.} {\em Buď $
\sum_{n=0}^\infty f_n(x)$ stejnoměrně konvergentní řada funkcí. Potom je součet spojitá funkce.}

\bigskip

Z 2.2 dostaneme, užitím toho, že derivace konečného součtu je součet derivací,

\medskip

{\bf 4.4. Tvrzení.} {\em Nechť řada $\sum_{n=0}^\infty f_n(x)$ konverguje a nechť
$\sum_{n=0}^\infty f'_n(x)$ konverguje stejnoměrně. Potom $f(x)$ má derivaci
$$
\left(\sum_{n=0}^\infty f_n(x)\right)'=\sum_{n=0}^\infty f'_n(x).
$$}

\bigskip

{\bf 4.5.} Následující rozšířeni kriteria III.2.2 bude velmi užitečné.

\medskip

{\bf Věta.} {\em Buďte  $b_n\geq 0$ taková, že $\sum_{n=0}^\infty b_n$ konverguje. Buďte $f_n(x)$ reálné funkce definované na oboru $D$ takové, že $|f_n(x)|\leq b_n$ pro všechna $x\in D$. Potom $\sum_{n=0}^\infty f_n(x)$ konverguje na $D$ absolutně a stejnoměrně.

Důkaz.} Ta absolutní konvergence je bezprostřední z definice. Nyní zvolme $\epsilon>0$. Posloupnost $(\sum_{k=0}^n b_k)_n$ je Cauchyovská a tedy existuje  $n_0$ takové, že pro $m,
 n+1\geq n_0$ je $\sum_n^m b_k<\epsilon$. Máme tedy pro $x\in D$,
$$
\left|\sum_{n+1}^m f_k(x)\right|\leq\sum_{n+1}^m |f_k(x)|\leq\sum_{n+1}^m b_k<\epsilon
$$
a tedy v $C(D)$
$$
d(\sum_{k=0}^m f_k,\sum_{k=0}^n f_k)=\sup\setof{\left|\sum_{n+1}^m f_k(x)\right|}{x\in D}\leq\epsilon.
$$
Posloupnost $(\sum_{k=0}^n f_k)_n$ je tedy Cauchyovská v $C(D)$ a podle 3.2 (a definice 2.2)
$\sum_{k=0}^\infty f_k(x)$ stejnoměrně konverguje. \sq

\medskip

{\bf 4.5.1. Důsledek.} {\em Nechť $f(x)=\sum_{n=0}^\infty f_n(x)$ konverguje a nechť 
$f_n(x)$ mají derivace. Nechť pro nějakou konvergentní řadu $\sum_{n=0}^\infty b_n$ platí, že $|f'_n(x)|\leq b_n$ pro všechna $n$ a $x$. Potom derivace $f$ existuje a máme
$$
\left(\sum_{n=0}^\infty f_n(x)\right)'=\sum_{n=0}^\infty f_n'(x).
$$}


\newpage

\centerline{\Large\bf XIX. Mocninné řady} 
 
 \vskip10mm
 
  
 {\large\bf 1. Limes superior.}
 
 \bigskip
 
 {\bf 1.1.} Budeme dovolovat též nekonečné limity posloupností reálných čísel, t.j.,
 $$\begin{aligned}
 &\lim_n a_n=+\infty \qtq{jestliže} \forall K\ \exists n_0\ (n\geq n_0\ \Rightarrow \ a_n\geq K),\\
&\lim_n a_n=-\infty \qtq{jestliže} \forall K\ \exists n_0\ (n\geq n_0\ \Rightarrow \ a_n\leq K),
\end{aligned}
 $$
a nekonečná suprema pro $M\sue\Rbb$,
 $$
 \sup M=+\infty \qtq{nemá-li $M$ žádnou horní mez.}
 $$
 Položíme
 $$
 \begin{aligned}
 &(+\infty)\cdot a= a\cdot(+\infty)=+\infty\ \ \text{pro kladná}\ a, \ \text{a}\\
 &(+\infty)+ a= a+(+\infty)=+\infty\ \ \text{pro konečná}\ a.
 \end{aligned}
 $$
 
 \bigskip
 
 {\bf 1.2.} Pro posloupnost $(a_n)_n$ reálných čísel definujeme  {\em limes superior} jako číslo
 $$
 \limsup_na_n= \lim_n\sup_{k\geq n}a_k=\inf_n\sup_{k\geq n}a_k.
 $$
 Druhá rovnost je zřejmá: posloupnost $(\sup_{k\geq n}a_k)_n$ neroste.
 
 \medskip
 
 Limes superior je definována pro libovolnou posloupnost. Dále máme
 
 \medskip
 
 {\bf 1.2.1. Pozorování.} {\em Pokud \ $\lim_na_n$ existuje, platí $\limsup_na_n=\lim_na_n$.}
 
 (Je-li $\lim_na_n=-\infty$ potom $(\sup_{k\geq n}a_k)_n$ nemá dolní mez, je-li
 $\lim_na_n=+\infty$ potom je $\sup_{k\geq n}a_k=+\infty$ pro všechna $n$. Buď tedy $a=\lim_na_n$ konečná a buď $\epsilon>0$. Potom $|a_n-a|<\epsilon$ implikuje, že $|\sup_{k\geq n}a_k-a|\leq \epsilon$.)
 
 \bigskip
 
 {\bf 1.3. Tvrzení.} {\em Nechť $a_n, b_n\geq 0$; položme $a=\limsup_na_n$. Nechť existuje konečná a kladná limita   $b=\lim_nb_n$.
  Potom
  $$
  \limsup_na_nb_n =ab.
  $$
  
  Důkaz.} I. Pro $\epsilon>0$ zvolme $n_0$ tak, aby
  $$
  n\geq n_0\quad\Rightarrow\quad  b_n<b+\epsilon\ \ \text{a}\ \ \sup_{k\geq n}a_k\leq a+\epsilon.
  $$
  Potom pro $n\geq n_0$ máme
  $$
  \sup_{k\geq n}a_kb_k\leq (\sup_{k\geq n}a_k)(b+\epsilon)\leq (a+\epsilon)(b+\epsilon)=ab+\epsilon(a+b+\epsilon)
  $$
  a jelikož $\epsilon>0$ bylo libovolné, vidíme, že $\limsup_na_nb_n\leq ab$ (což zahrnuje též případ $a=+\infty$ kde samozřejmě je odhad triviální).
  
  \smallskip
  
  II. Pro $\epsilon>0$ tak malé, aby bylo $b-\epsilon>0$.
 Zvolme $n_0$ takové, že
  $$
  n\geq n_0\quad\Rightarrow\quad  b_n>b-\epsilon.
  $$
  Jelikož  $\sup_{k\geq n}a_k\geq\inf_m\sup_{k\geq m}a_k=a$ pro každé $n$, existuje $k(n)\geq n$ takové, že
  $$
  \begin{aligned}
  &a_{k(n)}\geq a-\epsilon \ \ \text{je-li $a$ konečné, a}\\
  &a_{k(n)}\geq n \ \ \text{pokud $a=+\infty$.}
  \end{aligned}
  $$
  Potom je pro $n\geq n_0$,
  $$
  (a-\epsilon)(b-\epsilon)\leq a_{k(n)}b_{k(n)}\quad\text{resp.}\ \ n(b-\epsilon)\leq a_{k(n)}b_{k(n)}\ \text{je-li} \ a=+\infty
  $$
  takže
   $$
  ab-\epsilon(a+b-\epsilon)\leq \sup_ma_{m}b_{m}\quad\text{resp.}\ \ n(b-\epsilon)\leq \sup_ma_{m}b_{m}\
  \text{je-li} \ a=+\infty
  $$
  a jelikož $\epsilon>0$ bylo libovolné a jelikož je $n(b-\epsilon)$ libovolně velké,
   máme též $ab\leq\limsup_na_nb_n$. \sq
   
   \bigskip
   
   {\bf 1.4. Poznámka.} Podobně jako limes superior se definuje též
    {\em limes inferior}  pro libovolnou posloupnost $(a_n)_n$ reálných čísel jako
 $$
 \liminf_na_n= \lim_n\inf_{k\geq n}a_k=\sup_n\inf_{k\geq n}a_k.
 $$
Vlastnosti jsou zcela analogické.
 
 
 
 
 
   
   \vskip10mm
 
 {\large\bf 2. Mocninná řada a poloměr konvergence.}
 
 \bigskip
 
 Až do kapitoly XXI nebudeme systematicky zkoumat komplexní funkce komplexní proměnné, ale v této sekci bude výhodné uvažovat koeficienty
$a_n$, $c$  a proměnnou $x$ komplexní. Nejen proto, že důkaz věty o poloměru konvergence
je doslova stejný; v této chvíli může být ale ještě důležitější, že to vysvětlí zdánlivě paradoxní chování některých {\em reálných mocninných řad} (viz 2.4).
  
  \bigskip
 
 {\bf 2.1.} Buďte $a_n$ a $c$ komplexní čísla.  {\em Mocninná řada} s koeficienty $a_n$ a {\em středem} $c$ je řada
 $$
 \sum_{n=0}^\infty a_n(x-c)^n.
 $$
V této sekci bude chápána jako funkce komplexní proměnné $x$; definiční obor bude specifikován za okamžik.
 
 \bigskip
 
 {\bf 2.2.}  {\em Poloměr konvergence} mocninné řady $\sum_{n=0}^\infty a_n(x-c)^n$ je číslo
 $$
 \rho=\rho((a_n)_n)=\frac{1}{\limsup_n\sqrt[n]{|a_n|}}\ .
 $$
 
   \medskip
   
   {\bf 2.3.1. Věta.} {\em  Buď $\rho=\rho((a_n)_n)$ poloměr konvergence mocninné řady
	$\sum_{n=0}^\infty a_n(x-c)^n$ a buď $r<\rho$. Potom řada $\sum_{n=0}^\infty a_n(x-c)^n$ konverguje stejnoměrně a absolutně na množině
   $\setof{x}{|x-c|\leq r}$. 
   
   Je-li $|x-c|>\rho$, řada nekonverguje vůbec.
   
   Důkaz.} I. Pro pevné    $r<\rho$ zvolme $q$ takové, že
   $$
   r\cdot\inf_n \sup_{k\geq n}\sqrt[k]{|a_k|}<q<1.
   $$
   Potom existuje $n$ takové že pro všechna $k\geq n$ platí,
   $$
   r\cdot \sup_{k\geq n}\sqrt[k]{|a_k|}<q \qtq{a tedy} r\cdot\sqrt[k]{|a_k|}< q.
   $$
  Pro dostatečně velké $K\geq 1$ navíc platí $r^k\cdot|a_k|<Kq^k$ pro všechna $k\leq n$ a tedy
   $$
   \text{je-li}\ |x-c|\leq r\ \text{potom} \ \ |a_k(x-c)^k|\leq Kq^k\ \text{pro všechna}\ k
   $$
   a podle XVIII.3.5 vidíme, že
  $\sum_{n=0}^\infty a_n(x-c)^n$ konverguje stejnoměrně a absolutně na $\setof{x}{|x-c|\leq r}$.
  
  \smallskip
  
  II. Je-li $|x-c|>\rho$ je  $|x-c|\cdot \inf_n \sup_{k\geq n}\sqrt[k]{|a_k|}>1$ a tedy máme
  $|x-c|\cdot \sup_{k\geq n}\sqrt[k]{|a_k|}>1$ pro všechna $n$. Následkem toho pro každé $n$ existuje $k(n)\geq n$ takové, že
 $|x-c|\cdot \sqrt[k(n)]{|a_{k(n)}|}>1$ a tedy $|a_{k(n)}(x-c)^{k(n)}|>1$, takže sčítance této řady ani nekonvergují k nule.\sq

\medskip

Z 2.3.1 a XVIII.1.5 
dostáváme

\smallskip


{\bf 2.3.2. Důsledek.} {\em  Mocninná řada $\sum_{n=0}^\infty a_n(x-c)^n$ lokálně stejnoměrně konverguje na otevřeném kruhu  $D=\setof{x}{|x-c|<\rho((a_n)_n)}$ a nekonverguje v žádném $x$ s $|x-c|>\rho$. Speciálně je funkce
$f(x)=\sum_{n=0}^\infty a_n(x-c)^n$  spojitá na $D$.}

\bigskip

{\bf 2.4. Poznámky.} 1. Věta 2.3.1 je v úvodních textech reálné analysy často interpretována jako tvrzení o konvergenci reálné mocninné řady na intervalu $(c-\rho,c+\rho)$. Důkazy v reálném a komplexním kontextu jsou doslova stejné (i když samozřejmě silně využitá trojúhelníková nerovnost pro absolutní hodnotu komplexního čísla je mnohem hlubší fakt než tato nerovnost v $\Rbb$).

2. Definiční obor $D$  (konvergence) mocninné řady je omezen mezi otevřeným a uzavřeným kruhem
$$
\setof{x}{|x-c|<\rho}\sue D\sue\setof{x}{|x-c|\leq\rho}
$$ 
v komplexní rovině a nad ten uzavřený se rozšířit nemůže. To vysvětluje zdánlivě paradoxní chování konvergence na reálné přímce. Vezměme např. reálnou funkci
$$
f(x)=\frac1{1+x^2}.
$$
V intervalu $(-1,1)$ může být napsána jako mocninná řada
$$
1-x^2+x^4-x^6+x^8-\cdots
$$
která náhle přestane konvergovat v  $+1$ (a pro $x<-1$ samozřejmě již nekonverguje). 
Myslíme-li v termínech reálné analysy není pro to viditelný důvod: $f(x)$ se za těmito mezemi jen zmenšuje. Ale v komplexní rovině kruhy $\setof{x}{|x|<r}$ jako definiční obory $f(x)$ musí zastavit svoji expansi při $r=1$: překážky jsou body $i$ a $-i$, na reálné ose ovšem žádná překážka není.

3. Věta 2.3.1 mluví o konvergenci v bodech mnořiny $\setof{x}{|x|<\rho}$ a divergenci pro  $|x|>\rho$. Pro body na kružnici $C=\setof{x}{|x|=\rho}$ žádné obecné pravidlo není.

\bigskip

{\bf 2.5. Tvrzení.} {\em Poloměr konvergence řady
$\sum_{n=1}^\infty na_n(x-c)^{n-1}$je týž jako poloměr konvergence řady $\sum_{n=0}^\infty a_n(x-c)^n$.

Důkaz.} Pro $x\neq 0$ řada  $\mathcal S=\sum_{n=1}^\infty na_n(x-c)^{n-1}$ zřejmě konverguje právě když konverguje $\mathcal S_1=\sum_{n=}^\infty na_n(x-c)^{n}=x(\sum_{n=1}^\infty na_n(x-c)^{n-1})$. Podle 1.3 máme
$$
\limsup_n\sqrt[n]{n|a_n|}=\limsup_n\sqrt[n]{n}\sqrt[n]{|a_n|}=
\lim_n\sqrt[n]{n}\cdot\limsup_n\sqrt[n]{|a_n|}=\limsup_n\sqrt[n]{|a_n|}
$$
jelikož $\lim_n\sqrt[n]{n}=\lim_n\text{e}^{\frac1n\lg n}=\text{e}^0=1$. Tedy je poloměr konvergence $\mathcal S_1$,   roven opět číslu $\rho((a_n)_n)$.\sq

\bigskip

{\bf 2.5.1.} Z XVIII.3.5.1 nyní plyne
\smallskip

{\bf Věta.} {\em Řada $f(x)=\sum_{n=0}^\infty a_n(x-c)^n$ má derivaci
$$
f'(x)=\sum_{n=1}^\infty na_n(x-c)^{n-1}
$$
 a též primitivní funkci
$$
(\int f)(x)=C+\sum_{n=0}^\infty \frac{a_n}{n+1}(x-c)^{n+1}
$$
v celém intervalu $J=(c-\rho,c+\rho)$ kde $\rho=\rho((a_n)_n)$.

Jinými slovy, mocninou řadu můžeme derivovat i integrovat po jednotlivých sčítancích.}

 
   \vskip10mm
 
 {\large\bf 3. Taylorovy řady.}
 
 \bigskip

{\bf 3.1.} Připomeňme si VIII.7.3. Nechť má funkce $f$ derivace všech řádů $f^{(n)}$ v intervalu $J=(c-\Delta,c+\Delta)$. Potom pro každé $n$ a $x\in J$,
$$
 f(x)=\sum_{k=0}^n\frac{f^{(k)}(c)}{k!}(x-c)^k+R_n(f,x)
 $$ 
kde $R_n(f,x)=\frac{f^{(n+1)}(\xi)}{(n+1)!}(x-c)^{n+1}$ s číslem $\xi$ mezi $c$ a $x$.

\medskip

{\bf 3.1.1. Tvrzení a definice.} {\em Nechť má funkce $f$ derivace všech řádů $f^{(n)}$ v intervalu $J=(c-\Delta,c+\Delta)$.
Nechť pro zbytek $R_n(f,x)=f(x)-\sum_{k=0}^n\frac{f^{(k)}(c)}{k!}(x-c)^k$ platí, že
$$
\lim_n R_n(f,x)=0\ \ \text{pro každé $x\in J$}. 
$$
Potom může být funkce $f(x)$ vyjádřena v  $J$ mocninnou řadou
$$
\sum_{n=0}^\infty\frac{f^{(n)}(c)}{n!}(x-c)^n.
$$
Tato mocniná řada se nazývá {\em Taylorova řada} funkce $f$.


Důkaz.} Máme
$$
\lim_n\sum_{k=0}^n\frac{f^{(k)}(c)}{k!}(x-c)^k=\lim_n(f(x)-R_n(f,x)=
f(x)-\lim_nR_n(f,x)=f(x).
$$
\sq

\bigskip

{\bf 3.2. Příklady.} 1. Pro libovolně velké $K$ platí
$$
\lim_n\frac{K^n}{n!} =0
$$
(položíme-li $k_n=\frac{K^n}{n!}$ je pro $n>2K$, $k_{n+1}<\frac{k_n}{2}$ a tedy $k_{n+m}<2^{-m}k_n$). Následkem toho, pro libovolné  $x$ zbytek v  Taylorově formuli VIII.7.3 pro $e^x$, $\sin  x$ a $\cos x$ konverguje k nule a máme tedy  Taylorovy řady 
$$
\begin{aligned}
&e^x=1+\frac{x}{1!}+\frac{x^2}{2!}+ \cdots +\frac{x^n}{n!}+\cdots,\\
&\sin x= \frac{x}{1!}-\frac{x^3}{3!}+\frac{x^5}{5!}-\cdots\pm\frac{x^{2n+1}}{(2n+1)!}\mp\cdots, \ \text{a}\\
&\cos x= 1-\frac{x^2}{2!}+\frac{x^4}{4!}-\frac{x^6}{6!}+\cdots\pm\frac{x^{2n+2}}{(2n+2)!}\mp\cdots \\
\end{aligned}
$$
všechny s poloměrem konvergence rovným $+\infty$.

\smallskip

2. Samotná existence derivací všech řádů by nestačila:  zbytek nekonverguje automaticky k nule. Uvažme příklad z VIII.7.4,
$$
f(x)=\begin{cases} e^{-\frac{1}{x^2}}\ \ \text{for}\ x\neq 0,\\
                       0\ \ \text{for}\ x=0\end{cases}
$$                       
 kde $f^{(k)}(0)=0$ for all $k$.
 
 \bigskip

{\bf 3.3.} Buď $f(x)=\sum_{n=0}^\infty a_n(x-c)^n$ mocninná řada s poloměrem konvergence $\rho$. Potom podle 2.5.1
\begin{equation}
\begin{aligned}
f^{(k)}(x)&=\sum_{n=k}^\infty n(n-1)\cdots(n-k+1)a_n(x-c)^{n-k}=\\
&=k!a_k+\sum_{n=k+1}^\infty n(n-1)\cdots(n-k+1)a_n(x-c)^{n-k}.
\end{aligned} \tag{$*$}
\end{equation}
 

\medskip

{\bf 3.3.1. Tvrzení.} 1. {\em Koeficienty mocninné řady
$f(x)=\sum_{n=0}^\infty a_n(x-c)^n$ jsou jednoznačně určeny funkcí
$f$.}

2. {\em Mocninná řada je svá vlastní Taylorova řada.

Důkaz.} 1.Podle $(*)$ je $a_k=\frac{f^{(k)}(x)}{k!}$.

2. Řada $f(x)=\sum_{n=0}^\infty a_n(x-c)^n$ konverguje a máme
$$
f(x)=\sum_{n=0}^k a_n(x-c)^n+\sum_{n=k+1}^\infty a_n(x-c)^n,
$$
a $R_k(f,x)=\sum_{n=k+1}^\infty a_n(x-c)^n$ konverguje k nule následkem konvergence $\sum_{n=0}^\infty a_n(x-c)^n$. Nadto, jak jsme již pozorovali, máme $a_k=\frac{f^{(k)}(x)}{k!}$.\sq

\bigskip

{\bf 3.4.} Není vždy snadné počítat koeficienty
$\frac{f^{(n)}(c)}{n!}$ Taylorovy řady funkce $f$ opakovaným derivováním. Někdy ale můžeme Taylorovu řadu určit velmi snadno užitím
tvrzení 3.3.1 a věty 2.5.1.

\medskip

{\bf 3.4.1. Příklad: logaritmus.} Máme $(\lg(1-x))'=\frac{1}{x-1}$. Jelikož
$$
\frac{1}{x-1}=-1-x-x^2-x^3-\cdots
$$
máme podle 2.5.1 (a 3.3.1)
$$
\lg(1-x)= C-x-\frac12 x^2-\frac13 x^3-\frac14 x^4 -\cdots
$$
a je jelikož $\lg 1=\lg(1-0)=0$ máme $C=0$ a získáváme známou formuli $\lg(1-x)=-\sum_{n=1}^\infty\frac{x^n}{n}$.

\medskip

{\bf 3.4.2. Příklad: arkustangens.} Máme $\arctan(x)'=\frac{1}{1+x^2}$. Jelikož
$$
\frac{1}{1+x^2}=1-x^2+x^4-x^6+x^8-\cdots
$$
použiyím primitivní funkce dostáváme
\begin{equation}
\arctan(x)=x-\frac13 x^3+\frac15 x^5-\frac17 x^7+\frac19 x^9-\cdots \tag{$*$}
\end{equation}
Aditivní konstanta je 0 protože $\arctan(0)=0$.

\medskip

{\bf 3.4.3. Nepříliš efektivní ale elegantní formule pro $\pi$.} Formule ($*$)  napovídá, že
$$
\frac{\pi}{4}=\arctan(1)=1-\frac13 +\frac15 -\frac17 +\frac19 -\cdots.
$$
Tato rovnice skutečně platí, ale úplně bezprostřední není. Proč: poloměr konnvergence mocninné řady
 $f(x)=x-\frac13 x^3+\frac15 x^5-\frac17 x^7+\frac19 x^9-\cdots$ je 1 takže argument 1 je na hranici kruhu konvergence $\setof{x}{|x|<1}$ o kterém obecné tvrzení nic neříká  (připomeňte si 2.4). Funkce $\arctan$ je spojitá a pro $|x|<1$ máme $\arctan(x)=f(x)$. Takže je potřeba dokázat, že
 $$
 \lim_{x\to 1-}f(x)=1-\frac13 +\frac15 -\frac17 +\frac19 -\cdots.
 $$
 Vezměmě $\epsilon>0$. Řada $1-\frac13 +\frac15 -\frac17 +\frac19 -\cdots$ konverguje (i když neabsolutně) a
existuje tedy $n$ takové, že $|P_n|<\epsilon$
 pro $P_n=\frac{1}{2n+1}-\frac{1}{2n+3}+\frac{1}{2n+5}-\cdots$. Zvolme nyní $\delta>0$ takové, že $1-\delta<x<1$ a pro $P_n(x)=\frac{1}{2n+1}x^{2n+1}-\frac{1}{2n+3}x^{2n+3}+\frac{1}{2n+5}x^{2n+5}-\cdots$ máme
 $$
 \begin{aligned}
 &|P_n(x)|<\epsilon\ \  \text{a}\\
 &|(x-\frac13 x^3+\frac15 x^5-\cdots\pm\frac{1}{2n-1}x^{2n-1})-
 (1-\frac13 +\frac15 -\cdots\pm\frac{1}{2n-1})|<\epsilon. 
 \end{aligned}
 $$
 Teď můžeme najít odhad pro $1-\delta<x<1$ rozdílu mezi $f(x)$ a alternující řadou $1-\frac13 +\frac15 -\frac17 +\frac19 -\cdots$ :
 $$
 \begin{aligned}
 |f(x)&-(1-\frac13 +\frac15 -\frac17 +\frac19 -\cdots)|=\\
 &|(x-\frac13 x^3+\frac15 x^5-\cdots\pm\frac{1}{2n-1}x^{2n-1}\mp P_n(x)) -\\
 &\qquad\qquad\qquad-(1-\frac13 +\frac15 -\cdots\pm\frac{1}{2n-1}\mp P_n)|\leq\\
 &\leq|(x-\frac13 x^3+\frac15 x^5-\cdots\pm\frac{1}{2n-1}x^{2n-1})-\\
 &\qquad\qquad\qquad-(1-\frac13 +\frac15 - \cdots\pm\frac{1}{2n-1})|+|P_n(x)|+|P_n|<3\epsilon.
 \end{aligned}
 $$
 Všimněte si že to skutečně je jen jednostranná limita: pro $f(x)$ nedává smysl $x>1$. 



 

 
 \newpage
.
 
  
\newpage 
 
  
 \centerline{\Large\bf XX. Fourierovy řady} 
 
 \vskip10mm
 
  
 {\large\bf 1. Periodické a po částech hladké funkce.}
 
 \bigskip
 
 {\bf 1.1. Po částech spojité a hladké funkce.} Reálná funkce $f:\langle a,b\rangle:\to\Rbb$ je {\em po částech spojitá} existují-li čísla
$$
a=a_0<a_1< a_2<\cdots <a_n=b
$$
taková, že
\begin{itemize}
\item $f$ je spojitá na každém otevřeném intervalu $(a_j,a_{j+1})$ a
\item existují jednostranné konečné limity $\lim_{x\to a_j+}f(x)$, $j=0,\dots,n-1$ a $\lim_{x\to a_j-}f(x)$, $j=1,\dots,n$.
\end{itemize}
Je {\em po částech hladká} má-li navíc
\begin{itemize}
\item $f$ spojité derivace na každém otevřeném intervalu  $(a_j,a_{j+1})$ a
\item existují jednostranné limity $\lim_{x\to a_j+}f'(x)$, $j=0,\dots,n-1$ a $\lim_{x\to a_j-}f'(x)$, $j=1,\dots,n$.
\end{itemize}
Pro $y\in\langle a,b\rangle$ položme
$$
f(y+)=\lim_{x\to y+}f(x),\quad f(y-)=\lim_{x\to y-}f(x)\qtq{a}
f(y\pm)=\frac{f(y+)+f(y-)}{2}.
$$
O číslech $a_i$ budeme mluvit jako o {\em výjimečných bodech} funkce  $f$.


\medskip

{\bf 1.1.1. Poznámky a pozorování.} 1. Po částech spojitá funkce
 $f$ může být rozšířena na spojitou funkci na každém intervalu $\langle a_j,a_{j+1}\rangle$. Má tedy Riemannův integrál.

2. Pokud $y\notin\set{a_0,a_1,\dots,a_n}$ je $f(y+)=f(y-)=f(y\pm)=f(y)$. Pokud $y=a_i$ toto nemusí platit. Rozdělující body $a_i$ v nichž
$f(a_i+)=f(a_i-)=f(a_i\pm)=f(a_i)$ mohou být považovány za přebytečné v případě, že mluvíme jen o spojitosti po částech, ne však mluvíme-li o hladkosti počástech: bereme v úvahu též funkce bez derivací v některých bodech, kde jsou spojité.

3. Je možné se ptát zda body v nichž $f(y+)=f(y-)\neq f(y)$ mají nějaký speciální status. Pro nás zde ne: budeme se zajímat o integrály po částech spojitých funkcí a hodnoty v isolovaných bodech nebudou hrát roli. 

4. Připomeňte si VII.3.2.1. Poslední podmínka pro hladkost po částech je totéž jako existence jednostranných derivací ve výjimečných bodech.

\bigskip

{\bf 1.2. Periodické funkce.} Říkáme, že reálná funkce $f:\Rbb\to\Rbb$ je {\em periodická} s {\em periodou} $p$ platí-li
$$
\forall x\in\Rbb, \quad f(x+p)=f(x).
$$

\medskip

{\bf 1.2.1. Úmluva.} Říkáme, že periodická funkce je po částech spojitá resp. počástech hladká
je-li restrikce $f|\langle 0,p\rangle$ po částech spojitá resp. po částech hladká.


\bigskip

{\bf 1.3. Funkce na kompaktních intervalech representované jako periodické funkce (a také  opačně).} V této kapitole bude výhodné representovat reálnou funkci
 $f:\langle a,b\rangle\to\Rbb$ jako periodickou funkci $\wt f:\Rbb\to\Rbb$ s periodou $p=b-a$ definovanou předpisem
$$
\begin{aligned}
&\wt f(x+kp)=f(x) \qtq{pro} x\in (a,b) \ \ \text{a každé celé číslo} \ k,\\
&\wt f(a+kp)= \frac12(f(a)+f(b)).
\end{aligned}
$$
Je-li tato záměna zřejmá, píšeme prostě  $f$ místo $\wt f$;  typicky při počítání integrálů, kdy na možné záměně hodnot
v $a$ a $b$ nezáleží.

Na druhé straně neztratíme žádnou informaci budeme-li periodické funkce  s periodou $p$ studovat v restrikci na některý z intervalů $\langle a, a+p\rangle$.


\bigskip

{\bf 1.4. Tvrzení.} {\em Buď $f$ po částech spojitá periodická s periodou $p$. Potom
$$
\int_0^p f(x)\d x=\int_a^{p+a} f(x)\d x\ \ \text{pro každé} \ \ a\in\Rbb.
$$

Důkaz.} Zřejmě je $\int_b^{c}f=\int_{b+p}^{c+p}f$ a tedy rovnice platí pro $a=kp$ s celým číslem $k$. Buď nyní
$a$ obecné. Zvolme celé číslo $k$ takové, že $a\leq kp\leq a+p$. Potom je
$$
\begin{aligned}
\int_{a}^{p+a}f=\int_{a}^{kp}f&+\int_{kp}^{p+a}f=
\int_{p+a}^{(k+1)p}f+\int_{kp}^{p+a}f=\\
&=\int_{kp}^{p+a}f+\int_{p+a}^{(k+1)p}f=\int_{kp}^{(k+1)p}f=\int_{0}^{p}f.
\end{aligned}
$$
\sq

\medskip

 Substitucí $y=x+C$ a užitím XI.5.5 dostaneme

\smallskip

{\bf 1.4.1. Důsledek.} {\em Pro libovolné reálné $C$ platí
$$
\int_{0}^{p}f(x+C)\d x=\int_{0}^{p}f(x)\d x.
$$}

 \vskip10mm
 
 {\large\bf 2. Něco jako skalární součin.}
 
 \bigskip

Abychom mohli pracovat se $\sin kx$ a $\cos kx$ omezíme se v dalším, až do bodu 4.4.1, na periodické funkce s periodou $2\pi$.

\bigskip

{\bf 2.1.} Jsou-li $f,g$ po částech hladké na $\langle-\pi,\pi\rangle$ potom zřejmě totéž platí o $f+g$ a kterékoli $\alpha f$ s reálným $\alpha$.
Tedy množina všech po částech hladkých funkcí na $\langle-\pi,\pi\rangle$
tvoří vektorový prostor
$$
\text{PSF}(\langle-\pi,\pi\rangle).
$$

\bigskip

{\bf 2.2.} Pro $f,g\in\text{PSF}(\langle-\pi,\pi\rangle)$ definujme
$$
[f,g]=\int_{-\pi}^{\pi}f(x)g(x)\d x.
$$
Tato funkce $[-,-]:\text{PSF}(\langle-\pi,\pi\rangle)\times\text{PSF}(\langle-\pi,\pi\rangle)\to\Rbb$
se chová skoro jako skalární součin. Viz následující

\medskip

{\bf 2.2.1. Tvrzení.} {\em Platí
\begin{enumerate}
\item $[f,f]\geq 0$ a $[f,f]=0$ právě když $f(x)=0$ ve všech $x$ až na výjimečné,
\item $[f+g,h]=[f,h]+[g,h]$, a
\item $[\alpha f,g]=\alpha[f,g]$.
\end{enumerate}

Důkaz} je triviální; jediné, co snad potřebuje vysvětlení je druhá část bodu (1).  Pokud $f(y)=a\neq 0$ není výjimečný potom pro nějaké $\delta>0$, je $f(x)>\frac{a}{2}$   pro $y-\delta < x <y-\delta$ a máme
$$
[f,f]=\int_{-\pi}^{pi}f^2(y)\d x\geq\int_{y-\delta}^{y+\delta}f^2(x)\d x\geq\delta\frac{a^2}{2}.
$$
\sq
 

 \medskip

{\bf 2.2.2. Poznámka.} Jediná drobná vada krásy je v tom, že
 $[f,f]$ tak úplně neimplikuje $f\equiv 0$. To se ale týká jen konečně mnoha argumentů a pro naše účely je to zcela nepodstatné.

\bigskip

{\bf 2.3. Několi formulí, které si potřebujeme připomenout.} Ze standardních
$$
\begin{aligned}
&\sin(\alpha+\beta)=\sin\alpha\cos\beta+\sin\beta\cos\alpha\ \ \text{and}\\
&\cos(\alpha+\beta)=\cos\alpha\cos\beta-\sin\alpha\sin\beta
\end{aligned}
$$ 
okamžitě dostáváme (stejně standardní)
$$
\begin{aligned}
&\sin\alpha\cos\beta=\frac12(\sin(\alpha+\beta)-\sin(\alpha-\beta)),\\
&\sin\alpha\sin\beta=\frac12(\cos(\alpha-\beta)-\cos(\alpha+\beta)),\\
&\cos\alpha\cos\beta=\frac12(\cos(\alpha+\beta)+\cos(\alpha-\beta)).
\end{aligned}
$$
\bigskip

{\bf 2.4. Tvrzení.} {\em Pro každá dvě přirozená $m,n\in \Nbb$ máme  $[\sin mx,\cos nx]=0$. Pokud $m\neq n$ platí $[\sin mx,\sin nx]=0$ a
$[\cos mx,\cos nx]=0$. Dále, $[\cos 0x,\cos 0x]=[1,1]=2\pi$ a
 $[\cos nx,\cos nx]=[\sin nx,\sin nx]=\pi$ pro každé $n>0$. 
Systém funkcí
$$
\frac{1}{2\pi},\ \frac{1}{\pi}\cos{x},\ \frac{1}{\pi}\cos{2x},\ \frac{1}{\pi}\cos{3x},\ \dots,\ \frac{1}{\pi}\sin{x},\ \frac{1}{\pi}\sin{2x},\ \frac{1}{\pi}\sin{3x},\ \dots    
$$
je tedy orthonormální v $(\text{PSF}(\langle-\pi,\pi\rangle),[-,-])$.

Důkaz.} Podle 2.3 máme $\sin mx\cos nx=\frac12(\sin(m+n)x-\sin(m-n)x)$,
$\sin mx\sin nx=\frac12(\cos(m-n)x-\cos(m+n)x)$ a
$\cos mx\cos mx=$\newline $\frac12(\cos(m+n)x+\cos(m-n)x)$. Primitivní funkce pro
$\sin kx$ resp. $\cos kx$ je $-\frac1k\cos kx$ resp. $\frac1k\sin kx$ a hodnoty získáme snadno z
 XI.4.3.1.\sq


 \vskip10mm
 
 {\large\bf 3. Dvě užitečná lemmata.}

\bigskip

{\bf 3.1. Lemma.} {\em Buď $g$ po částech spojitá funkce na $\langle a,b\rangle$. Potom
$$
\lim_{y\to+\infty}\int_a^b g(x)\sin (yx) \d x=0.
$$

Důkaz.} Jsou-li $a_0,a_1,\dots ,a_n$ výjimečné body $g$ máme $\int_a^bg=
\sum_{i=0}^{n-1}\int_{a_i}^{a_{i+1}}g$ a tedy stačí tvrzení dokázat pro spojité (a tedy stejnoměrně spojité) $g$.

Jelikož primitivní funkce k $\sin (yx)$ je $-\frac1y\cos (yx)$ máme pro libovolné meze
 $u,v$,
$$
\left|\int_u^v \sin (yx) \d x\right|=\left|\left[-\frac1k\cos (yx)\right]_u^v\right|\leq \frac2y.
$$
Zvolme $\epsilon>0$. Funkce $g$ je stejnoměrně spojitá a tedy
existuje $\delta>0$ takové, že pro $|x-z|<\delta$ je $|g(x)-g(z)|<\epsilon$.
Zvolme rozklad $a=t_1<t_2<\cdots<t_n=b$ intervalu $\langle a,b\rangle$ s jemností $<\delta$, tedy takový, že $t_{i+1}-t_i<\delta$ pro všchna $i$.

Nyní buď
$$
y> \frac4{\epsilon}\sum_{i=1}^n|g(t_i)|.
$$
Potom máme
$$
\begin{aligned}
&\left|\int_a^bg(x)\sin (yx) \d x\right|=\\
&\quad\left|\sum_{i=1}^n\left(\int_{t_{i-1}}^{t_i}(g(x)-g(t_{i}))\sin (yx) \d x+
g(t_i)\int_{t_{i-1}}^{t_i}\sin (yx) \d x\right)\right|\leq\\
&\quad\leq \sum_{i=1}^n\int_{t_{i-1}}^{t_i}\frac{\epsilon}{2(b-a)}\d x +
\sum_{i=1}^n|g(t_i)|\cdot\left|\int_{t_{i-1}}^{t_i}\sin (yx)\d x\right|\leq
\frac{\epsilon}{2}+\sum|g(t_i)|\frac2y\leq \epsilon.
\end{aligned}
$$
\sq

\medskip

{\bf 3.1.1. Poznámka.} Lemma 3.1
je ve skutečnosti velmi názorný fakt. Přepokládejme, že počítáme $\int_a^b C\sin (yx) \d x$ s konstantou $C$. Je-li potom $y$ velké, je
přibližně stejně mnoho hodnot  nad a pod  $x$-ovou osou.
Nadto, je-li  $y$ ještě mnohem větší, děje se to již na krátkých podintervalech $\langle a,b\rangle$ kde $g$ se již chová ``skoro jako konstanta''.

\bigskip

{\bf 3.2. Lemma.} {\em Buď $\sin\frac{\alpha}{2}\neq 0$. Potom
$$
\frac12+\sum_{k=1}^n \cos k\alpha=\frac{\sin(2n+1)\frac{\alpha}{2}}{2\sin\frac{\alpha}{2}}.
$$

Důkaz.} Podle první formule v 2.3 máme
$$
2\sin\frac\alpha{2}\cos k\alpha=\sin\left(k\alpha+\frac{\alpha}2\right)-
\sin\left((k-1)\alpha+\frac{\alpha}2\right).
$$
Tedy,
$$
\begin{aligned}
2\sin\frac{\alpha}2&\left(\frac12+\sum_{k=1}^n \cos k\alpha\right)=
\sin\frac{\alpha}2 +\sum_{k=1}^n 2\sin\frac\alpha{2} \cos k\alpha=\\
&=\sin\frac{\alpha}2 +\sum_{k=1}^n \left(\sin\left(k\alpha+\frac{\alpha}2\right)-
\sin\left((k-1)\alpha+\frac{\alpha}2\right)\right)=\\ 
&=\sin(2n+1)\frac{\alpha}2.
\end{aligned}
$$
\sq


\vskip10mm

 {\large\bf 4. Fourierovy řady.}

\bigskip

{\bf 4.1.} Z lineární algebry si připomňme representaci obecného vektoru jako lineární kombinace prvků orthonormální base:

 Buď 
$$
\ve{u}_1,\ve{u}_2,\dots,\ve{u}_n
$$
orthonormální base, to jest base pro kterou $\ve{u}_i\ve{u}_j=\delta_{ij}$, vektorového prostoru $V$ se skalárním součinem $\ve{u}\ve{v}$. Potom je obecný vektor $\ve{a}$ vyjádřen jako
$$
\ve{a}=\sum_{i=1}^na_i\ve{u}_i \qtq{kde} a_i=\ve{a}\ve{u}_i.
$$
Uvidíme, že něco podobného se stane s orthonormálním systémem z 2.4.

\bigskip

{\bf 4.2.} Buď $f$ po částech hladká funkce s periodou $2\pi$. Položme 
$$
\begin{aligned}
&a_k=[f,\frac1\pi\cos kx]=\frac1{\pi}\int_{-\pi}^{\pi}f(t)\cos kt\d t \quad \text{for}\ k\geq 0, \ \ \ \text{and}\\
&b_k=[f,\frac1\pi\sin kx]=\frac1{\pi}\int_{-\pi}^{\pi}f(t)\sin kt\d t \quad \text{for}\ k\geq 1.
\end{aligned}
$$
Budeme směřovat k   důlazu, že $f$ se téměř shoduje s
$$
\frac{a_0}{2}+\sum_{k=1}^\infty(a_k\cos kx+ b_k\sin kx).
$$
Tedy se orthonormální systém z 2.3 chová podobně jako orthonormální base. Je zde samozřejmě ten rozdíl, že k tomu budeme potřebovat 
 {\em nekonečné součty} (``nekonečné lineární kombinace'') abychom representovali $f\in\text{PSF}(\langle-\pi,\pi\rangle)$ (což je podstatné) a že $f$ bude representováno až na konečně mnoho hodnot (což je nepodstatné).

\bigskip

{\bf 4.3.} Položme
$$
s_n(x)=
\frac{a_0}{2}+\sum_{k=1}^n(a_k\cos kx+ b_k\sin kx).
$$

\medskip

{\bf 4.3.1. Lemma.} {\em Pro každé $n$ je 
$$
s_n(x)=\frac1\pi\int_0^\pi (f(x+t)+f(x-t))\cdot\frac{\sin(n+\frac12)t}{2\sin\frac{1}{2}t}\d t.
$$

Důkaz.} Použitím vzorců pro $a_n$ a $b_n$ a standardní formule pro $\cos k(x-t)=\cos(kx-kt)$, a potom užitím rovnosti z 3.2 dostaneme
$$
\begin{aligned}
s_n&(x)=\frac1\pi\int_{-\pi}^\pi\left(\frac12+\sum_{k=1}^n(\cos kt\cdot\cos kx+\sin kt\cdot\sin  kx)\right)f(t)\d t=\\
&\frac1\pi\int_{-\pi}^\pi\left(\frac12+\sum_{k=1}^n\cos k(x-t)\right)f(t)\d t=
\frac1\pi\int_{-\pi}^\pi\left(f(t)\frac{\sin(n+\frac12)(x-t)}{2\sin\frac{x-t}{2}}\right)\d t
\end{aligned}
$$
Nyní užijme substituce $t=x+z$. Potom je $\d t =\d z$ a $z=t-x$, a jelikož $\sin(-u)=-\sin u$ můžeme pokračovat (užívajíce též 1.4)
$$
\cdots=\frac1\pi\int_{-\pi}^\pi\left(f(x+z)\frac{\sin(n+\frac12)z}{2\sin\frac12z}\right)\d z =
\frac1\pi\left(\int_{0}^\pi \cdots+\int_{-\pi}^0\cdots\right).
$$
Substitucí $y=-z$ v druhém sčítanci dostaneme
$$
\cdots=\frac1\pi\int_{0}^\pi\left(f(x+z)\frac{\sin(n+\frac12)z}{2\sin\frac12z}\right)\d z+
\frac1\pi\int_{-\pi}^\pi\left(f(x-y)\frac{\sin(n+\frac12)y}{2\sin\frac12y}\right)\d y
$$
a nahradíme-li proměnné v obou integrálech proměnnou $t$ dostaneme
$$
\cdots=\frac1\pi\int_{0}^\pi(f(x+t)+f(x-t))\frac{\sin(n+\frac12)t}{2\sin\frac12t}\d t.
$$
\sq

\medskip

{\bf 4.3.2. Důsledek.} {\em Pro každé $n$ je,
$$
\frac1\pi\int_0^\pi\frac{\sin(n+\frac12)t}{\sin\frac{1}{2}t}\d t = 1.
$$

Důkaz.} Uvažme konstantní funkci $f=(x\mapsto 1)$. Potom je $a_0=2$ a $a_k=b_k=0$ pro všechna $k\geq 1$. \sq

\bigskip

{\bf 4.4. Věta.} {\em Buď $f$ po částech hladká periodická funkce s periodou $2\pi$. Potom (protože $f(x\pm)=\frac12(f(x+)+f(x-)$) řada
$
\sum_{k=1}^\infty(a_k\cos kx+ b_k\sin kx)
$ konverguje v každém $x\in \Rbb$ a máme (viz 1.1)
$$
f(x\pm)=\frac{a_0}{2}+\sum_{k=1}^\infty(a_k\cos kx+ b_k\sin kx).
$$

 Důkaz.}  Podle  4.3.1 a 4.3.2 máme
$$
\begin{aligned}
&s_n(x)=\\
&=\frac1\pi\int_0^\pi(2f(x\pm)+f(x+t)-f(x+)+ f(x-t)-f(x-))\frac{\sin(n+\frac12)t}{\sin\frac{1}{2}t}\d t=\\
&=f(x\pm)\cdot\frac1\pi\int_0^\pi\frac{\sin(n+\frac12)t}{\sin\frac{1}{2}t}\d t\ +\\
&+\frac1\pi\int_0^\pi\left(\frac{f(x+t)-f(x+)}{t}+ \frac{f(x-t)-f(x-)}{t}\right)\frac{\frac12t}{\sin\frac12t}\sin\left(n+\frac12\right)t\d t.
\end{aligned}
$$
Položme
$$
g(t)=\left(\frac{f(x+t)-f(x+)}{t}+ \frac{f(x-t)-f(x-)}{t}\right)\frac{\frac12t}{\sin\frac12t}.
$$
Funkce $g$ je po částech spojitá na intervalu $\langle 0,\pi\rangle$: to je zřejmé pro $t>0$ a v $t=0$  máme konečnou limitu vzhledem k levým a pravým derivacím  $f$ v $x$ a standardní $\lim_{t\to 0}\frac{\frac12t}{\sin\frac12t}=1$. Můžeme tedy užít Lemma 3.1 (a Důsledek
4.3.2) k závěru
$$
\lim_{\to\infty}s_n(x)=f(x\pm).
$$\sq


\bigskip

{\bf 4.4.1.} Věta 4.4 může být snadno upravena pro po částech hladké periodické funkce s obecnou periodou  $p$. Pro takové $f$ dostaneme

\smallskip

{\em $$
f(x\pm)=\frac{a_0}{2}+\sum_{k=1}^\infty(a_k\cos\frac{2\pi}{p}kx+ b_k\sin\frac{2\pi}{p}kx)
$$
kde
$$
\begin{aligned}
&a_k=\frac2{p}\int_{0}^{p}f(t)\cos\frac{2\pi}{p}kt\d t \ \ \text{for}\ k\geq 0, \ \ \text{a}\\
&b_k=\frac2{p}\int_{0}^{p}f(t)\sin\frac{2\pi}{p}kt\d t \ \ \text{for}\ k\geq 1.
\end{aligned}
$$
Užitím representace z 1.3 to můžeme aplikovat pro po částech hladké funkce na intervalu $\langle a,b\rangle$ položíme-li $p=b-a$.}

\medskip

{\bf 4.4.2.} Řada 
$\frac{a_0}{2}+\sum_{k=1}^\infty(a_k\cos kx+ b_k\sin kx)$ resp.
 $\frac{a_0}{2}+\sum_{k=1}^\infty(a_k\cos kx+ b_k\sin kx)$ se nazývá Fourierova řada funkce $f$. Připomínám, že její součet je roven $f(x)$ ve všech nevýjimečných bodech.

\vskip10mm

 {\large\bf 5. Poznámky.}

\bigskip

{\bf 5.1.} Součty $s_n(x)$ jsou spojité, ale výsledná 
$f$ spojitá být nemusí. Konvergence Fourierovy řady z 4.4 často není stejnoměrná
(viz XIX.1.3). 

Pokud součty $\sum |a_n|$ a $\sum |b_n|$ konvergují, potom samozřejmě Fourierova řada  konverguje stejnoměrně a absolutně, a pokud i  $\sum n|a_n|$ a $\sum n|b_n|$ konvergují, můžeme ji derivovat člen po členu.

\bigskip

{\bf 5.2.} Derivování člen po členu může být nekorektní i když výsledný součet derivaci má. Tady je příklad. Vezměme $f(x)=x$ na $(-\pi,\pi\rangle$ rozšířenou na periodickou funkci s periodou $2\pi$. Zde dostáváme
$$
f(x\pm)=2(\sin x-\frac12\sin 2x+\frac13\sin3x-\frac14\sin 4x+\cdots).
$$
$f(x)$ s derivací 1 ve všech $x\neq(2k+1)\pi$. Formální derivace člen po členu
ale dává
$$
g(x)=2(\cos x-\cos 2x+\cos3x-\cos 4x+\cdots)
$$
a píšeme-li $g_n(x)$ pro částečný součet $n$ sčítanců dostáváme $g_n(0)=
2(1-1+1-\cdots+(-1)^{n+1})$, tedy $g_n(0)=0$ pro $n$ sudé a $g_n(x)=2$ pro $n$ liché.

\bigskip

{\bf 5.3.} Všimněte si, že pro $f$ s $f(-x)=f(x)$ jsou všechna $b_n$ nuly, a je-li $f(-x)=-f(x)$ jsou nuly všechna $a_n$.

\bigskip

{\bf 5.4.} Fourierovy řady mají zajímavou interpretaci v akustice. Tón je popsán  periodickou funkcí $f$. Jeho výška je určena periodou $p$ (přesněji,  {\em frekvencí} $\frac1p$). Funkce $f$ je při tom zřídka sinusoidální .
Konkretní tvar $f$ určuje {\em kvalitu} (barvu) tónu charakteristickou pro ten který hudební nástroj. Ve Fourierově representaci vidíme u prvního sčítance
 základní frekvenci, určující výšku, a zároveň s tím znějí tóny v dvojnásobné, trojnásobné, atd. frekvenci.  
Tak např. hrajete-li na flétnu dostanete se o oktávu výše  ``odfouknutím prvního basického tónu'' následkem čehož ten který je nyní první má dvojnásobnou frekvenci.


\newpage


 \def\ver#1{\text{\boldgreek#1}}
 
 \centerline{\Large\bf XXI. Křivky a křivkové integrály} 
 
 \vskip10mm
 
  
 {\large\bf 1.  Křivky.}
 
 \bigskip 
 
V aplikacích v následujících kapitolách budeme užívat jen rovinné křivky. Ale pro materál v prvních dvou sekcích v této kapitole by omezení dimense nic nezjednušilo. 
 
\bigskip 
 
 {\bf 1.1. Parametrizovaná křivka.}  {\em Parametrizovaná křivka} v $\Ebb_n$ je spojité zobrazení
 $$
 \ver{\phi}=(\phi_1,\dots,\phi_n):\langle a,b\rangle\to\Ebb_n
 $$
(kde kompaktní interval  $\langle a,b\rangle$ bude vždy netriviální, t.j. s $a<b$).
 
 \bigskip  
 
 {\bf 1.2.  Dvě ekvivalence.} Parametrizované křivky $\ver{\phi}=(\phi_1,\dots,\phi_n):\langle a,b\rangle\to\Ebb_n$ a 
  $\ver{\psi}=(\psi_1,\dots,\psi_n):\langle c,d\rangle\to\Ebb_n$ jsou {\em slabě ekvivalentní} existuje-li homeomorfismus
   $\alpha:\langle a,b\rangle\to\langle c,d\rangle$ takový, že $\ver{\psi}\circ\alpha=\ver{\phi}$. Píšeme
  $$
  \ver{\phi}\sim\ver{\psi}.
  $$
  (Tato relace je zřejmě reflexivní, je symetrická protože inverse homeomorfismu je homeomorfismus, a transitivní protože složení homeomorphismů je homeomorphismus.)
  
  Křivky
  $\ver{\phi}$ a
  $\ver{\psi}$ jsou {\em equivalenní} existuje-li
   {\em rostoucí}  homeomorphismus $\alpha:\langle a,b\rangle\to\langle c,d\rangle$ takový, že $\ver{\psi}\circ\alpha=\ver{\phi}$. Píšeme
  $$
  \ver{\phi}\approx\ver{\psi}.
  $$
	
	\medskip
	
	{\bf 1.2.1.} Zejména budeme pracovat s
	\begin{itemize}
	\item křivkami representovanými prostými $\ver{\phi}$, říká se jim {\em jednoduché oblouky}, a
	\item křivkami representovanými $\ver{\phi}$ které jsou prosté s výjimkou $\ver{\phi}(a)=\ver{\phi}(b)$, a těm se říká {\em jednoduché uzavřené křivky}.
	\end{itemize}
  
  \medskip
  
  {\bf 1.2.2. Tvrzení.} {\em $\sim$-třída ekvivalence jednoduchého oblouku nebo jednoduché uzavřené křivky
	je disjunkní sjednocení přesně dvou 
   $\approx$-tříd ekvivalence.
  
  Důkaz.} Jelikož $\ver{\phi}\approx\ver{\psi}$ implikuje $\ver{\phi}\sim\ver{\psi}$,  $\sim$-třída je nutně (disjunktní) sjednocení $\approx$-tříd. Homeomorfismus $\alpha$ v $\ver{\psi}\circ\alpha=\ver{\phi}$ je (podle požadavku na $\ver{\phi}$) jednoznačně určen (je jednoznačně určen na $(a,b)$ a tedy na celém kompaktním intervalu podle IV.5.1 - existují posloupnosti v $(a,b)$ konvergující k $a$ resp. $b$) a tedy je např. $\ver{\phi}$ a  $\ver{\phi}\circ\iota$ kde $\iota(t)=-t+b+a$, jsou $\sim$-equivalentní ale ne  $\approx$-equivalenní. Nyní buď $\ver{\phi}\sim\ver{\psi}$ s $\alpha$ takovým, že $\ver{\psi}\circ\alpha=\ver{\phi}$. Potom podle IV.3.4 $\alpha$ buď roste nebo klesá. V prvním případě $\ver{\psi}\approx\ver{\phi}$, ve druhém je 
  $\ver{\psi}\circ\alpha\circ\iota=\ver{\phi}\circ\iota$ a $\alpha\circ\iota$ roste, takže $\ver{\psi}\approx\ver{\phi}\circ\iota$.\sq
  
   \bigskip
 
 {\bf 1.3.}  $\sim$-třída ekvivalence $L=[\phi]_\sim$ se nazývá {\em křivka}.  $\approx$-třídy asociované s touto křivkou representují její orientace; mluvíme pak o  {\em orientovaných křivkách}
 $L=[\phi]_\approx$. 

Podle 1.2.2, mají jednoduchý oblouk a jednoduchá uzavřená křivka  dvě orientace.

Parametrizovaná křivka $\ver{\phi}$ taková, že $L=[\phi]_\sim$ resp. $L=[\phi]_\approx$ se nazývá {\em parametrizace} křivky $L$.
 
Často prostě mluvíme o parametrizované křivce  $\ver{\phi}:\langle a,b\rangle\to\Ebb_n$ jako o křivce resp. orientované křivce $\ver{\phi}$. Máme při tom samozřejmě na mysli s ní spojenou $\sim$-  resp. $\approx$-třídu.
 

 \medskip
 
 {\bf 1.3.1. Poznámky.} 1. Parametrizovanou křivku si můžeme představovat jako putování po nějaké cestě při čemž $\ver{\phi}(t)$ říká, kde zrovna jsme v okamžiku
 $t$.  $\sim$-ekvivalence nás zbavuje této informace navíc (jsou zde jen koleje a žádná informace o po nich jedoucím vlaku). Orientace zachycuje směr putování. 
 
  Čtenář může mít na mysli také jednodušší representaci křivky jako množiny
 $\ver{\phi}[\langle a,b\rangle]$, tedy ``geometrický tvar'' té $\ver{\phi}$. Skutečně je, pokud $\ver{\phi}$ a $\ver{\psi}$ parametrizují jednoduchý oblouk nebo jednoduchou uzavřenou křivku, jednoduché ukázat, že
 $\ver{\phi}[\langle a,b\rangle]=\ver{\psi}[\langle c,d\rangle]$ právě když $\ver{\phi}\sim\ver{\psi}$.
  Ale užívání těch tříd ekvivalence má své výhody  (již orientování křivky je jednodušší).
 
 2. V definicích ekvivalencí $\sim$ resp. $\approx$ jsme užívali parametrické křivky  $\ver{\phi}:\langle a,b\rangle\to\Ebb_n$, 
  $\ver{\psi}:\langle c,d\rangle\to\Ebb_n$ s různými definičními obory. Zvolíme-li si pevný interval můžeme kanonicky transformovat $\ver{\psi}$ do $\ver{\psi}\circ\lambda:\langle a,b\rangle\to\Ebb_n$
  s $\lambda(t)=\frac1{b-a}((d-c)t+bc-da)$. Někdy (viz například dále definici  $\ver{\phi}*\ver{\psi}$ v 1.4) volně definiční obor posouváme podle potřeby. Zjednodušuje to formule a ničemu neuškodí.
 
 3. Tvrzení 1.2.2 platí jen pro jednoduché oblouky a jednoduché uzavřené křivky. Nakreslete si obrázek s $\ver{\phi}(x)=\ver{\phi}(y)$ pro nějaké
 $x\neq a,b$ a uvidíte, že je tam více možných orientací.
 
 4. Slovo ``uzavřená'' ve výrazu  ``jednoduchá uzavřená křivka'' nemá nic společného s uzavřeností podmnožiny v metrickém prostoru. Každá
$\ver{\phi}[\langle a,b\rangle]$ je samozřejmě kompaktní a tedy uzavřená v příslušném prostoru 
$\Ebb_n$.
 
 \bigskip
 
{\bf 1.4. Skládání orientovaných křivek.} Buďte $K,L$ orientované křivky representované parametrickými  $\ver{\phi}:\langle a,b\rangle\to\Ebb_n$, 
  $\ver{\psi}:\langle b,c\rangle\to\Ebb_n$ (pokud druhá původně nezačínala v  $b$ transformujme ji tak jak je naznačeno v 3.3.1.2)
 takové, že $\ver{\phi}(b)=\ver{\psi}(b)$. Položme
    $$
  (\ver{\phi}*\ver{\psi})(t)=\begin{cases}\ver{\phi}(t)\ \ \text{pro}\ \ t\in\langle a,b\rangle\ \ \text{a}\\
                                      \ver{\psi}(t)\ \ \text{pro}\ \ t\in\langle b,c\rangle.
                                      \end{cases}
  $$ 
  Zřejmě je $\ver{\phi}*\ver{\psi}$ spojité zobrazení  $\langle a,c\rangle\to\Ebb_n$ a vidíme, že pokud                    $\ver{\phi}\approx\ver{\phi}_1:\langle a_1,b_1\rangle\to\Ebb_n$ a  
  $\ver{\psi}\approx\ver{\psi}_1:\langle b_1,c_1\rangle\to\Ebb_n$   potom $\ver{\phi}*\ver{\psi}  \approx
  \ver{\phi}_1*\ver{\psi}_1$ (všimněte si, že je podstatné, že $K,L$ jsou {\em orientované křivky}, ne jen křivky).
   Orientovaná křivka (popsaná zobrazením) $\ver{\phi}*\ver{\psi}$  tedy  závisí jen na $K$ a $L$; budeme ji označovat
  $$
  K+L.
  $$
  (Všimněte si ještě, že operace $K+L$ je asociativní.)          
 
 \bigskip
 
{\bf 1.5. Opačná orientace.} Pro orientovanou křivku $L$ representovanou pomocí
$\ver{\phi}:\langle a,b\rangle\to\Ebb_n$ definujeme  {\em orientovanou křivku s opačnou orientací}
$$
-L
$$
jako $\approx$-třídu určenou $\ver{\phi}\circ\iota:\langle a,b\rangle\to\Ebb_n$ s $\iota(t)=-t+b+a$ (připomeňte si důkaz 1.2.2). Zřejmě je $-L$ určena orientovanou křivkou $L$.

\bigskip

{\bf 1.6. Po částech hladké křivky.} Připomeťe si XX.1.1. Parametrizovaná křivka (orientovaná křivka, nebo křivka)
$\ver{\phi}=(\phi_1,\dots,\phi_n):\langle a,b\rangle\to\Ebb_n$ je  {\em po částech hladká}
může-li být v každé $\phi_j$ system výjimečných bodů
$
a=a_0<a_1< a_2<\cdots <a_n=b
$
vybrán tak, že
\begin{itemize}
\item pro každý z intervalů $J=(a_i,a_{i+1})$, existuje $j$ takové, že $\phi'_j(t)$ je buď kladná, nebo záporná na celém $J$.
\end{itemize}
Na druhé straně ale oslabíme požadavek hladkosti po částech tím, že dovolíme jednostranné limity
$\lim_{t\to a_j+}\phi'_j(t)$ a $\lim_{t\to a_j-}\phi'_j(t))$ (tedy jednostranné limity ve výjimečných bodech  -- viz VII.3.2) také nekonečné.

Budeme psát
$$
\ver{\phi}'\qtq{pro} (\phi'_1,\dots,\phi'_n)
$$
 (takže v konečně mnoha bodech $t\in\langle a,b\rangle$, hodnota $\ver{\phi}'(t)$ nemusí být definována; hodnoty derivací se však objeví jen pod integrálem,  takže na tom nezáleží).

\medskip

{\bf 1.6.1. Pozorování.} {\em Nechť jsou křivky $\ver{\phi}=(\phi_1,\dots,\phi_n):\langle a,b\rangle\to\Ebb_n$ and $\ver{\psi}=(\psi_1,\dots,\psi_n):\langle c,d\rangle\to\Ebb_n$ po částech hladké, a nechť je $\alpha$ takové, že $\ver{\psi}=\ver{\phi}\circ\alpha$, poskytující $\sim$- či $\approx$-eqkvivalence těch dvou parametrizací. Potom je $\alpha$ spojité a po částech hladké.}

\smallskip

(Skutečně, mezi kterýmikoli dvěma výjimečnými body, je některé z $\phi_j$ prosté. Máme potom na příslušném intervalu $\alpha=\phi_j^{-1}\circ\psi_j$.)

 
 
\vskip10mm

 {\large\bf 2. Křivkové integrály.}


\bigskip

{\bf Úmluva.} V dalším budou křivky vždy po částech hladké.

\bigskip

{\bf Note.} Čtenáři bude asi divné, že budeme nejprve mluvit o integrálu druhého druhu a teprve později o integrálu prvního druhu.
Terminologie určující ``první'' resp. ``druhý druh'' je tradiční. Důvod může být v poněkud zřejmějším geometrickém smyslu v případě prvního druhu.
Ale křivkový integrál druhého druhu je základnější (a ve skutečnosti integrál prvního druhu přes něj lze vyjádřit, což naopak možné není).



\bigskip

{\bf 2.1. Křivkový integrál druhého druhu.} Buď $\ver{\phi}=
(\phi_1,\dots,\phi_n):\langle a,b\rangle\to\Ebb_n$ parametrizace orientované křivky $L$ a buď
$\ve{f}:(f_1,\dots,f_n):U\to\Ebb_n$ spojitá vektorová funkce definovaná na otevřené $U\supe
\ver{\phi}[\langle a,b\rangle]$.
 {\em Křivkový integrál druhého druhu} přes orientovanou křivku $L$ je číslo
$$
\sint_L\ve{f}=\int_a^b\ve{f}(\ver{\phi}(t))\cdot\ver{\phi}'(t)\,\d t=
\sum_{j=1}^n\int_a^bf_j(\ver{\phi}(t))\phi_j'(t)\d t.
$$
(Tedy zde tečka v $\int_a^b\ve{f}(\ver{\phi}(t))\cdot\ver{\phi}'(t)\,\d t$
indikuje standardní skalární součin $n$-tic of reálných čísel.) Není-li nebezpečí nedorozumění, píšeme prostě $\int_L$ místo $\sint_L$.

\medskip

{\bf Poznámka.}
Čtenář se může setkat s integrálem druhého druhu dejme tomu
vektorových funkcí $(P,Q)$ nebo $(P,Q,R)$ označovaném
$$
\int_LP \d x+Q \d y \qtq{nebo} \int_LP \d x+Q \d y +R \d z.
$$

\bigskip

{\bf 2.2. Tvrzení.} {\em Hodnota křivkového integrálu  $\int_L\ve{f}$ nezáleží na volbě parametrizace křivky $L$.

Důkaz.} Vezměme $\ver{\phi}=\ver{\psi}\circ\alpha$ s rostoucím homeomorfismem $\alpha:\langle a,b\rangle
\to\langle c,d\rangle$. Podle 1.6.1 je $\alpha$ po částech hladké. Potom podle
XI.5.5 
$$
\begin{aligned}
\sum_{j=1}^n\int_a^bf_j(\ver{\phi}(t))\phi_j'(t)\d t&=
\sum_{j=1}^n\int_a^bf_j(\ver{\psi}(\alpha(t)))\psi_j'(\alpha(t))\alpha'(t)\d t=\\
&=\sum_{j=1}^n\int_c^df_j(\ver{\psi}(t))\psi_j'(t)\d t.
\end{aligned}
$$\sq

\bigskip

{\bf 2.3. Tvrzení.} {\em Pro operace 1.5 z 1.4 máme
$$
\sint_{-L}\ve{f}=-\sint_L\ve{f} \qtq{a} \sint_{L+K}\ve{f}=
\sint_{L}\ve{f}+\sint_{K}\ve{f}.
$$

Důkaz.} V důkazu 2.2 nahoře jsme měli $\int_c^d$ protože $\alpha$ rostlo. Pro klesající $\alpha$ by substituce dala $\int_d^c=-\int_c^d$, 
tedy $\sint_{-L}\ve{f}=-\sint_L\ve{f}$. Druhá rovnost je zřejmá. \sq

\bigskip

{\bf 2.4. Křivkový integrál prvního druhu: jen pro informaci.} Někdy též nazývaný {\em křivkový integrál podle délky},
se definuje pro neorientovanou křivku parametrizovanou
 $\ver{\phi}=
(\phi_1,\dots,\phi_n):\langle a,b\rangle\to\Ebb_n$. Buď
$f:U\to\Rbb$ spojitá reálná funkce definovaná na $U\supe
\ver{\phi}[\langle a,b\rangle]$. Idea je v modifikaci Riemannova integrálu počítáním součtů podél
 (po částech hladké) křivky místo podél intervalu. 
Součty
$$
\sum_{i-1}^kf(\ver{\phi}(t_i))\|\ver{\phi}(t_i))-\ver{\phi}(t_{i-1}))\|
$$
uvažované pro rozklady $a=t_0<t_1<\cdots<t_k=b$ konvergují s jemností konvergující k nule k 
$$
\int_a^bf(\ver{\phi}(t))\|\ver{\phi}'(t))\|\,\d t.
$$
Tento integrál se nazývá {\em křivkový integrál prvního druhu} přes $L$ a označuje
$$
\fint_Lf \qtq{or} \fint_Lf(\ve{x})\|d \ve{x}\|.
$$
Má zřejmý geometrický smysl; zejména,

{\em  délka křivky $L$ může být vyjádřena jako
$$
\fint_L1=\int_a^b\|\ver{\phi}'(t)\|\d t.
$$.}

\smallskip

Je snadné vidět, že integrál prvního druhu může být representován jako integrál druhého druhu: máme
$$
\fint_Lf=\sint_L\ve{f}
$$
kde
$$
\ve{f}(\ver{\phi}(t))=\frac{\ver{\phi}'(t)}{\|\ver{\phi}'(t)\|}.
$$

\bigskip

{\bf 2.5. Komplexní křivkový interál.} Křivkový integrál prvního druhu v dalším textu neužijeme, ale následující komplexní integrál bude mít zásadní význam.

\medskip

{\bf 2.5.1. Komplexní funkce reálné proměnné.} Aniž bychom to dále příliš zdůrazňovali budeme v dalším běžně identifikovat komplexní rovinu $\Cbb$ s euklidovskou rovinou
 $\Ebb_2$ (na $x+iy$ se budeme dívat jako na $(x,y)$ a budeme brát v úvahu, že absolutní hodnota rozdílu
 $|z_1-z_2|$ se shoduje s běžnou euklidovskou vzdáleností). Jen nesmíme zapomínat na to, že struktura $\Cbb$ je bohatší, a zejména na násobení v  {\em tělese} $\Cbb$.

Komplexní funkci jedné reálné proměnné budeme rozepisovat jako dvě reálné funkce,
$$
f(t)=f_1(t)+i f_2(t)
$$
a budeme definovat (nepřekvapivě) její derivaci $f'(t)$ jako $f'_1(t)+if_2(t)$  a její Riemannův integrál jako
$$
\int_a^bf(t)\d t=\int_a^bf_1(t)\d t+i\int_a^bf_2(t)\d t.
$$ 
Křivka $\Cbb$ v parametrizované podobě je $\phi:\langle a,b\rangle\to\Cbb$, často psané jako  $\phi(t)=\phi_1(t)+i\phi_2(t)$. Budeme s ní jednat
  (v definicích ekvivalencí, hladkosti, atd.) jako s parametrizovanou křivkou $\ver{\phi}(t)=(\phi_1(t),\phi_2(t))$;  hodnoty  násobíme jako komplexní čísla v $\Cbb$.

\medskip

{\bf 2.5.2.} Pro orientovanou po částech hladkou křivku $\phi:\langle a,b\rangle\to\Cbb$ definujeme {\em komplexní křivkový integrál} komplexní funkce jedné komplexní proměnné formulí
$$
\int_Lf(z)\d z=\int_a^bf(\phi(t))\cdot\phi'(t)\d t.
$$
Pozor: Zde je násobení, značené opět tečkou $\cdot$, (na rozdíl od násobení na předchozích stránkách, zejména v 2.1) {\em násobení v tělese} $\Cbb$.

\smallskip

Nezávislost na volbě parametrizace bude zřejmá z následujícího 

\medskip

{\bf 2.5.3. Tvrzení.} {\em Mysleme na komplexní funkci 
$f(z)=f_1(z)+if_2(z)$ jako na vektorovou funkci $\ve{f}=(f_1,f_2)$. Potom může být komplexní integrál přes $L$ vyjádřen jako křivkobý integrál druhého druhu takto:
$$
\int_Lf(z)\d z=\sint_l(f_1,-f_2)+i\sint_L(f_2,f_1).
$$
Následkem toho, 
\begin{itemize}
\item $\int_Lf(z)\d z$ nezávisí na volbě parametrizace, a
\item máme $\int_{-L}f(z)\d z=-\int_Lf(z)\d z$ and $\int_{L+K}f(z)\d z=\int_Lf(z)\d z+\int_Kf(z)\d z$.
\end{itemize}

Důkaz.}  Máme
$$
\begin{aligned}
&\int_a^bf(\phi(t))\phi'(t)\d t=
\int_a^b(f_1(\phi(t))+if_2(\phi(t)))(\phi_1'(t)+i\phi_2'(t))\d t=\\
&=\int_a^b(f_1(\phi(t))\phi'_1(t)-f_2(\phi(t))\phi'_2(t))\d t+
i\int_a^b(f_1(t)\phi'_2(t)+f_2(t)\phi'_1(t))\d t=\\
&=\int_a^b(f_1(\phi(t)),-f_2(\phi(t)))(\phi_1(t),\phi_2(t))+
i\int_a^b(f_2(\phi(t)),f_1(\phi(t)))(\phi_1(t),\phi_2(t))
\end{aligned}
$$
(v posledním řádku jde o skalární součin dvojic čísel). Máme tedy
$$
\dots=\sint_L(f_1,-f_2)+i\sint_L(f_2,f_1).
$$
\sq

\vskip10mm

 {\large\bf 3. Greenova věta.}

\bigskip

{\bf 3.1.} Nejprve, jen pro informaci, uvedeme nekterá fakta, která jsou mimo naše technické možnosti. 
V aplikacích následujících kapitolách však budeme potřebovat jen velmi speciální případy, pro které
budeme moci provést důkazy korektně.


\smallskip

Jednoduchá uzavřená křivka $C$ dělí rovinu na dvě souvislé oblasti (slovo ``souvislá'' můžeme chápat tak, že kterékoli dva body té oblasti lze spojit křivkou, slovo ``dělit'' se vztahuje k tomu, že body různých oblastí křivkou spojit nelze), jednu omezenou a druhou neomezenou. To je slavná {\em Jordanova věta}, snadno srozumitelná a
visualisovatelná, ale nesnadno dokazatelná. Ta omezená z nich bude nazývána
 {\em oblast křivky $L$}. Křivka $L$ je její hranice, a uzávěr  té oblasti  $\ol U$  je roven  $U\cup L$ a jelikož je uzavřený a omezený, je kompaktní; o $\ol U$ budeme mluvit jako o  {\em uzavřené oblasti}.

\medskip

Také musíme předpokládat, že rozumíme výrazům ``ve směru nebo proti směru hodinových ručiček'' a ``křivka je orientována proti směru hodinových ručiček''. Tomu je možno dát přesný obecný smysl, ale my to budeme užívat jen pro velmi jednoduché křivky jako kružnice, hranice trojúhelníků a pod., kde význam je zcela jasný. Integrál
přes uzavřenou oblast bude chápán jako integrál přes
 interval $J$ obsahující oblast $M$, a to z funkce jejíž hodnoty jsou na $J\smin M$ doplněny nulami.

\smallskip
 
{\bf 3.1.1. Věta.} (Greenova Věta, Greenova Formule) {\em Buď $L$ jednoduchá uzavřená po částech hladká křivka
orientovaná proti směru hodinových ručiček a buď
 $M$ její uzavřená oblast. Buď  $\ve{f}=(f_1, f_2)$ vektorová funkce taková, že obě $f_j$ mají spojité parciální derivace na
 (otevřené) oblasti křivky $L$. Potom platí
$$
\sint_L\ve{f}=\int_M\left(\pad{f_2}{x_1}-\pad{f_1}{x_2}\right)\d x_1\d x_2 \ .
$$.}

\bigskip

{\bf 3.2. Lemma.} {\em Buď $g:\langle a,b\rangle\to\Rbb$ hladká funkce, buď $f(x)\geq c$ pro všechna $x$. Položme
$$
M=\setof{(x,y)}{a\leq x\leq b, \ c\leq y\leq g(x)}.
$$
Buď $L$ uzavřená křivka tvořící okraj $M$. Potom Greenova formule platí pro $L$ a $M$.


Důkaz.} Pišme  $L=L_1+L_2+L_3+L_4$ jak je popsáno na následujícím obrázku.


\vskip1mm

\centerline{
\xymatrix{(a,g(a))\ar[ddd]_{L_2}&&&\\
&&&\\
&&&(b,g(b))\ar@/_2pc/[uulll]^{L_1}_{y=g(x)}\\
(a,c)\ar[rrr]_{L_3}&&&\ar[u]_{L_4}(b,c)
}}
 \vskip1mm
 
 Parametrizujme křivky $L_j$ předpisy
 $$
 \begin{aligned}
 -&L_1\ : \ \phi_1:\langle a,b\rangle\to\Rbb_2, \ \phi_1(t)=(t,g(t)),\\
 -&L_2\ : \ \phi_2:\langle c,g(a)\rangle\to\Rbb_2, \ \phi_2(t)=(a,t),\\
 &L_3\ : \ \phi_3:\langle a,b\rangle\to\Rbb_2, \ \phi_3(t)=(t,c),\\
 &L_4\ : \ \phi_4:\langle c,g(b)\rangle\to\Rbb_2, \ \phi_4(t)=(b,t).
 \end{aligned}
 $$
 Je tedy $\phi'_1(t)=(1,g'(t))$, $\phi_2'(t)=\phi'_a(t)=(0,1)$ a $\phi'_3(t)=(1,0)$ a máme
 $$
 \begin{aligned}
 &\sint_{L_1}=-\int_a^bf_1(t,g(t))\d t-\int_a^bf_2(t,g(t))g'(t)\d t,\\
 &\sint_{L_2}= -\int_c^{g(a)}f_2(a,t)\d t, \quad \sint_{L_3}=\int_a^bf_1(t,c)\d t,\quad \sint_{L_4}=\int_c^{g(b)}f_2(b,t)\d t.
 \end{aligned}
 $$
Substituujeme $\tau=g(t)$ v druhém integrálu ve formuli pro
 $\sint_{L_1}$ a dostaneme
$$
 \sint_{L_1}=-\int_a^bf_1(t,g(t))\d t+\int_{g(b)}^{g(a)}f_2(h(\tau),\tau)\d \tau
 $$
 kde $h$ je inverse $g$. 

Nyní abychom se připravili na tvrzení lemmatu, začneme 
 první proměnnou psát $x_1$ a druhou $x_2$. Pro $\sint_{L}$ který napíšeme jako součet $\sint_{L_1}+\sint_{L_2}+\sint_{L_3}+\sint_{L_4}$ nyní dostaneme, když přitom píšeme $\int_c^{g(a)}$ ve formuli pro
 $\sint_{L_2}$ jako $\int_c^{g(b)}+\int_{g(b)}^{g(a)}$,
 $$
 \begin{aligned}
 \sint_L&=\int_c^{g(b)}(f_2(b,x_2)-f_2(a,x_2))\d x_2+\int_{g(b)}^{g(a)}(f_2(h(x_2),x_2)-f_2(a,x_2))\d x_2-\\
 &\quad-\int_a^b(f_1(x_1,g(x_1))-f_1(x_1,c))\d x_1.
 \end{aligned}
 $$
 Pro spočtení dvojrozměrného integrálu rozšíříme hodnoty funkcí $f_j$ na interval $J=\langle a,b\rangle\times\langle c,g(a)\rangle$ hodnotami $0$ v $J\smin M$ a dostaneme
 $$
 \begin{aligned}
 &f_2(b,x_2)-f_2(a,x_2)=\int_a^b\pad{f_2(x_1,x_2)}{x_1}\d x_1,\\ &f_2(h(x_2),x_2)-f(a,x_2)=\int_a^{h(x_2)}\pad{f_2(x_1,x_2)}{x_1}\d x_1=\int_a^{b}\pad{f_2(x_1,x_2)}{x_1}\d x_1,  \ \text{and}\\ 
&f_1(x_1,g(x_1))-f_1(x_1,c)=\int_c^{g(x_1)}\pad{f_1(x_1,x_2)}{x_2}\d x_2=\int_c^{g(a)}\pad{f_1(x_1,x_2)}{x_2}\d x_2
\end{aligned}
$$
takže formule nahoře se transformuje na
$$
\sint_L\ver{f}=\int_c^{g(a)}\left(\int_a^b\pad{f_2(x_1,x_2)}{x_1}\d x_1\right)\d x_2-
\int_a^b\left(\int_c^{g(a)}\pad{f_1(x_1,x_2)}{x_2}\d x_2\right)\d x_1
$$
a tvrzení dostaneme z Fubiniovy věty (XVI.4.1).\sq

\bigskip

{\bf 3.3.} Nyní máme  Greenovu formuli také speciálně pro obdélníky a pravoúhlé trojúhelníky s přeponou po případě zakřivenou. Užitím toho, že  $\sint_L=-\sint_{-L}$ nyní formuli získáme pro kerýkoli obrazec, který může být rozřezán na konečně mnoho z těchto obrazců. Tak např. pri rozkladu jako na následujícím obrázku 

\vskip1mm

\centerline{
\xymatrix@=0.7mm{&&&&\cdot\ar[dddddllll]\ar@<0.5ex>[ddddd]&&&&&&&\\
&&&&&&&&&&\\
&&&&&&&&&&&\\
&&&&&&&&&&\\
&&&&&&&&&&&\\
\cdot\ar[rrrr]&&&&\cdot\ar@<0.5ex>[uuuuu]\ar[rrrrrrr]&&&&&&&\cdot\ar[uuuuulllllll]
}}

\vskip1mm

\noindent zjistíme, že

\medskip

{\bf 3.3.1.} {\em  Greenova formule platí pro každý trojúhelník.}

\medskip

Nebo, rozložíme-li kruh následujícím způsobem

\vskip1mm

\centerline{
\xymatrix@=5mm{&&\cdot\ar@/_1.2pc/[ddll]_{L_2}\ar@<0.5ex>[dd]^{L_{11}}&&\\
&&&&\\
\cdot\ar@/_1.2pc/[ddrr]_{L_3}\ar@<0.5ex>[rr]^{L_{21}} &&\cdot\ar@<0.5ex>[ll]^{L_{32}}\ar@<0.5ex>[uu]^{L_{22}}\ar@<0.5ex>[dd]^{L_{42}}\ar@<0.5ex>[rr]^{L_{12}}&&\cdot\ar@<0.5ex>[ll]^{L_{41}}\ar@/_1.2pc/[uull]_{L_1} \\
&&&&\\
&&\cdot\ar@/_1.2pc/[uurr]_{L_4}\ar@<0.5ex>[uu]^{L_{41}}&&
}}

\vskip3mm

 \noindent zjistíme, že 
 
 \medskip

{\bf 3.3.2.} {\em  Greenova formule platí pro každý kruh.}



 

\noindent(Všimněte si, že pro ''křivé trojúhelníky'' které tu máme by parametrizace z 3.2 nefungovala: funkce $g$ by na jednom z konců neměla požadovanou derivaci. Místo toho můžeme užít např. $\phi(t)=(\cos t, \sin t)$. Nebo, samozřejmě, můžeme kruh rozřezat na víc nž čtyři části.)

\medskip

{\bf 3.3.3. Poznámka.} Ve skutečnosti může být oblast kerékoli po částech hladké jednoduché křivky rozložena na podoblasti pro které formule z  Lemmatu 3.2 vyplývá.  Je to dost názorné, ale v našich dalších úvahách budeme potřebovat jen jednoduché obrazce pro které jsou potřebné rozklady zřejmé, a tak podrobný důkaz obecnějšího tvrzení vynecháváme.





\bigskip

{\bf 3.4. Tvrzení.} {\em Buď $L$ kružnice se středem $c$ a buď $M$ její uzavřená oblast. Buď $\ve{f}$ omezená na $M$, nechť parciální derivace  funkcí $f_j$ existují a jsou spojité na $M\smin\set{c}$, a nechť
$\int_M\left(\pad{f_2}{x_1}-\pad{f_1}{x_2}\right)\d x_1\d x_2$ má smysl. Potom Greenova formule platí.

Důkaz.} 
Označme $K^n$ kružnici se středem  $c$ a poloměrem $\frac1n$ orientovanou {\em ve směru hodinových ručiček}, buď $N(n)$ její oblast. Buď  $n$ dost velké tak aby $K^n$  (a tedy též $N(n)$) byla obsažená v $M$. V následujícím obrázku vezměme

\vskip1mm

\centerline{
\xymatrix@=2mm{&&&&&&\cdot\ar@/_2.1pc/[ddddddllllll]_{L_2}\ar@<0.5ex>[ddddd]^{L_{11}^n}&&&&&&\\
&&&&&&&&&&&&\\
&&&&&&&&&&&&\\
&&&&&&&&&&&&\\
&&&&&&&&&&&&\\
&&&&&&\cdot\ar@/^0.4pc/[dr]^{K_1^n}\ar@<0.5ex>[uuuuu]^{L_{22}^n}&&&&&&\\
\cdot\ar@/_2.1pc/[ddddddrrrrrr]_{L_3}\ar@<0.5ex>[rrrrr]^{L_{21}^n}&&&&&\cdot\ar@<0.5ex>[lllll]^{L_{32}^n}\ar@/^0.4pc/[ur]^{K_2^n}&\cdot &\ar@/^0.4pc/[dl]^{K_4^n}\ar@<0.5ex>[rrrrr]^{L_{12}^n}\cdot&&&&&\cdot\ar@<0.5ex>[lllll]^{L_{41}^n}\ar@/_2.1pc/[uuuuuullllll]_{L_1} 
\\ 
&&&&&&\cdot\ar@<0.5ex>[ddddd]^{L_{42}^n}\ar@/^0.4pc/[ul]^{K_3^n}\\
&&&&&&&&&&&&\\
&&&&&&&&&&&&\\
&&&&&&&&&&&&\\
&&&&&&&&&&&&\\
&&&&&&\cdot\ar@/_2.1pc/[uuuuuurrrrrr]_{L_4}\ar@<0.5ex>[uuuuu]^{L_{31}^n}&&&&&&
}}

\vskip1mm

\noindent (proti směru hodinových ručiček orientované) jednoduché uzavřené křivky $\wt L^n_k=L_k+L_{k1}^n+K_k^n+L_{k2}^n$ s oblastmi $M_k(n)$. Pro tyto křivky Greenova formule zřejmě platí.
 

\vskip1mm

\centerline{
\xymatrix@=2mm{&&&\cdot\ar@/_1.5pc/[dddlll]\\
&&&\\
&&&\cdot\ar[uu]\\
\cdot\ar[rr]&&\cdot\ar@/^0.6pc/[ur]&
}}

\vskip1mm

\noindent a máme 
\begin{equation}
\sint_{\wt L^n_k}\ve{f}=\int_{M_k(n)}\left(\pad{f_2}{x_1}-\pad{f_1}{x_2}\right).\tag{$*$}
\end{equation}
Podle 2.3 je 
\begin{equation}
\sint_{\wt L^n_1}+\sint_{\wt L^n_2}+\sint_{\wt L^n_3}+\sint_{\wt L^n_4}=\sint_L+\sint_{K^n}. \tag{$**$}
\end{equation}
Vezměme $V=V(x_1,x_2)$. Předpokladame-li, že Riemannův integrál $\int_MV(x_1,x_2)$ existuje, $V$ je omezená, tedy  $|V(x_1,x_2)|<A$ pro nějaké $A$.
 Jelikož je $N(n)\sue\langle c-\frac1n,c+\frac1n\rangle\times\langle c-\frac1n,c+\frac1n\rangle$ máme 
$$
\left|\int_{N(n)}V\right|<\epsilon \ \ \text{pro dost velké $n$}.
$$
$\ve{f}$ je omezená podle předpokladu a tedy též máme ( $-K^n$ můžeme parametrizovat třeba jako $\phi(t)+\frac1n(\cos t,\sin t)$)
$$
\left|\sint_{K^n}\ve{f}\right|<\epsilon \ \ \text{pro dost velká $n$}.
$$
Nyní máme podle ($*$) a ($**$)
 $$
 \sint_L+\sint_{K^n}=\int_{M_1(k)}V+\int_{M_2(k)}V+\int_{M_3(k)}V+\int_{M_4(k)}V=\int_{M}V-\int_{N(k)}V
 $$
 a tedy
 $$
 |\sint_L\ve{f}-\int_MV|\leq\sint_{K^n}+\int_{N(k)}V
 $$
 a jelikož pravá strana je libovolně malá dostáváme naše tvrzení.\sq

\medskip

{\bf 3.4.1. Poznámka.} 1. Tvrzení 3.4 je jen velmi speciální případ obecného faktu. Platí  pro libovolnou po částech hladkou jednoduchou uzavřenou křivku $L$
s oblastí $M$ a výjimešným bodem $c\in M$.

2.  Omezenost funkce $\ve{f}$ je podstatná, jak uvidíme dále v
XXII.4.1.


\newpage

\centerline{\Large\bf XXII. Základy komplexní analysy} 
 
\vskip10mm

{\large\bf 1. Komplexní derivace.}
 
 \bigskip
 
 {\bf 1.1.} V tělese $\Cbb$ komplexních čísel máme nejen aritmeticcké operace, ale také metrickou strukturu 
dovolující mluvit o limitách. Takže, je-li dána funkce $f$ definovaná v okolí $U\sue\Cbb$
 bodu $z$ můžeme se ptát existuje-li limita
$$ 
 \lim_{h\to 0}\frac{f(z+h)-f(z)}{h}.
 $$ 
 Pokud ano, budeme mluvit o {\em derivaci} funkce $f$ v $z$, a označovat získanou hodnotu
 $$
 f'(z), \quad \frac{\d f(z)}{\d z}, \quad \frac{\d f}{\d z}{z}, \qtq{atd.,}
 $$
podobně jako v reálném kontextu. Tak např. jako pro reálnou mocninu $x^n$
máme
 $$
 \begin{aligned}
  (z^n)'&= \lim_{h\to 0}\frac{(z+h)^n- z^n}{h}=\lim_{h\to 0}\frac{\sum_{k=1}^n\binom{n}{k}x^{n-k}h^k}{h}=\\
 &=\lim_{h\to 0}(nz^{n-1}+h\sum_{k=2}^n\binom{n}{k}x^{n-k}h^{k-2})\ = \ nz^{n-1}.
 \end{aligned}
 $$
 
 \medskip
 
 Podobně jako v VI.1.5 platí
 
 \medskip
 
 {\bf 1.1.2. Tvrzení.} {\em Funkce $f$ má derivaci $A$ v bodě $z\in\Cbb$ právě když existuje pro dost malé  $\delta>0$ komplexní funkce $\mu:\setof{h}{[h|<\delta}\to\Cbb$  taková, že
\begin{enumerate}
\item $\lim_{h\to 0}\mu(h)=0$, a
\item pro $0<|h|<\delta$,
$$
f(z+h)-f(z)=Ah+\mu(h)h.
$$
\end{enumerate}
($|h|$ je samozřejmě absolutní hodnota v $\Cbb$).}

\smallskip

(Skutečně, podobně jako v VI.1.5, jestliže $A=\lim_{h\to 0}\frac{f(z+h)-f(z)}{h}$ existuje,  m\'a
$
\mu(h)=\frac{f(x+h)-f(x)}{h}- A
$
 požadované vlastnosti, a
existuje-li $\mu$ taková, že (1) a (2) potom máme pro malá $|h|$,
$
\frac{f(z+h)-f(x)}{h}=A+\mu(h)
$,
a limita $f'(x)$ existuje a je rovna $A$.)

\medskip

{\bf 1.1.3. Důsledek.} {\em Nechť má $f$ derivaci v $z$. Potom je v tomto bodě spojitá.}

\bigskip

{\bf 1.2. Trochu překvapující příklad.} Tvrzení 1.1.2 se zdá naznačovat, že podobně jako v reálném případě je možno existenci derivace interpretovat jako ``geometrickou tečnu'' ktrá vyjadřuje jakousi hladkost. Ale ve skutečnosti je to mnohem výlučnější vlastnost.

 Vezměme $f(z)=\ol z$ (číslo komplexně sdružené) a počítejme derivaci. Píšeme-li  $h=h_1+ih_2$ dostaneme
 $$
 \frac{\ol{z+h}-\ol z}{h}=\frac{\ol{z}+\ol{h}-\ol z}{h}=\frac{\ol{h}}{h}=\begin{cases} 1\ \ \text{pro}\ \ h_1\neq 0=h_2\\
                                                               -1\ \ \text{pro}\ \ h_1= 0\neq h_2.
\end{cases} 
$$                                                              
 Limita $\lim_{h\to 0}\frac{\ol{z+h}-\ol z}{h}$ tedy neexistuje a naše $f$ nemá limitu v žádném $z$, 
zatím co sotva nějaké zobrazení $\Cbb\to\Cbb$ je hladší než toto zrdcadlení podél reálné osy.
 
 \bigskip
 

 {\bf 1.3.} {\em Komplexní parciální derivace}, označované
 $$
 \pad{f(x,\zeta)}{z} \qtq{resp.} \pad{f(x,\zeta)}{\zeta}
 $$ 
 jsou stejně jako v reálném kontextu derivace funkcí získaných fixováním $\zeta$ resp. $z$.
 
 
 
 \vskip10mm
 
 {\large\bf 2. Cauchy-Riemannovy podmínky.}
 
 \bigskip
 
 Pišme komplexní $z$ jako $x+iy$ s reálnými $x,y$ a vyjádřeme funkci $f(z)$ jedné komplexní proměnné jako dvě reálné funkce dvou reálných proměnných 
 $$
 f(z)=P(x,y)+iQ(x,y).
 $$
 
 \medskip
 
 {\bf 2.1. Věta.} {\em Nechť má $f$ derivaci v $z=x+iy$. Potom $P$ a $Q$ mají parciální derivace v $(x,y)$ a ty vyhovují rovnicím
 $$
 \pad{P}{x}(x,y)=\pad{Q}{y}(x,y)\qtq{a}\pad{P}{y}(x,y)=-\pad{Q}{x}(x,y).
 $$
 Pro derivaci $f'$ potom máme formuli
 $$
 f'=\pad{P}{x}+i\pad{Q}{x}=\pad{Q}{y}-i\pad{P}{y}.
 $$
 
 Důkaz.} Máme
 $$
 \begin{aligned}
 \frac1h(f(z+h)-f(z))& =\frac1{h_1+ih_2}(P(x+h_1,y+h_2)-P(x,y))+\\
 &\quad\quad +i\frac{1}{h_1+ih_2}(Q(x+h_1,y+h_2)-Q(x,y)).
 \end{aligned}
 $$
Existuje-li limita $L=\lim_{h\to 0}\frac1h(f(z+h)-f(z))$ existují speciálně také limity
 $L=\lim_{h_1\to 0}\frac1{h_1}(f(z+h_1)-f(z))$ a $L=\lim_{h_2\to 0}\frac1{ih_2}(f(z+ih_2)-f(z))
 =-i\lim_{h_2\to 0}\frac1{h_2}(f(z+ih_2)-f(z))$. Tedy je
 $$
 \begin{aligned}
 L&=\lim_{h_1\to 0}\frac1{h_1}(P(x+h_1,y)-P(x,y))+i\lim_{h_1\to 0}\frac1{h_1}(Q(x+h_1,y)-Q(x,y))=\\
  &=\pad{P}{x}(x,y)+i\pad{Q}{x}(x,y)
  \end{aligned}
  $$ 
  a v druhém případě
  $$
 \begin{aligned}
 L&=-i\lim_{h_2\to 0}\frac1{h_2}(P(x,y+h_2)-P(x,y))+i\lim_{h_2\to 0}\frac1{ih_2}(Q(x,y+h_2)-Q(x,y))=\\
  &=\pad{Q}{y}(x,y)-i\pad{P}{y}(x,y).
  \end{aligned}
  $$ \sq
  
  \medskip
  
  {\bf 2.1.1.}  (Parciální diferenciální) rovnice
  $$
 \pad{P}{x}=\pad{Q}{y}\qtq{a}\pad{P}{y}=-\pad{Q}{x}
 $$
se nazývají {\em Cauchy-Riemannovy rovnice} nebo  {\em Cauchy-Riemannovy podmínky}. Dokázali jsme, že jsou pro existenci derivace nutné. Nyní ukážeme, že  budeme-li navíc požadovat spojitost, jsou i postačující. 
 
 \bigskip
 
 {\bf 2.2. Věta.} {\em Nechť komplexní funkce $f(z)=P(x,y)+iQ(x,y)$ splňuje v otevřené $U\sue\Cbb$  Cauchy-Riemannovy podmínky a nechť jsou všechny zúčastněné parciální derivace  spojité v $U$. Potom má $f$ derivaci v  $U$.
 
 Důkaz.} Podle věty o střední hodnotě pro reálné derivace  je pro vhodná čísla $0<\alpha,\beta,\gamma,\delta<1$,
 $$
 \begin{aligned}
 &\frac1h(f(z+h)-f(z))=\\
 &\quad=\frac1h(P(x+h_1,y+h_2)-P(x,y)+i(Q(x+h_1,y+h_2)-Q(x,y)))=\\
 &\quad=\frac1h(P(x+h_1,y+h_2)-P(x+h_1,y)+P(x+h_1,y)-P(x,y))+\\
 &\quad\quad +i\frac1h(Q(x+h_1,y+h_2)-Q(x+h_1,y)+Q(x+h_1,y)-Q(x,y))=\\
 &\quad=\frac1h\Big(\pad{P(x+h_1,y+\alpha h_2)}{y}h_2+\pad{P(x+\beta h_1,y)}{x}h_1 +\\
 &\quad\quad+
 i\pad{Q(x+h_1,y+\gamma h_2)}{y}h_2+i\pad{Q(x+\delta h_1,y)}{x}h_1 \Big)
 \end{aligned}
 $$
 a použijeme-li Cauchy-Riemannovy podmínky můžeme pokračovat
 $$
 \begin{aligned} 
 &\dots=\frac1h\Big(-\pad{Q(x+h_1,y+\alpha h_2)}{x}h_2+\pad{P(x+\beta h_1,y)}{x}h_1 +\\
 &\quad\quad+
 i\pad{P(x+h_1,y+\gamma h_2)}{x}h_2+i\pad{Q(x+\delta h_1,y)}{x}h_1 \Big)=\\
 &=\pad{P(x+\beta h_1,y)}{x}+F(h_1,h_2,\beta,\gamma)\frac{i h_2}{h}+
 i\pad{Q(x+\delta h_1,y)}{x}+G(h_1,h_2,\alpha,\delta)\frac{h_2}{h} 
 \end{aligned}
 $$
kde
 $$
 \begin{aligned}
 &F(h_1,h_2,\beta,\gamma)=\pad{P(x+h_1,y+\gamma h_2)}{x}-\pad{P(x+\beta h_1,y)}{x}\ \ \text{a}\\
 &G(h_1,h_2,\alpha,\delta)=\pad{Q(x+h_1,y+\alpha h_2)}{x}-\pad{Q(x+\delta h_1,y)}{x}.
 \end{aligned}
 $$
 Jelikož $|h_2|\leq|h|$ a $F(\cdots)$ a  $G(\cdots)$ konvergují k nule  pro $h\to 0$ podle spojitosti, výraz konverguje k $\pad{P}{x}(x,y)+i\pad{Q}{x}(x,y)$.\sq
 
 \bigskip
 
 {\bf 2.3.} Komplexní funkce $f:U\to\Cbb$, $U\sue\Cbb$ se spojitými parciálními rovnicemi splňujícími   Cauchy-Riemannovy podmínky se nazývají  {\em funkce holomorfní} (v $U$).

 \vskip10mm

\newpage
 
 {\large\bf 3. Víc o komplexním křivkovém integrálu.}

\medskip

{\large\bf \hskip7mm Primitivní funkce.}

\bigskip

Připomeňte si komplexní křivkový integrál z XXI.2.5.2
\begin{equation}
\int_Lf(z)\d z=\int_a^bf(\phi(t))\cdot\phi'(t)\d t \tag{$*$}
\end{equation}
a jeho representaci jako křivkového integrálu druhého druhu (XXI.2.5.3)
$$
\int_Lf(z)\d z=\sint_L(f_1,-f_2)+i\sint_L(f_2,f_1).
$$

\bigskip

{\bf 3.1. Věta.} {\em Nechť je $f(z,\gamma)$ spojitá komplexní funkce dvou komplexních proměnných definovaná v $V\times U$ kde $U$ je otevřená, a nechť je pro každé pevné
$z\in V$ funkce $f(z,-)$  holomorfní v $U$. Buď $L$ po částech hladká orientovaná křivka ve
$V$. Potom je pro $\gamma\in U$
$$
\frac{\d}{\d \gamma}\int_Lf(z,\gamma)\d z=\int_L\pad{f(z,\gamma)}{\gamma}\d z.
$$

Důkaz.} Pišme $z=x+iy$, $\gamma=\alpha+i\beta$ a
$$
f(z,\gamma)=P(x,y,\alpha,\beta)+iQ(x,y,\alpha,\beta).
$$
Podle XXI.2.5.3 máme pro $F(\gamma)
=\int_Lf(z,\gamma)\d z$ z definice komplexního křivkového integrálu
$$
F(\gamma)=\Pc(\alpha,\beta)+i\Qc(\alpha,\beta)
$$
kde
$$
\begin{aligned}
&\Pc(\alpha,\beta)=\sint_L(P(x,y,\alpha,\beta),-Q(x,y,\alpha,\beta)),\\
&\Qc(\alpha,\beta)=\sint_L(Q(x,y,\alpha,\beta),P(x,y,\alpha,\beta)).
\end{aligned}
$$
Jelikož je $f$ holomorfní v $\gamma$, splňuje rovnice
$\pad{P}{\alpha}=\pad{Q}{\beta}$ a $\pad{P}{\beta}=-\pad{Q}{\alpha}$
a z definice komplexního křivkového integrálu
a jeho vyjádření podle XXI.2.5.3, a z
XVIII.2.4.2, dostáváme že
\begin{equation}
\begin{aligned}
&\pad{\Pc}{\alpha}=\sint_L\left(\pad{P}{\alpha},-\pad{Q}{\alpha}\right)=
                    \sint_L\left(\pad{Q}{\beta},\pad{P}{\beta}\right)=\pad{\Qc}{\beta},\\
&\pad{\Pc}{\beta}=\sint_L\left(\pad{P}{\beta},-\pad{Q}{\beta}\right)=
                   - \sint_L\left(\pad{Q}{\alpha},\pad{P}{\alpha}\right)=-\pad{\Qc}{\alpha}
\end{aligned}  \tag{$*$}
\end{equation}
a tedy je funkce  $F(\gamma)$ holomorfní v $U$. Užitím formule pro derivaci z 2.1 konečně dostáváme, že
$$
\int_L\pad{f(z,\gamma)}{\gamma}\d z=
\sint_L\left(\pad{P}{\alpha},-\pad{Q}{\alpha}\right)+i\sint\left(\pad{Q}{\alpha},\pad{P}{\alpha}\right)=
\pad{\Pc}{\alpha}+i\pad{\Qc}{\alpha}=\frac{\d F}{\d \gamma}.
$$ \sq
\bigskip

{\bf 3.2. Věta.} {\em Buď $L$ po částech hladká orientovaná křivka parametrizovaná pomocí $\phi$ a buďte $f_n$ spojité komplexní funkce definované (aspoň) na $L$. Pokud $f_n$ stejnoměrně konvergují k  $f$ platí
$$
\int_Lf=\lim_n\int_Lf_n.
$$
Speciálně pokud $\sum_{n=1}^\infty g_n$ je stejnoměrně konvergující řada spojitých funkcí definovaných na $L$ platí
$$
\int_L\left(\sum_{n=1}^\infty g_n\right)=\sum_{n=1}^\infty \int_Lg_n.
$$

Důkaz.}  Jelikož je $\phi$ po částech hladká je $\phi$ na $L$ omezená,  dejme tomu číslem $A$. Následkem toho je
$$
\begin{aligned}
|f_n(\phi(t))&\cdot\phi'(t)-f(\phi(t))\cdot\phi'(t)|=
|(f_n(\phi(t))-f(\phi(t)))\cdot\phi'(t)|=\\
&=|f_n(\phi(t))-f(\phi(t))|\cdot|\phi'(t)|\leq
|f_n(\phi(t))-f(\phi(t))|\cdot A
\end{aligned}
$$
a tedy $f_n\rrar f$ implikuje, že $(f_n\circ\phi)\cdot\phi'\rrar
(f\circ\phi)\cdot\phi'$ a můžeme užít
 XVIII.4.1 a formuli $(*)$.

Pro druhé tvrzení si nyní stačí uvědomit, že
$\int_L(f+g)=\int_Lf+\int_Lg$. \sq
 
 \bigskip

Následující věta bude formulována (podobně jako XXI.3.1) v obecnosti v níž ji ve skutečnosti nebudeme dokazovat. Budeme ji však užívat jen pro křivky se snadno rozdělitelnými oblastmi
 (vzpomeňťe si na XXI.3.3 a dále až do XXI.3.4.1) pro které přesné důkazy máme. 

\medskip

{\bf 3.3. Věta.} 1. {\em Nechť má $f$ derivaci na otevřené množině $U\sue\Cbb$
a nechť L je orientovaná po částech hladká jednoduchá uzavřená křivka jejíž oblast je obsažena v
 $U$. Potom je
$$
\int_Lf(z)\d z=0.
$$}

2. {\em Tato formule platí též v případě, že  $f$ není definována v jednom bodě
 oblasti křivky pokud je $f$ omezená.

Důkaz.} Podle XXI.2.5.3 máme pro $f(z)=P(x,y)+iQ(x,y)$, 
$$
\int_Lf=\sint_L(P,-Q)+i\sint_L(Q,P)
$$
a podle Greenovy formule (ať již máme na mysli situaci z bodu 1
 či z bodu 2)
 dostáváme
$$
\int_Lf=\int_M\left(-\pad{Q}{x}-\pad{P}{y}\right)+i\int_M\left(\pad{P}{x}-\pad{Q}{y}\right)=0
$$
protože podle  Cauchy-Riemannových rovnic jsou funkce pod integrály $\int_M$ nulové. \sq

\bigskip

{\bf 3.4.} Připomeňme si, že podmnožina $U\sue \Cbb$ je  {\em konvexní}
je-li pro libovolné dva body $a,b\in U$ celá úsečka
$\setof{z}{z=a+t(b-a),\ 0\leq t\leq 1}$ obsažená v $U$.

\medskip

Nechť má $f$ derivaci v konvexní otevřené $U$. Zvolme $a\in U$ a pro libovolné $u\in U$ deinujme
$$
L(a,u) 
$$
jako orientovanou křivku parametrizovanou $\phi(t)=a+t(u-a)$. Položme
$$
F(u)=\int_{L(a,u)}f(z)\d z.
$$

\medskip

{\bf 3.4.1. Tvrzení.} {\em Funkce $F$ je primitivní funkce k funkci $f$ v U. To jest, pro každé $u\in U$  (komplexní) derivace $F'(u)$ existuje a je rovna $f(u)$.

Důkaz.} Buď $h$ takové, že $u+h\in U$. Máme po částech hladkou jednoduchou uzavřenou křivku
$$
L(a,u)+L(u,u+h)-L(a,u+h)
$$
a tedy je podle 3.3.1 a XXI.2.4,
$$
F(u+h)-F(u)=\int_{L(a,u+h)}f-\int_{L(a,u)}f= \int_{L(u,u+h)}f.
$$
Užijeme-li parametrizaci $\phi$ jako nahoře (a píšeme-li $f=P+iQ$) dostaneme
$$
\begin{aligned}
\frac1h(&F(u+h)-F(u))=\frac1h\int_0^1f(u+th)\d t=\\
&=\frac1h\int_0^1P(u+th)\d t+i\frac1h\int_0^1Q(u+th)\d t=
P(u+\theta_1h)+iQ(u+\theta_2h)
\end{aligned}
$$
(pro poslední užíváme integrální větu o střední hodnotě  XI.3.3)
a toto konverguje k $f(u)=P(u)+iQ(u)$. \sq

\medskip

{\bf 3.4.2. Poznámka.} Konvexní $U$ jsme brali jen pro pohodlí.
Obecněji to je možno dokázat pro jednoduše souvislé $U$ (``otevřené množiny bez děr''). Místo $L(a,u)$ je možno brát orientované jednoduché oblouky $L$ začínající v $a$ a končící v $u$; integrál přes takovou $L$  závisí jen na $a$ a $u$  (což je důsledek 3.3.1: Pokud se takové dvě křivky
$L_1$, $L_2$ dotknou jen v $a$ a $u$ užijme jednoduchou uzavřenou křivku $L_1-L_2$; ale dá se to dokázat i pro oblouky které se protnou). Pro souvislé, ale ne jednoduše souvislé, $U$ je však situace jiná.

 \vskip10mm

 
 {\large\bf 4. Cauchyova formule.}
 
 \bigskip

{\bf 4.1. Lemma}  {\em Buď $K$ kružnice se středem $z$ a libovolným poloměrem $r$,
orientovaná proti směru hodinových ručiček. Potom je
$$
\int_K\frac{\d \zeta}{\zeta-z}= 2\pi i.
$$

Důkaz.} Parametrizujme $K$ pomocí $\phi(t)=z+r(\cos t+i\sin t)$, $0\leq t\leq 2\pi$.
Potom je $\phi'(t)=r(-\sin t+i\cos t)$ a tedy
$$
\int_K\frac{\d \zeta}{\zeta-z}=\int_0^{2\pi}\frac{r(-\sin t+i\cos t)}{r(\cos t+i\sin t)}\d t=\int_0^{2\pi}i\d t=2\pi i,
$$
jelikož $-\sin t+i\cos t=i(\cos t+i\sin t)$.\sq

\medskip

{\bf 4.1.1. Poznámka.} Srovnejte tuto rovnost s případem s nulou v 3.3.2. Funkce v tomto integrálu je holomorfní všude s výjimkou jediného bodu.  Věta 3.3.2 ale nemůže být užita protože $f$ není omezená v oblasti křivky $K$.

\bigskip

{\bf 4.2. Věta.} (Cauchyova Formule) {\em Nechť má komplexní funkce jedné komplexní proměnné $f$ derivaci v množině $U$ obsahující uzavřenou oblast kružnice $K$se středem $z$ orientované proti směru hodinových ručiček. Potom je
$$
\frac{1}{2\pi i}\int_K\frac{f(\zeta)}{\zeta-z}\d \zeta= f(z).
$$

Důkaz.}   Máme
$$
\begin{aligned}
\int_K&\frac{f(\zeta)}{\zeta-z}\d \zeta= 
\int_K\frac{f(z)}{\zeta-z}\d \zeta+\int_K\frac{f(\zeta)-f(z)}{\zeta-z}\d\zeta=\\
&=f(z)\int_K\frac{\d\zeta}{\zeta-z}+\int_K\frac{f(\zeta)-f(z)}{\zeta-z}\d\zeta=
2\pi i f(z)+\int_K\frac{f(\zeta)-f(z)}{\zeta-z}\d\zeta
\end{aligned}
$$
Funkce $g(\zeta)=\frac{f(\zeta)-f(z)}{\zeta-z}$
je holomorfní pro $\zeta\neq z$. V bodě $z$ má limitu,
totiž derivaci $f'(z)$. Může tedy být doplněna na spojitou funkci, tedy je omezená, a můžeme použít 3.3.2 a vidíme, že integrál je 0.\sq

\medskip

{\bf 4.2.1. Poznámka.} Cauchyova formule hraje v komplexním diferenciálním počtu centrální roli podobnou roli věty o střední hodnotě v reálné analyse. Něco z toho uvidíme v příští kapitole.

\bigskip

{\bf 4.3. Věta.} {\em Má-li komplexní funkce derivaci v okolí bodu $z$ potom v tomto okolí má derivace všech řádů. Konkretněji, máme
$$
f^{(n)}(z)=\frac{n!}{2\pi i}\int_K\frac{f(\zeta)}{(\zeta-z)^{n+1}}\d \zeta.
$$

Důkaz.} To je bezprostřední důsledek Cauchyovy formule a věty 3.1:
stačí opakovaně derivovat za integračním znaménkem.\sq
\medskip

{\bf 4.3.1. Poznámka.} Už jsme si všimli, že existence derivace v komplexním kontextu se liší od derivovatelnosti v reálné analyse. Teď vidíme, že  je to mnohem silnější vlastnost. V další kapitole uvidíme, že jen mocninné řady mají komplexní derivace.


\bigskip

{\bf 4.4. Důsledek.} {\em Funkce $f$ je holomorfní v otevřené množině $U$ právě když
tam má derivaci 
právě když tam má spojité parciální derivace splňující Cauchy-Riemannovy rovnice.

Důkaz.} Má-li $f$ derivaci $f'$ má též druhou derivaci $f''$ a tedy  $f'$ musí být spojitá. Druhá implikace je zřejmá.\sq

\medskip

{\bf 4.4.1. Poznámka.} Jinými slovy, větu 2.2 je možno obrátit. 

Přirozeně vzniká otázka, je-li možno obrátit větu 2.1, t.j., zda postačují samotné Cauchy-Riemannovy podmínky (zda je spojitost automaticky splněna). Odpověď je záporná.

 \bigskip
 
 {\bf 4.5. Tvrzení.} {\em Komplexní funkce má primitivní funkci v konvexní otevřené množině  $U$ právě když tam má derivaci.
 
 Důkaz.} Má-li derivaci, má primitivní funkci podle 3.4.1. Naopak je-li  $F$ funkce primitivní k funkci $f$, má podle 4.3 druhou derivaci $F''=f'$.\sq
 
\medskip
(To je další fakt silně kontrastující s reálnou analysou.)




\newpage

\centerline{\Large\bf XXIII. Několik dalších fakt z komplexní analysy} 
 
\vskip10mm

{\large\bf 1. Taylorova formule.}
 
 \bigskip
 
 {\bf 1.1. Věta.} (Věta o Taylorových řadách v komplexním oboru) {\em Buď $f$ holomorfní v nějakém okolí $V$ bodu $a$. Potom v dostatečně malém okolí  $U$ bodu $a$ může být psána jako řada
 $$
 f(z)=f(a)+\frac1{1!}f'(a)(z-a)+\frac1{2!}f''(a)(z-a)^2+\dots+\frac1{n!}f^{n}(a)(z-a)^n+\dots \ .
 $$
 
 Důkaz.} Máme
 \begin{equation}
 \frac1{\zeta-z}=\frac1{\zeta-a}\cdot\frac1{1-\frac{z-a}{\zeta-a}}. \tag{$*$}
 \end{equation}
 Vezměme kružnici $K$ se středem $a$ a poloměrem $r$ takovým, že celý příslušný kruh (oblast křivky $K$) je obsažen ve $V$. Zvolme $q$ s $0<q<1$ a okolí $U$ bodu $a$ dost malé aby pro $z\in U$ bylo $|z-a|<rq$. Potom máme
 \begin{equation}
 \zeta\in K\ \ \Rightarrow\ \ \left|\frac{z-a}{\zeta-a}\right|< q<1. \tag{$**$}
 \end{equation}
 Nyní získáme pro $x\in U$ z $(*)$ 
 $$
  \frac1{\zeta-z}= \frac1{\zeta-a}\left(\sum_{n=0}^\infty\left(\frac{z-a}{\zeta-a}\right)^n\right)
 $$
 a tedy
 $$
  \frac{f(\zeta)}{\zeta-z}= \sum_{n=0}^\infty\frac{f(\zeta)}{\zeta-a}\left(\frac{z-a}{\zeta-a}\right)^n.
 $$
 Spojitá funkce $f$ je omezená na kompaktní kružnici $K$ takže podle $(**)$ je pro vhodné $A$,
 $$
 \left|\frac{f(\zeta)}{\zeta-a}\left(\frac{z-a}{\zeta-a}\right)^n\right|<\frac{A}{r}\cdot q^n
  $$
  a tedy podle XVIII.4.5 řada $\sum_{n=0}^\infty\frac{f(\zeta)}{\zeta-a}\left(\frac{z-a}{\zeta-a}\right)^n$ stejnoměrně konverguje
  a můžeme užít XXII.3.2 a dostaneme
   $$
  \int_K\frac{f(\zeta)}{\zeta-z}\d \zeta= \sum_{n=0}^\infty\int_K\frac{f(\zeta)}{\zeta-a}\left(\frac{z-a}{\zeta-a}\right)^n\d\zeta= 
 \sum_{n=0}^\infty(z-a)^n\int_K\frac{f(\zeta)}{(\zeta-a)^{n+1}}\d\zeta .
 $$
Užijme Cauchyovu formuli pro první integrál a formuli XXII.4.3 pro poslední. Tím konečně dostaneme
$$
f(z)=\sum_{n=0}^\infty\frac{f^{(n)}(a)}{n!}(z-a)^n.
$$ \sq

\medskip

{\bf 1.1.1. Poznámky.}  1. Takže všechny komplexní funkce s derivacemi v okolích bodů mohou být (lokálně) vyjádřeny mocninnými řadami.

2. Srovnejte důkaz věty 1.1 s jeho protějškem v reálné analyse. Komplexní varianta je vlastně mnohem jednodušší: píšeme prostě $\frac1{\zeta-z}$ jako vhodnou mocninnou řadu a vezmeme integrály jednotlivých sčítanců (jen musíme vědět, že to smíme udělat), a potom aplikujeme Cauchyovu formuli (a její derivace). Samozřejmě, Cauchyova formule je velmi silný nástroj,  ale to není všechno. V reálném oboru dokazujeme svým způsobem obecnější větu: věta hovoří také o té spoustě funkcí, které mají jen několik derivací.

\bigskip

{\bf 1.2. Exponenciela a goniometrické funkce.} Užitím techniky komplexní analysy můžeme nyní dokázat existenci goniometrických funkcí, o nichž jsme dosud jen předpokládali, že existují. Nejprve {\em definujeme} exponenciální funkci pro komplexní proměnnou jako řadu
$$
e^z=\sum_{n=1}^\infty\frac1{n!}z^n.
$$
Tu již máme v reálném kontextu. Existenci (reálného) logaritmu již máme dokázánu (viz XII.4),  $e^x$ je její inverse a může být napsána jako (reálná) Taylorova řada stejnou formulí. 

Budeme potřebovat formuli pro součet $e^{u+v}=e^ue^v$ pro obecné komplexní $u$ a $v$. To je snadné:
$$
\begin{aligned}
e^ue^v&=\left(\sum_{n=0}^\infty\frac1{n!}u^n\right)\left(\sum_{n=0}^\infty\frac1{n!}v^n\right)=
\sum_{n=0}^\infty\left(\sum_{k+r=n}\frac1{k!}u^k\frac1{r!}v^r\right)=\\
&=\sum_{n=0}^\infty\left(\sum_{k=0}^n\frac1{k!}\frac1{(n-k)!}u^kv^{(n-k)}\right)=
\sum_{n=0}^\infty\frac1{n!}\left(\sum_{k=0}^n\frac{n!}{k!(n-k)!}u^kv^{(n-k)}\right)=\\
&=\sum_{n=1}^\infty\frac{1}{n!}\left(\sum_{k=0}^n{{n}\choose{k}}u^kv^{(n-k)}\right)=
\sum_{n=1}^\infty\frac1{n!}(u+v)^n.
\end{aligned}
$$

\medskip

{\bf 1.2.1.} Nyní definujme (pro obecné komplexní $z$)
$$
\begin{aligned}
&\sin z= \frac{e^{iz}-e^{-iz}}{2i}=z-\frac{z^3}{3!}+\frac{z^5}{5!}-\frac{z^7}{7!}+\cdots,\ \ \text{a}\\
&\cos z= \frac{e^{iz}+e^{-iz}}{2i}=1-\frac{z^2}{2!}+\frac{z^4}{4!}-\frac{z^6}{6!}+\cdots.
\end{aligned}
$$
Zřejmě máme
$$
\lim_{z\to 0}\frac{\sin z}{z}=1
$$
a vše co ještě potřebujeme jsou součtové formule. Dokážme dejme tomu formuli pro sinus:
$$
\begin{aligned}
&\sin u\cos v+\sin v\cos u=\frac1{4i}((e^{iu}-e^{-iu})(e^{iv}+e^{-iv})+
(e^{iv}-e^{-iv})(e^{iu}+e^{-iu}))=\\
&=\frac1{4i}(e^{iu}e^{iv}+e^{iu}e^{-iv}
-e^{-iu}e^{iv}-e^{-iu}e^{-iv}+e^{iv}e^{iu}+e^{iv}e^{-iu}
-e^{-iv}e^{iu}-e^{-iv}e^{-iu})=\\
&=\frac1{4i}(2e^{iu}e^{iv}
-2e^{-iu}e^{-iv})=\frac1{2i}(e^{i(u+v)}-
e^{-i(u+v)})=\sin(u+v).
\end{aligned}
$$




\vskip10mm
 
 {\large\bf 2.  Věta o jednoznačnosti.}
 
 \bigskip

{\bf 2.1.} Připomeňte si, že polynomy stupně
 $n$ shodující se v $n+1$ argumentech jsou si rovné. Podíváme-li se na mocninné  řady jako na ``polynomy spočetného supně'' můžeme si na okamžik myslet že dvě  řady shodující se v nekonečně mnoha argumentech se už také budou shodovat všude. Taková hypotéza je okamžitě odmítnuta: vezměte $\sin x$ a konstantní 0.

Ve skutečnosti ale tato hypotéza není až tak úplně špatná. Takové tvrzení totiž platí za předpokladu, že množina bodů v vichž se funkce shodují má hromadný bod
 (připomeňte si XVII.3.1).

\bigskip

{\bf 2.2.} Nejprve dokážeme lokální variantu věty o jednoznačnosti.

\medskip 

{\bf Lemma.} {\em Buďte $f$ a $g$  holomorfní na otevřené množině $U$ a buď $c$ v $U$. Nechť $c_n\neq c$, $c=\lim_n c_n$ a $f(c_n)=g(c_n)$ pro všechna $n$. Potom se $f$ shoduje s $g$ na nějakém okolí bodu $c$.

Důkaz.} Stačí dokázat, že pokud $f(c_n)=0$ pro všechna $n$ je $f(z)=0$
na nějakém okolí bodu $c$.

Jelikož je $c\in U$, derivace $f$ v $c$ existuje a tedy podle 1.1 je v dostatečně malém okolí $V$ bodu $c$
$$
f(z)=\sum_{k=0}^\infty a_k(z-c)^k.
$$
Nechť $f$ není na $V$ konstantně nula, takže některé z $a_k$ musí být nenulové. Buď $a_n$ první z nich. Je tedy
$$
f(z)=(z-c)^n(a_n+ a_{n+1}(z-c)+a_{n+2}(z-c)^2+\cdots)
$$
Řada $g(z)=a_n+ a_{n+1}(z-c)+a_{n+2}(z-c)^2+\cdots$ je spojitá funkce a $g(0)=a_n\neq 0$, a tedy   $g(z)\neq 0$ v nějakém okolí  $W$ bodu $c$, a $f(z)=(z-c)^ng(z)$ je ve $W$ rovna $0$ jen v $c$. Pro dost velké
 $n$ je ale $c_n$ ve $W$ -- spor.\sq

\bigskip

{\bf 2.3. Souvislost: jen několik fakt.} Neprázdný metrický prostor $X$ je {\em nesouvislý} existují-li v něm disjunktní neprázdné otevřené množiny  $U$, $V$ takové, že $X=U\cup V$. Je {\em souvislý} není-li nesouvislý.

Řekneme, že $X$ je {\em obloukově souvislý} jestliže pro kterékoli dva body $x,y\in X$ existuje spojité zobrazení
 $\phi:\langle a,b\rangle\to X$ takové, že $\phi(a)=x$ a
$\phi(b)=y$.

Samozřejmě mluvíme o souvislé či obloukově souvislé  {\em podmnožině} metrického prostoru je li odpovídající {\em podprostor} souvislý či obloukově souvislý.

\medskip

{\bf 2.3.1. Poznámky.} 1. Jsou dobré důvody pro to, aby prázdný prostor byl považován za nesouvislý. Naše prostory ale stejně budou všechny neprázdné.

2. Jelikož uzavřené podmnožiny jsou přesně doplňky otevřených, vidíme, že
$X$ je {\em nesouvislý} právě když existují uzavřené $A$, $B$ takové, že $X=A\cup B$.

3. Oblouková souvislost znamená, samozřejmě, spojení libovolných dvou bodů křivkami, zobecníme-li pojem křivky  v $\Ebb_n$ na libovolný metrický prostor.

4. Víme-li, že prostor  $X$ je souvislý, můžeme dokázat tvrzení $\mathcal V(x)$ o prvcích $x\in X$ tak, že dokážeme, že množina
$$
\setof{x}{\mathcal V(x)\ \text{platí}}
$$
je neprázdná, otevřená a uzavřená.

\medskip

{\bf 2.3.2. Fakt.} {\em Kompaktní interval $\langle a,b\rangle$ je souvislý.

Důkaz.} Předpokládejme, že je $\langle a,b\rangle=A\cup B$ s $A,B$ disjunkními uzavřenými, a buď, dejme tomu, $a\in A$. Položme
$$
s=\sup\setof{x}{\langle a,x\rangle\sue A}.
$$
Jelikož  $x\in A$ mohou být libovolně blízko bodu $s$, a $s\in\ol A=A$. Pokud $s<b$ existuje $x\in B$ libovolně blízko k $s$ což ale  znamená, že $s\in\ol B=B$ ve sporu s disjunktností. Tedy je $s=b$ a $B$ je prázdné.\sq

\medskip

{\bf 2.3.3. Fakt.} {\em Každý obloukově souvislý (neprázdný) prostor je souvislý.

Důkaz.} Nechť je $X$ obloukově souvislý ale ne souvislý. Potom existují
neprázdné otevřené $U$, $V$ takové, že $X=U\cup V$. Vyberme $x\in U$ a $y\in V$. Existuje spojité $\phi:\langle a,b\rangle\to X$ takové, že $\phi(a)=x$ a
$\phi(b)=y$. Potom jsou $U'=\phi^{-1}[U]$, $V'=\phi^{-1}[V]$ neprázdné disjunktní otevřené množiny takové, že $U'\cup V'=\langle a,b\rangle$ ve sporu s 2.3.2. \sq

\medskip

{\bf 2.3.4. Fact.} {\em Otevřená podmnožina prostoru $\Ebb_n$ je souvislá právě když je obloukově souvislá.

Důkaz.} Buď $U\sue\Ebb_n$ neprázdná otevřená. Pro $x\in U$ definujme
$$
U(x)=\setof{y\in U}{\exists \phi:\langle a,b\rangle\to U,\ \phi(a)=x,\ \psi(b)=y}.
$$
Množiny $U(x)$ a $U(y)$ jsou buď disjunktní nebo stejné (je-li $z\in U(x)\cap U(y)$ zvolte orientované křivky $L_1$, $L_2$ spojující $x$ se $z$ a $z$ s $y$; potom $L_1+L_2$ z XXI.1.4 dokazuje, že $y\in U(x)$ a užijeme-li XXI.1.4  znovu, vidíme, že
$U(y)\sue U(x)$).

Dále, každá $U(x)$ je otevřená. Skutečně, buď $y\in U(x)$ a buď $L$ orientovaná křivka spojující $x$ s $y$. Jelikož je $U$ otevřená, existuje $\epsilon>0$ takové,že $\Omega(y,\epsilon)\sue U$. Pro libovolné $z\in
\Omega(y,\epsilon)$ máme orientovanou úsečku $K$ parametrizovanou jako
$\psi=(t\mapsto y+t(z-y)):\langle 0,1\rangle\to\Omega(y,\epsilon)$ a tedy
$L+K$ spojující $x$ se $z$. Je tedy $\Omega(y,\epsilon)\sue U(x)$.

Jestliže nyní $U$ není obloukově souvislá existují $x,y$ takové, že $U(x)\cap U(y)=\ems$,
 množina $V=\bigcup\setof{U(y)}{y\in U,\ U(x)\cap U(y)=\ems}$ je neprázdná otevřená, $U(x)\cup V=U$ a $U$ je tedy není souvislá.
 \sq

\bigskip

{\bf 2.4. Věta.} {\em Buďte $f$ a $g$ holomorfní na souvislé otevřené množině $U$ a nechť existují $c$ a $c_n\neq c$ v $U$ takové, že $c=\lim_n c_n$ a $f(c_n)=g(c_n)$ pro všechna $n$. Potom $f=g$.

Důkaz.} Položme
$$
V=\setof{z}{z\in U, \ f(u)=g(u)\ \text{pro všechna}\ u\ \text{v nějakém okolí bodu}\ z}.
 $$
Potom je  $V$ z definice otevřená, a podle 2.2 a předpokladu o $c$ neprázdná. Jestliže nyní $z_n\in V$ a $\lim_n z_n=z$ je podle 2.2, $z\in V$ takže
$V$ je též uzavřená, a tedy $V=U$ ze souvislosti (viz 2.3.1.4).\sq


\vskip10mm
 
 {\large\bf 3. Liouvilleova věta a 

\medskip

\hskip7mm  Základní věta algebry.}
 
 \bigskip
 
 {\bf 3.1. Lemma.} {\em Buď $f$ komplexní funkce definovaná na kružnici $
K$ s poloměrem $r$. Je-li $|f(z)|\leq A$ pro všechna $z$ je
 $$
 \left|\int_Lf(z)\d z\right|\leq 8A\pi r.
 $$
 
 Důkaz.} Parametrizujme $L$ pomocí $\phi:\langle 0,2\pi\rangle\to\Cbb$
definované předpisem $\phi(t)=c+r\cos t+ir\sin t$ ak\v e $\phi'(t)=-r\sin t+ ir\cos t$
a tedy $|\phi'_1|,|\phi'_1|\leq r$. Buď $f=f_1+if_2$. Potom máme
 $$
 \begin{aligned}
 \left|\int_Lf\right|&=\left|\int_0^{2\pi}f(\phi(t))\phi'(t)\d t\right|=
 \left|\int_0^{2\pi}f_1\phi'_1-\int_0^{2\pi}f_2\phi'_2+i\int_0^{2\pi}f_1\phi'_2-i\int_0^{2\pi}f_2\phi'_1\right|\leq\\
  &\leq\left|\int_0^{2\pi}f_1\phi'_1\right|+\left|\int_0^{2\pi}f_2\phi'_2\right|+\left|\int_0^{2\pi}f_1\phi'_2\right|+\left|\int_0^{2\pi}f_2\phi'_1
  \right|\leq\\
  &\leq
   \int_0^{2\pi}|f_1||\phi'_1|+\int_0^{2\pi}|f_2||\phi'_2|+\int_0^{2\pi}|f_1||\phi'_2|+\int_0^{2\pi}|f_2||\phi'_1|\leq\\
   &\leq 4\int_0^{2\pi}Ar\d t
 =4Ar\int_0^{2\pi}\d t=4Ar2\pi.
 \end{aligned}
 $$
 \sq

\medskip

{\bf Poznámka.} Tento odhad je velmi hrubý, ale pro naše potřeby stačí.

\bigskip


{\em Důkaz.} Podle XXII.4.3 máme pro libovolnou kružnici $K$ se středem $z$
$$
f'(z)=\frac{2!}{2\pi i}\int_K\frac{f(\zeta)}{(\zeta-z)^2}\d \zeta.
$$
Buď $|f(\zeta)|<A$ pro všechna $\zeta$. Zvolíme li kružnici  $K$ s poloměrem  $r$ bude $(\zeta-z)^2=r^2$  pro $\zeta$ na $K$, a tedy
$$
\left|\frac{f(\zeta)}{(\zeta-z)^2}\right|<\frac{A}{r^2}.
$$
Tedy podle lemmatu 3.1,
$$
|f'(z)|<\frac{2!}{2\pi}8\frac{A}{r^2}\pi r=\frac{8A}{r}.
$$
Jelikož  $r$ mohlo být libovolně veliké, je $f'(z)$ konstantně nula, a tedy je $f$ konstanta.\sq

\bigskip

{\bf 3.3. Věta.} (Základní  Věta Algebry) {\em Každý polynom $p$ stupně $\deg(p)>0$ s komplexními koeficienty má komplexní kořen.

Důkaz.} Nechť polynom
$$
p(z)= z^n+a_{n-1}z^{n-1}+\cdots+a_1z+a_0
$$
nemá kořeny. Potom je holomorfní funkce
$$
f(z)=\frac1{p(z)}
$$
definována na celé $\Cbb$. Položme
$$
R=2n\max\set{|a_0|,|a_1|,\dots,|a_{n-1}|,1}. 
$$
Potom pro $|z|\geq R$ máme
$$
\begin{aligned}
|p(z)|&\geq |z|^n-|a_{n-1}z^{n-1}+\cdots+a_1z+a_0(z)|\geq\\
&\geq |z|^n-|z|^{n-1}\frac12 R\geq  R|z|^{n-1}-|z|^{n-1}\frac12 R
=|z|^{n-1}\frac12 R\geq\frac12 R^{n}.
\end{aligned}
$$
Tedy je
$$
|z|\geq R\ \ \Rightarrow\ \ |f(z)|\leq\frac{2}{R^n}.
$$
Konečně, jelikož je množina $\setof{x}{|x|\leq R}$ kompaktní, je spojitá funkce $f$ omezen\'a tak\'e pro $|x|\leq R$, a tedy všude. Tedy je podle Liouvilleovy věty $f$ konstantní a tedy je konstantní také $p$.\sq

\vskip10mm
 
 {\large\bf 4. Poznámky o konformním zobrazení.}
 
 \bigskip
 
 {\bf 4.1.} Připomeňme si z analytické geometrie formuli pro kosinus úhlu
$\alpha$ mezi dvěma (nenulovými) vectory $\ve{u}$ a $\ve{v}$:
$$
\cos\alpha=\frac{\ve{u}\ve{v}}{\|\ve{u}\| \|\ve{v}\|}.
$$
V zhledem k této formuli budeme v této sekci rozumět výraz  ``zachovávání úhlu mezi $\ve{u}$  a $\ve{v}$'' jako zachovávání hodnoty
$\frac{\ve{u}\ve{v}}{\|\ve{u}\| \|\ve{v}\|}$.

\bigskip

{\bf 4.2.} Buď $U$ souvislá otevřená podmnožina $\Cbb$. Budeme se především zajímat o holomorfní funkce
 $f$ a budeme užívat (jako dříve) značení $f(z)=f(x+iy)=P(x,y)+iQ(x,y)$  pro $f:U\to \Cbb$ s parciálními derivacemi. V tomto značení již budeme pokračovat. 

\medskip

{\bf 4.2.1.}  Připomeňte si Jakobián z XV.4 a také to, že zobrazení $f:U\to \Cbb$ s parciálními derivacemi se nazývá regulární jestliže
\begin{equation}
\frac{\Ds(f)}{\Ds(z)}=\frac{\Ds(P,Q)}{\Ds(x,y)}=\det\left(\begin{matrix}\pad{P}{x},\pad{P}{y}
\\ \pad{Q}{x},\pad{Q}{y}
\end{matrix}\right)=\pad{P}{x}\pad{Q}{y}-\pad{Q}{x}\pad{P}{y}\neq 0.
\tag{reg}
\end{equation}

\medskip



{\bf 4.2.2.} Buď nyní $f:U\to\Cbb$ holomorfní. Potom se podle Cauchy-Riemannových podmínek formule (reg) transformuje na
$$
\pad{P}{x}\pad{Q}{y}-\pad{Q}{x}\pad{P}{y}=\pad{P}{x}^2+\pad{P}{y}^2=
\pad{Q}{x}^2+\pad{Q}{y}^2
$$
a vidíme, že

\medskip

 {\em holomorfní $f$ je regulární  na otevřené množině   $U$ právě když pro
všechna $z\in U$ je $f'(z)\neq 0$.}

\bigskip

{\bf 4.3.} Řekneme, že $f:U\to\Cbb$ je
 {\em konformní zobrazení} je-li regulární a zachovává-li úhly, čímž máme na mysli úhly mezi tečnými vektory křivek transformovaných zobrazením  $f$.

Ukážeme, že konformní zobrazení úzce souvisejí s holomorfními. 

\bigskip


{\bf 4.4.} Buďte $\phi$, $\psi$ křivky v $U$. Regulární zobrazení
$f:U\to\Cbb$ je transformuje do křivek
$$
\Phi=f\circ\phi\qtq{a}\Psi=f\circ\psi
$$
v rovině $\Cbb$.

\medskip


{\bf 4.4.1. Lemma.} {\em Nechť je  $f$ holomorfní. Potom máme pro skalární součin
 $\ve{u}\ve{v}$ tečných vektorů (zde tečka $\cdot$ znamená násobení reálných čísel)
$$
\Phi'\Psi'=\frac{\Ds(f)}{\Ds(z)}\cdot\phi'\psi'.
$$

Důkaz.} S použitím Cauchy-Riemannových podmínek dostaneme
$$
\begin{aligned}
\Phi'_1\Psi'_1+\Phi'_2\Psi'_2&=
\left(\pad{P}{x}\phi'_1+\pad{P}{y}\phi'_2\right)\left(\pad{P}{x}\psi'_1+\pad{P}{y}\psi'_2\right)+\\
&+\left(-\pad{P}{y}\phi'_1+\pad{P}{x}\phi'_2\right)\left(-\pad{P}{y}\psi'_1+\pad{P}{x}\psi'_2\right)=\\
&=\left(\phi'_1\psi'_1+\phi'_2\psi'_2\right)\left(\left(\pad{P}{x}\right)^2+\left(\pad{P}{y}\right)^2\right).
\end{aligned}
$$\sq

\medskip


{\bf 4.4.2. Věta.} {\em Holomorfní zobrazení  $f:U\to\Cbb$ takové, že $f'(z)\neq 0$ pro všechna $z\in U$ je konformní.

Důkaz.} Z lemmatu 4.4.1  zjišťujeme též, že pro normu platí $\|\Phi'\|^2= \Psi'\Psi'=
\frac{\Ds(f)}{\Ds(z)}\cdot\phi'\phi'=\frac{\Ds(f)}{\Ds(z)}\|\phi'\|^2$ takže
$$
\frac{\Phi'\Psi'}{\|\Phi'\| \|\Psi'\|}=
\frac{\frac{\Ds(f)}{\Ds(z)}\phi'\psi'}{\sqrt{\frac{\Ds(f)}{\Ds(z)}}\|\phi'\|\sqrt{\frac{\Ds(f)}{\Ds(z)}} \|\psi'\|}=\frac{\phi'\psi'}{\|\phi'\| \|\psi'\|}.
$$
Připomeňte si 4.1. \sq

\medskip 

{\bf Poznámka.} Podmínka regularity, t.j., že $f'(z)\neq 0$, je podstatná.
Například zobrazení $f(z)=z^2$ úhly v  $z=0$ zdvojnásobuje.

\bigskip

{\bf 4.5.} Je, naopak, konformní zobrazení nutně holomorfní? Ne, například zobrazení
$$
\text{\sf conj}=(z\mapsto\ol z):\Cbb\to\Cbb
$$
je konformní (dokonce isometrické), ale holomorfní není (viz XXII.1.2).
Byla by to ale trochu laciná odpověď, kdybychom to na tom nechali.
Ve skutečnosti se nic horšího než intervence zobrazení {\sf conj} nemůže stát.
Platí

\medskip

{\bf Věta.} {\em Buď $U$ otevřená podmnožina roviny $\Cbb$ a buď $f:U\to \Cbb$ regulární zobrazení. Potom jsou následující tvrzení ekvivalentní.
\begin{enumerate}
\item $f$ je konformní.
\item $f$ zachovává kolmost.
\item Buď $f$ nebo {\sf conj}$\ \circ f$ je holomorfní.
\end{enumerate}

Důkaz.} (1)$\Rightarrow$(2) je triviální a (3)$\Rightarrow$(1) je v 4.4.2 (modifikace zobrazením\ {\sf conj}\ je zřejmá).

\smallskip

(2)$\Rightarrow$(3): Pišme $(u,v)$ pro tečný vektor $\phi'(t)$ parametrizace křivky $\phi$. Po transformaci zobrazením  $f$ z něj dostaneme
$$
\left(\pad{P}{x}u+\pad{P}{y}v,\pad{Q}{x}u+\pad{Q}{y}v\right).
$$
Uvažujme nyní pro $(u,v)$ dva kolmé vektory $(a,b)$ a $(-b,a)$. Potom skalární součin transformovaných vektorů
$$
\begin{aligned}
&\left(\pad{P}{x}a+\pad{P}{y}b,\pad{Q}{x}a+\pad{Q}{y}b\right)
\left(-\pad{P}{x}b+\pad{P}{y}a,-\pad{Q}{x}b+\pad{Q}{y}a\right)=\\
&=(a^2-b^2)\left(\pad{P}{x}\pad{P}{y}+\pad{Q}{x}\pad{Q}{y}\right)+\\
&\quad\quad\quad\quad\quad\quad+ab\left(\left(\pad{P}{y}\right)^2+\left(\pad{Q}{y}\right)^2-\left(\pad{P}{x}\right)^2-\left(\pad{Q}{x}\right)^2\right)
\end{aligned}
$$
by měl být nula. Speciálně pro $(a,b)=(1,0)$ to dá
\begin{equation}
\pad{P}{x}\pad{P}{y}+\pad{Q}{x}\pad{Q}{y}=0
\tag{1}
\end{equation}
a pro $(a,b)=(1,1)$ dostaneme
\begin{equation}
\left(\pad{P}{y}\right)^2+\left(\pad{Q}{y}\right)^2-\left(\pad{P}{x}\right)^2-\left(\pad{Q}{x}\right)^2=0.\tag{2}
\end{equation}
Nyní, protože $f$ je regulární, některá z parciálních derivací, dejme tomu $\pad{Q}{x}(z)$, je nenulová (v argumentu na který se soustřeďujeme). Položme
$$
\lambda=\pad{P}{x}\left(\pad{Q}{x}\right)^{-1}
$$ 
takže máme $\pad{P}{x}=\lambda\pad{Q}{x}$ a rovnice (1) dává
$
\lambda\pad{P}{y}+\pad{Q}{y}=0
$, 
a po substituci těchto dvou rovnic do (2) získáme
$$
(1+\lambda^2)\left(\pad{P}{y}\right)^2=(1+\lambda^2)\left(\pad{Q}{x}\right)^2
$$
a jelikož $\lambda$ je reálné, $1+\lambda^2\neq0$ vidíme, že
$$
\left(\pad{P}{y}\right)^2=\left(\pad{Q}{x}\right)^2.
$$
Nyní je buď $\pad{P}{y}=-\pad{Q}{x}$ a potom z (1) plyne, že
 $\pad{P}{x}=\pad{Q}{y}$, a $f$ splňuje Cauchy-Riemannovy podmínky;
jelikož ty parciální derivace jsou spojité je
$f$ holomorfní. Nebo je $\pad{P}{y}=\pad{Q}{x}$ a pak (1) dává
$\pad{P}{x}=-\pad{Q}{y}$. Potom  podle řetězcového pravidla ${\sf conj}\circ f$ splňuje Cauchy-Riemannovy podmínky a tedy je holomorfní. \sq






\end{document}



 
 
 
 





 
 
 


